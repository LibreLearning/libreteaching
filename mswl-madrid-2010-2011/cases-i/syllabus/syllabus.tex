\documentclass[a4paper]{article}
%\usepackage[T1]{fontenc}
\usepackage[utf8]{inputenc}
\usepackage{url}
\usepackage[hypertex,colorlinks]{hyperref}
%\usepackage{html}
%\usepackage{hthtml}
\usepackage{geometry}

% Comments (optional argument is author of comment)
\newcommand{\comments}[2][?]{
  \begin{quote}
    \textbf{Comment (#1):} {\em #2}
  \end{quote}
  }

% \name is for ``special names'', like procedure or variable names.
\newcommand{\name}[1]{\texttt{\hturl{#1}}}

% Just a shortcut for links where the url appears as footnote.
\newcommand{\htflink}[2]{\htmladdnormallinkfoot{#1}{#2}}

\title{MSWL Case Studies I \\
Master on libre software \\
URJC - GSyC/Libresoft \\
\url{http://master.libresoft.es}}

\author{Santiago Dueñas}
\date{December 2010}

\sloppy
\begin{document}
\maketitle

\begin{abstract}
Course syllabus and learning program for the course ``Case Studies I'', 
of the Master on libre software of the Universidad Rey Juan Carlos (Móstoles, Spain).

[This is an evolving document, until the course is finished and graded]
\end{abstract}

\tableofcontents

\section{Grading plan}

Each activity contributing to the grading of the course has its own evaluation criteria, 
as described below. Each of these activities has a minimum and maximum grading. If the 
minimum grading is 0, the activity is \textit{optional}. Otherwise, the activity is \textit{mandatory}, and 
has to be graded al least with the minimum to pass the course. Each activity has also a 
description, and when possible, some general grading criteria. In any case, the final grade 
for the course will also depend on the continuous observation of the instructors on the 
outcomes and progress of students.

Students should ask instructors about any detail which may not be clear to them, 
either about the general grading plan, or about specific aspects of the activities. 
As a general rule, evaluation will have into account how the activity and its results 
show that the student has come close to the competences, knowledge and skills expected 
for the course.

The student can consider that the next table will be used as a (minimum) guideline for 
assigning marks:

\begin{itemize}
\item Pass (``aprobado''): 50
\item Good (``notable''): 75
\item Excellent (``sobresaliente''): 100
\end{itemize}

\subsection{Graded activities}

\begin{itemize}

\item \textbf{Blog entries}. \\
  Minimum: 50 points, maximum: 100 points

  Blog entries specifically related to the course such as opinion articles, experiences, 
recipes or other similar topics. A minimum of 5 articles is required. The tag used to mark
these entries is \textit{mswl-cases}.

 Note: Entries can be reused from other courses: ``Introduction'', ``Development and Tools'',
``Legal Aspects'', ``Developers and Motivation'', ``Economic Aspects'' and ``Project Evaluation''. 
Please make sure to remark these entries as described above.

\item \textbf{Write a paper}. \\
  Minimum: 0 points, maximum: 50 points

  Paper related to any of the topics dealt with in this course. If you are really interested, ask the teachers for more information.

\end{itemize}

\subsection{Deadline}

The deadline for completing all the activities, described on the previuos section, is Monday, January 3, 2011.

\section{Sessions}

\subsection{Session 1 (Fri. September 17, 2010)}

\begin{itemize}
 \item \textbf{Speaker}: Jesús M. González-Barahona

 \item \textbf{Content}:

  \begin{enumerate}
   \item Tools and systems useful for the master's program
   \item Introduction to Moodle
   \item Brief introduction to Git
  \end{enumerate}

\end{itemize}

\subsection{Session 2 (Fri. September 24, 2010)}

\begin{itemize}
 \item \textbf{Speaker}: Miquel Vidal

 \item \textbf{Content}:

  \begin{enumerate}
   \item Introduction to LaTeX
  \end{enumerate}

\end{itemize}

\subsection{Session 3 (Fri. October 1, 2010)}

  \begin{itemize}
   \item \textbf{Speaker}: Santiago Dueñas.

   \item \textbf{Content}:

   \begin{enumerate}
    \item Useful MySQL commands
   \end{enumerate}

  \end{itemize}

\subsection{Session 4 (Fri. October 8, 2010)}

\begin{itemize}
 \item \textbf{Speaker}: Felipe Ortega

  \item \textbf{Content}:

  \begin{enumerate}
   \item A short introduction to GNU R
  \end{enumerate}

\end{itemize}

\subsection{Session 5 (Fri. October 22, 2010)}

\begin{itemize}
 \item \textbf{Professor}: José Francisco Castro.

 \item \textbf{Content}:

 \begin{enumerate}
  \item Some useful shell commands that you should know
 \end{enumerate}

\end{itemize}

\subsection{Session 6 (Fri. October 29, 2010)}

\begin{itemize}
 \item \textbf{Professor}: José Francisco Castro.

 \item \textbf{Content}:

 \begin{enumerate}
  \item Brief introduction to Vim, an advanced text editor
 \end{enumerate}

\end{itemize}

\subsection{Session 7 (Fri. November 5, 2010)}

\begin{itemize}
 \item \textbf{Professor}: José Francisco Castro and Miguel Vidal.

 \item \textbf{Content}:

 \begin{enumerate}
  \item The Illumos community
 \end{enumerate}

\end{itemize}

\subsection{Session 8 (Fri. November 12, 2010)}

\begin{itemize}
 \item \textbf{Professor}: Israel Herraiz.

 \item \textbf{Content}:

 \begin{enumerate}
  \item Research papers How To
 \end{enumerate}

\end{itemize}

\subsection{Session 9 (Fri. November 19, 2010)}

\begin{itemize}
 \item \textbf{Professor}: Roberto Calvo.

 \item \textbf{Content}:

 \begin{enumerate}
  \item The Android project
 \end{enumerate}

\end{itemize}

\subsection{Session 10 (Fri. November 26, 2010)}

\begin{itemize}
 \item \textbf{Professor}: José Gato.

 \item \textbf{Content}:

 \begin{enumerate}
  \item How To compile the Linux kernel
 \end{enumerate}

\end{itemize}

\subsection{Session 11 (Fri. December 10, 2010)}

\begin{itemize}
 \item \textbf{Professor}: Israel Herraiz.

 \item \textbf{Content}:

 \begin{enumerate}
  \item Public key cryptography - a practical approach
 \end{enumerate}

\end{itemize}

\subsection{Session 12 (Fri. December 17, 2010)}

\begin{itemize}
 \item \textbf{Professor}: Israel Herraiz.

 \item \textbf{Content}:

 \begin{enumerate}
  \item Key signing party
 \end{enumerate}

\end{itemize}

\end{document}
