

\section{Economic impact}

%%---------------------------------------------------------------

\begin{frame}
\frametitle{The (business, economic) role of libre software}

\begin{itemize}
\item Functionality: \\
  Easy access to cutting-edge functionality
\item Acquisition of technology: \\
  Libre software is a transfer of technology engine
\item Economic efficiency: \\
  In many cases, libre software case is the most efficient
\item New opportunities: \\
  Libre software changes rules and permits new opportunities
\item Service economy: \\
  Software as a service, instead of software as a product
\item Adaptability, conformance to needs: \\
  Any change is possible
\item Impact: \\
  Free distribution facilitates a massive impact
\end{itemize}
\end{frame}

%%---------------------------------------------------------------

\begin{frame}
\frametitle{Specific impact in the business fabric}

\begin{itemize}
\item Creation of production fabric: \\
  Expenditure in services, network of companies of different sizes
\item Creation of quality employment: \\
  Jobs related to development, maintenance, management, innovation
\item Cost savings, better use of public expenditure (funds for development):
  More and better impact of public expenditure
\item Improved competitiveness: \\
  Use of specific libre software, incremental development in IT sectors
\item Promotion of the value of innovation: \\
  Common base, added value provided by innovation
\end{itemize}

\end{frame}

%%---------------------------------------------------------------

\begin{frame}
\frametitle{Libre software as a disruptive factor}

\begin{center}
{\LARGE
Libre software means new opportunities \\
with disruptive potentials, \\
creation of a sustainable production fabric \\
and provision of means for improvements in many sectors
}
\end{center}

\end{frame}

%%---------------------------------------------------------------

\begin{frame}
\frametitle{Some consequences of software ``freedom''}

\begin{itemize}
\item \textbf{Cost}: model very different from proprietary software
\item \textbf{Openness}: can be modified, inspected, studied
\item \textbf{Distribution}: new channels, new methods
\item \textbf{Development}: `surprising' development models
\item \textbf{Maintenance and support}: Real competition
\end{itemize}

Mixture of two powerful mechanisms:

\begin{itemize}
\item Competition (with the same software base)
\item Cooperation (even involuntary)
\end{itemize}

\end{frame}

%%---------------------------------------------------------------

\begin{frame}
\frametitle{Some business arguments}

\begin{itemize}
\item Advantages of public scrutiny and chances for improvements
\item Real competition in development and  maintenance
\item Technical feasibility vs. marketing
\item Easy testing, deployment, update
\item Lower barriers to business opportunities
\item Non-formal collaboration and pooling of resources
\item Technology transfer
\item Libre software as an strategic tool
\end{itemize}

\end{frame}


%%---------------------------------------------------------------

\begin{frame}
\frametitle{What disappears when a business uses libre software?}

\begin{itemize}
\item Dependence on monopolies: \\
  Real competition, better products, better services
\item Importance of vendor reliability: \\
  Future depends on product acceptance
\item Decisions taken based on few elements: \\
  Software can be tested in its real environment, at a very low cost
\item Dependence on the strategy of providers: \\
  Decisions on the evolution of a product taken for those contributing with resources
\item Confidence on ``black boxes'': \\
  Software can be studied in detail
\end{itemize}
\end{frame}

%%---------------------------------------------------------------

\begin{frame}
\frametitle{Conclusions?}

\begin{itemize}
\item Libre software is a viable business option
\item Allows for new business opportunities...
\item ...but it is important to know how to take advantage of those
\end{itemize}

\end{frame}





