%
% $Id: $

\section{Sustainability}

%% Hecker classification, etc.



%%---------------------------------------------------------------

\begin{frame}
\frametitle{Funding options for libre software}

\begin{itemize}
\item External funding
  \begin{itemize}
    \item Who funds decides on resource allocation
    \item Usually, targeted at producing software
    \item It is a kind of sponsorship
  \end{itemize}
\item Self-funded
  \begin{itemize}
  \item Income comes from activities of the organization
  \end{itemize}
\item Developments without direct funding
\item Developments for internal use
\item Mixed models
\end{itemize}

\end{frame}

%%---------------------------------------------------------------

\begin{frame}
\frametitle{Financiación externa: pública}

\begin{itemize}
\item Financiación similar a la de los proyectos de I+D
\item La financiación suele venir de entidades promotoras de I+D
\item La entidad financiadora no suele buscar recuperar la inversión
  de forma directa
\item Sólo en casos muy particulares se hace de forma explícita, pero
  en muchos más casos es un ``subproducto'' de un contrato
\end{itemize}

\end{frame}

%%---------------------------------------------------------------

\begin{frame}
\frametitle{Financiación externa: pública (2)}

\begin{itemize}
\item Motivación ``científica'': que el software necesario para
  producir unos resultados esté disponible, para poder reproducirlos
\item Motivación ``precompetitiva'': que todo el tejido industrial se
  pueda beneficiar de los resultados precompetitivos
\item Motivación ``promoción de estándares'': que se implementen
  versiones de referencia de un estándar
\item Motivación ``social'': que se financie al creación de
  infraestructura básica común para la sociedad de la información
\item Caso de estudio: Gnat (compilador de Ada), alrededor de 1 M USD
  asignado por el Gobierno de EE.UU. a NYU para
    su desarrollo
\end{itemize}

\end{frame}

%%---------------------------------------------------------------

\begin{frame}
\frametitle{Financiación externa: privada sin ánimo de lucro}

\begin{itemize}
\item Normalmente realizada por fundaciones o ONGs
\item Motivación ``directa'': producir software libre
\item Motivación ``indirecta'': contribuir a resolver un problema
  mediante la producción de software libre
\item En general, mecanismos similares a los de la financiación
  pública
\item Casos de estudio: 
  \begin{itemize}
  \item Free Software Foundation
  \item Open Bioinformatics Foundation
  \end{itemize}
\end{itemize}

\begin{flushright}
\url{http://fsf.org} / \url{http://open-bio.org}
\end{flushright}
\end{frame}

%%---------------------------------------------------------------

\begin{frame}
\frametitle{Financiación externa: por quien necesita mejoras}

\begin{itemize}
\item Alguien necesita mejoras en un producto libre
\item Se financia el desarrollo por un grupo o una empresa
\item Caso de estudio:
  \begin{itemize}
  \item Corel quería portar sus productos a Linux
  \item Wine le podía permitir muchos ahorros si era mejorado
  \item Corel financió contribuciones de Macadamian a Wine, que
    incluían las mejoras que necesitaba Corel
  \end{itemize}
\end{itemize}

\begin{flushright}
\url{http://www.macadamian.com/news/wine.html}
\end{flushright}
\end{frame}

%%---------------------------------------------------------------

\begin{frame}
\frametitle{Financiación externa: indirecta}

\begin{itemize}
\item Financiación de creación de software libre buscando beneficios
  en productos relacionados con él
\item Normalmente las ventajas (incluso para el tipo de productos que
  se venden) no son exclusivas 
\item Las ventajas son mayores cuanto mayor es la cuota de mercado en
  el nicho en cuestión
\item Ejemplo: ventas de libros, hardware, distribuciones
\end{itemize}
\end{frame}

%%---------------------------------------------------------------

\begin{frame}
\frametitle{Financiación externa: indirecta (2)}

\begin{itemize}
\item O'Reilly financia desarrollo en Perl, y es el
  principal editor sobre él
\item VA Software (en sus comienzos como VA Research, VA Linux) colabora con el
  desarrollo del kernel Linux para asegurar su continuidad, vende
  equipos con Linux preinstalado
\item RedHat financia desarrollo de Gnome para tener un entorno de
  escritorio para su distribución
\end{itemize}

\begin{flushright}
\url{http://ora.com} \\
\url{http://vasoftware.com} \\
\url{http://redhat.com}
\end{flushright}
\end{frame}

%%---------------------------------------------------------------

\begin{frame}
\frametitle{Autofinanciado: mejor conocimiento}

\begin{itemize}
\item Se venden servicios basados en el conocimiento de un producto
\item El conocimiento se consigue fundamentalmente trabajando en el
  producto y colaborando a su desarrollo
\item Desarrollar el producto ayuda también a la imagen
\item Pero no es imprescindible participar en el desarrollo
\item Normalmente se venden servicios de consultoría, adaptación,
  integración, etc.
\end{itemize}
\end{frame}

%%---------------------------------------------------------------

\begin{frame}
\frametitle{Autofinanciado: mejor conocimiento}

Casos de estudio
\begin{itemize}
\item Levanta (antes LinuxCare): consultoría y soporte para GNU/Linux
  y software libre en EE.UU.
\item Alcove: consultoría y consultoría estratégica para software
  libre en Europa (fundamentalmente Francia), quebró en 2004
\end{itemize}

\begin{flushright}
\url{http://levanta.com} \\
\url{http://www.alcove.com}
\end{flushright}

\end{frame}

%%---------------------------------------------------------------

\begin{frame}
\frametitle{Autofinanciado: mejor conocimiento con limitaciones}

\begin{itemize}
\item Estos modelos tratan de limitar el efecto de la competencia
\item Típicamente se usan patentes o licencias propietarias
\item Normalmente se usan estos mecanismos en una parte pequeña (pero
  fundamental) del producto desarrollado
\item En muchos casos, la comunidad del software libre desarrolla su
  propia versión para ese componente
\end{itemize}


\end{frame}

%%---------------------------------------------------------------

\begin{frame}
\frametitle{Autofinanciado: fuente de un programa}

\begin{itemize}
\item Similar a los basados en el conocimiento
\item La ventaja competitiva se aumenta al ser los desarrolladores del
  producto, y tenerlo antes que la competencia
\item Muy interesante en términos de imagen

\end{itemize}

\end{frame}

%%---------------------------------------------------------------

\begin{frame}
\frametitle{Autofinanciado: fuente de un programa}

Casos de estudio:
\begin{itemize}
\item Abiword
\item Evolution, RedCarpet (Ximian, hoy Novell)
\item Zope (Zope Corporation, antes Digital Creations)
\end{itemize}

\begin{flushright}
\url{http://abiword.com/} \\
\url{http://ximian.com/} \\
\url{http://www.zope.com/}
\end{flushright}

\end{frame}

%%---------------------------------------------------------------

\begin{frame}
\frametitle{Autofinanciado: fuente de un programa con limitaciones}

\begin{itemize}
\item Se toman medidas para limitar la competencia
\item Ejemplo: distribución propietaria primero, luego libre
\item Ejemplo: distribución limitada durante un tiempo
\item Estos modelos se benefician menos de las ventajas del software
  libre para desarrolladores (modelos menos cooperativos)
\item Casos de estudio:
  \begin{itemize}
  \item Alladin Ghostscript (un año entre versiones AFPL y GPL) 
  \item Gnat (AdaCore) (versiones de desarrollo sólo para clientes)
  \end{itemize}
\end{itemize}

\begin{flushright}
\url{http://www.ghostscript.com} / \url{http://www.aladdin.com} \\
\url{http://adacore.com} 
\end{flushright}

\end{frame}

%%---------------------------------------------------------------

\begin{frame}
\frametitle{Negocios autofinanciado: licencias especiales}

\begin{itemize}
\item Habitualmente, basados en distribuciones bajo dos o más
  licencias
\item Normalmente, complementados con consultoría sobre el producto
\item Ejemplo: distribución libre bajo GPL, distribución propietaria
  para quien quiere otras condiciones
\item Caso de estudio:
  \begin{itemize}
    \item dbm (SleepyCat), distribución libre obliga a incluir el
      fuente de cualquier aplicación que use dbm, si se redistribuye
  \end{itemize}
\end{itemize}

\begin{flushright}
\url{http://sleepycat.com/} 
\end{flushright}

\end{frame}

%%---------------------------------------------------------------

\begin{frame}
\frametitle{Autofinanciado: venta de marca}

\begin{itemize}
\item Si el nombre es suficientemente reconocido, se puede vender casi
  cualquier cosa
\item Ejemplo: fabricantes de distribuciones GNU/Linux que se dedican
  a otras cosas
\item Caso de estudio:
  \begin{itemize}
  \item RedHat: comenzó dedicándose fundamentalmente a su
    distribución, hoy hace fundamentalmente consultoría, formación,
    certificación, etc.
  \end{itemize}
\end{itemize}

\begin{flushright}
  \url{http://www.redhat.com}
\end{flushright}
\end{frame}

%%---------------------------------------------------------------

\begin{frame}
\frametitle{Desarrollos sin financiación directa}

\begin{itemize}
\item La mayor parte de los proyectos se están desarrollando
  fundamentalmente de esta forma
\item En muchos casos hay financiación indirecta:
  \begin{itemize}
  \item Empresas que permiten que sus empleados colaboren a tiempo
    parcial con proyectos
  \item Contribuciones por organizaciones que quieren cierta
    funcionalidad
  \item Contribuciones con infraestructura para el proyecto
  \item Donaciones
  \end{itemize}
\end{itemize}
\end{frame}

%%---------------------------------------------------------------

\begin{frame}
\frametitle{Sin financiación directa: casos de estudio}

Muchos grandes proyectos han establecido fundaciones que les den
cobertura legal y (en parte) económica

\begin{itemize}
\item Apache Software Foundation
\item Gnome Foundation
\item KDE e. V.
\item Mozilla Foundation
\end{itemize}

\begin{flushright}
  \url{http://apache.org} \\
  \url{http://foundation.gnome.org/} \\
  \url{http://www.kde.org/areas/kde-ev/} \\
  \url{http://www.mozilla.org/foundation/}
\end{flushright}

\end{frame}

%%---------------------------------------------------------------

\begin{frame}
\frametitle{Desarrollos para uso interno}

\begin{itemize}
\item Al menos al comienzo, se desarrolla un producto sólo para uso interno
\item El desarrollador se beneficia de las ventajas del desarrollo
  libre (contribuciones, informes de error, parches, etc.)
\item La empresa se beneficia también: continuidad de soporte,
  inspección por terceras partes, documentación, etc.
\item Si hay aceptación de mercado, puede decidirse buscar ingresos
  relacionados con él
\item Caso de estudio: Cisco Enterprise Print System (CEPS)
\end{itemize}

\begin{flushright}
\url{http://ceps.sourceforge.net}
\end{flushright}
\end{frame}

%%---------------------------------------------------------------

\begin{frame}
\frametitle{Otra clasificación (por OSI, Hecker)}

\begin{itemize}
\item ``Support sellers'', venta de servicios relacionados
  con el producto
\item ``Loss leader'',  venta de otros productos propietarios
\item ``Widget frosting'', venta de hardware
\item ``Accessorizing'', venta de productos físicos (libros, etc.)
\item ``Service enabler'', venta de servicios en línea proporcionados
  por el programa
\item ``Sell it, free it'', como ``loss leader'' cíclico
\item ``Brand licensing'', venta de derechos de marca
\item ``Software franchising'', franquicia de marca
\end{itemize}

\end{frame}

%%---------------------------------------------------------------

\begin{frame}
\frametitle{Otros modos de financiación}

\begin{itemize}
\item Sitios donde se ponen en contacto desarrolladores con clientes
  (SoureXchange)
\item Venta de bonos para financiar un proyecto
\item Cooperativas de desarrolladores (Free Developers)
\end{itemize}

\begin{flushright}
\url{http://www.collab.net/sites/sxc_redirect/}
\url{http://freedevelopers.net/}
\end{flushright}

\end{frame}

%%---------------------------------------------------------------

\begin{frame}
\frametitle{Modelos mixtos}

\begin{itemize}
\item Casi todas las empresas usan en realidad modelos mixtos
\item Medidas que fortalezcan la marca
\item Pueden contribuir o no a la creación de software
\item En pocos casos se dedican exclusivamente a software libre
\item Muchas empresas ``tradicionales'' están probando nuevas líneas
  relacionadas con software libre
\end{itemize}

\end{frame}

%%---------------------------------------------------------------

\begin{frame}
\frametitle{Nuevas formas de colaboración}

\begin{itemize}
\item El software libre puede ser usado como medio para nuevas formas
  de colaboración entre empresas
\item Posibilidades más variadas (colaboración sin contratos, sin
  acuerdos formales)
\item Creación de ecosistemas de empresas (cadena trófica basada en
  producción / consumo de software libre)
\item Potenciación de otros negocios (software como posibilitador)
\end{itemize}

\end{frame}

%%---------------------------------------------------------------

\begin{frame}
\frametitle{Modelos basados en servicios especializados}

\begin{itemize}
\item Instalación (y actualización transparente)
\item Integración
\item Certificación (legal, de personas, de productos)
\item Formación (genérica, específica)
\item Mantenimiento (quizás con garantías contractuales)
\item Migración
\item Mediación (con comunidades, especialmente si no hay empresas
  comercializando soluciones)
\end{itemize}
\end{frame}

%%---------------------------------------------------------------

\begin{frame}
\frametitle{Conservación de los beneficios atractivos}

{\em ``Cuando desaparecen los beneficios atractivos en una etapa de la
  cadena de valor porque un producto se hace modular y ``commodity'',
  la oportunidad para conseguir beneficios atractivos con productos
  privativos normalmente aparecerá en una etapa adyacente''}

\begin{itemize}
\item ``Ley'' enunciada por Clayton Christensen
\item El software libre es un factor de ``comoditización''
\item Transferencia de oportunidades de beneficio a su alrededor
\item Caso especialmente interesante: empresas que no venden software,
  sino servicios o productos basados en software
\item Consorcios (quizás informales) interesados en hacer libre un
  eslabón de la cadena de valor
\end{itemize}
\end{frame}


%%---------------------------------------------------------------

\begin{frame}
\frametitle{Algunos ejemplos interesantes}

\begin{itemize}
\item AdaCore (antes ACT y ACT Europe)
\item Matra Datavision y OpenCascade
\item Sun, StarOffice y OpenOffice
\end{itemize}

\begin{flushright}
  \url{http://www.adacore.com/}  \\
  \url{http://libre.act-europe.fr/} \\
  \url{http://www.opencascade.org/} \\
  \url{http://www.opencascade.com/} \\
  \url{http://staroffice.com} \\
  \url{http://openoffice.org}
\end{flushright}

\end{frame}


%%---------------------------------------------------------------

\begin{frame}
\frametitle{Referencias}

\emph{``Free Software, Open Source: Information Society Opportunities for
Europe?''}, European Working Group on Libre Software \\
\emph{``Setting Up Shop''}, Franck Hecker \\
\emph{``Open Source Software''}, Naomi Hoffman \\
\emph{``Open Source Case for Business''},  OSI \\

\begin{flushright}
{\small
\url{http://eu.conecta.it} \\
\url{http://www.hecker.org/writings/setting-up-shop.html} \\
\url{http://public.kitware.com/VTK/pdf/oss.pdf} \\
\url{http://www.opensource.org/advocacy/case_for_business.html}
}
\end{flushright}

\end{frame}

%%---------------------------------------------------------------

\begin{frame}
\frametitle{Referencias (2)}

\emph{``The Magic Cauldron''}, Eric Raymond \\
\emph{``The Wall Street Performer Protocol''}, Chris Rasch \\
\emph{``Free/Libre Open Source Software: a guide for SMEs''}

\begin{flushright}
{\small
\url{http://www.tuxedo.org/~esr/writings/magic-cauldron/} \\
\url{http://firstmonday.dk/issues/issue6_6/rasch/index.html} \\
\url{http://guide.conecta.it}
}
\end{flushright}

\end{frame}

