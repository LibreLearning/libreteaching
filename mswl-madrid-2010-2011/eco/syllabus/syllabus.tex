\documentclass[a4paper]{article}
%\usepackage[T1]{fontenc}
\usepackage[utf8]{inputenc}
\usepackage{url}
\usepackage[pdfborder=0 0 0]{hyperref}
%\usepackage{html}
%\usepackage{hthtml}
\usepackage{geometry}

% Comments (optional argument is author of comment)
\newcommand{\comments}[2][?]{
  \begin{quote}
    \textbf{Comment (#1):} {\em #2}
  \end{quote}
  }

% \name is for ``special names'', like procedure or variable names.
\newcommand{\name}[1]{\texttt{\hturl{#1}}}

% Just a shortcut for links where the url appears as footnote.
\newcommand{\htflink}[2]{\htmladdnormallinkfoot{#1}{#2}}

\title{MSWL Economic Aspects \\
Master on libre software \\
URJC - GSyC/Libresoft \\
\url{http://master.libresoft.es}}

\author{Jesus M. Gonzalez-Barahona, Felipe Ortega}
\date{November 2010}

\sloppy
\begin{document}
\maketitle

\begin{abstract}
Course syllabus and learning program for the course ``Economic aspects of libre software'', of the Master on libre software of the Universidad Rey Juan Carlos (Móstoles, Spain).

[This is an evolving document, until the course is finished and graded]
\end{abstract}

\tableofcontents


\section{Sessions}

\subsection{Session 1 (Friday November 19, 2010)}

\begin{itemize}
 \item \textbf{Lecturers}: Jesús M. González-Barahona and Felipe Ortega.

 \item \textbf{Content}:

    \begin{enumerate}
     \item Presentation of the course.
     \item Introduction to economic aspects.
	\begin{itemize}
	 \item Video: Charles Leadbeater on open innovation.
	 \item First aspect: macroeconomic and strategic perspectives.
	  \begin{itemize}
	   \item Macroeconomic aspects.
	   \item Top-level strategies, market niches.
	   \item Influence on firms outside the software development business.
	   \item Open innovation: added-values for the enterprise.
	  \end{itemize}

	 \item Second aspect: business models and business viability.
	   \begin{itemize}
	    \item Taxonomy of business models based on FLOSS (C. Daffara).
	    \item Designing a FLOSS business model.
	   \end{itemize}

	 \item Third aspcet: FLOSS ecosystems and project sustainability.
	    \begin{itemize}
	     \item Hecker taxonomy of FLOSS projects.
	     \item FLOSS influence on stakeholders and markets.
	     \item The role of FLOSS foundations and leader firms in the FLOSS marketplace.
	    \end{itemize}

	\end{itemize}

     \item Activity: 10 statements about economic aspects of FLOSS.
	  \begin{itemize}
	   \item We visit 10 statements about economic aspects of FLOSS.
	   \item Working in groups, use gobby to prepare your answers collaboratively.
	   \item \textit{Forum assignment}: Comments about recent paper on the negative aspects of FLOSS for industry markets.
	  \end{itemize}

     \item Video and debate: The new open source economics.

	  \begin{itemize}
	   \item Video exhibition: Yockai Benkler on the new open source economics.
	   \item Open debate.
	   \item \textit{Forum assignment}: Comment or make critics to the video.
	  \end{itemize}

    \end{enumerate}

\end{itemize}


\subsection{Session 2 (Friday, November 26, 2010)}

\begin{itemize}
 \item \textbf{Lecturers}: Jesús M. González-Barahona and Felipe Ortega.
 \item \textbf{Guest speaker}: Sergio Ramos (UPM).

 \item \textbf{Content}:

    \begin{itemize}
     \item Invited talk: Open innovation and FLOSS.
     \item \textit{Forum assignment}: Identifying open innovation practices within specific firms or markets.

     \item Debate: strategic aspects of FLOSS outside the software development business.
        \item Presentation: Adoption of FLOSS in companies: TCO and other financial and 
	operational perspectives for SMEs and firms.
	\item Criteria for an informed selection of FLOSS solutions.
	\item Video?
	\item Open debate. Concrete case examples: Android.
        \item \textit{Assignment}: Choose a specific sector and try to introduce a novel business strategy
	based on FLOSS.
    \end{itemize}

\end{itemize}

\subsection{Session 3 (Friday, December 3, 2010)}

\begin{itemize}
 \item \textbf{Lecturers}: Jesús M. González-Barahona and Felipe Ortega.

 \item \textbf{Content}:

    \begin{itemize}
     \item Macroeconomic aspects.
      \begin{itemize}
       \item Implications for global and country-level finances.
       
      \end{itemize}

     \item Software patents.
	\begin{itemize}
	 \item Motivations behind software patents.
	 \item Implications for development of new markets and solutions.
         \item Case study: [FIXME: introduce your favourite case study]
	\end{itemize}

    \end{itemize}

\end{itemize}

\subsection{Session 4 (Friday, December 10, 2010)}

\begin{itemize}
 \item \textbf{Lecturers}: Felipe Ortega.

 \item \textbf{Content}:

    \begin{itemize}
     \item FLOSS business models.
      \begin{itemize}
       \item Taxonomy of FLOSS business models from FLOSS guide to SMEs (C. Daffara).
       \item Analysis: case examples of companies running FLOSS business models.
       \item \textit{Practice and assignment}: How to design a FLOSS business model.
      \end{itemize}

  \end{itemize}
\end{itemize}

\subsection{Session 5 (Friday, December 17, 2010)}

\begin{itemize}
 \item \textbf{Lecturers}: Felipe Ortega.

 \item \textbf{Content}:

    \begin{itemize}
     \item Creating and joining FLOSS ecosystems.
	\begin{itemize}
	 \item A framework for management styles in FLOSS projects and their impact on FLOSS ecosystems.
	\end{itemize}

     \item Case study: Oracle acquires Sun Microsystems.
	\begin{itemize}
	 \item Former strategies and FLOSS policies in Sun Microsystems.
	 \item Acquisition by Oracle. Implications for several FLOSS ecosystems.
	 \item Case example: the OpenSolaris community and the Illumos project.
	 \item Case example: MySQL and implications for market shares.
	 \item \textit{Forum assigment}: Analyze the rationale and implications of the Document Foundation and LibreOffice.
	\end{itemize}

    \end{itemize}

\end{itemize}

\section{Grading plan}

Each activity contributing to the grading of the course has its own evaluation criteria, as described below. Each of these activities has a minimum and maximum grading. If the minimum grading is 0, the activity is optional. Otherwise, the activity is mandatory, and has to be graded al least with the minimum to pass the course. Each activity has also a description, and when possible, some general grading criteria. In any case, the final grade for the course will also depend on the continuous observation of the instructors on the outcomes and progress of students.

Students should ask instructors about any detail which may not be clear to them, either about the general grading plan, or about specific aspects of the activities. As a general rule, evaluation will have into account how the activity and its results show that the student has come close to the competences, knowledge and skills expected for the course.

The student can consider that the next table will be used as a (minimum) guideline for assigning marks:

\begin{itemize}
\item Pass (``aprobado''): 150
\item Good (``notable''): 250
\item Excellent (``sobresaliente): 350
\end{itemize}

\begin{itemize}
\item \textbf{Exercises (answered in forum)}. \\
  Minimum: 20 points, maximum: 100 points.

  Exercises proposed and answered in the the forum of the course.

\item \textbf{Blog entries}. \\
  Minimum: 20 points, maximum: 80 points

  Blog entries specifically related to the course, and marked as such. The tag used for that is mswl-eco.

\item \textbf{Collaborative notebook}. \\
  Minimum: 0 points, maximum: 40 points

  Based on work in class (in real time) and afterwards (complementing the work, using git).

\item \textbf{Business plan}. \\
  Minimum: 20 points, maximum: 80 points

  Business plan for a company in which libre software clearly has an impact. The plan has to be detailed enough

\item \textbf{Specific report}. \\
  Minimum: 20 points, maximum: 80 points

Specific report about a certain business model or strategy based on FLOSS, showing its general aspects, but also analyzing companies already putting it into place, discussing advantages and drawbacks of the model, etc. A detailed DAFO analysis has to be a part of the report.

\begin{itemize}
\item A `traditional' written report.
\item A video or audio presentation (10 min. maximum).
\item A set of slides supporting the presentation.
\end{itemize}

 It is important to detail all the references, and to heavily root the report on data and/or specific works publicly available.

\item \textbf{Other activities}. \\
  Minimum: 0 points, maximum: 100 points

  These activities have to be agreed in advance with the instructors.
\end{itemize}

\end{document}
