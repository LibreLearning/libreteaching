\documentclass[a4paper]{article}

% Paquetes utilizados
\usepackage[utf8]{inputenc}
\usepackage[spanish]{babel}
\usepackage{url}
% Fueras a que las notas al pie aparezcan abajo
\usepackage[bottom]{footmisc}

% Metadatos
\title{\textbf{Casos de estudio para la práctica final}}
\author{Jose Gato Luis, Jose Castro}
\date{March 2011}
\vspace{0.5cm}
%--- \\
% \vspace{0.5cm}
%\normalsize{URJC - GSyC/LibreSoft} \\

% Inicio del documento
\begin{document}

 \section{Pritek Solutions}
La empresa priteksolutions es líder en el mercado de soluciones inform ́ticas. La empresa cuenta
a
con alrededor de 30 trabajadores, de los cuales 20 son de perfil puramente técnico. Su producto
e
estrella e-Services se trata de un portal electr ́nico para los ciudadanos. e-Services est ́ desar-
o
a
rollado usando .net con varias tecnolog ́ web y Oracle para el soporte de base de datos. En su
ıas
momento, un comercial de Oracle les ofreci ́ esta soluci ́n al tener como objetivo un proyecto con
o
o
una muy fuerte carga de datos, ya que deb ́ ser capaz de manejar  ́gilmente la informaci ́n de
ıa
a
o
millones de usuarios. Desde que form ́ la empresa, el IDE oficial de desarrollo es Microsoft Visual
o
Studio y Microsoft Visual Source Safe como herramienta de desarrollo colaborativa o control de
versiones.
Con la entrada en la empresa del gerente de soluciones inform ́ticas Jesus Nahoraba, un
a
nuevo aire fresco entra en la empresa. Este nuevo gerente conoce perfectamente las virtudes
del software libre y quiere introducirlo de forma progresiva en la empresa. De manera que las
soluciones ofrecidas por pritekSolutions tiendan a usar software libre. De hecho, e-Services est ́
a
licenciado de forma privativa y Nahoraba quiere plantear la posibilidad de ofrecerlo con alguna
licencia libre, ya sea el producto entero, o alguna de sus partes.



\secion{Empresa gestora/oficinas LiberSol}

Una empresa gestora con oficinas en el centro de madrid trabaja desde hace muchos a ̃os usando
n
las siguientes herramientas:
• Microsoft Office: Word, PowerPoint, Excell y Outlook
1
• Internet Explorer 6
• Contaplus
 ́
Ultimamente han o ́ hablar de las virtudes del software libre y del ahorro en licencias que
ıdo
podr ́ conseguir. Desgraciadamente, nadie en LiberSol tiene un conocimiento m ́
ıan
ınimo como
para poder acometer una migraci ́n sin problemas. Por ello, se ponen en contacto con el grupo
o
Libresoft para que les asesore con los pasos a seguir
 

\section{Design4Future}
Design4Future es una empresa de marketing y dise ̃o con sede en Sevilla. Esta empresa se
n
encarga de ofrecer distintas soluciones de dise ̃o: logos, dise ̃os web, anuncios comerciales e
n
n
incluso soporte en pel ́
ıculas y cortos de animaci ́n. En la actualidad, quieren introducirse en el
o
mundo de la animaci ́n en 3D.
o
Esto les llevar ́ a un fuerte cambio en su modelo de negocio trat ́ndose de una gran apuesta.
a
a
Desgraciadamente, no poseen nuevas fuentes de ingresos para acometer tal labor, ya que se trata
de una apuesta personal. Una forma de abaratar costes para entrar en este mundo, es el ahorro
en las car ́
ısimas licencias que poseen y el software libre les ha abiertos los ojos. Algunas de las
herramientas que utilizan en la actualidad:
• Photoshoop
• 3D Studio
• Adobe Premiere
• Adobe Flash
• Adobe Ilustrator
• Pro tools
Design4Future ha puesto a Jose ManStall, uno de sus trabajadores, a cargo de realizar un
estudio de alternativas libres, que cumplan con todos los requisitos que tienen en la actualidad.
Con un ahorro en licencias suficiente, podr ́ apostar por el mundo de la animaci ́n 3D, por lo
ıan
o
que tambi ́n necesitar ́ una herramienta libre para tal cometido.



  \section{El druida auténtico}
  La asociación \textit{El druida auténtico} es una asociación de jugadores de juegos de mesa y juegos de rol de Barcelona.
  Esta asociación cuenta con 5 trabajadores con responsabilidades administrativas.
  
  Afortunadamente, esta asociación cuenta con 1 servidor y 6 ordenadores de sobremesa.
  
  El servidor tiene como sistema operativo un Windows Server ofreciendo varios servicios: web, correo electrónico, FTP,
  compartición de ficheros con Samba y es donde está conectada la impresora compartida.
  
  Los puestos de trabajo tienen instalado como sistema operativo Windows 7, la suite ofimática Micosoft Office, el cliente
  de correo Outlook y el navegador Explorer 7.
  
  La asociación se está planteando contratar un informático para migrar los servicios a software libre.
  
  
  \section{Instituto Francisco de Goya}
  El instituto \textit{Francisco de Goya} está situado en Zamora y actualmente imparte cursos de primaria, secundaria y algunos
  ciclos formativos de formación profesional.
  
  Este instituto cuenta con servidores propios y una aula con 25 equipos de sobremesa.
  
  Para ponerse al día en las nuevas tecnologías en la educación, el equipo directivo apuesta por tener una plataforma de
  \textit{e-learning} que facilite la comunicación entre profesores y alumnos.
  
  También se han planteado tener su página web y el servidor de correo coorporativo del instituto.
  
  Por suerte, los alumnos del ciclo formativo de Informática tuvieron una sesión sobre software libre el mes pasado y se
  han ofrecido voluntarios para hacer el diseño y el despliegue.
  
  
  \section{Libre Hosting}
  La empresa \textit{Libre Hosting} será un empresa que proveerá servicios de hosting. Aún está en fase de creación y
  el equipo técnico está haciendo el diseño de la plataforma.
  
  Cuentan con un parque de 200 servidores de tipo \textit{blade} y 5 cabinas de almacenamiento con fibra óptica.
  
  Es ahora cuando están evaluando la posibilidad de desplegar una infraestructura virtualizada (y quizás con 
  \textit{cloud computing}) para minimizar los costes en licencias y poder adaptar las soluciones a sus necesidades
  específicas.
  
  Para poder ofrecer un servicio de hosting de calidad también se ven en la necesidad de un sistema de monitorización
  de cada una de las máquinas tanto físicas como virtuales.
  
  
  \section{Private University, (Esteban Carreras }

La empresa Private University está bien posicionada en el mundo de la
enseñanza universitaria oficial.

El departamento de Desarrollo de la empresa, cuenta con unos 10 trabajadores
técnicos repartidos entre el director del departamento, 4 analistas y 6
desarrolladores. Estos realizan aplicaciones a medida para las gestión de
los recursos de la Universidad, como son los alumnos, estudios, grupos,
calificaciones, etc., entre ellos se encuentra un CRM corporativo
desarrollado en Visual Basic 6.0, un CMS y un sistema de gestión de alumnos
desarrollados en PHP y JavaScript, y varios informes en MS Access para la
parte de reporting de los departamentos. Todas las aplicaciones utilizan un
servidor MS SQL Server 2005, utilizando su BBDD, sus procedimientos
almacenados y sus funciones. Desde hace años se utiliza como IDE oficial
para PHP NuSphere y para el CRM en Microsoft Visual Studio. También se
utiliza Microsoft Visual Source Safe como gestor de versiones. En cuanto a
las herramientas ofimáticas, se utiliza el paquete de MS Office, incluyendo
MS Project para las planificaciones.

 Desde que uno de los miembros del equipo de desarrollo está realizando un
Máster en Software Libre, se ha extendido en el departamento la idea de que
se pueden cambiar las herramientas con las que se realiza el trabajo diario
a Software Libre. Además, pretenden que sirva como piloto para migrar todas
las herramientas de la Universidad, o el mayor numero posible, a Software
Libre, e incluso a más largo plazo desarrollar una aplicación de gestión de
Universidades liberada bajo una licencia de Software Libre.

 \section{Empresa aerospacial, Esther Parilla}

Esta empresa se dedica al desarrollo de software para el sector Aeroespacial
y por lo tanto sus principales clientes son organismos públicos como la
Agencia Espacial Europea, la Agencia Espacial Francesa, el Ministerio de
Defensa etc... Trabajar con este tipo de cliente implica que los proyectos
suelen ser de larga duración (2 a 3 años mínimo) y con una complejidad muy
grande que obliga a llevar una gestión integrada de los mismos.
La empresa comenzó siendo muy pequeña pero ha crecido de forma exponencial
en los últimos años, por lo tanto los mecanismos de gestión de proyectos que
se usaban al principio son ahora insuficientes.
Para resolver este problema se ha creado un grupo de expertos que va a
investigar qué herramientas pueden usarse para mejorar esta gestión, entre
este grupo hay dos tendencias, unos quieren apostar por una solución
privativa y otros quieren apostar por Software Libre. Mi estudio se basará
en intentar justificar por qué la solución Libre es mejor.

\end{document}
