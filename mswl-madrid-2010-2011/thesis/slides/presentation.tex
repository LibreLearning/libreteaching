%% presentation.tex
%%
%% Presentation of the course ``Master Thesis" of the Official Master on Libre Software (URJC)
%% http://master.libresoft.es
%%

%%---------------------------------------------------------------------
%%---------------------------------------------------------------------

\section{Presentation of the Master Thesis Course}

%%---------------------------------------------------------------

\begin{frame}
\frametitle{Administrative data}

\begin{itemize}
\item Both semesters, 12 ECTS credits
\item Teachers:
  \begin{itemize}
  \item Gregorio Robles (grex at gsyc.urjc.es)
  \item Jesus M. Gonzalez-Barahona (jgb at gsyc.urjc.es)
  \item Departamento de Sistemas Telem�ticos y Computaci�n (GSyC)
  \item Rooms 109 and 120 Departamental II (M�stoles campus)
  \item Room 103 Biblioteca (Fuenlabrada campus)
  \end{itemize}
\item Schedule: see Calendar
\item Sessions:
  \begin{itemize}
  \item Classroom 215, Aulario II, Fuenlabrada campus
  \end{itemize}
\item Moodle course (please, join it as soon as possible): \\
  \url{http://docencia.etsit.urjc.es/moodle/course/view.php?id=134}
\end{itemize}
\end{frame}

%%---------------------------------------------------------------

\begin{frame}
\frametitle{Goals of the Master Thesis}

\begin{center}
Primary Goal of the Master Thesis

\vspace{0.25cm}

{\LARGE {\bf To apply the lessons and practices learned in this master
to a real problem}}

\vspace{1cm}

Secondary goal

\vspace{0.25cm}

{\large {\bf To do it in one academic year}}
\end{center}

\end{frame}

%%---------------------------------------------------------------

%%---------------------------------------------------------------

\begin{frame}
\frametitle{Recommended path}

Case study

\begin{itemize}
    \item Specific software sector
    \item Large project
    \item Ecosystem
\end{itemize}

from several points of view.

\vspace{0.5cm}

We are open to ways to link master thesis and \emph{Practicum}.

\end{frame}

%%---------------------------------------------------------------

\begin{frame}
\frametitle{Points of view}

The case study should have several points of view:

\begin{itemize}
  \item History
  \item Philosophy
  \item Main stakeholders and important people
  \item Legal aspects
  \item Organization and community
  \item Technical, process, etc.
  \item Business models
  \item Interviews
\end{itemize}

\end{frame}


%%---------------------------------------------------------------

\begin{frame}
\frametitle{Example I: LMS and libre software}

LMS: learning management systems

\begin{itemize}
  \item History
  \item (Philosophy)
  \item Main stakeholders and important people
  \item Legal aspects
  \item Organization and community
  \item Study of solutions
  \item Technical, process, etc.
  \item Business models
  \item Interviews
  \item Use of LMS
  \item Literature
\end{itemize}

\end{frame}


%%---------------------------------------------------------------

\begin{frame}
\frametitle{Example II: The MySQL Ecosystem}

MySQL is far more than a software product...

\begin{itemize}
  \item History
  \item (Philosophy)
  \item Main stakeholders and important people
  \item Legal aspects
  \item Organization and community
  \item Companies
  \item Technical, process, etc.
  \item Business models
  \item Interviews
  \item Literature
\end{itemize}

\end{frame}


%%---------------------------------------------------------------

\begin{frame}
\frametitle{Personal path}

\begin{itemize}
\item Master thesis in the ``traditional'' way
\item Mind the dates!
\item May have some of the previous parts from the recommended path
\item Mainly thought for the development of libre software
\item Strong focus on community development and development with a community
\item It is not only about programming! There should be at least efforts
to gain a community around the software
\item Integration in an already existing project is possible
\item Interactions with the community have to be documented
\end{itemize}

\end{frame}

%%---------------------------------------------------------------

\begin{frame}
\frametitle{Evaluation}

\begin{itemize}
\item The output of the master thesis will be:
  \begin{itemize}
     \item A report (mandatory): up to 50 pages, in \LaTeX, with a free license
     \item A software (optional): with a free license, a repository, some website and other community-development tools
     \item A video presentation (mandatory): up to 6 minutes, with a free license
  \end{itemize}
\item All this will be evaluated by a committee (lecturers and external experts)
\item There may be request for changes (and several rounds)
\item All materials will be published and publicized in the master website
\end{itemize}

\end{frame}

%%---------------------------------------------------------------

\begin{frame}
\frametitle{Looking for excellence!}

The master thesis supposes the best opportunity for master students
to show their acquired knowledge and abilities during the master

\begin{itemize}
\item Award to the best master thesis of the year (to be given in the graduation gala)
\item Master thesis will be presented in next year graduation gala
\item If it is software, to be presented ``Concurso Universitario de Software Libre''
\item Submission to scientific workshops and conferences are encouraged (and well seen)
\end{itemize}

\end{frame}

%%---------------------------------------------------------------

\begin{frame}
\frametitle{Calendar}

\begin{itemize}
\item 30.09: Presentation of the subject
\item 11.11: Proposal by students (5 min presentation)
\item Milestones (with on-site lectures):
\begin{itemize}
  \item 27.01: First milestone
  \item 24.02: Second milestone
  \item 24.03: Third milestone
  \item 14.04: Fourth milestone - Preparation for the defense
\end{itemize}
\item May 2011: Defense (first round)
\begin{itemize}
  \item 02.05: Submission 
  \item week starting 09.05: Defense and review
  \item 17.05: First closing date
\end{itemize}
\item June 2011: Defense (first round)
\begin{itemize}
\item 10.06 (may be changed): Second submission
\item week starting 17.06 (may be changed): Defense and review
\item 01.07 (may be changed): Second closing date
\end{itemize}
\end{itemize}

\end{frame}

%%---------------------------------------------------------------

\begin{frame}
\frametitle{Some references}

\begin{itemize}
\item Master Thesis Moodle Course \\
  \url{http://docencia.etsit.urjc.es/moodle/course/view.php?id=134}
\item Introduction to libre software (book) \\
  \url{http://curso-sobre.berlios.de/introsobre}
\item Concurso Universitario de Software Libre \\
  \url{http://www.concursosoftwarelibre.org/}
\end{itemize}

\end{frame}
