\documentclass[a4paper]{article}
%\usepackage[T1]{fontenc}
\usepackage[utf8]{inputenc}
\usepackage{url}
\usepackage[pdfborder=0 0 0]{hyperref}
%\usepackage{html}
%\usepackage{hthtml}
\usepackage{geometry}

% Comments (optional argument is author of comment)
\newcommand{\comments}[2][?]{
  \begin{quote}
    \textbf{Comment (#1):} {\em #2}
  \end{quote}
  }

% \name is for ``special names'', like procedure or variable names.
\newcommand{\name}[1]{\texttt{\hturl{#1}}}

% Just a shortcut for links where the url appears as footnote.
\newcommand{\htflink}[2]{\htmladdnormallinkfoot{#1}{#2}}

\title{MSWL Projects Evaluation \\
Master on libre software \\
URJC - GSyC/Libresoft \\
\url{http://master.libresoft.es}}

\author{Felipe Ortega, Pedro Coca, Daniel Izquierdo}
\date{November 2010}

\sloppy
\begin{document}
\maketitle

\begin{abstract}
Course syllabus and learning program for the course ``Evaluation of libre software projects'', of the Master on libre software of the Universidad Rey Juan Carlos (Móstoles, Spain).

[This is an evolving document, until the course is finished and graded]
\end{abstract}

\tableofcontents


\section{Sessions}

\subsection{Session 1 (Thursday November 18, 2010)}

\begin{itemize}
 \item \textbf{Lecturers}: Felipe Ortega
 \item \textbf{Summary}: Introduction to quality in traditional software engineering.
 \item \textbf{Content}:

    \begin{itemize}
      \item Presentation of the course
      \item Activity:
        \begin{itemize}
          \item What does quality mean?
        \end{itemize}
      \item Activity:
        \begin{itemize}
          \item Work in groups to analyze the quality assurance of different 
          libre software projects and recommendation guides provided by them 
          (e.g.: Ubuntu and Mozilla)
          \item Give a small talk by group with the results and the important 
          quality aspects addressed by each of the communities.
        \end{itemize}
      \item Introduction to ISO 9126
      \item Activity:
        \begin{itemize}
          \item Discussion about the different quality attributes and their possible
          meaning.
          \item Get an agreement among students for each of the definitions of 
          those quality attributes.
        \end{itemize}
       \item Introduction to quality in Software Engineering
       \item Introduction to quality models
       \item Activity:
         \begin{itemize} 
           \item Create an entry in the blog related to one of the shown quality
           models (Homeworks).
         \end{itemize}
    \end{itemize}

   
\end{itemize}


\subsection{Session 2 (Thursday, November 25, 2010)}

\begin{itemize}
 \item \textbf{Lecturers}: Pedro Coca.
 \item \textbf{Summary}: Introduction to light-weight quality models in libre software.
 \item \textbf{Content}:

 \begin{itemize}
      \item Activity:
        \begin{itemize}
          \item Create a collaborative document where important aspect of the libre software world
          have not been taken in previous quality models seen in session 1. This document should 
          be updated with new ideas during the rest of the subject.
        \end{itemize}
      \item Introduction to OpenBRR
      \item Activity: 
        \begin{itemize}
          \item Divided students in groups of two/three people and work on a specified quality attribute from OpenBRR
          in a given libre software project.
          \item Students, should later aggregate all the results in a final spreadsheet with all the results.
        \end{itemize}
       \item Introduction to QSoS
       \item Activity:
         \begin{itemize}
           \item Discussion about main differences between openbrr and qsos.
           \item Students, should later aggregate all the results in a final spreadsheet with all the results.
         \end{itemize}
       \item Activity:
         \begin{itemize}
          \item Read paper where a comparison between OpenBRR and QSoS is done.
        \end{itemize}

 \end{itemize}    

\end{itemize}

\subsection{Session 3 (Thursday, December 2, 2010)}

\begin{itemize}
 \item \textbf{Lecturers}: Felipe Ortega
 \item \textbf{Summary}: Introduction to QualOSS, SQO-OSS
 \item \textbf{Content}:

 \begin{itemize}
      \item Introduction to QualOSS
      \item Introduction to SQO-OSS and Alitheia Core
      \item Activity:
        \begin{itemize}
          \item Community metrics in Ohloh
        \end{itemize}
      \item Activity:
        \begin{itemize}
          \item Tools to analyze FLOSS projects
        \end{itemize}
 \end{itemize}

\end{itemize}

\subsection{Session 4 (Thursday, December 9, 2010)}

\begin{itemize}
 \item \textbf{Lecturers}: Felipe Ortega.
 \item \textbf{Summary}: Introduction to automation in quality models
 \item \textbf{Content}:

 \begin{itemize}
     \item Introduction to FLOSS meta-repositories
     \item Activity:
      \begin{itemize}
        \item Metrics for FLOSS projects
        \item Tools for FLOSS projects
      \end{itemize}
     \item Activity:
      \begin{itemize}
        \item Automation of FLOSS projects metrics
      \end{itemize}
      \begin{itemize}
        \item GNU R metrics receipts
      \end{itemize}

 \end{itemize}


\end{itemize}

\subsection{Session 5 (Thursday, December 16, 2010)}

\begin{itemize}
 \item \textbf{Lecturers}: Felipe Ortega, Pedro Garcia and Pedro Coca.
 \item \textbf{Summary}: Introduction to Libre software testing. Case study: Mozilla Community: Thunderbird
 \item \textbf{Content}:

    \begin{itemize}

     \item Free Software QA and Free software testing

      \begin{itemize}
       \item Tools in Free Software Testing
      \end{itemize}

     \item Case Study: Mozilla Thunderbird

        \begin{itemize}
         \item Mozilla QA and Testing
         \item Thunderbird testing process
         \item Thunderbird testing tools and example
        \end{itemize}

    \end{itemize}
   

\end{itemize}

\section{Grading plan}

Each activity contributing to the grading of the course has its own evaluation criteria, as described below. Each of these activities has a minimum and maximum grading. If the minimum grading is 0, the activity is optional. Otherwise, the activity is mandatory, and has to be graded at least with the minimum to pass the course. Each activity has also a description, and when possible, some general grading criteria. In any case, the final grade for the course will also depend on the continuous observation of the instructors on the outcomes and progress of students.

Students should ask instructors about any detail which may not be clear to them, either about the general grading plan, or about specific aspects of the activities. As a general rule, evaluation will have into account how the activity and its results show that the student has come close to the competences, knowledge and skills expected for the course.

The student can consider that the next table will be used as a (minimum) guideline for assigning marks:

\begin{itemize}
\item Pass (``aprobado''): 120
\item Good (``notable''): 250
\item Excellent (``sobresaliente): 350
\end{itemize}

\begin{itemize}
\item \textbf{Exercises (answered in forum)}. \\
  Minimum: 20 points, maximum: 100 points.

  Exercises proposed and answered in the the forum of the course.

\item \textbf{Blog entries}. \\
  Minimum: 20 points, maximum: 80 points

  Blog entries specifically related to the course, and marked as such. The tag used for that is mswl-eval.

\item \textbf{Quality. Case Study. Libre SCM Tools: Bazaar, Mercurial and Git}. \\
  Minimum: 0 points, maximum: 100 points

\item \textbf{Other activities}. \\
  Minimum: 0 points, maximum: 100 points

\end{itemize}

\end{document}
