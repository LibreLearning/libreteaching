% Created 2011-03-16 Wed 19:07
\documentclass[11pt]{beamer}

      \mode<presentation>

      \usetheme{Warsaw}

      \usecolortheme{lily}

\usebackgroundtemplate{\includegraphics[width=\paperwidth]{format/libresoft-bg-soft.png}}
      \beamertemplateballitem

      \setbeameroption{show notes}
      \usepackage[utf8]{inputenc}

      \usepackage[T1]{fontenc}

      \usepackage{hyperref}

      \usepackage{color}
      \usepackage{listings}
      \lstset{numbers=none,language=[ISO]C++,tabsize=4,
  frame=single,
  basicstyle=\small,
  showspaces=false,showstringspaces=false,
  showtabs=false,
  keywordstyle=\color{blue}\bfseries,
  commentstyle=\color{red},
  }

      \usepackage{verbatim}

      \institute{israel.herraiz@upm.es \\ Universidad Politécnica de Madrid}

       \subject{Advanced Software Development}

\usepackage[utf8]{inputenc}
\usepackage[T1]{fontenc}
\usepackage{fixltx2e}
\usepackage{graphicx}
\usepackage{longtable}
\usepackage{float}
\usepackage{wrapfig}
\usepackage{soul}
\usepackage{textcomp}
\usepackage{marvosym}
\usepackage{wasysym}
\usepackage{latexsym}
\usepackage{amssymb}
\usepackage{hyperref}
\tolerance=1000
\providecommand{\alert}[1]{\textbf{#1}}

\title{Documenting and releasing Python applications}
\author{Israel Herraiz}
\date{16 March 2011}

\begin{document}

\maketitle

\begin{frame}
\frametitle{Outline}
\setcounter{tocdepth}{3}
\tableofcontents
\end{frame}

\lstset{ %
language=python,
basicstyle=\footnotesize\ttfamily,       % the size of the fonts that are used
numbers=none,                   % where to put the line-numbers
numberstyle=\ttfamily,      % the size of the fonts that are used
stepnumber=2,                   % the step between two
numbersep=5pt,                  % how far the line-numbers are from
showspaces=false,               % show spaces adding particular
showstringspaces=false,         % underline spaces within strings
showtabs=false,                 % show tabs within strings adding
frame=none,                   % adds a frame around the code
tabsize=2,                      % sets default tabsize to 2 spaces
captionpos=b,                   % sets the caption-position to bottom
breaklines=true,                % sets automatic line breaking
breakatwhitespace=false,        % sets if automatic breaks should only
title=\lstname,                 % show the filename of files included
escapeinside={\%*}{*)},         % if you want to add a comment within
morekeywords={*,...}            % if you want to add more keywords to
}
\setbeamertemplate{blocks}[rounded][shadow=true]

\section{How to document Python programs?}
\label{sec-1}
\begin{frame}[fragile]\frametitle{Listen to the classics}
\label{sec-1_1}


\hfill
\begin{block}{}
    Programs must be written for people to read, and only incidentally
    for machines to execute.
\begin{flushright}
    —Abelson \& Sussman,\\ Structure and Interpretation of Computer
    Programs
\end{flushright}
\end{block}
\hfill
\end{frame}
\begin{frame}[fragile]\frametitle{Guides to read to become a Pythonista}
\label{sec-1_2}


\begin{itemize}
\item PEP 8: Style guide for Python code

\begin{itemize}
\item \href{http://www.python.org/dev/peps/pep-0008/}{http://www.python.org/dev/peps/pep-0008/}
\item PEP means \emph{Python Enhancement Proposal}
\end{itemize}

\item Code like a Pythonista: Idiomatic Python

\begin{itemize}
\item \href{http://python.net/~goodger/projects/pycon/2007/idiomatic/}{http://python.net/\~goodger/projects/pycon/2007/idiomatic/}
\end{itemize}

\end{itemize}
\end{frame}
\section{Conventions}
\label{sec-2}
\begin{frame}[fragile]\frametitle{Code style matters}
\label{sec-2_1}


\begin{itemize}
\item Four space indentation

\begin{itemize}
\item Forget tabs!
\item But if you cannot forget, \textbf{do not mix tab and spaces}
\end{itemize}

\item One blank line between functions
\item Two blank lines between classes
\item Add a space after ``,'' in dicts, lists, tuples, argument lists, and
  after ``:'' in dicts, but not before.
\item Put spaces around assignments and comparisons (except in argument
  lists).
\item No spaces just inside parentheses or just before argument lists.
\item No spaces just inside docstrings.
\end{itemize}
\end{frame}
\begin{frame}[fragile]\frametitle{Wrong example}
\label{sec-2_2}


\begin{lstlisting}
class Wrong:
   def __init__(self,x,y,z):
      self.x=x
      self.y=y
      self.z=z
   def another_function(self):
     print "Hi there!"
\end{lstlisting}
\end{frame}
\begin{frame}[fragile]\frametitle{Correct example}
\label{sec-2_3}


\begin{lstlisting}
class Right:
   def __init__(self, x, y, z):
      self.x = x
      self.y = y
      self.z = z

   def another_m(self):
     print "Hi there!"

   def yet_another_m(self):
     pass


class AnotherClass:
   def __init__(self):
     pass
\end{lstlisting}
\end{frame}
\begin{frame}[fragile]\frametitle{How to name your entities}
\label{sec-2_4}


\begin{itemize}
\item \texttt{joined\_lower} for functions, methods, attributes
\item \texttt{ALL\_CAPS} for constants
\item \texttt{InitialCaps} for classes
\item Attributes:

\begin{itemize}
\item \texttt{public} (never use public attributes!)
\item \texttt{\_internal}
\item \texttt{\_\_private} (use discouraged)
\end{itemize}

\end{itemize}
\end{frame}
\begin{frame}[fragile]\frametitle{Long lines}
\label{sec-2_5}


\begin{itemize}
\item Lines above 80 characters should be cut
\end{itemize}

\begin{lstlisting}
def __init__(self, first, second, third,
             fourth, fifth, sixth):
    output = (first + second + third
              + fourth + fifth + sixth)


VeryLong.left_hand_side \
    = even_longer.right_hand_side()
\end{lstlisting}
\end{frame}
\section{Documenting and commenting your code}
\label{sec-3}
\begin{frame}[fragile]\frametitle{Docstrings and comments}
\label{sec-3_1}


\begin{itemize}
\item Docstrings

\begin{itemize}
\item Tell how to use your code
\end{itemize}

\item Comments

\begin{itemize}
\item Tell how your code works
\end{itemize}

\item Docstrings explain external behavior, comments explain internal
  behavior
\item Docstrings conventions are included in the PEP 257

\begin{itemize}
\item \href{http://www.python.org/dev/peps/pep-0257/}{http://www.python.org/dev/peps/pep-0257/}
\end{itemize}

\end{itemize}
\end{frame}
\begin{frame}[fragile]\frametitle{Docstrings}
\label{sec-3_2}


\begin{itemize}
\item String literal that occurs as the first statement of a module, class,
 function, method
\item It becomes the \texttt{\_\_doc\_\_} attribute of that element

\begin{itemize}
\item And therefore it is the text that appears after calling \texttt{help()}
\end{itemize}

\item No blank line after the docstring
\item Period at the end of the help text
\item Can also be multiline

\begin{itemize}
\item Multiline docstrings in scripts are the usage info of that script
\end{itemize}

\end{itemize}
\end{frame}
\begin{frame}[fragile]\frametitle{Sample docstrings}
\label{sec-3_3}


\begin{lstlisting}
def add(x, y):
   """Add two numbers x and y."""
   return x + y

def complex(real=0.0, imag=0.0):
  """Form a complex number.

  Keyword arguments:
  real -- the real part (default 0.0)
  imag -- the imaginary part (default 0.0)

  """
  if imag == 0.0 and real == 0.0: 
    return complex_zero
  else:
    return -1
\end{lstlisting}
\end{frame}
\begin{frame}[fragile]\frametitle{Comments}
\label{sec-3_4}


\begin{itemize}
\item Comment lines start with \texttt{\#}
\item Don't write comments beyond column 80
\end{itemize}

\begin{lstlisting}
def add(x, y):
   """Silly function."""
   # This is a comment alone in a line
   xx = 0
   yy = 0
   xx = x # This is a comment after some code
   yy = y # Yet another comment
   return x + y
\end{lstlisting}
\end{frame}
\section{Releasing your app}
\label{sec-4}
\begin{frame}[fragile]\frametitle{External documentation: essential files}
\label{sec-4_1}


\begin{itemize}
\item Provide always the following documentation with your code

\begin{itemize}
\item \texttt{README}
\item \texttt{INSTALL} (optionally, it can be included in the \texttt{README})
\item \texttt{NEWS}
\item \texttt{AUTHORS} (give credit to contributors)
\item \texttt{COPYING} (always include the license)
\item \texttt{ChangeLog} (summary of the most important changes)
\item \texttt{TODO} (if there is work to do so others can help)
\end{itemize}

\end{itemize}
\end{frame}
\begin{frame}[fragile]\frametitle{Release process}
\label{sec-4_2}


\begin{itemize}
\item Make sure your code is ready for a release
\item Decide your release number and update your code and doc accordingly
\item Update your external docs
\item Commit everything to the main branch of your repo
\item Tag your head commit with the release number
\item Provide tarballs in a downloads site

\begin{itemize}
\item Probably with checksums or cryptographic signatures
\end{itemize}

\item Spread the word!
\end{itemize}
\end{frame}

\end{document}