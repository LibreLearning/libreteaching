\documentclass[a4paper]{article}
%\usepackage[T1]{fontenc}
\usepackage[utf8]{inputenc}
\usepackage{url}
\usepackage[hypertex,colorlinks]{hyperref}
%\usepackage{html}
%\usepackage{hthtml}
\usepackage{geometry}

% Comments (optional argument is author of comment)
\newcommand{\comments}[2][?]{
  \begin{quote}
    \textbf{Comment (#1):} {\em #2}
  \end{quote}
  }

% \name is for ``special names'', like procedure or variable names.
\newcommand{\name}[1]{\texttt{\hturl{#1}}}

% Just a shortcut for links where the url appears as footnote.
\newcommand{\htflink}[2]{\htmladdnormallinkfoot{#1}{#2}}

\title{MSWL Introduction \\
Master on libre software \\
URJC - GSyC/Libresoft \\
\url{http://master.libresoft.es}}

\author{Israel Herraiz}
\date{February 2011}

\sloppy
\begin{document}
\maketitle

\begin{abstract}
Course syllabus and learning program for the course ``Advanced libre
software development'', of the Master on libre software of the
Universidad Rey Juan Carlos.

[This is an evolving document, until the course is finished and graded]
\end{abstract}

\tableofcontents

%%----------------------------------------------------------------
%%----------------------------------------------------------------
%%----------------------------------------------------------------
\section{Course topics and schedule}

%%----------------------------------------------------------------
%%----------------------------------------------------------------
\subsection{00 - Presentation of the course}

Presentation of the main aspects of the course, and specially those
related to administrative issues, evaluation, etc.

%%----------------------------------------------------------------
\subsubsection{February 17, 2011 (0.5 hours)}

\subsection{01 - Debian packaging}

How to package a program for the Debian GNU/Linux distribution and
derivatives (such as Ubuntu).

%%----------------------------------------------------------------
\subsubsection{February 17, 2011 (3.5 hours)}

\begin{itemize}
\item \textbf{Lecturers:} Jose Gato
\item \textbf{Presentation:} ``Debian packaging''
  \begin{itemize}
  \item \textbf{Discussion:} Issues about packaging different kinds of
    software for Debian (standalone programs, libraries, etc)
  \item \textbf{Supporting material:} Slides ``Debian packaging''
  \end{itemize}
\item \textbf{Assignment (results in website):} Any kind of activity
  related to this talk (blog post, video, etc)
\end{itemize}


\subsection{02 - Accessibility}

 How can we create software that can be used by any person regardless
of any kind of sensorial impairments. 

%%----------------------------------------------------------------
\subsubsection{February 24, 2011 (4 hours)}

\begin{itemize}
\item \textbf{Lecturers:} Fernando Herrera
\item \textbf{Presentation:} ``Accessibility''
  \begin{itemize}
  \item \textbf{Discussion:} Issues about accessibility. Do you value
    accessibility in the software you use?
  \item \textbf{Supporting material:} Slides ``Accessibility''
  \end{itemize}
\item \textbf{Assignment (results in website):} Any kind of activity
  related to this talk (blog post, video, etc)
\end{itemize}


\subsection{03 - Localization and Internationalization (l10n, i18n)}

How to create applications that can be used in any language and any
local environment.

%%----------------------------------------------------------------
\subsubsection{March 3, 2011 (4 hours)}

\begin{itemize}
\item \textbf{Lecturers:} Fernando Herrera and Daniel Mustieles
\item \textbf{Presentation:} ``L10N and I18N''
  \begin{itemize}
  \item \textbf{Discussion:} Issues about translating and localizing software
  \item \textbf{Supporting material:} Slides ``L10N and I18N''
  \end{itemize}
\item \textbf{Assignment (results in website):} Any kind of activity
  related to this talk (blog post, video, etc)
\end{itemize}


\subsection{04 - Advanced Python}

How to create a Python application from scratch, and how to deploy it
so it can be easily installed in any system. How to document and how
to test.

%%----------------------------------------------------------------
\subsubsection{March 10, 2011 (4 hours)}

\begin{itemize}
\item \textbf{Lecturers:} Israel Herraiz
\item \textbf{Presentation:} ``How to create and deploy Python applications''
  \begin{itemize}
\item \textbf{Requirements:} Running Python 2.x development environment
  \item \textbf{Discussion:} Have you ever tried to install a Python app?
  \item \textbf{Supporting material:} Slides ``How to create and
    deploy Python applications'', sample source code
  \end{itemize}
\item \textbf{Assignment (results in website):} Create a spider to
  track updates of a web page I (assignment~\ref{sub:python})
\end{itemize}

\subsubsection{March 17, 2011 (4 hours)}

\begin{itemize}
\item \textbf{Lecturers:} Israel Herraiz
\item \textbf{Requirements:} Running Python 2.x development environment
\item \textbf{Presentation:} ``How to document and test Python apps''
  \begin{itemize}
  \item \textbf{Discussion:} About reading some real-world Python code
  \item \textbf{Supporting material:} Slides ``Documenting and testing
    Python apps'', sample source code
  \end{itemize}
\item \textbf{Assignment (results in website):} Create a spider to
  track updates of a web page II (assignment~\ref{sub:python})
\end{itemize}



\section{Grading}

This section details the criteria for grading the course, the
deadlines for the different activities, and the submission details for
the activities that require them.

\subsection{Evaluation criteria}
\label{sub:evaluation-criteria}

Each activity contributing to the grading of the course has its own
evaluation criteria, as described below. Each of these activities has
a minimum and maximum grading. If the minimum grading is 0, the
activity is optional. Otherwise, the activity is mandatory, and has to
be graded al least with the minimum to pass the course. Each activity
has also a description, and when possible, some general grading
criteria. In any case, the final grade for the course will also depend
on the continuous observation of the instructors on the outcomes and
progress of students.

Students should ask instructors about any detail which may not be
clear to them, either about the general grading plan, or about
specific aspects of the activities. As a general rule, evaluation will
have into account how the activity and its results show that the
student has come close to the competences, knowledge and skills
expected for the course.

The student can consider that the next table will be used as a
(minimum) guideline for assigning marks:

\begin{itemize}
\item Pass (``aprobado''): 150
\item Good (``notable''): 250
\item Excellent (``sobresaliente''): 350
\end{itemize}

\begin{itemize}
\item \textbf{Blog entries}. \\
  Minimum: 40 points, maximum: 150 points

  Blog entries specifically related to the course, and marked as such. The tag used for that is mswl-adv-dev.

\item \textbf{Python application}. \\
  Minimum: 60 points, maximum: 350 points

Source code of the ``spider'' assignment. Grading will vary taking in
account the different topics covered in the lectures (easy deployment,
documentation, testing, clarity of the code, etc).

\end{itemize}

\subsection{Submission deadlines}

All activities to be graded in May must be completed and submitted by April 30th 2011.

\subsection{Submission details}

Please, consider the details below for submitting the different activities for evaluation (for those not specified in this list, nothing special is needed for submission).

\begin{itemize}
\item As a summary of all the activities, a ``Summary of activities for evaluation'' should be sent. This summary should be uploaded to the corresponding resource in the Moodle site for this course, and should include the following data:
  \begin{itemize}
  \item \textbf{Name:} Full name of the student (as ``family name'', ``given name'')
  \item \textbf{Blog entries:} Url of the blog entries for this course (HTML, not RSS version).
  \item \textbf{Contributions to Python application:} Id for commits
    to the repository where the source code is hosted, and summary of the main contributions to it related to this course, including links to the repository and commit ids if appropriate.  \end{itemize}
\end{itemize}

%%----------------------------------------------------------------
%%----------------------------------------------------------------
%%----------------------------------------------------------------
\section{Assignments and activities}

%%----------------------------------------------------------------
%%----------------------------------------------------------------
\subsection{Spider to track the updates of a web page}
\label{sub:python}

You will have to write a Python application that get the current
version of a web page, compare against a local cache of the page, and
if changed, retrieve the new version of the page and write in the
standard output a summary of the changes.

The spider must visit all the links below the current page. The log of
changes displayed in the standard output will contain a list of all
the links that have been changed, and the number of lines of
difference between the two versions.

The application must be easily installable using Python standard
deployment methods, must be properly document and must include a
battery of tests to check that it is working as expected.

\textbf{Supporting material}

\begin{itemize}
\item Slides about advanced Python development
\item Snippets of sample source code
\end{itemize}


%%----------------------------------------------------------------
%\subsubsection{Statements about economic aspects of libre software}
%\label{sub:statements-eco}

\end{document}
