\section{Python}

\begin{frame}
  \frametitle{Introduction}

  \begin{itemize}
  \item High level interpreted language created by Guido van Rossum
  \item First public release in 1991
  \item Designed to be simple to use
  \item Readability
  \item Object Oriented
  \item ``Multiplatform''
  \item The interpreter is extensible (written in C)
  \item Free Software!
  \end{itemize}
\end{frame}

\begin{frame}
  \frametitle{History}

  \begin{itemize}
  \item In the early 80s, Guido van Rossum worked at CWI (Centrum voor
    Wiskunde \& Informatica) in Netherlands, developing the ABC
    programming language.
  \item In 1986 van Rossum is moved to a different project, the Amoeba
    project, a distributed operating system.
  \item By the late 80s they realized they need and scripting language
    for Amoeba, and Guido started to work on his own programming
    language (in his free time taking advantage of the Christmas holidays)
  \item During 1990 he continued working on Python in his own time.
  \item In February 1991 the 0.9 version is released.
  \item The version 1.0 is released in January 1994.
  \end{itemize}
\end{frame}

\begin{frame}
  \frametitle{History (Cont.)}

  \begin{itemize}
  \item In 1995 Guido van Rossum leaves CWI and moves to Corporation
    for National Research Initiatives (CNRI) in Reston, Virginia where
    he continues developing Python.
  \item In 2000, the Python core development team moved to BeOpen.com
    to form the BeOpen PythonLabs team.
  \item In October 2000 version 2.0 is released by BeOpen.com
  \item After 2.0 was released, the PythonLabs developers joined
    Digital Creations (later renamed as Zope Coorporation)
  \item In March 2001, the Python Software Foundation (PSF) is founded
    as a non-profit organization in order to protect and promote the
    Python development.
  \item Today, the project has an important community around and
    Python is the language selected by many different projects.
  \item Python 3.0 (a.k.a Python 3000) is the latest major release 
    and it breaks backwards compatibility with Python 2.
  \end{itemize}
\end{frame}

\begin{frame}[fragile]
  \frametitle{Hello World}
{\huge
\begin{verbatim}
print "Hello World"
\end{verbatim}
}
\end{frame}

\begin{frame}[fragile]
  \frametitle{The interpreter}

  \begin{itemize}
  \item Interactive mode
{\scriptsize
\begin{verbatim}
Python 2.5.2 (r252:60911, Oct  5 2008, 19:24:49) 
[GCC 4.3.2] on linux2
Type "help", "copyright", "credits" or "license" for more information.
>>> 2 + 2
4
>>> print "Hello World"
Hello World
>>>
\end{verbatim}
}
\item Running scripts
{\scriptsize
\begin{verbatim}
#!/usr/bin/env python

print "Hello World"
\end{verbatim}
}

\item ipython: An Enhanced Interactive Python
\end{itemize}
\end{frame}

\begin{frame}
  \frametitle{Builtin data types}
  \begin{itemize}
  \item Elemental types: int, float, bool, long, ...
  \item Strings: str, unicode
  \item Compound types: list, tuple, dict, ...
  \end{itemize}
\end{frame}

\begin{frame}[fragile]
  \frametitle{Strings}
  \begin{itemize}
  \item Single or double quotes
{\scriptsize
\begin{verbatim}
>>> "double quote" == 'double quote'
True
>>> 'single \' quote'
"single ' quote"
>>> "single ' quote"
"single ' quote"
\end{verbatim}
}

\item Even triple quotes: """ or '''
{\scriptsize
\begin{verbatim}
>>> print "We need to scape\n\
... new lines"
We need to scape
new lines
>>> print """No need to scape
... new lines"""
No need to scape
new lines
\end{verbatim}
}  

\item Raw string
{\scriptsize
\begin{verbatim}
>>> print "Not\t a raw\n string"
Not a raw
 string
>>> print r"It's\t a raw\n string"
It's\t a raw\n string
\end{verbatim}
}
\end{itemize}
\end{frame}

\begin{frame}[fragile]
  \frametitle{Strings (Cont.)}
  \begin{itemize}
  \item Concatenate and repeat
{\scriptsize
\begin{verbatim}
>>> "Hello" + "World"
'HelloWorld'
>>> "Hello"*5
'HelloHelloHelloHelloHello'
\end{verbatim}
}

\item Indexing
{\scriptsize
\begin{verbatim}
>>> s = "Hello World"
>>> s[6]
'W'
\end{verbatim}
}

\item Unicode
{\scriptsize
\begin{verbatim}
>>> print u'Danke schön für alles!'
Danke schön für alles!
>>> u'Danke schön für alles!'
u'Danke sch\xf6n f\xfcr alles!'
>>> print u'Danke sch\xf6n f\xfcr alles!'
Danke schön für alles!
>>> print 'Danke sch\xf6n f\xfcr alles!'
Danke schn fr alles!
\end{verbatim}
}

\end{itemize}
\end{frame}

\begin{frame}[fragile]
  \frametitle{Lists and tuples}
  \begin{itemize}
  \item Ordered sequence of items
{\scriptsize
\begin{verbatim}
>>> ['string', 5.0, u'unicode', 2, ['a', "b"], r'raw']
['string', 5.0, u'unicode', 2, ['a', 'b'], 'raw']
\end{verbatim}
}

\item Indexing
{\scriptsize
\begin{verbatim}
>>> print l[0], l[-1], l[-2]
string raw ['a', 'b']
>>> print l[3:], l[:3], l[:-4]
[2, ['a', 'b'], 'raw'] ['string', 5.0, u'unicode'] ['string', 5.0]
>>> print l[:]
['string', 5.0, u'unicode', 2, ['a', 'b'], 'raw']
\end{verbatim}
}

\item Other operations
{\scriptsize
\begin{verbatim}
>>> [1,2,3] + [4,5,6]
[1, 2, 3, 4, 5, 6]
>>> len ([1,2,3])
3
\end{verbatim}
}

\item A tuple is an immutable list
{\scriptsize
\begin{verbatim}
>>> (1,2)
(1, 2)
>>> t[0]
1
>>> t[0] = 0
Traceback (most recent call last):
  File "<stdin>", line 1, in <module>
TypeError: 'tuple' object does not support item assignment
\end{verbatim}
}
\end{itemize}

\end{frame}

\begin{frame}[fragile]
  \frametitle{Dictionaries}
  \begin{itemize}
  \item Unordered collection of items. Hash table
{\scriptsize
\begin{verbatim}
>>> rooms = { 'carlosgc' : 120, 'jfcogato' : 126 }
>>> rooms['carlosgc']
120
>>> rooms['islafuente'] = 118
>>> rooms
{'islafuente': 118, 'jfcogato': 126, 'carlosgc': 120}
>>> rooms.keys ()
['islafuente', 'jfcogato', 'carlosgc']
\end{verbatim}
}
\end{itemize}
\end{frame}

\begin{frame}[fragile]
  \frametitle{Control Flow}
  \begin{itemize}
  \item Code blocks are defined by its indentation.
  \item The colon (:) is used when a new code block is expected
{\scriptsize
\begin{verbatim}
if n % 2 == 0:
    print "Even number"
else:
    print "Odd number"
\end{verbatim}
}
  \item if, else, elif. There isn't case or switch
  \item while, for, break, continue
  \item Functions
{\scriptsize
\begin{verbatim}
>>> def next (n):
...     return n + 1
... 
>>> print next (2)
3
>>>
\end{verbatim}
}
  \end{itemize}
\end{frame}

\begin{frame}[fragile]
  \frametitle{Object Oriented Programming}
  \begin{itemize}
  \item Everything is an object
{\scriptsize
\begin{verbatim}
>>> isinstance (25, int)
True
\end{verbatim}
}

\item Inheritance
\item Polymorphism
\item Methods overriding (always virtual)
\item Methods and operators redefinition
\item Everything is public
\item Exceptions: try, except, finally, raise
\end{itemize}
\end{frame}

\begin{frame}[fragile]
  \frametitle{Example}
{\scriptsize
\begin{verbatim}
>>> class Figure:
...     def __init__ (self, name):
...             self.name = name
...     def what_am_i (self):
...             print "I'm a geometric figure named %s" % (self.name)
...     def get_area (self):
...             raise NotImplementedError
... 
>>> class Rectangle (Figure):
...     def __init__ (self, width=0.0, height=0.0):
...             Figure.__init__ (self, "Rectangle")
...             self.width = width
...             self.height = height
...     def get_area (self):
...             return self.height * self.width
... 
>>> r = Rectangle (10, 20)
>>> r.what_am_i ()
I'm a geometric figure named Rectangle
>>> r.get_area ()
200
\end{verbatim}
}
\end{frame}

\begin{frame}[fragile]
  \frametitle{Methods and operators redefinition}
{\scriptsize
\begin{verbatim}
>>> class Number:
...     def __init__ (self, n=None):
...             if n is not None:
...                     self.n = n
...             else:
...                     self.n = 0
...     def __add__ (self, other):
...             return self.n + other.n
... 
>>> n1 = Number (10)
>>> n2 = Number ()
>>> n1 + n2
10
>>> n3 = Number (30)
>>> n1 + n3
40
>>> n3 - n1
Traceback (most recent call last):
  File "<stdin>", line 1, in <module>
TypeError: unsupported operand type(s) for -: 'instance' and 'instance'
\end{verbatim}
}
\end{frame}

\begin{frame}[fragile]
  \frametitle{Everything is public}
  \begin{itemize}
  \item Attributes and methods starting with \_\_ are supposed to be
    private
  \end{itemize}
{\scriptsize
\begin{verbatim}
>>> class Foo:
...     def __init__ (self, p1, p2):
...             self.__p1 = p1
...             self.p2 = p2
... 
>>> f = Foo (1, 2)
>>> f.p2
2
>>> f.p1
Traceback (most recent call last):
  File "<stdin>", line 1, in <module>
AttributeError: Foo instance has no attribute 'p1'
>>> f.__p1
Traceback (most recent call last):
  File "<stdin>", line 1, in <module>
AttributeError: Foo instance has no attribute '__p1'
>>> f._Foo__p1
1
\end{verbatim}
}
\end{frame}

\begin{frame}[fragile]
  \frametitle{Exceptions}
{\scriptsize
\begin{verbatim}
>>> class NotPositiveException (Exception):
...     pass
... 
>>> class Positive:
...     def __init__ (self, n):
...             if n <= 0:
...                     raise NotPositiveException
...             self.n = n
...     def __add__ (self, other):
...             return self.n + other.n
...
>>> p = Positive (-1)
Traceback (most recent call last):
  File "<stdin>", line 1, in <module>
  File "<stdin>", line 4, in __init__
__main__.NotPositiveException
>>> try:
...     p = Positive (-1)
... except NotPositiveException:
...     p = Positive (2)
... 
>>> p.n
2
\end{verbatim}
}
\end{frame}

\begin{frame}[fragile]
  \frametitle{Modules and Packages}
  \begin{itemize}
  \item A module is a file with .py extension containing
    definitions. The name of the module is the file name without the
    extension. (i.e os.py)
{\scriptsize
\begin{verbatim}
>>> import os
>>> os.uname ()
('Linux', 'charmaleon', '2.6.28', '#1 SMP PREEMPT Wed Dec 31 14:33:16
CET 2008', 'i686')
>>> from os import uname
>>> uname ()
('Linux', 'charmaleon', '2.6.28', '#1 SMP PREEMPT Wed Dec 31 14:33:16
CET 2008', 'i686')
>>> from os import *
>>> getpid ()
6011
\end{verbatim}
}

\item A package is a directory that contains modules. In order to be
  considered a package the directory must have also a file called
  \_\_init\_\_.py (it might be empty). 

\item When a module is imported, its code is executed.
\item When a package is imported, the \_\_init\_\_.py file code is
  executed. 
\end{itemize}

\end{frame}

\begin{frame}[fragile]
  \frametitle{Some tips}

  \begin{itemize}
  \item Unpacking of lists, tuples, ...
{\scriptsize
\begin{verbatim}
>>> name, age = ('Carlos', 28)
>>> print name
Carlos
>>> print age
28
>>> a, b = b, a
\end{verbatim}
}

\item Use 'in' where possible: it's generally faster and it's available for
  containers such as lists, tuples, dicts, ...

\item Truth values: use if x: instead of if x == True: since it's
  efficient to take advantage of intrinsic truth values of objects.

\item Default parameter values
{\scriptsize
\begin{verbatim}
>>> def append (item, l = []):
...     l.append (item)
...     return l
... 
>>> print append (25)
[25]
>>> print append (30)
[25, 30]
\end{verbatim}
}
\end{itemize}
\end{frame}

\begin{frame}[fragile]
  \frametitle{Some tips (Cont.)}
  \begin{itemize}
  \item {\bf Don't use wild-card imports!}
  \item Variables are tags, not 'boxes'
  \item Portability: use the 'os' module
  \item Dictionaries: get and setdefault
{\tiny
\begin{verbatim}
>>> cache = {}
>>> for i, c in enumerate ("text to count the chars"):
...     if c not in cache:
...             cache[c] = []
...     cache[c].append (i)
>>> cache
{'a': [20], ' ': [4, 7, 13, 17], 'c': [8, 18], 'e': [1, 16], 'h': [15,
19], 'o': [6, 9], 'n': [11], 's': [22], 'r': [21], 'u': [10], 't': [0,
3, 5, 12, 14], 'x': [2]}
\end{verbatim}
\pause
\begin{verbatim}
>>> cache = {}
>>> for i, c in enumerate ("text to count the chars"):
...     cache.setdefault (c, []).append (i)
>>> cache
{'a': [20], ' ': [4, 7, 13, 17], 'c': [8, 18], 'e': [1, 16], 'h': [15,
19], 'o': [6, 9], 'n': [11], 's': [22], 'r': [21], 'u': [10], 't': [0,
3, 5, 12, 14], 'x': [2]}
\end{verbatim}
}    
  \end{itemize}
\end{frame}

\begin{frame}
  \frametitle{References}
  \begin{itemize}
  \item The Python Tutorial (\url{http://docs.python.org/tutorial/})
  \item Code Like a Pythonista: Idiomatic Python
    (\url{http://python.net/~goodger/projects/pycon/2007/idiomatic/handout.html})
  \end{itemize}
\end{frame}


