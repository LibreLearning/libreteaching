\documentclass[a4paper]{article}
%\usepackage[T1]{fontenc}
\usepackage[utf8]{inputenc}
\usepackage{url}
\usepackage[hypertex,colorlinks]{hyperref}
%\usepackage{html}
%\usepackage{hthtml}
\usepackage{geometry}

% Comments (optional argument is author of comment)
\newcommand{\comments}[2][?]{
  \begin{quote}
    \textbf{Comment (#1):} {\em #2}
  \end{quote}
  }

% \name is for ``special names'', like procedure or variable names.
\newcommand{\name}[1]{\texttt{\hturl{#1}}}

% Just a shortcut for links where the url appears as footnote.
\newcommand{\htflink}[2]{\htmladdnormallinkfoot{#1}{#2}}

\title{MSWL Introduction \\
Master on libre software \\
URJC - GSyC/Libresoft \\
\url{http://master.libresoft.es}}

\author{Jesus M. Gonzalez-Barahona, Gregorio Robles}
\date{September 2010}

\sloppy
\begin{document}
\maketitle

\begin{abstract}
Course syllabus and learning program for the course ``Introduction to libre software'', of the Master on libre software of the Universidad Rey Juan Carlos (Móstoles, Spain).

[This is an evolving document, until the course is finished and graded]
\end{abstract}

\tableofcontents

%%----------------------------------------------------------------
%%----------------------------------------------------------------
%%----------------------------------------------------------------
\section{Course topics and schedule}

%%----------------------------------------------------------------
%%----------------------------------------------------------------
\subsection{00 - Presentation of the course}

Presentation of the main aspects of the course, and specially those related to administrative issues, evaluation, etc.

%%----------------------------------------------------------------
\subsubsection{November 19, 2010 (0.5 hours)}


\section{Grading}

This section details the criteria for grading the course, the deadlines for the different activities, and the submission details for th activities that require them.

\subsection{Evaluation criteria}
\label{sub:evaluation-criteria}

Each activity contributing to the grading of the course has its own evaluation criteria, as described below. Each of these activities has a minimum and maximum grading. If the minimum grading is 0, the activity is optional. Otherwise, the activity is mandatory, and has to be graded al least with the minimum to pass the course. Each activity has also a description, and when possible, some general grading criteria. In any case, the final grade for the course will also depend on the continuous observation of the instructors on the outcomes and progress of students.

Students should ask instructors about any detail which may not be clear to them, either about the general grading plan, or about specific aspects of the activities. As a general rule, evaluation will have into account how the activity and its results show that the student has come close to the competences, knowledge and skills expected for the course.

The student can consider that the next table will be used as a (minimum) guideline for assigning marks:

\begin{itemize}
\item Pass (``aprobado''): 150
\item Good (``notable''): 250
\item Excellent (``sobresaliente): 350
\end{itemize}

\begin{itemize}
\item \textbf{Exercises (answered in forum)}. \\
  Minimum: 20 points, maximum: 80 points.

  Exercises proposed and answered in the the forum of the course.

\item \textbf{Blog entries}. \\
  Minimum: 20 points, maximum: 80 points

  Blog entries specifically related to the course, and marked as such. The tag used for that is mswl-intro.

\item \textbf{Collaborative notebook}. \\
  Minimum: 0 points, maximum: 40 points

  Based on work in class (in real time) and afterwards (complementing the work, using git).

\item \textbf{Video presentation}. \\
  Minimum: 20 points, maximum: 80 points

  Can be an screencast, or a more ellaborate kind of video. It has to explain some topic covered by the course. The most focused, the better: try to explain only one issue, but explain it as well as possible. The video wil be uploaded to some video web hosting site which allows for download of the whole video file, not only streaming (eg, blip.tv).

\item \textbf{Specific report}. \\
  Minimum: 20 points, maximum: 120 points

  Specific report about a relevant aspect of libre software, related to the topics dealt with in this course. Can be a traditional written report, but can also be a presentation (recorded in video, in this case), a video, a podcast, etc. It is important to detail all the references, and to heavily root the report on data and/or specific works publicly available.

\item \textbf{Other activities}. \\
  Minimum: 0 points, maximum: 100 points

  These activities have to be agreed with the instructors.
\end{itemize}

\subsection{Submission deadlines}

All activities to be graded in January must be completed and submitted by December 31st 2010.

\subsection{Submission details}

Please, consider the details below for submitting the different activities for evaluation (for those not specified in this list, nothing special is needed for submission).

\begin{itemize}
\item As a summary of all the activities, a ``Summary of activities for evaluation'' should be sent. This summary should be uploaded to the corresponding resource in the Moodle site for this course, and should include the following data:
  \begin{itemize}
  \item \textbf{Name:} Full name of the student (as ``family name'', ``given name'')
  \item \textbf{Blog entries:} Url of the blog entries for this course (HTML, not RSS version).
  \item \textbf{Contributions to the collaborative notebook:} Id for commits to the repository where the collaborative notebook is hosted, and summary of the main contributions to it related to this course, including links to the repository and commit ids if appropriate.
  \item \textbf{Video presentation:} Title and url where the video can be seen/downloaded (remember that the video should be uploaded to some video hosting site).
  \item \textbf{List of other activities:} If any, list of other activities submitted for evaluation (those that would fit in the ``other activities'' item in the ``Evaluation criteria'' (subsection~\ref{sub:evaluation-criteria}). The results of those activities should be uploaded to the ``other activities'' resource in the Moodle site for this course, when appropriate. In some specific cases (such as streaming videos) it will be enough to include in this list the url to the external site where the result is hosted.
  \end{itemize}
\end{itemize}

%%----------------------------------------------------------------
%%----------------------------------------------------------------
%%----------------------------------------------------------------
%\section{Assignments and activities}

%%----------------------------------------------------------------
%%----------------------------------------------------------------
%\subsection{01 - Introduction and motivation}


%%----------------------------------------------------------------
%\subsubsection{Statements about economic aspects of libre software}
%\label{sub:statements-eco}

\end{document}
