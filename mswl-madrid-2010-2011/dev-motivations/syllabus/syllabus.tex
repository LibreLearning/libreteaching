\documentclass[a4paper]{article}
%\usepackage[T1]{fontenc}
\usepackage[utf8]{inputenc}
\usepackage{url}
\usepackage[hypertex,colorlinks]{hyperref}
%\usepackage{html}
%\usepackage{hthtml}
\usepackage{geometry}

% Comments (optional argument is author of comment)
\newcommand{\comments}[2][?]{
  \begin{quote}
    \textbf{Comment (#1):} {\em #2}
  \end{quote}
  }

% \name is for ``special names'', like procedure or variable names.
\newcommand{\name}[1]{\texttt{\hturl{#1}}}

% Just a shortcut for links where the url appears as footnote.
\newcommand{\htflink}[2]{\htmladdnormallinkfoot{#1}{#2}}

\title{MSWL Developers and motivation \\
Master on libre software \\
URJC - GSyC/Libresoft \\
\url{http://master.libresoft.es}}

\author{Felipe Ortega}
\date{October 2010}

\sloppy
\begin{document}
\maketitle

\begin{abstract}
Course syllabus and learning program for the course ``Developers and their motivations'', 
of the Master on libre software of the Universidad Rey Juan Carlos (Móstoles, Spain).

[This is an evolving document, until the course is finished and graded]
\end{abstract}

\tableofcontents

\section{Grading plan}

Each activity contributing to the grading of the course has its own evaluation criteria, 
as described below. Each of these activities has a minimum and maximum grading. If the 
minimum grading is 0, the activity is \textit{optional}. Otherwise, the activity is \textit{mandatory}, and 
has to be graded al least with the minimum to pass the course. Each activity has also a 
description, and when possible, some general grading criteria. In any case, the final grade 
for the course will also depend on the continuous observation of the instructors on the 
outcomes and progress of students.

Students should ask instructors about any detail which may not be clear to them, 
either about the general grading plan, or about specific aspects of the activities. 
As a general rule, evaluation will have into account how the activity and its results 
show that the student has come close to the competences, knowledge and skills expected 
for the course.

The student can consider that the next table will be used as a (minimum) guideline for 
assigning marks:

\begin{itemize}
\item Pass (``aprobado''): 150
\item Good (``notable''): 250
\item Excellent (``sobresaliente): 350
\end{itemize}

\subsection{Graded activities}

\begin{itemize}
\item \textbf{Exercises (answered in forum)}. \\
  Minimum: 20 points, maximum: 100 points.

  Exercises proposed and answered in the the forum of the course.

\item \textbf{Blog entries}. \\
  Minimum: 20 points, maximum: 80 points

  Blog entries specifically related to the course, and marked as such. The tag used for that is mswl-intro.

\item \textbf{Collaborative notebook}. \\
  Minimum: 0 points, maximum: 40 points

  Based on work in class (in real time) and afterwards (complementing the work, using git).

\item \textbf{Specific report on leadership}. \\
  Minimum: 20 points, maximum: 140 points

  Specific written report (5-10 pages) about a prominent leader in FLOSS. An initial list of candidate profiles to choose
from will be published in short on the course website. It is important to detail all the references, and to 
heavily root the report on data and/or specific works publicly available.

  Some aspects that must be considered include:
  \begin{itemize}
   \item Biography and relevant details to contextualize their contributions in FLOSS.
   \item Annotated description of the evolution of the FLOSS project they lead, or have led in the past.
   \begin{itemize}
    \item Founder? Arrived lately?
    \item If possible, try to quantify their contributions (number of commits, bug fixes or other relevant 
    activities to boost the project).
    \item Is this person still leading the project? If that is not the case, why did he/she left?
    \item Was he/she essential for the success of the project or the surrounding community? Why?
   \end{itemize}

   \item It is possible that they have lead more than one project. Then, students can either focus on just
   one of these projects or enlighten the transitions between projects and their reasons.
   \item A thorough description of the main leadership traits exhibited by the analyzed figure is essential.
  \end{itemize}


\item \textbf{Other activities}. \\
  Minimum: 0 points, maximum: 100 points

  These activities have to be agreed with the instructors.
\end{itemize}

\section{Sessions}

\subsection{Session 1 (Thu. October 14, 2010)}

\begin{itemize}
 \item \textbf{Professors}: José Gato and Pedro Coca.

 \item \textbf{Content}:

  \begin{enumerate}
   \item Characterization of libre software developers.
   \item Analysis of the paper \textit{Geographic origin of libre software developers}, by J.M. González-Barahona,
  G. Robles, R. Andradas-Izquierdo and R. Gosh. Information Economics and Policy Volume 20, Issue 4, December 2008, Pages 356-363
  Empirical Issues in Open Source Software.

    \item Analysis of the essay \textit{The Cathedral and the bazaar}, by Eric. S. Raymond. Questions and open debate in class.
  \end{enumerate}

\end{itemize}

\subsection{Session 2 (Thu. October 21, 2010)}

\begin{itemize}
 \item \textbf{Professor}: Felipe Ortega.

 \item \textbf{Content}:

  \begin{enumerate}
   \item Practice: The OpenSSL affair in Debian. Working in groups, we analyzed in detail the case study of the security problem
originated by a flaw in OpenSSL package in Debian. We went through all details of the story, documenting each step with mailing
lists messages, excerpts from commits, and comments and posts from developers and experts involved in this issue.

   \item Open debate: We divided the class in two groups. One group assumed the role of Debian, while the others acted as OpenSSL. Each
team prepared a short slide presentation (including as many concrete proofs as possible) to show that the other group was liable for
this OpenSSL affair in Debian.
  \end{enumerate}

\end{itemize}

\subsection{Session 3 (Thu. October 28, 2010)}

  \begin{itemize}
   \item \textbf{Professor}: Felipe Ortega.

   \item \textbf{Content}:

   \begin{enumerate}
    \item Roles in FLOSS projects and open communities: Theoretical presentation summarizing the main profiles found in FLOSS and other
open communties, and the necessary skills to match each profile.

    \item Exercise: In two groups, we analyzed the roles in Ubuntu and Fedora communities. Explain the most salient traits found in each main
role, and how roles are linked with teams, groups, management and technical plans.

    \item Organization in FLOSS and open communities: We explored 2 case studies (Apache SF and Mozilla Found.) to learn different approaches
follow for community organization and governance.

    \item Exercise: Search for information about internal organization in the Wikipedia communty.

    \item Homework: Search for details about organization and different roles in Debian. Answer some concrete questions posted in the Q/A forum.
   \end{enumerate}

  \end{itemize}

\subsection{Session 4 (Thu. November 4, 2010)}

\begin{itemize}
 \item \textbf{Professors}: Juan José Amor and Miquel Vidal.

  \item \textbf{Content}:

  \begin{enumerate}
   \item FLOSS leaders: Short introduction on the story and origin of FLOSS leaders, their commont traits and skill.

   \item Group dynamic: Create a short presentation about a FLOSS leader (10 different profiles presented). For each leader, find out some
details about bio, projects that founded or led, current activities, what they do now for a living, etc.

   \item \textbf{Main assignment}: Ellaborate a short work summarizing in 10 pages the complete profile of a FLOSS leader of your choice.
  \end{enumerate}

\end{itemize}

\subsection{Session 5 (Thu. November 11, 2010)}

\begin{itemize}
 \item \textbf{Professor}: Felipe Ortega.

 \item \textbf{Content}:

 \begin{enumerate}
  \item Social structure of FLOSS communities: We introduced the paper \textit{The Social Structure of Free and Open Source Software Development}, by
K. Crowston and J. Howison, First Monday Vol. 10 (2), 2005. Then, we presented different methodologies to analyze the structure of FLOSS communities.

  \item Exercise: We explored the quantitative data from the SVN and bug tracking system of the \textit{brasero} project in GNOME. Retrieve the information
already available in Flossmetrics (\texttt{http://melquiades.flossmetrics.org}). Answer 4 concrete questions to study whether the onion model matches this community.
 \end{enumerate}

\end{itemize}

\end{document}
