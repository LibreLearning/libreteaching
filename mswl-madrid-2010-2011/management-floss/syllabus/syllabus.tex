\documentclass[a4paper]{article}
%\usepackage[T1]{fontenc}
\usepackage[utf8]{inputenc}
\usepackage{url}
\usepackage[pdfborder=0 0 0]{hyperref}
%\usepackage{html}
%\usepackage{hthtml}
\usepackage{geometry}

% Comments (optional argument is author of comment)
\newcommand{\comments}[2][?]{
  \begin{quote}
    \textbf{Comment (#1):} {\em #2}
  \end{quote}
  }

% \name is for ``special names'', like procedure or variable names.
\newcommand{\name}[1]{\texttt{\hturl{#1}}}

% Just a shortcut for links where the url appears as footnote.
\newcommand{\htflink}[2]{\htmladdnormallinkfoot{#1}{#2}}

\title{MSWL Project Management \\
Master on libre software \\
URJC - GSyC/Libresoft \\
\url{http://master.libresoft.es}}

\author{Felipe Ortega, Daniel Izquierdo Cortázar}
\date{January 2011}

\sloppy
\begin{document}
\maketitle

\begin{abstract}
Course syllabus and learning program for the course ``Project Management'', of the Master on libre software at Universidad Rey Juan Carlos (Móstoles, Spain).

[This is an evolving document, until the course is finished and graded]
\end{abstract}

\tableofcontents

%%----------------------------------------------------------------
%%----------------------------------------------------------------
%%----------------------------------------------------------------
\section{Course topics and schedule}

%%----------------------------------------------------------------
%%----------------------------------------------------------------
\subsection{00 - Presentation of the course}

Presentation of the main aspects of the course, and specially those related to administrative issues, evaluation, etc.

%%----------------------------------------------------------------
\subsubsection{November 19, 2010 (1 hour)}

\begin{itemize}
\item \textbf{Lecturer:} Daniel Izquierdo Cortázar
\item \textbf{Presentation:} ``Presentation of the course''
  \begin{itemize}
  \item \textbf{Supporting material:} Slides ``Presentation of the course''
  \end{itemize}
\item \textbf{Main assignment (results in website):} Assessment report on best practices to release a software project as FLOSS (assignment~\ref{sub:report-FLOSS-release})

\item \textbf{Supporting materials:} 
  \begin{itemize}
  \item Text: ``Producing OSS'', by Karl Fogel (O'Reilly Media, 2005)\\
    \url{http://producingoss.com}
  \item Text: ``Art of Community'', by Jono Bacon (O'Reilly Media, 2009)\\
    \url{http://www.artofcommunityonline.org} 
  \end{itemize}
\end{itemize}

%%----------------------------------------------------------------
%%----------------------------------------------------------------
\subsection{01 - Introduction to infrastructure used in Libre Software Communities}

Introduction to the basic infrastructure and services that are essential for correct management of FLOSS projects.

%%----------------------------------------------------------------
\subsubsection{January 14, 2011 (3 hours)}

\begin{itemize}
\item \textbf{Lecturer:} Daniel Izquierdo
\item \textbf{Presentation:} ``Introduction to infrastructure: tips and clues''
\item \textbf{Presentation:} ``Tools for documentation and communication''
  \begin{itemize}
  \item \textbf{Supporting material:} Slides ``Introduction to infrastructure: tips and clues''
  \item \textbf{Supporting material:} Slides ``Tools for documentation and communication''
  \end{itemize}
\item \textbf{Discussion and assignment (results in website):} The origin of GNOME and KDE (assignment~\ref{sub:gnome-kde})
\item \textbf{Discussion and assignment (results in website):} Community forks: Compiz and Beryl (assignment~\ref{sub:compiz-beryl})
\end{itemize}

%%----------------------------------------------------------------
\subsubsection{January 21, 2011 (1 hour)}

Review and discussion of pending assignments.

\begin{itemize}
\item \textbf{Lecturer:} Felipe Ortega
 \item \textbf{Presentation:} ``The role of community managers''.
    \begin{itemize}
      \item \textbf{Supporting material:} Slides ``Community manager''.
    \end{itemize}
\item \textbf{Discussion and assignment (results in website):} Profiles of featured community managers (assignment~\ref{sub:community-managers})
\end{itemize}

%%----------------------------------------------------------------
%%----------------------------------------------------------------
\subsection{02 - Forges for software development}

Introduction to forges for software development: services, community support and strategies for correct project management.

%%----------------------------------------------------------------
\subsubsection{January 21, 2011 (3 hours)}

\begin{itemize}
\item \textbf{Lecturers}: José Francisco Castro, Felipe Ortega
\item \textbf{Guest presentation:} ``Forges for software development''.
  \begin{itemize}
  \item \textbf{Supporting material:} Slides ``Forges for software development''.
  \end{itemize}
\item \textbf{Discussion and assignment (results in website):} Analysis of 3 relevant forges for software development (assignment~\ref{sub:forges-analysis})
\item \textbf{Assignment (results in website):} Choosing the best forge to release our software project. (assignment~\ref{sub:report-FLOSS-release})
\end{itemize}

%%----------------------------------------------------------------
%%----------------------------------------------------------------
\subsection{03 - Case studies in project management}

Invited talks by relevant leaders of prominent FLOSS projects, dealing with different aspects of project management in their respective communities.

%%----------------------------------------------------------------
\subsubsection{January 28, 2011 (4 hours)}

\begin{itemize}
\item \textbf{Lecturers}: Adriaan de Groot (KDE) and Santiago Gala (ASF).
\item \textbf{Guest presentation:} ``The KDE community''.
  \begin{itemize}
  \item \textbf{Supporting material:} Slides ``The KDE community''.
  \end{itemize}
\item \textbf{Guest presentation:} ``Apache Software Foundation''.
  \begin{itemize}
  \item \textbf{Supporting material:} Slides ``Apache Software Foundation''.
  \end{itemize}
\item \textbf{Discussion and assignment (results in website):} Social skills and developers (assignment~\ref{sub:social-skills})
\item \textbf{Assignment (results in website):} Legal issues and project management. (assignment~\ref{sub:legal-manage})
\end{itemize}

%%----------------------------------------------------------------
%%----------------------------------------------------------------
\subsection{04 - Communication, outreach and netiquette}

Effective outreach strategies in FLOSS projects, netiquette.

%%----------------------------------------------------------------
\subsubsection{February 4, 2011 (4 hours)}

\begin{itemize}
\item \textbf{Lecturer}: Felipe Ortega.
\item \textbf{Presentation:} ``Communication and outreach in FLOSS projects''.
  \begin{itemize}
  \item \textbf{Supporting material:} Slides ``Communication and outreach''.
  \end{itemize}
\item \textbf{Discussion and assignment (results in website):} On-line personalities (assignment~\ref{sub:on-line-personalities})
\item \textbf{Assignment (results in website):} Study of netiquette policies and guidelines (assignment~\ref{sub:netiquette})
\end{itemize}

%%----------------------------------------------------------------
%%----------------------------------------------------------------
\subsection{05 - Volunteer management}

Introduction to the management of volunteers in FLOSS projects, community anti-patterns.

%%----------------------------------------------------------------
\subsubsection{February 11, 2011 (4 hours)}

\begin{itemize}
\item \textbf{Lecturer}: Teo Romera.
\item \textbf{Debate:} Managing volunteer activity in free software projects (assignment~\ref{sub:manage-volunteer})
\item \textbf{Discussion:} Bikeshed \url{http://bikeshed.org} (assignment~\ref{sub:bikeshed})
\item \textbf{Presentation:} ``Managing volunteers''.
  \begin{itemize}
  \item \textbf{Supporting material:} Slides ``Managing volunteers''.
  \end{itemize}
\item \textbf{Video and discussion:} David Neary ``Community anti-patterns''. (assignment~\ref{sub:anti-patterns})
\end{itemize}

%%----------------------------------------------------------------
%%----------------------------------------------------------------

\section{Grading}

%%----------------------------------------------------------------

This section details the criteria for grading the course, the deadlines for the different activities, and the submission details for th activities that require them.

\subsection{Evaluation criteria}
\label{sub:evaluation-criteria}

Each activity contributing to the grading of the course has its own evaluation criteria, as described below. Each of these activities has a minimum and maximum grading. If the minimum grading is 0, the activity is optional. Otherwise, the activity is mandatory, and has to be graded at least with the minimum to pass the course. Each activity has also a description, and when possible, some general grading criteria. In any case, the final grade for the course will also depend on the continuous observation of the instructors on the outcomes and progress of students.

Students should ask instructors about any detail which may not be clear to them, either about the general grading plan, or about specific aspects of the activities. As a general rule, evaluation will have into account how the activity and its results show that the student has come close to the competences, knowledge and skills expected for the course.

The student can consider that the next table will be used as a (minimum) guideline for assigning marks:

\begin{itemize}
\item Pass (``aprobado''): 130
\item Good (``notable''): 190
\item Excellent (``sobresaliente''): 250
\end{itemize}

\begin{itemize}
\item \textbf{Exercises (answered in forum)}. \\
  Minimum: 20 points, maximum: 100 points.

  Exercises proposed and answered in the the forum of the course.

\item \textbf{Blog entries}. \\
  Minimum: 20 points, maximum: 90 points

  Blog entries specifically related to the course, and marked as such. The tag used for that is \textit{mswl-manage}.

\item \textbf{Collaborative notebook}. \\
  Minimum: 0 points, maximum: 40 points

  Based on work in class (in real time) and afterwards (complementing the work, using git).

\item \textbf{Specific report}. \\
  Minimum: 30 points, maximum: 100 points

Specific report presenting a complete assessment on how to release an software project from an SME company as FLOSS.

As a result of this activity, the student should produce a written report (please, check specific requirements in section~\ref{sub:report-FLOSS-release}).

It is important to detail all the references, and to heavily root the report on data and/or specific works publicly available.

\item \textbf{Other activities}. \\
  Minimum: 0 points, maximum: 100 points

  These activities have to be agreed in advance with the instructors.
\end{itemize}

\subsection{Submission deadlines}

All activities to be graded in January must be completed and submitted \textbf{by March 15th, 2010}.

\subsection{Submission details}

Please, consider the details below for submitting the different activities for evaluation (for those not specified in this list, nothing special is needed for submission).

\begin{itemize}
\item As a summary of all the activities, a ``Summary of activities for evaluation'' should be sent. This summary should be uploaded to the corresponding resource in the Moodle site for this course, and should include the following data:
  \begin{itemize}
  \item \textbf{Name:} Full name of the student (as ``family name'', ``given name'')
  \item \textbf{Blog entries:} Url of the blog entries for this course (HTML, not RSS version).
  \item \textbf{Contributions to the collaborative notebook:} Id for commits to the repository where the collaborative notebook is hosted, and summary of the main contributions to it related to this course, including links to the repository and commit ids if appropriate.
  \item \textbf{List of other activities:} If any, list of other activities submitted for evaluation (those that would fit in the ``other activities'' item in the ``Evaluation criteria'' (subsection~\ref{sub:evaluation-criteria}). The results of those activities should be uploaded to the ``other activities'' resource in the Moodle site for this course, when appropriate. In some specific cases (such as streaming videos) it will be enough to include in this list the url to the external site where the result is hosted.
  \end{itemize}
\end{itemize}

%%----------------------------------------------------------------
%%----------------------------------------------------------------
%%----------------------------------------------------------------
\section{Assignments and activities}

%%----------------------------------------------------------------
%%----------------------------------------------------------------
\subsection{00 - Main assignment: Report}

%%----------------------------------------------------------------
\subsubsection{Assessment on how to release a FLOSS project}
\label{sub:report-FLOSS-release}

\begin{itemize}
 \item \textbf{Group size}: Individual or in pairs.
 \item \textbf{Max. length}: 30 pages.
\end{itemize}

You work for an SME company evaluating how to release one of their products as FLOSS effectively.

For sure, we'd like to create community around it and attract new users and contributors. A target business model similar to MySQL (dual licensing) is our main goal.

Students will write a report summarizing the main aspects to be considered for the release plan, including:

\begin{itemize}
 \item Current competitors.
 \item Technical Infrastructure needed.
 \item Social and political organization of the project/community.
 \item Communication strategy and channels.
 \item Development plan (good practices for source code development) and roadmap.
 \item Managing volunteers and attracting new users.
 \item Brief discussion about licenses (your company has heard about some BSD or GPL, but they are not sure!).
\end{itemize}

We will tackle individual sections of the whole report as we move along the subject.

%%----------------------------------------------------------------
%%----------------------------------------------------------------
\subsection{01 - Introduction to infrastructure used in Libre Software Communities}

%%----------------------------------------------------------------
\subsubsection{The origin of GNOME and KDE}
\label{sub:gnome-kde}

Search for information about the beginning of both communities. Present it to the rest of the class.

As a second question, do you think that this division of efforts is successful enough, nowadays? 

%%----------------------------------------------------------------
\subsubsection{Community forks: Compiz and Beryl}
\label{sub:compiz-beryl}

Following the previous assignment: 
\begin{itemize}
 \item Why did these two communities decide to divide their efforts?
 \item Why did these two communities decide to combine efforts later? 
\end{itemize}

%%----------------------------------------------------------------
\subsubsection{Profiles of featured community managers}
\label{sub:community-managers}

Split up in groups of 3 persons, and find out information about the profile of any of these prominent community managers:

\begin{itemize}
 \item Adam Williamson.
 \item Jono Bacon.
 \item Joe Brockmeier.
 \item Greg Dekoenigsberg.
 \item Jim Grisanzio.
\end{itemize}

\textbf{Supporting material}

\begin{itemize}
\item Slides ``Community manager''
\end{itemize}

%%----------------------------------------------------------------
%%----------------------------------------------------------------
\subsection{02 - Forges for software development}

%%----------------------------------------------------------------
\subsubsection{Analysis of 3 relevant forges for software development}
\label{sub:forges-analysis}

Split up in groups of 3 persons, analyze the main characteristics, services and workflows found in the following 3 featured forges:

\begin{itemize}
 \item GitHub.
 \item SourceForge.
 \item Fusion Forge.
\end{itemize}

\textbf{Supporting material}

\begin{itemize}
\item Slides ``Forges for software development''
\end{itemize}

%%----------------------------------------------------------------
%%----------------------------------------------------------------
\subsection{03 - Case studies in project management}

%%----------------------------------------------------------------
\subsubsection{Social skills and developers}
\label{sub:social-skills}

One of the featured comments from Adriaan De Groot's talk was about the need to consider social skills 
in developers profiles. In particular, he stressed how much benefit they can get from them as for their 
relationship with the community around a FLOSS project.

\begin{itemize}
 \item Can you find examples, documents, blog posts or any other sources supporting this opinion?
 \item What do you think about this issue, from the project management point of view?
 \item Who should be responsible for preserving the good mood and fruitful relationships among project members?
\end{itemize}

%%----------------------------------------------------------------
\subsubsection{Legal issues and project management}
\label{sub:legal-manage}

Santiago Gala referred to the recent patents lawsuit between Oracle and Google, regarding the license 
of some source code files included in the Android project. One reference was a recent post by 
Florian Mueller on this topic:

\url{http://fosspatents.blogspot.com/2011/01/new-evidence-supports-oracles-case.html}

\begin{itemize}
 \item Which should be the main responsibilities for a FLOSS project manager regarding licensing and legal issues?
 \item Are there any legal risks that could be "inherited" by adopters of external code released under a wrong license, 
and included in our FLOSS project?
 \item What kind of mechanisms can be adopted to avoid these situations?
\end{itemize}

%%----------------------------------------------------------------
%%----------------------------------------------------------------
\subsection{04 - Communication, outreach and netiquette}

%%----------------------------------------------------------------
\subsubsection{On-line personalities}
\label{sub:on-line-personalities}

We study in this activity the case of a well-known admin and bureaucrat in the English Wikipedia that was involved in a highly
controversial case about on-line profiles.

Split up in 3-4 groups to answer the following questions:

\begin{itemize}
 \item What was the whole story regarding Essjay digital profile?
 \item What happened with him after he left Wikipedia?
 \item What was the opinion about him from project leaders?
 \item And the opinion about him from other users?
 \item Compare this case with the hints presented in this class.
 \item Did this cause any change in Wikipedia policies and guidelines?
\end{itemize}

%%----------------------------------------------------------------
\subsubsection{Netiquette rules}
\label{sub:netiquette}

We organize the class in 4 different groups. For each one, analyze the netiquette rules that one should follow in the following
scenarios:

\begin{itemize}
 \item Mailing lists.
 \item Instant messaging, IRC.
 \item Version Control Systems, Issue Tracking Systems.
 \item Forum and wikis.
\end{itemize}

%%----------------------------------------------------------------
%%----------------------------------------------------------------
\subsection{05 - Volunteer management}

%%----------------------------------------------------------------
\subsubsection{Netiquette rules}
\label{sub:manage-volunteer}

Debate about the content of the paper ``Managing volunteer activity in free software projects'', by Martin
Mychlmayr.

%%----------------------------------------------------------------
\subsubsection{Bikeshed}
\label{sub:bikeshed}

Visit the website \url{http://bikeshed.org}. What can we learn from this story about on-line relationships?

%%----------------------------------------------------------------
\subsubsection{Community anti-patterns}
\label{sub:anti-patterns}

The activity has two phases:

\begin{itemize}
 \item Watch the video ``Community anti-patterns'', talk by David Neary at Meego Conference 2010.
 \item Discuss the applicability and accuracy of the anti-patterns for relationships in on-line communities described in this talk.
\end{itemize}

\textbf{Supporting material}

\begin{itemize}
\item Slides ``Community anti-patters'': \url{http://www.slideshare.net/nearyd/community-antipatterns}
\end{itemize}

%%----------------------------------------------------------------
%%----------------------------------------------------------------
\end{document}
