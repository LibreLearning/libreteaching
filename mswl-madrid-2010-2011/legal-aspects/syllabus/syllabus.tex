\documentclass[a4paper]{article}
%\usepackage[T1]{fontenc}
\usepackage[utf8]{inputenc}
\usepackage{url}
\usepackage[hypertex,colorlinks]{hyperref}
%\usepackage{html}
%\usepackage{hthtml}
\usepackage{geometry}

% Comments (optional argument is author of comment)
\newcommand{\comments}[2][?]{
  \begin{quote}
    \textbf{Comment (#1):} {\em #2}
  \end{quote}
  }

% \name is for ``special names'', like procedure or variable names.
\newcommand{\name}[1]{\texttt{\hturl{#1}}}

% Just a shortcut for links where the url appears as footnote.
\newcommand{\htflink}[2]{\htmladdnormallinkfoot{#1}{#2}}

\title{MSWL Legal Aspects \\
Master on libre software \\
URJC - GSyC/Libresoft \\
\url{http://master.libresoft.es}}

\author{Miguel Vidal}
\date{October-November 2010}

\sloppy
\begin{document}
\maketitle

\begin{abstract}
Course syllabus and learning program for the course ``Legal Aspects'', 
of the Master on libre software of the Universidad Rey Juan Carlos (Móstoles, Spain).

[This is an evolving document, until the course is finished and graded]
\end{abstract}

\tableofcontents

\section{Grading plan}

Each activity contributing to the grading of the course has its own evaluation criteria, 
as described below. Each of these activities has a minimum and maximum grading. If the 
minimum grading is 0, the activity is \textit{optional}. Otherwise, the activity is \textit{mandatory}, and 
has to be graded al least with the minimum to pass the course. Each activity has also a 
description, and when possible, some general grading criteria. In any case, the final grade 
for the course will also depend on the continuous observation of the instructors on the 
outcomes and progress of students.

Students should ask instructors about any detail which may not be clear to them, 
either about the general grading plan, or about specific aspects of the activities. 
As a general rule, evaluation will have into account how the activity and its results 
show that the student has come close to the competences, knowledge and skills expected 
for the course.

The student can consider that the next table will be used as a (minimum) guideline for 
assigning marks:

\begin{itemize}
\item Pass (``aprobado''): 150
\item Good (``notable''): 250
\item Excellent (``sobresaliente''): 350
\end{itemize}

\subsection{Graded activities}

\begin{itemize}
\item \textbf{Exercises (answered in forum)}. \\
  Minimum: 20 points, maximum: 100 points.

  Exercises proposed and answered in the the forum of the course.

\item \textbf{Blog entries}. \\
  Minimum: 20 points, maximum: 80 points

  Blog entries specifically related to the course, and marked as such. The tag used for that is mswl-legal.

\item \textbf{Collaborative notebook}. \\
  Minimum: 0 points, maximum: 40 points

  Based on work in class (in real time) and afterwards (complementing the work, using git).

\item \textbf{Specific work about a Legal Aspect}. \\
  Minimum: 20 points, maximum: 140 points

  Specific work about a relevant legal aspect of libre software, related to the topics dealt with in this course. Can be a traditional written report, but can also be a presentation (recorded in video, in this case), a video, a podcast, etc. It is important to detail all the references, and to heavily root the report on data and/or specific works publicly available.

Some ideas:

\begin{itemize}
\item IP (copyrights, patents) lawsuits against FLOSS
\item Infringement and violations of FLOSS licenses.
\item How to affect the patent system to libre software.
\item How to affect SaaS (software on demand, cloud computing, etc.) to FLOSS licensing.
\item Analysis of a specific FLOSS license or comparison between FLOSS licenses.
\item Choosing the right license: considering business models, product architecture,  IP ownership, license compatibily issues, relicensing, etc.
\item Thinking about derivate works: linking and licensing, dailylife with FLOSS licenses, etc.
\item Etc.

\end{itemize}



\item \textbf{Other activities}. \\
  Minimum: 0 points, maximum: 100 points

  These activities have to be agreed with the instructors.
\end{itemize}

\end{document}
