%% introPI.tex
%%
%% Presentation of the course ``Legal Issues'' of the Official Master on Libre Software (URJC)
%% http://master.libresoft.es
%%

%%---------------------------------------------------------------------
%%---------------------------------------------------------------------

\begin{frame}
  \frametitle{Course Contents}

%  \begin{itemize}[<+->]
  \begin{itemize}
    \item Lesson 1: Intellectual Property: basic concepts and legal framework
    \item Lesson 2: Legal Aspects of Libre Software
    \item Lesson 3: Libre software licenses
    \item Lesson 4: Free cultural works licenses
    \item Lesson 5: Case studies
  \end{itemize}

\end{frame}


%%%%%%%%%%%%%%%%%%%%%%%%%%%%%%%%%%%%%%%%%%%%%%%%%%%%%%%%%%%%%%%%%%%%%%%
\section{Lesson I: Intellectual Property: basic concepts and legal framework}
%%%%%%%%%%%%%%%%%%%%%%%%%%%%%%%%%%%%%%%%%%%%%%%%%%%%%%%%%%%%%%%%%%%%%%%



%%---------------------------------------------------------------




%%%%%%%%%%%%%%%%%%%%%%%%%%%%%%%%%%%%%%%%%%%%%%%%%%%%%%%%%%%%%%%%%%%%%%%

\begin{frame}
\frametitle{The importance of licenses}
\textit{``\alert{I} \alert{A}m \alert{N}ot \alert{A} \alert{L}awyer (IANAL) and I never read licenses... why should I care about licenses?''}

\pause

\begin{itemize}[<+->]
\item Licenses provides terms of use of a work.
\item Licenses enable the opportunity to free a work.
\item Free licenses are not just another license: they are a declaration of principles, a social contract.
\end{itemize}
\bigskip

\pause

\alert{Licenses (free or not) are based on every country's \textit{copyright} law.}

\end{frame}

%%%%%%%%%%%%%%%%%%%%%%%%%%%%%%%%%%%%%%%%%%%%%%%%%%%%%%%%%%%%%%%%%%%%%%%


\begin{frame}
\frametitle{Law and Code}

\begin{itemize}
\item Tiny SCO group sued the huge IBM in 2005 put forward a cluster of complaints: trademarks, copyright infringements and theft of trade secrets...
\item Software patents lawsuits. 
\end{itemize}

\end{frame}




%%%%%%%%%%%%%%%%%%%%%%%%%%%%%%%%%%%%%%%%%%%%%%%%%%%%%%%%%%%%%%%%%%%%%%%

\begin{frame}
\frametitle{What is Intellectual Property?}

\begin{itemize}
\item Intellectual Property (IP) refers to creations of the human mind.
\item IP is divided into two categories: Industrial Property and Copyright.  
\end{itemize}

\end{frame}

%%%%%%%%%%%%%%%%%%%%%%%%%%%%%%%%%%%%%%%%%%%%%%%%%%%%%%%%%%%%%%%%%%%%%%%

\begin{frame}
\frametitle{IP: Concepts}

\begin{itemize}
\item IP: refers to several issues, depending on its context.
\item Today used referring to the privileges of non-physical goods with economic value.
\end{itemize}

\end{frame}


%%%%%%%%%%%%%%%%%%%%%%%%%%%%%%%%%%%%%%%%%%%%%%%%%%%%%%%%%%%%%%%%%%%%%%%

\begin{frame}
\frametitle{What is Intellectual Property}

\textsc{WIPO} gives an ``common law'' definition of IP:

\begin{block}{WIPO Definition}
  Intellectual property refers to creations of the mind:
  inventions, literary and artistic works, and symbols, names, images,
  and designs used in commerce.
\end{block}

\end{frame}



%%%%%%%%%%%%%%%%%%%%%%%%%%%%%%%%%%%%%%%%%%%%%%%%%%%%%%%%%%%%%%%%%%%%%%%

\begin{frame}
\frametitle{Origins}

\begin{itemize}
\item With the invention of the press, works became commercial objects.
\item First forms of plagiarism appeared, so that editors forced legislators
to regulate and protect the original works.
\item Regulation was also conceived as a way of controlling information (i.e.
censorship)
\end{itemize}


\end{frame}

%%%%%%%%%%%%%%%%%%%%%%%%%%%%%%%%%%%%%%%%%%%%%%%%%%%%%%%%%%%%%%%%%%%%%%%

\begin{frame}
\frametitle{IP vs. physical property}

Some differences between Intellectual Property and ``Physical''
Property:
\begin{itemize}
\item Expiration date.
\item When you copy the IP object, you do not harm to the owner of
  copied object. However, a copy can harm the author.
\end{itemize}

Also, it is difficult to see the difference between ``copy'' and
``inspiration''. Sometimes you can compose some music piece inspired
by others, and some guys could say you that you are plagiarizing.


\end{frame}



%%%%%%%%%%%%%%%%%%%%%%%%%%%%%%%%%%%%%%%%%%%%%%%%%%%%%%%%%%%%%%%%%%%%%%%

\begin{frame}
\frametitle{IP Categories (Continental Law)}

IP is divided into two categories (Continental Law):  

\begin{itemize}
\item \alert{Industrial property}: inventions, patents, trademarks, industrial designs, and geographic indications of source. 
\item \alert{Copyright} (''Author's Rights''): literary and artistic works such as novels, poems and plays, films, musical works, artistic works such as drawings, paintings, photographs and sculptures, and architectural designs.  
\end{itemize}

Rights related to copyright include those of performing artists in their performances (``neighboring rights'').  

\end{frame}


%%%%%%%%%%%%%%%%%%%%%%%%%%%%%%%%%%%%%%%%%%%%%%%%%%%%%%%%%%%%%%%%%%%%%%%

\begin{frame}
\frametitle{IP Categories (Common Law)}

\alert{Common law} system:
\begin{itemize}
\item \alert{Copyrights}: Protect from unauthorized copy: artistic or literary
  works, computer programs, data collections, industrial designs, etc.
\item \alert{Trademarks}: Protect company symbols and names.
\item \alert{Trade secrets}: Protect access to some industrial secrets.
\item \alert{Patents}: Protect the rights of exploiting inventions as monopolies.
\end{itemize}

\end{frame}



%%%%%%%%%%%%%%%%%%%%%%%%%%%%%%%%%%%%%%%%%%%%%%%%%%%%%%%%%%%%%%%%%%%%%%%

\begin{frame}
\frametitle{Trade secrets}

\begin{itemize}
\item A trade secret is a way to protect investments in industrial area,
through Industrial Property laws.

\item Under trade secrets, there are several goods such as chemical or pharmaceutical
formulas, but also software.

\item Proprietary software enterprises hide the source code of their
software products as a way to protect their investment in creating
such software.

\item One of the objectives, for example, is to avoid the creation of
derivative works.

\item However, in some countries, reverse engineering is permitted in order to create
compatible software products.
\end{itemize}

\end{frame}

%%%%%%%%%%%%%%%%%%%%%%%%%%%%%%%%%%%%%%%%%%%%%%%%%%%%%%%%%%%%%%%%%%%%%%%

\begin{frame}
\frametitle{Trademarks}

\begin{itemize}
\item A trade name, is the name which a business trades under for commercial
purposes.

\item Trading names are sometimes registered as trademarks or are regarded
as brands.

\item Sometimes, the names are not registered in most countries and this
implied some problems. For example, in the US somebody registered the
trademark ``Linux'' and tried to obtain money for its use.

\item In Free Software, there are not very important, probably because
registering a trademark is not free and most developers do not pay
attention on them. However, there are some well known trademarks in
this world, such as GNOME, GNU, Debian.

\end{itemize}

\end{frame}



%%%%%%%%%%%%%%%%%%%%%%%%%%%%%%%%%%%%%%%%%%%%%%%%%%%%%%%%%%%%%%%%%%%%%%%

\begin{frame}
\frametitle{Patents}

\begin{itemize}
\item Patents?
\begin{itemize}
\item The invention is not protected by secret. On the contrary, the
  invention is publicly available.
\item However, for a certain period (20 years) for
  exploiting the invention the interested company must pay a license.
\end{itemize}

\item From other point of view, the patent is not a right to use the
invention, but provides the right to exclude others from making,
using, selling, offering for sale, or importing the patented; for the
signed period.

\item Patents, good or bad?
\end{itemize}

\end{frame}


%%%%%%%%%%%%%%%%%%%%%%%%%%%%%%%%%%%%%%%%%%%%%%%%%%%%%%%%%%%%%%%%%%%%%%%

\begin{frame}
\frametitle{IP: Concepts}

IP Laws are coordinated in nearly all the world, thanks to several
organizations and initiatives:
\begin{itemize}
\item WIPO: Promotes both property types.
\item TRIPS: Establishes minimal conditions to all countries of WTO.
\item International agreements: Bern and Geneva Convention. 
\end{itemize}

\begin{block}{Universal Declaration on Human Rights, art. 27.2}
Everyone has the right to the protection of the moral and material interests resulting from any scientific, literary or artistic production of which he is the author.
\end{block}

\small
But usually IP rights are transferred to enterprises where creators
work (for example, the companies where the programmers work).

\normalsize

\end{frame}

%%%%%%%%%%%%%%%%%%%%%%%%%%%%%%%%%%%%%%%%%%%%%%%%%%%%%%%%%%%%%%%%%%%%%%%

\begin{frame}
\frametitle{Spanish Legal framework}

\begin{itemize}
\item Ley de Propiedad Intelectual (LPI): Continental law. 
\item Derechos de autor vs. Derechos afines (o ` conexos'' , or ``vecinos'') 
\item Derechos morales vs. Derechos de explotación (``patrimoniales'')
\end{itemize}

\end{frame}


%%%%%%%%%%%%%%%%%%%%%%%%%%%%%%%%%%%%%%%%%%%%%%%%%%%%%%%%%%%%%%%%%%%%%%%

\begin{frame}
\frametitle{IP: Works protected}

The type of works considered include:

\begin{itemize}
\item Literature works (novels, poems, theater works, reference documents, newspapers and software)
\item Artistic works 
\item Scientific works 
\item Databases, movies 
\item Musical compositions and choreographies, architectonic works, publicity 
\item Maps and technical paintings
\end{itemize}

Like literature and music, \alert{software} is protected primarily by copyright law: It is a \alert{literary work}.
\end{frame}



%%%%%%%%%%%%%%%%%%%%%%%%%%%%%%%%%%%%%%%%%%%%%%%%%%%%%%%%%%%%%%%%%%%%%%%

\begin{frame}
\frametitle{Authors and copyright holders}

\begin{itemize}

\item Authors is a (physical or juridical) person that creates
a work. 

\item \alert{Collaborative work}: unitary result of the collaboration
of several authors where the input of each author may be
identified and exploited independently.

\item \alert{Collective work} (art. 8 LPI): under the initiative and coordination of
a physical or juridical person. It groups the input
of several authors that cannot be identified independently
and that compose a unique and autonomous creation. Examples: GNOME,
Mozilla, FSF, etc.

\item The work produced by an employee or by means of 
a contract is owned by the company.
But the moral rights are still retained by the programmer.

\end{itemize}

\end{frame}

%%%%%%%%%%%%%%%%%%%%%%%%%%%%%%%%%%%%%%%%%%%%%%%%%%%%%%%%%%%%%%%%%%%%%%%

%\begin{frame}
%\frametitle{The Copyright: Protection for Authors}
%
%Copyright, a set of exclusive rights:
%\begin{itemize}
%\item To regulate the use of a particular expression of an idea or
%  information.
%\item In most cases, with limited duration.
%\end{itemize}
%
%The copyright is granted to all intellectual publications without any other
%requirements:
%\begin{itemize}
%\item Automatic copyright when the work is published.
%\item (Almost) global scope.
%\end{itemize}
%
%\end{frame}

%%%%%%%%%%%%%%%%%%%%%%%%%%%%%%%%%%%%%%%%%%%%%%%%%%%%%%%%%%%%%%%%%%%%%%%

\begin{frame}
\frametitle{The Copyright: Protection for Authors}

The rights protected by copyright laws:
\begin{itemize}
\item \alert{Moral rights}. Guarantee work dissemination and author attribution.
Only in Continental law.
\item \alert{Economic rights} (\textbf{copyright} \textit{per se} in Common Law). Property rights, guarantee economic exploitation.
\end{itemize}

Economic rights have time expiration, depending on local laws. For
example, in EU they expire 70 years after the author's death.

\begin{center}
{\large How can we grant some rights to users of copyrighted works?}
\end{center}

\end{frame}

%%%%%%%%%%%%%%%%%%%%%%%%%%%%%%%%%%%%%%%%%%%%%%%%%%%%%%%%%%%%%%%%%%%%%%%

\begin{frame}
\frametitle{Moral rights}

\begin{itemize}
\item Disclosure of the work
\item Way of publication: with his name, a pseudonym or anonymously
\item The right of attribution
\item The right to the integrity of the work (distortion or mutilation)
\item The withdrawal of his work (addressing compensation if needed)
\end{itemize}

\end{frame}

%%%%%%%%%%%%%%%%%%%%%%%%%%%%%%%%%%%%%%%%%%%%%%%%%%%%%%%%%%%%%%%%%%%%%%%

\begin{frame}
\frametitle{Moral rights (2)}

\begin{itemize}
\item These rights cannot be withdrawn, cannot be transferred, are inalienable
and some even perpetual.
\item Included in the Bern Convention in 1928. 
\item The US do not completely recognize moral rights as part of copyright law, but rather as part of other bodies of law, such as defamation, academic fraud or unfair competition (plagiarism).
\end{itemize}


\end{frame}




%%%%%%%%%%%%%%%%%%%%%%%%%%%%%%%%%%%%%%%%%%%%%%%%%%%%%%%%%%%%%%%%%%%%%%%

\begin{frame}
\frametitle{What is Copyright}

\begin{itemize}
\item Limitation on the \alert{expression} of an idea (other expressions of the same idea are possible!).
\item Gives exclusive rights to the owner.
\item There are exceptions (\emph{fair use})
\item By default, all rights are reserved.
\item In software the expression is given by the code; algorithms are
not protected.
\item There are neighboring rights.
\end{itemize}

\end{frame}


%%%%%%%%%%%%%%%%%%%%%%%%%%%%%%%%%%%%%%%%%%%%%%%%%%%%%%%%%%%%%%%%%%%%%%%

\begin{frame}
\frametitle{What is Copyright}

Gives its owner an ``exclusive right'' to:

\begin{itemize}
\item To make and sell copies of the work (including,
typically, electronic copies).
\item To make derivative works
\item to publicly perform/display the work
\item To sell or assign these rights to others
\end{itemize}

\end{frame}

%%%%%%%%%%%%%%%%%%%%%%%%%%%%%%%%%%%%%%%%%%%%%%%%%%%%%%%%%%%%%%%%%%%%%%%

\begin{frame}
\frametitle{The Copyright Term}

Copyright expires after:

\begin{itemize}
\item Minimum: 50 years after death of person. 
\item In general (USA, Europe): 70 years \textit{post mortem}.
\item When its terms expire, a work goes into the Public Domain.

\end{itemize}

\end{frame}

%%%%%%%%%%%%%%%%%%%%%%%%%%%%%%%%%%%%%%%%%%%%%%%%%%%%%%%%%%%%%%%%%%%%%%%

\begin{frame}
\frametitle{Economic rights (copyright)}

\begin{itemize}
\item Reproduction (includes communication and copying): loading,
presentation on the screen, execution, transmission and storage.
\begin{itemize}
\item Even for using a program you require the author's approval!
\item The right to copy/reproduce is fundamental in licenses; else
the software cannot be run.
\item If somebody steals a book, he is attempting against the owner
of the book but not the owner of the IP. 
\end{itemize}
\item Distribution: Public disposal of physical copies (i.e. offering
the software over the Internet is not included). Software is not
sold as this could make re-selling possible. What is sold is the CD; the
software is licensed!
\item Public performance (there is no distribution of physical copies; What is public and private
on the Internet?)
\item Transformation (for instance, translation)
\end{itemize}


\end{frame}


%%---------------------------------------------------------------
%%%%%%%%%%%%%%%%%%%%%%%%%%%%%%%%%%%%%%%%%%%%%%%%%%%%%%%%%%%%%%%%%%%%%%%
% \section{References}
%%%%%%%%%%%%%%%%%%%%%%%%%%%%%%%%%%%%%%%%%%%%%%%%%%%%%%%%%%%%%%%%%%%%%%%

\begin{frame}
\frametitle{References}

\begin{itemize}
\item \textsc{Van Lindberg}, \textit{Intellectual Property and Open Source}, O'Reilly, July 2008.
\item \textsc{Malcolm Bain} et al. \textit{Aspectos legales y de explotación del software libre}, UOC, February 2007. \\
\url{http://ocw.uoc.edu/informatica-tecnologia-y-multimedia/aspectos-legales-y-de-explotacion-del-software-libre/materiales/}
\item \textsc{Lawrence Rose}, \textit{Open Source Licensing}, Prentice Hall, July 2004 
% \item Text for CC Attribution-ShareAlike 3.0 License \\ 
% \url{http://creativecommons.org/licenses/by-nc-sa/3.0/legalcode}

\end{itemize}

\end{frame}

%%%%%%%%%%%%%%%%%%%%%%%%%%%%%%%%%%%%%%%%%%%%%%%%%%%%%%%%%%%%%%%%%%%%%%%




