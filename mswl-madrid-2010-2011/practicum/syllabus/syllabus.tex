\documentclass[a4paper]{article}
%\usepackage[T1]{fontenc}
\usepackage[utf8]{inputenc}
\usepackage{url}
\usepackage[pdfborder=0 0 0]{hyperref}
%\usepackage{html}
%\usepackage{hthtml}
\usepackage{geometry}

% Comments (optional argument is author of comment)
\newcommand{\comments}[2][?]{
  \begin{quote}
    \textbf{Comment (#1):} {\em #2}
  \end{quote}
  }

% \name is for ``special names'', like procedure or variable names.
\newcommand{\name}[1]{\texttt{\hturl{#1}}}

% Just a shortcut for links where the url appears as footnote.
\newcommand{\htflink}[2]{\htmladdnormallinkfoot{#1}{#2}}

\title{MSWL Practicum\\
Master on libre software \\
URJC - GSyC/Libresoft \\
\url{http://master.libresoft.es}}

\author{Jesus M. Gonzalez-Barahona, Pedro Coca}
\date{December 2010}

\sloppy
\begin{document}
\maketitle

\begin{abstract}
Course syllabus and learning program for the course ``Practicum'', of the Master on libre software of the Universidad Rey Juan Carlos (Móstoles, Spain).

[This is an evolving document, until the course is finished and graded]
\end{abstract}

\tableofcontents

%%----------------------------------------------------------------
%%----------------------------------------------------------------
%%----------------------------------------------------------------
\section{Course topics and schedule}

%%----------------------------------------------------------------
%%----------------------------------------------------------------
\subsection{Presentation of the course}

Presentation of the main aspects of the course, and specially those related to administrative issues, evaluation, etc, will be carried out with each student case to case, as this course has no lectures assigned.


\section{Grading plan}

Each practicum has its own evaluation criteria, as described below, nevertheless, there are common deliverables. Each of these deliverables and activities has a minimum and maximum grading. If the minimum grading is 0, the activity is optional. Otherwise, the activity is mandatory, and has to be graded at least with the minimum to pass the course. Each activity has also a description, and when possible, some general grading criteria. In any case, the final grade for the course will also depend on the continuous observation of the instructors on the outcomes and progress of students.

Students should ask instructors about any detail which may not be clear to them, either about the general grading plan, or about specific aspects of the activities. As a general rule, evaluation will have into account how the activity and its results show that the student has come close to the competences, knowledge and skills expected for the course.

The student can consider that the next table will be used as a (minimum) guideline for assigning marks:

\begin{itemize}
\item Pass (``aprobado''): 150
\item Good (``notable''): 250
\item Excellent (``sobresaliente''): 350
\end{itemize}

\begin{itemize}
\item \textbf{Initial Report)}. \\
  Minimum: 20 points, maximum: 100 points.

  An initial report detailing the objectives, activities within the practicum, internal deadlines, deliverables and progress estimation. 

\item \textbf{Progress Report)}. \\
  Minimum: 20 points, maximum: 100 points.

  A progress report detailing the objectives (met/on track/overdue), on going activities, internal deadlines (met or overdue), deliverables status and progress done.

\item \textbf{Final Report)}. \\
  Minimum: 20 points, maximum: 100 points.

  A final report detailing the objectives (met/overdue), issues found, activity results, project plan, deliverables status and conclusions


\item \textbf{Blog entries}. \\
  Minimum: 0 points, maximum: 50 points

  Blog entries specifically related to the course, and marked as such. The tag used for that is mswl-prac.


As a result of this activity, the student should produce:

\begin{itemize}
\item Three written reports (initial, progress and final)
\end{itemize}

 It is important to detail all the references, and to heavily root the report on data and/or specific works publicly available.

\item \textbf{Other activities}. \\
  Minimum: 0 points, maximum: 100 points

  These activities have to be agreed in advance with the instructors and could include interviews with the university tutor or/and the company/project contact.
\end{itemize}

\subsection{Deliverables deadlines}

\begin{itemize}
  \item item The initial report must be handed in two weeks after the practicum start.
  \item The progress report must be handed in two weeks after the half of the time assigned to the practicum has been consumed.
  \item The final report must be handed in two weeks after the end of the practicum.
  \item Activities to be graded in May must be completed by May 17th 2011.
  \item All activities to be graded at the end of the masters must be completed by June 30th 2011.
\end{itemize}

%%----------------------------------------------------------------
%%----------------------------------------------------------------
%%----------------------------------------------------------------
\section{Assignments and activities}

 No Assignments or activities are required in addition to the written reports
%%----------------------------------------------------------------

\end{document}
