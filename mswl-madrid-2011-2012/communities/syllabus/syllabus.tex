\documentclass[a4paper]{article}
%\usepackage[T1]{fontenc}
\usepackage[utf8]{inputenc}
\usepackage{url}
\usepackage[pdfborder=0 0 0]{hyperref}
%\usepackage{html}
%\usepackage{hthtml}
\usepackage{geometry}

% Comments (optional argument is author of comment)
\newcommand{\comments}[2][?]{
  \begin{quote}
    \textbf{Comment (#1):} {\em #2}
  \end{quote}
  }

% \name is for ``special names'', like procedure or variable names.
\newcommand{\name}[1]{\texttt{\hturl{#1}}}

% Just a shortcut for links where the url appears as footnote.
\newcommand{\htflink}[2]{\htmladdnormallinkfoot{#1}{#2}}

\title{MSWL Libre Software Communities \\
Master on libre software \\
URJC - GSyC/Libresoft \\
\url{http://master.libresoft.es}}

\author{Daniel Izquierdo Cortázar}
\date{September 2011}

\sloppy
\begin{document}
\maketitle

\begin{abstract}
Course syllabus and learning program for the course ``Libre Software Communities'', 
of the Master on libre software of the Universidad Rey Juan Carlos (Móstoles, Spain).

[This is an evolving document, until the course is finished and graded]
\end{abstract}

\tableofcontents

%%----------------------------------------------------------------
%%----------------------------------------------------------------
%%----------------------------------------------------------------
\section{Course topics and schedule}

%%----------------------------------------------------------------
%%----------------------------------------------------------------
\subsection{00 - Presentation of the course}

Presentation of the main aspects of the course, and specially those related to administrative issues, evaluation, etc.

%%----------------------------------------------------------------
\subsubsection{Session 1 (0.5 hours)}

\begin{itemize}
\item \textbf{Lecturers:} Daniel Izquierdo and Felipe Ortega
\item \textbf{Presentation:} ``Presentation of the course''
  \begin{itemize}
  \item \textbf{Supporting material:} Slides ``Presentation of the course''
  \end{itemize}
\end{itemize}

%%----------------------------------------------------------------
%%----------------------------------------------------------------
\subsection{01 - Introduction and motivation}



%%----------------------------------------------------------------
\subsubsection{Session 1 (3.5 hours)}

\begin{itemize}
\item \textbf{Lecturers:} Daniel Izquierdo
\item \textbf{Presentation:} ``Introduction to libre software communities''
  \begin{itemize}
  \item \textbf{Discussion:} Brief introduction about data sources
  \item \textbf{Supporting material:} Slides ``Introduction to data sources''
  \end{itemize}
\item \textbf{Exercises} Questions will be addressed.

\end{itemize}



%%----------------------------------------------------------------
%%----------------------------------------------------------------
\subsection{02 - Data mining in libre software repositories}

How libre software can be used to support open innovation practices, and how open innovation practices are used in libre software projects and communities.

%%----------------------------------------------------------------
\subsubsection{Session 2 (4 hours)}

\begin{itemize}
\item \textbf{Lecturers}: Daniel Izquierdo Cortázar
  \begin{itemize}
  \item \textbf{Discussion:} Measuring activity in libre software projects
  \item \textbf{Supporting material:} Slides
  \end{itemize}
\item \textbf{Exercises} Questions will be addressed)

\end{itemize}

%%----------------------------------------------------------------
%%----------------------------------------------------------------
\subsection{03 - Effort distribution}

%%----------------------------------------------------------------
\subsubsection{Session 3 (4 hours)}


\begin{itemize}
\item \textbf{Lecturers}: Felipe Ortega
  \begin{itemize}
  \item \textbf{Discussion:} Case study: effort distribution in libre software communities
  \item \textbf{Supporting material:} Slides
  \end{itemize}
\item \textbf{Exercises} Questions will be addressed)

\end{itemize}



%%----------------------------------------------------------------
%%----------------------------------------------------------------
\subsection{04 - Developer turnover and knowledge management}

%%----------------------------------------------------------------
\subsubsection{Session 4 (4 hours)}

\begin{itemize}
\item \textbf{Lecturers}: Daniel Izquierdo
\item \textbf{Presentation:} Developer turnover and knowledge management
  \begin{itemize}
  \item \textbf{Supporting material:} Slides
  \end{itemize}
\end{itemize}
  
\end{itemize}



%%----------------------------------------------------------------
%%----------------------------------------------------------------
\subsection{05 - Territoriality and experience}


%%----------------------------------------------------------------
\subsubsection{December 17, 2010 (4 hours)}

\begin{itemize}
  \item \textbf{Lecturers}: Daniel Izquierdo
  
  \item \textbf{Presentation:} ``Territoriality and experience''
\end{itemize}
  


\section{Grading}

This section details the criteria for grading the course, the deadlines for the different activities, 
and the submission details for th activities that require them.

\subsection{Evaluation criteria}
\label{sub:evaluation-criteria}

Each activity contributing to the grading of the course has its own evaluation criteria, as described below. 
Each of these activities has a minimum and maximum grading. If the minimum grading is 0, the activity is 
optional. Otherwise, the activity is mandatory, and has to be graded at least with the minimum to pass 
the course. Each activity has also a description, and when possible, some general grading criteria. 
In any case, the final grade for the course will also depend on the continuous observation of the 
instructors on the outcomes and progress of students.

Students should ask instructors about any detail which may not be clear to them, either about the general 
grading plan, or about specific aspects of the activities. As a general rule, evaluation will have into 
account how the activity and its results show that the student has come close to the competences, knowledge 
and skills expected for the course.

The student can consider that the next table will be used as a (minimum) guideline for assigning marks:

\begin{itemize}
\item Pass (``aprobado''): 150
\item Good (``notable''): 250
\item Excellent (``sobresaliente''): 350
\end{itemize}

\begin{itemize}
\item \textbf{Exercises (answered in forum)}. \\
  Minimum: 20 points, maximum: 50 points.

  Exercises proposed and answered in the the forum of the course.

\item \textbf{Blog entries}. \\
  Minimum: 30 points, maximum: 80 points

  Blog entries specifically related to the course, and marked as such. The tag used for that is mswl-comm.

\item \textbf{General exercises}. \\
  Minimum: 150 points, maximum: 300 points

  These exercises will be requested in each session. There will be at least one exercise to be (mandatory) per day
  and each of them should be passed in order to pass the subject. Each session's exercise(s) will be marked with a maximum of
  60 points.

  It is important to detail all the references, and to heavily root the report on data and/or specific 
works publicly available. 

\item \textbf{Other activities}. \\
  Minimum: 0 points, maximum: 20 points

  These activities have to be agreed in advance with the instructors.
\end{itemize}

\subsection{Submission deadlines}

To be done...





\end{document}
