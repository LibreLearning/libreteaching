\documentclass[a4paper]{article}
%\usepackage[T1]{fontenc}
\usepackage[utf8]{inputenc}
\usepackage{url}
\usepackage[hypertex,colorlinks]{hyperref}
%\usepackage{html}
%\usepackage{hthtml}
\usepackage{geometry}

% Comments (optional argument is author of comment)
\newcommand{\comments}[2][?]{
  \begin{quote}
    \textbf{Comment (#1):} {\em #2}
  \end{quote}
  }

% \name is for ``special names'', like procedure or variable names.
\newcommand{\name}[1]{\texttt{\hturl{#1}}}

% Just a shortcut for links where the url appears as footnote.
\newcommand{\htflink}[2]{\htmladdnormallinkfoot{#1}{#2}}

\title{MSWL Legal Aspects \\
Master on libre software \\
URJC - GSyC/Libresoft \\
\url{http://master.libresoft.es}}

\author{Miguel Vidal \and Gregorio Robles}
\date{October-November 2011}

\sloppy
\begin{document}
\maketitle

\begin{abstract}
Course syllabus and learning program for the course ``Legal Aspects'', 
of the Master on libre software of the Rey Juan Carlos University (Madrid, Spain). 

Supporting materials and references can be found on the Moodle page for this course. \\

[This is an evolving document, until the course is finished and graded]
\end{abstract}

\tableofcontents

\section{Administrivia}

\begin{tabular}{ll}
\textbf{Credits:} 3 ECTS \\
\textbf{Schedule:} \\
& 13th October, Thu, 17:00-21:00 \\
& 20th October, Thu, 17:00-21:00 \\
& 27th October, Thu, 17:00-21:00 \\
& 3rd November, Thu, 17:00-21:00 \\
& 10th November, Thu, 17:00-21:00 \\
\textbf{Lecturers:} \\
& Miguel Vidal \\
& \hspace{1cm} email: \texttt{mvidal @ gsyc.urjc.es} \\
& \hspace{1cm} Twitter: \texttt{@mvidallopez}, \url{http://twitter.com/mvidallopez} \\
& \hspace{1cm} Room 102, Biblioteca (Library), campus at Fuenlabrada \\
& Gregorio Robles \\
& \hspace{1cm} email: \texttt{grex @ gsyc.urjc.es} \\
& \hspace{1cm} Room 110, Departamental II, campus at Fuenlabrada \\
\end{tabular}{l}

%%----------------------------------------------------------------
%%----------------------------------------------------------------
%%----------------------------------------------------------------


\section{Grading}

This section details the criteria for grading the course, the deadlines for the different activities, and the submission details for th activities that require them.

\subsection{Evaluation criteria}

Each activity contributing to the grading of the course has its own evaluation criteria, 
as described below. Each of these activities has a minimum and maximum grading. If the 
minimum grading is 0, the activity is \textit{optional}. Otherwise, the activity is \textit{mandatory}, and 
has to be graded al least with the minimum to pass the course. Each activity has also a 
description, and when possible, some general grading criteria. In any case, the final grade 
for the course will also depend on the continuous observation of the instructors on the 
outcomes and progress of students.

Students should ask instructors about any detail which may not be clear to them, 
either about the general grading plan, or about specific aspects of the activities. 
As a general rule, evaluation will have into account how the activity and its results 
show that the student has come close to the competences, knowledge and skills expected 
for the course.

The student can consider that the next table will be used as a guideline for 
assigning marks:

\begin{itemize}
\item \textbf{0-49: Fail} (``Suspenso'', SS)
\item \textbf{50-69: Pass} (``Aprobado'', AP)
\item \textbf{70-89: Good} (``Notable'', NT)
\item \textbf{90-100: Excellent} (``Sobresaliente'', SB)

\end{itemize}

\subsection{Graded activities}

\begin{itemize}
\item \textbf{Exercises} (mandatory, answered in Moodle forum) \\
  Minimum: 10 points, maximum: 35 points.

  Exercises proposed and answered in the the forum of the course.

\item \textbf{Blog entries} (optional) \\
  Minimum: 0 points, maximum: 20 points

  Blog entries specifically related to the course, and marked as such. The tag used for that is mswl-legal.

\item \textbf{Specific work about a Legal Aspect} (mandatory) \\
  Minimum: 10 points, maximum: 45 points

  Specific written (7-10 pages) report about a relevant legal aspect of free/open source software, related to the topics dealt with in this course. It can also be accompanied by a presentation (recorded in video, in this case), a video, a podcast, etc. It is important to detail all the references, and to heavily root the report on data and/or specific works publicly available.

\medskip

Some ideas:

\begin{itemize}
\item IP (copyrights, patents) lawsuits against FLOSS
\item Infringement and violations of FLOSS licenses.
\item How to affect the patent system to libre software.
\item How to affect SaaS (software on demand, cloud computing, etc.) to FLOSS licensing.
\item Analysis of a specific FLOSS license or comparison between FLOSS licenses.
\item Choosing the right license: considering business models, product architecture,  IP ownership, license compatibily issues, relicensing, etc.
\item Thinking about derivate works: linking and licensing, dailylife with FLOSS licenses, etc.
\item Intellectual Property debate: Defense and opposition to pat­ents and copyrights. 
\item Etc.

\end{itemize}



\item \textbf{Other activities} (optional) \\
  Minimum: 0 points, maximum: 15 points

  These activities have to be agreed with the instructors.
\end{itemize}

\subsection{Submission deadlines}

All activities to be graded in January must be completed and submitted by \textbf{December 23th 2011}.

\subsection{Submission details}

Please, consider the details below for submitting the different activities for evaluation (for those not specified in this list, nothing special is needed for submission).

[These details could change before the actual submission dates, please check for updates before submitting.]

\begin{itemize}
\item As a summary of all the activities, a ``Summary of activities for evaluation'' should be sent. This summary should be uploaded to the corresponding resource in the Moodle site for this course, and should include the following data:
  \begin{itemize}
  \item \textbf{Name:} Full name of the student (as ``family name'', ``given name'')
  \item \textbf{Blog entries:} Url of the blog entries for this course (HTML, not RSS version).
  \item \textbf{List of other activities:} If any (for example contributions to the collaborative notebook), list of other activities submitted for evaluation (those that would fit in the ``other activities'' item in the ``Evaluation criteria''. The results of those activities should be uploaded to the ``other activities'' resource in the Moodle site for this course, when appropriate. In some specific cases (such as streaming videos) it will be enough to include in this list the url to the external site where the result is hosted.
  \end{itemize}
\end{itemize}


\end{document}
