%% copyrightONsoftware.tex
%%
%% Presentation of the course ``Legal Issues'' of the Official Master on Libre Software (URJC)
%% http://master.libresoft.es
%%


%%%%%%%%%%%%%%%%%%%%%%%%%%%%%%%%%%%%%%%%%%%%%%%%%%%%%%%%%%%%%%%%%%%%%%%
\section{Lesson III: Free/Open Source Software Licenses}
%%%%%%%%%%%%%%%%%%%%%%%%%%%%%%%%%%%%%%%%%%%%%%%%%%%%%%%%%%%%%%%%%%%%%%%

%%%%%%%%%%%%%%%%%%%%%%%%%%%%%%%%%%%%%%%%%%%%%%%%%%%%%%%%%%%%%%%%%%%%%%

\begin{frame}
\frametitle{Why Do I Need a License?}

\pause

\begin{center}
\Large{If you don't license your code,\\ it can't be used (legally) by other people.}
\end{center}

\end{frame}



%%%%%%%%%%%%%%%%%%%%%%%%%%%%%%%%%%%%%%%%%%%%%%%%%%%%%%%%%%%%%%%%%%%%%%%

\begin{frame}
\frametitle{FLOSS License Example}

Implement a basic free license is very easy: 

\begin{block}{Free License Example}
Copyright (c) 2010 Foobar Developers. All rights reserved. 

\medskip
Redistribution and use in source and binary forms, with or without modification, are permitted provided that the redistributions of source code must retain the above copyright notice.
\end{block}

\textit{That's all!!}

\end{frame}


%%%%%%%%%%%%%%%%%%%%%%%%%%%%%%%%%%%%%%%%%%%%%%%%%%%%%%%%%%%%%%%%%%%%%%%

\begin{frame}
\frametitle{FLOSS Licensing}

From least to greatest complexity (and strict):
\begin{itemize}
\item \alert{Academic Licenses}
\item \alert{Permissive Licenses}
\item \alert{Partially Closable Licenses} (weak copyleft)
\item \alert{Reciprocal Licenses} (strong copyleft)
\end{itemize}

\end{frame}

%%%%%%%%%%%%%%%%%%%%%%%%%%%%%%%%%%%%%%%%%%%%%%%%%%%%%%%%%%%%%%%%%%%%%%%

\begin{frame}
\frametitle{Recommended licenses}

\begin{itemize}
\item	Academic/permissive
	\begin{itemize}
	\item The 2-clause BSD License 
	\item The Apache License 2.0 
	\end{itemize}
\item Weak copyleft
	\begin{itemize}
	\item The Mozilla Public License (MPL) // CDDL
	\item The Lesser GPL (LGPL), version 2 or 3
	\end{itemize}
\item Strong copyleft
	\begin{itemize}
	\item The GNU GPL, version 2 or 3 
	\item The Affero GPL, version 3
	\end{itemize}
\end{itemize}

\end{frame}


\subsection{Academic and permissive licenses}
%%%%%%%%%%%%%%%%%%%%%%%%%%%%%%%%%%%%%%%%%%%%%%%%%%%%%%%%%%%%%%%%%%%%%%%

\begin{frame}
\frametitle{BSD License}

Origins:
\begin{itemize}
\item BSD (Berkeley Software Distribution) is a Unix flavor developed
  by University of Berkeley (CA).
\item BSD Unix was licensed under a ``minimalistic'' license which
  permits both source or binary redistribution; also modifications,
  but without any other restriction.
\item Several revisions: it's a \textit{template}
\end{itemize}

\end{frame}


%%%%%%%%%%%%%%%%%%%%%%%%%%%%%%%%%%%%%%%%%%%%%%%%%%%%%%%%%%%%%%%%%%%%%%%

\begin{frame}
\frametitle{BSD License}

\begin{itemize}
\item Based in original BSD license.
\item Very popular (BSD userland, PF, TCP/IP, OpenSSH, TCL/Tk...).
\item You may redistribute the work, in any form (source or binary)
  but with all remaining copyright notes (authorship attribution).
\item There is a ``no warranty'' clause. 
BSD: No restrictions on future behavior. 
\end{itemize}

\end{frame}

%%%%%%%%%%%%%%%%%%%%%%%%%%%%%%%%%%%%%%%%%%%%%%%%%%%%%%%%%%%%%%%%%%%%%%%

\begin{frame}
\frametitle{BSD License}

\begin{itemize}
\item Based in original BSD license.
\item Very popular (BSD userland, PF, TCP/IP, OpenSSH, TCL/Tk...).
\item You may redistribute the work, in any form (source or binary)
  but with all remaining copyright notes (authorship attribution).
\item There is a ``no warranty'' clause. 
\end{itemize}

\end{frame}

%%%%%%%%%%%%%%%%%%%%%%%%%%%%%%%%%%%%%%%%%%%%%%%%%%%%%%%%%%%%%%%%%%%%%%%

\begin{frame}
\frametitle{BSD License. Advantages}

\begin{itemize}
\item BSD license places minimal restrictions on future behavior.
\item This allows BSD code to remain Open Source or become integrated into commercial solutions.
\item No legal complexity (unlike GPL or LGPL licenses). 
\item It allows developers and companies to spend their time creating and promoting good code rather than worrying if that code violates licensing.
\end{itemize}

\end{frame}


%%%%%%%%%%%%%%%%%%%%%%%%%%%%%%%%%%%%%%%%%%%%%%%%%%%%%%%%%%%%%%%%%%%%%%%

\begin{frame}
\frametitle{The 2-clause BSD License}

Copyright (C) (owner). All rights reserved.

Redistribution and use in source and binary forms, with or without
modification, are permitted provided that the following conditions
are met:

\begin{enumerate}
\item Redistributions of source code must retain the above copyright notice, this list of  conditions and the following disclaimer.
\item Redistributions in binary form must reproduce the above copyright notice, this list of conditions and the following disclaimer in the documentation and/or other materials provided with the distribution.
\end{enumerate}
 
\end{frame}

%%%%%%%%%%%%%%%%%%%%%%%%%%%%%%%%%%%%%%%%%%%%%%%%%%%%%%%%%%%%%%%%%%%%%%%

\begin{frame}
\frametitle{The BSD License: No Warranty}

{\small
THIS SOFTWARE IS PROVIDED BY THE AUTHOR AND CONTRIBUTORS ``AS IS'' 
AND ANY EXPRESS OR IMPLIED WARRANTIES, INCLUDING, BUT NOT LIMITED TO, 
THE IMPLIED WARRANTIES OF MERCHANTABILITY AND FITNESS FOR A PARTICULAR 
PURPOSE ARE DISCLAIMED. IN NO EVENT SHALL THE AUTHOR OR CONTRIBUTORS 
BE LIABLE FOR ANY DIRECT, INDIRECT, INCIDENTAL, SPECIAL, EXEMPLARY, 
OR CONSEQUENTIAL DAMAGES (INCLUDING, BUT NOT LIMITED TO, PROCUREMENT 
OF SUBSTITUTE GOODS OR SERVICES; LOSS OF USE, DATA, OR PROFITS; 
OR BUSINESS INTERRUPTION) HOWEVER CAUSED AND ON ANY THEORY OF LIABILITY, 
WHETHER IN CONTRACT, STRICT LIABILITY, OR TORT (INCLUDING NEGLIGENCE 
OR OTHERWISE) ARISING IN ANY WAY OUT OF THE USE OF THIS SOFTWARE, 
EVEN IF ADVISED OF THE POSSIBILITY OF SUCH DAMAGE.
}

\end{frame}

%%%%%%%%%%%%%%%%%%%%%%%%%%%%%%%%%%%%%%%%%%%%%%%%%%%%%%%%%%%%%%%%%%%%%%%

\begin{frame}
\frametitle{BSD-like licenses}

\begin{itemize}

\item {\bf Internet Systems Consortium (ISC)}. 
	\begin{itemize}
	\item Equivalent to the 2-clause BSD license.
	\item Language ``made unnecessary by the Berne convention'' removed.
	\item BIND, DHCP and preferred license of OpenBSD project.
	\end{itemize}
\item {\bf X-Window R11 License (X11)}. 
	\begin{itemize}
	\item Used graphical subsystem in Unix systems. 
	\item Also known as ``MIT license''. 
	\end{itemize}
\end{itemize}

\end{frame}

%%%%%%%%%%%%%%%%%%%%%%%%%%%%%%%%%%%%%%%%%%%%%%%%%%%%%%%%%%%%%%%%%%%%%%%

\begin{frame}
\frametitle{ISC: the shortest license}

{\small Copyright (c) Year(s), Company or Person's Name $<$E-mail address$>$} \\

\medskip

\alert{Permission} to \alert{use}, \alert{copy}, \alert{modify}, and/or \alert{distribute} this software for any
purpose with or without fee is hereby granted, provided that the above
copyright notice and this permission notice appear in all copies.

\end{frame}


%%%%%%%%%%%%%%%%%%%%%%%%%%%%%%%%%%%%%%%%%%%%%%%%%%%%%%%%%%%%%%%%%%%%%%%

\begin{frame}
\frametitle{Other BSD-like licenses}

{\small
\begin{itemize}

\item {\bf Zope Public License 2.0}. This license is used by the Zope
  distribution (an application server) and some related products. Near
  BSD license, also prohibits the use of Zope Corporation trademarks.

\item {\bf Public domain}. An intellectual work in the public domain
  is neither under any IP law nor a license. Most public domain works
  retains, however, the authorship. For this reason, it is very
  similar to have the program under PD or under a BSD license.

\end{itemize}
}

\end{frame}


%%%%%%%%%%%%%%%%%%%%%%%%%%%%%%%%%%%%%%%%%%%%%%%%%%%%%%%%%%%%%%%%%%%%%%%

\begin{frame}
\frametitle{The Apache License v2}

\begin{itemize}
\item Old versions: 1.0 (original) and 1.1 (ASF, 2000).
\item Apache License 2.0 (January 2004): permissive license.
\item Make the license easier for non-ASF projects to use
\item Allow the license to be included by reference instead of listed in every file
\item Require a patent license from a contributor's own patents.
\item Over 5000 non-ASF projects located at SourceForge are available under Apache License (2009). 
\item 25\% from Google Code (including Android)
\item Compatibility with GPLv3 (only one-way). (But incompatible with GPLv2)

\end{itemize}

\end{frame}

\subsection{Weak copyleft licenses}
%%%%%%%%%%%%%%%%%%%%%%%%%%%%%%%%%%%%%%%%%%%%%%%%%%%%%%%%%%%%%%%%%%%%%%%

\begin{frame}
\frametitle{The Mozilla Public License}

\begin{itemize}
\item Weak copyleft: the code under the MPL may be combined with proprietary files in one program (``Larger Work'')
\item Source code copied or changed under the MPL must stay under the MPL.
\item Some complex restrictions make it incompatible with the GPL.
\item A module covered by the GPL and a module covered by the MPL cannot legally be linked together.
\item For this reason, Firefox have been relicensed under multiple licenses (MPL, GPL, LGPL).
\end{itemize}

\end{frame}

%%%%%%%%%%%%%%%%%%%%%%%%%%%%%%%%%%%%%%%%%%%%%%%%%%%%%%%%%%%%%%%%%%%%%%%

\begin{frame}
\frametitle{The GNU LGPL}

\begin{center}
LGPL = Lesser GPL
\end{center}

\begin{itemize}
\item LGPL started as ``Library GPL''. Later renamed to ``Lesser
  GPL''.
\item LGPL maintain all freedoms and restrictions to the licensed
  software, but with one exception:
\begin{itemize}
\item LGPL Software can be integrated/linked with any other software, without
  limitations (including proprietary software).
\end{itemize}
\end{itemize}

\end{frame}

%%%%%%%%%%%%%%%%%%%%%%%%%%%%%%%%%%%%%%%%%%%%%%%%%%%%%%%%%%%%%%%%%%%%%%%

\begin{frame}
\frametitle{Why LGPL?}


\begin{itemize}
\item Created for promoting the use of GPL libraries in any other
  software (ex: GNU libc).
\item Later, FSF checked that LGPL did not helped to the creation of
  new Free software, so they decided to rename it to ``lesser'' and
  discourage its use.
\end{itemize}

\end{frame}


\subsection{Strong copyleft licenses}
%%%%%%%%%%%%%%%%%%%%%%%%%%%%%%%%%%%%%%%%%%%%%%%%%%%%%%%%%%%%%%%%%%%%%%%

\begin{frame}
\frametitle{GNU GPL License}

\begin{center}
\item GPL = GNU General Public License.
\end{center}

GPL Concepts:
\begin{itemize}
\item Created by FSF for the GNU Project.
\item Very often used in non-GNU free software.
\item Probably, the most popular Free Software license: around 70\%
  Freshmeat projects licensed under GPL.
\item Some popular software licensed under GPL: Linux, GNOME, Emacs,
 GCC\ldots
\end{itemize}

\end{frame}


%%%%%%%%%%%%%%%%%%%%%%%%%%%%%%%%%%%%%%%%%%%%%%%%%%%%%%%%%%%%%%%%%%%%%%%
\begin{frame}
\frametitle{The GNU GPL}
What makes the GPL so special?
\pause
\begin{itemize}
\item It was the first license to outline the copyleft principle.
\item All copyleft licenses have been based on the GPL, including the Wikipedia license.
\item Without the GPL, copyleft would be just an idea. 
\pause
\item Designed to prevent the proprietary commercialization of Open Source code.
\end{itemize}

\end{frame}

%%%%%%%%%%%%%%%%%%%%%%%%%%%%%%%%%%%%%%%%%%%%%%%%%%%%%%%%%%%%%%%%%%%%%%%

\begin{frame}
\frametitle{GNU GPL License (II)}

GPL Characteristics:
\begin{itemize}
\item This license guarantees the four FLOSS freedoms.
\item It is though to always guarantee code freedom. This is the
  meaning of the ``copyleft'' clause: all derivative works should be
  licensed also under the same license.
\item Since in USA software patents are admissible, GPL includes a
  clause for avoiding GPL licensing of patented software or
  algorithms.
\item GPL has three versions. The most known is GPL Version 2 because
  has been on the market for more than 10 years. 
\item GPL code can not be mixed with other code under
  ``GPL-incompatible'' license.
\end{itemize}

\end{frame}


%%%%%%%%%%%%%%%%%%%%%%%%%%%%%%%%%%%%%%%%%%%%%%%%%%%%%%%%%%%%%%%%%%%%%%%
\begin{frame}
\frametitle{GPL versions}

\begin{itemize}
\item Based in Emacs license, first copyleft license (1986).
\item GPL version 1 (1989). Generics, program-independent, ``version 1 or later''.
\item GPL version 2 (1991). ``Liberty or dead'' clause (it prevents from patents threats).
\item GPL version 3 (2007). Tivoization, patents and DRMs. Community discuss.
\end{itemize}


\end{frame}


%%%%%%%%%%%%%%%%%%%%%%%%%%%%%%%%%%%%%%%%%%%%%%%%%%%%%%%%%%%%%%%%%%%%%%%

\begin{frame}
\frametitle{GPL, version 2 (GPLv2)}

\begin{itemize}
    \item Written by Richard Stallman and the FSF. It was published in 1991.
    \item The most popular free software license: It covers 50-70 \% of all free software.
    \item It's more than a software license: it is a social contract.


\end{itemize}

\pause

{\centerline{\large \alert{Why does it update it?}}}

\pause

After 15 years, needed updating in order to remain effective against the technological challenges.
\end{frame}


%%%%%%%%%%%%%%%%%%%%%%%%%%%%%%%%%%%%%%%%%%%%%%%%%%%%%%%%%%%%%%%%%%%%%%%
\begin{frame}
\frametitle{GPLv3 elaboration process}

Public consultation process:
\begin{itemize}
\item It lasted eighteen months: from January 16, 2006 (first draft) to June 29, 2007 (final version).
\item Four drafts. 
\item Five International Conferences (Boston, Porto Alegre, Barcelona, Tokyo and Brussels)
\end{itemize}

\pause

The most important change, compared to previous versions, was the re-elaboration of the license, since it was discussed and agreed by the community.
\end{frame}

%%%%%%%%%%%%%%%%%%%%%%%%%%%%%%%%%%%%%%%%%%%%%%%%%%%%%%%%%%%%%%%%%%%%%%%
\begin{frame}
\frametitle{Changes in GPLv3}
The newest GPL version does not invalidate previous versions or requires software to be licensed under the new version.
\begin{itemize}
\item Major changes
    \begin{itemize}
        \item It DOES NOT prevent DRM implementations with GPL software, but it DOES allow interoperable software to be written with it.
        \item {More protection related to software patents}
        \item It neutralizes WIPO (\textit{anti-circumvention}) laws which ban libre software (DMCA and EUCD).
        \item It clarifies license compatibility (additional permissions)
    \end{itemize}
\item Minor changes
    \begin{itemize}
        \item Adaptation to technological innovations.
        \item Clarifications to make it easier to use and understand.
        \item Better internationalization (\textit{convey/distribution})
    \end{itemize}
\end{itemize}

\pause

Many changes, but fundamental principles remain.

\end{frame}



%%%%%%%%%%%%%%%%%%%%%%%%%%%%%%%%%%%%%%%%%%%%%%%%%%%%%%%%%%%%%%%%%%%%%%%

\begin{frame}
\frametitle{GPLv3: Digital Rights Management (DRM)}

Also known as Digital Restrictions Mismanagement. 

How GPLv3 works:

\begin{itemize}
\item Neutralize laws that prohibit (write or share) libre software (such as DMCA, EUCD)
\item But not forbidding DRM with GPLed software.
\item It's always possible to use GPLed code to write software that implements DRM 
\item But it's possible write interoperable software and bypass restrictions. 
\item Neutralize tivoization: require to provide with information or necessary data to install modified software on the embedded device.
\end{itemize}

\textbf{GNU GPL does not restrict what people do in software; it just stops them from restricting others.}

\end{frame}

%%%%%%%%%%%%%%%%%%%%%%%%%%%%%%%%%%%%%%%%%%%%%%%%%%%%%%%%%%%%%%%%%%%%%%%

\begin{frame}
\frametitle{GPLv3: Software Patents}
Protection against patent threats is implemented in GPLv2 through clause ``Liberty or Death'' (sec. 7).

\begin{itemize}
\item If GPLed code includes patents with incompatible restrictions, can't be distributed.
\item Avoid ``zombie'' libre software (software would be free if patents won't exist anymore).
\end{itemize}

{\centerline{This clause remains in GPLv3.}}


\end{frame}

%%%%%%%%%%%%%%%%%%%%%%%%%%%%%%%%%%%%%%%%%%%%%%%%%%%%%%%%%%%%%%%%%%%%%%%

\begin{frame}
\frametitle{GPLv3: Protecting From Anti-Circumvention Law}
Protecting Users' Legal Rights From Anti-Circumvention Law (sec. 3):

\begin{block}{GPLv3, Section 3. Protecting From Anti-Circumvention}
No covered work shall be deemed part of an effective technological measure. [...]      When you convey a covered work, you waive any legal power to forbid circumvention of
technological measures to the extent such circumvention is effected by exercising rights
under this License 
 \end{block}


\end{frame}


%%%%%%%%%%%%%%%%%%%%%%%%%%%%%%%%%%%%%%%%%%%%%%%%%%%%%%%%%%%%%%%%%%%%%%%

\begin{frame}
\frametitle{Software Patents}
GPLv3 adds stronger protection against patent threats through legal-engineering:

\begin{itemize}
\item Who distribute GPLed software must provide any patent rights to exercise the freedoms that the GPL grants him.
\item If anyone intends to exercise a patent, your license is finished.
\item Users and developers can work with GPLv3 software without worrying about anybody can sue for patent infringement.
\end{itemize}


\end{frame}

%%%%%%%%%%%%%%%%%%%%%%%%%%%%%%%%%%%%%%%%%%%%%%%%%%%%%%%%%%%%%%%%%%%%%%%

\begin{frame}
\frametitle{Compatibility}

Compatibility == merge source code from different libre software licenses.

\begin{itemize}
\item GPLv3 increases compatibility with several free licenses (Apache, Affero).
\item Allow additional requirements:
\begin{itemize}
\item Responsibility: Allows add disclaimers or warranty notes.
\item Allows add restrictions about trademarks.
\end{itemize}
\end{itemize}

GPLv3 is more modular, more compatible, and will be compatible with different copyleft licenses.

\end{frame}

%%%%%%%%%%%%%%%%%%%%%%%%%%%%%%%%%%%%%%%%%%%%%%%%%%%%%%%%%%%%%%%%%%%%%%%

\begin{frame}
\frametitle{GPL Caveats}

\begin{itemize}
\item If GPL source is required for a program to compile, the program must be under the GPL. 
\item Linking statically to a GPL library requires a program to be under the GPL.
\item GPL requires that any patents associated with GPLed software must be licensed for everyone's free use.
\item Simply aggregating software together (i.e. distros) does not count as including GPLed programs in non-GPLed programs.
\item Output of a program does not count as a derivative work. This enables the gcc compiler to be used in commercial environments without legal problems.
\item Since the Linux kernel is under the GPL, any code statically linked with the Linux kernel must be GPLed. 
\item This requirement can be circumvented by dynamically linking loadable kernel modules. Disadvantage: they will only work for particular versions of the Linux kernel.
\end{itemize}

\end{frame}

%%%%%%%%%%%%%%%%%%%%%%%%%%%%%%%%%%%%%%%%%%%%%%%%%%%%%%%%%%%%%%%%%%%%%%%

\begin{frame}
\frametitle{What is a derivative work}

\begin{itemize}
\item A work based upon a preexisting work.
\item The preexisting work is modified, translated, recasted, 
transformed, adapted so to create an improved (or different) work.
\item \alert{Substantial similarity}: it's not enough to identify a derivated software work. 
\item Complex problem... only related to copyleft?
\end{itemize}
\end{frame}

%%%%%%%%%%%%%%%%%%%%%%%%%%%%%%%%%%%%%%%%%%%%%%%%%%%%%%%%%%%%%%%%%%%%%%%

\begin{frame}
\frametitle{What is a derivative work}

\begin{itemize}
\item  Is Linux a derivative work of Unix?
\item Is implementation of a industry standard a derivative work of that specification?
\item How much copying of source code is required to create a derivative work?
\item Does linking create a derivative work?
\end{itemize}
\end{frame}


%%%%%%%%%%%%%%%%%%%%%%%%%%%%%%%%%%%%%%%%%%%%%%%%%%%%%%%%%%%%%%%%%%%%%%%

\begin{frame}
\frametitle{What is a derivative work: linking}

\begin{itemize}
\item FSF \textit{doctrine}
\item Copyright law and linking (What about web pages?)
\item Translate, modify, revisions... are derivative works. But it's not linking that made the difference!
\item Output data (compiling)
\item Pipes vs. code
\item Copyright holder (not FSF!) and judge: valid meaning. 
\end{itemize}
\end{frame}



%%%%%%%%%%%%%%%%%%%%%%%%%%%%%%%%%%%%%%%%%%%%%%%%%%%%%%%%%%%%%%%%%%%%%%%

\begin{frame}
\frametitle{Other reciprocal licenses: Affero GPL (AGPL)}

\begin{itemize}
\item It is a \alert{derived} license from GPL. 
\item Published by the Free Software Foundation (version 3: 2007).
\item It contains a clause requiring distribution of any modified source code of applications \alert{running in a computer network} (SaaS).
\item It aims to cover the case of modified GPL software which is not distributed because the GPL license does not require to do so (web services or online applications). 
\end{itemize}
\end{frame}



%%%%%%%%%%%%%%%%%%%%%%%%%%%%%%%%%%%%%%%%%%%%%%%%%%%%%%%%%%%%%%%%%%%%%%%

\begin{frame}
\frametitle{Discussion: GPL vs BSD}


\begin{itemize}
\item ``BSD code is free, but GPL code stays free.'' 

\pause

\item GPL Advantages and Disadvantages

\pause

\item BSD Advantages and Disadvantages

\end{itemize}
\end{frame}


%%%%%%%%%%%%%%%%%%%%%%%%%%%%%%%%%%%%%%%%%%%%%%%%%%%%%%%%%%%%%%%%%%%%%%%



