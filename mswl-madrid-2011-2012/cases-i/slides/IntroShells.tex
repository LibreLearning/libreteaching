\documentclass{beamer}

\mode<presentation> {
  \usetheme{Warsaw}
  \setbeamercovered{transparent}
}

\usepackage[utf8]{inputenc}
\usepackage{graphics}

%% Metadatos del PDF.
\hypersetup{  
  pdftitle={Introducción a las shells},
  pdfauthor={José Castro},
  pdfcreator={GSyC/Libresoft},
  pdfproducer=PDFLaTeX,
  pdfsubject={Máster Software Libre 2011-2012},
}

\begin{document}

\title{Introducción a las shells}
\subtitle{Máster Software Libre 2011-2012}
\institute{GSyC/Libresoft -- URJC} 
\author{José Castro \\ --- \\ \tiny{jfcastro@libresoft.es \\ @jfcastroluis}}
\date{October 14th, 2011}

\frame{
\maketitle
}

\frame{
~
\vspace{4cm}

\begin{flushright}
{\footnotesize
\copyright{2011} José Castro. \\
  Algunos derechos reservados. Este trabajo está licenciado bajo Creative
  Commons Reconocimiento 3.0 España, disponible en
  \url{http://creativecommons.org/licenses/by/3.0/es/}
 

\bigskip

}
\end{flushright}
}


%%%%%%%%%%%%%%%%%%%%%%%%%%%%%%%%%%%%%%%%%%%%%%%%%%%%%%%%%%%%%%%%%%%%%%%
\section{Introducción}
%%%%%%%%%%%%%%%%%%%%%%%%%%%%%%%%%%%%%%%%%%%%%%%%%%%%%%%%%%%%%%%%%%%%%%%
\begin{frame}
  \begin{center}
    \huge{Introducción}
  \end{center}
\end{frame}

\subsection{Inicio del sistema}
%%%%%%%%%%%%%%%%%%%%%%%%%%%%%%%%%%%%%%%%%%%%%%%%%%%%%%%%%%%%%%%%%%%%%%%

\begin{frame}
  \frametitle{Inicio del sistema}
  \begin{enumerate}
    \item Arrancamos el PC
    \item La BIOS toma el control: comprueba dispositivos y busca instrucciones en el MBR
    \item El MBR apunta al cargador de arranque (GRUB o LILO)
    \item El cargador de arranque inicia el kernel
    \item Lo primero que hace el kernel es ejecutar el proceso \texttt{init}
    \item En función del nivel de ejecución, se van lanzando más scripts y servicios
    \item Por último, aparece la pantalla de \texttt{login}
  \end{enumerate}
\end{frame}

\subsection{Fichero /etc/passwd}
%%%%%%%%%%%%%%%%%%%%%%%%%%%%%%%%%%%%%%%%%%%%%%%%%%%%%%%%%%%%%%%%%%%%%%%

\begin{frame}
  \frametitle{Fichero /etc/passwd}
  \begin{block}{/etc/passwd}
    \texttt{\tiny{jfcastro:x:1000:1000:Jose Castro,,,:/home/jfcastro:/bin/bash}}
  \end{block}
  \medskip
  \begin{itemize}
    \item El \texttt{login} lee este fichero para proporcionar (o no) acceso
    \medskip
    \item Comprueba el usuario y la pass en \texttt{/etc/shadow}
    \item Si son correctos, ejecuta la aplicación de shell en \texttt{\$HOME}
  \end{itemize}
\end{frame}

\subsection{Definición de shell}
%%%%%%%%%%%%%%%%%%%%%%%%%%%%%%%%%%%%%%%%%%%%%%%%%%%%%%%%%%%%%%%%%%%%%%%

\begin{frame}
  \frametitle{Definición de shell}
  \begin{itemize}
    \item Simplemente una aplicación más
    \item Primera aplicación que se ejecuta cuando se inicia sesión
    \medskip
    \item Permite la interacción entre el usuario y el sistema operativo
    \item Ejecuta otras aplicaciones en primer o segundo plano
    \medskip
    \item Las shells modernas permiten sentencias elaboradas (pipes, redirecciones, variables...) y programación para ellas mismas (scripts de shells)
    \medskip
    \item También se conoce como intérprete de comandos o intérprete de órdenes
  \end{itemize}
\end{frame}


\subsection{Código shell}
%%%%%%%%%%%%%%%%%%%%%%%%%%%%%%%%%%%%%%%%%%%%%%%%%%%%%%%%%%%%%%%%%%%%%%%

\begin{frame}
  \frametitle{Código shell}

  \begin{block}{pyshell.py}
    \small{\texttt{python pyshell.py}}\\
    \medskip
    \small{\texttt{\#!/usr/bin/env python\\
    \medskip
    import os\\
    \medskip
    prompt = "\$ "\\
    \medskip
    while True:\\
    \ \ line = raw\_input(prompt)\\
    \ \ os.system(line)}}
  \end{block}
\end{frame}

\subsection{Shell vs GUI}
%%%%%%%%%%%%%%%%%%%%%%%%%%%%%%%%%%%%%%%%%%%%%%%%%%%%%%%%%%%%%%%%%%%%%%%

\begin{frame}
  \frametitle{Shell vs GUI}
    \includegraphics[width=5cm]{camara-reflex.jpg}
    \includegraphics[width=5cm]{camara.png}
  \begin{center}
    \huge{¿Están las shells obsoletas?}
  \end{center}
\end{frame}


%%%%%%%%%%%%%%%%%%%%%%%%%%%%%%%%%%%%%%%%%%%%%%%%%%%%%%%%%%%%%%%%%%%%%%%
\section{Tipos de shells}
%%%%%%%%%%%%%%%%%%%%%%%%%%%%%%%%%%%%%%%%%%%%%%%%%%%%%%%%%%%%%%%%%%%%%%%
\begin{frame}
  \begin{center}
    \huge{Tipos de shells}
  \end{center}
\end{frame}

\subsection{Thompson shell}
%%%%%%%%%%%%%%%%%%%%%%%%%%%%%%%%%%%%%%%%%%%%%%%%%%%%%%%%%%%%%%%%%%%%%%%

\begin{frame}
  \frametitle{Thompson shell}
  \begin{itemize}
    \item Desarrollada por Ken Thompson en 1971
    \item Fue la primera shell de Unix
    \medskip
    \item Era un simple intérprete de comandos
    \item No permitía variables ni scripting
    \medskip
    \item Era posible hacer redirecciones y \texttt{pipes}
    \medskip
    \item Precursora de las futuras shells de Unix
  \end{itemize}
\end{frame}

\subsection{Bourne shell (sh)}
%%%%%%%%%%%%%%%%%%%%%%%%%%%%%%%%%%%%%%%%%%%%%%%%%%%%%%%%%%%%%%%%%%%%%%%

\begin{frame}
  \frametitle{Bourne shell (sh)}
  \begin{itemize}
    \item Desarrollada por Stephen Bourne en AT\&T Bell Laboratories en 1977
    \item Sustituye a Thompson shell en Unix version 7
    \medskip
    \item Permitía scripting, variables, control de señales...
    \medskip
    \item Actualmente es la shell por defecto del usuario \texttt{root} en muchos Unix
  \end{itemize}
\end{frame}

\subsection{C shell (csh)}
%%%%%%%%%%%%%%%%%%%%%%%%%%%%%%%%%%%%%%%%%%%%%%%%%%%%%%%%%%%%%%%%%%%%%%%

\begin{frame}
  \frametitle{C shell (csh)}
  \begin{itemize}
    \item Desarrollada por Bill Joy en la Universidad de California a finales de los 70
    \item Independiente de la plataforma
    \medskip
    \item Añade mejoras: historial de comandos, aliases, completado...
    \medskip
    \item Licencia BSD
  \end{itemize}
\end{frame}

\subsection{Tenex C shell (tcsh)}
%%%%%%%%%%%%%%%%%%%%%%%%%%%%%%%%%%%%%%%%%%%%%%%%%%%%%%%%%%%%%%%%%%%%%%%

\begin{frame}
  \frametitle{Tenex C shell (tcsh)}
  \begin{itemize}
    \item Desarrollada por Ken Green en la Universidad Carnegie Mellon en 1975
    \item Basada en C shell con mejoras: autocompletado, Unicode...
    \medskip
    \item Es la shell por defecto de FreeBSD, DragonFly BSD y DesktopBSD
    \medskip
    \item Licencia BSD
  \end{itemize}
\end{frame}

\subsection{Korn shell (ksh)}
%%%%%%%%%%%%%%%%%%%%%%%%%%%%%%%%%%%%%%%%%%%%%%%%%%%%%%%%%%%%%%%%%%%%%%%

\begin{frame}
  \frametitle{Korn shell (ksh)}
  \begin{itemize}
    \item Desarrollada por David Korn en AT\&T Bell Laboratories en 1983
    \item Respeta el estándar \textit{Shell Language Standard} (POSIX 1003.2)
    \medskip
    \item Hasta el año 2000, era software propietario de AT\&T
    \item Se crearon varias alternativas libres: pdksh, mksh...
    \medskip
    \item Actualmente \texttt{pdksh} es la shell por defecto de OpenBSD
  \end{itemize}
\end{frame}

\subsection{Bourne-again shell (bash)}
%%%%%%%%%%%%%%%%%%%%%%%%%%%%%%%%%%%%%%%%%%%%%%%%%%%%%%%%%%%%%%%%%%%%%%%

\begin{frame}
  \frametitle{Bourne-again shell (bash)}
  \begin{itemize}
    \item Escrita por Bryan Fox en 1987
    \item Forma parte del proyecto GNU
    \medskip
    \item Incluye ideas de \texttt{ksh} y \texttt{csh}: historial, autocompletado...
    \item Es posible programar funciones, argumentos por defecto...
    \medskip
    \item Ampliamente utilizada porque es la shell por defecto en sistemas GNU/Linux
    \medskip
    \item Licencia GPL
  \end{itemize}
\end{frame}

\subsection{Z shell (zsh)}
%%%%%%%%%%%%%%%%%%%%%%%%%%%%%%%%%%%%%%%%%%%%%%%%%%%%%%%%%%%%%%%%%%%%%%%

\begin{frame}
  \frametitle{Z shell (zsh)}
  \begin{itemize}
    \item Escrita por Paul Flastad en la Universidad de Princeton en 1990
    \item El nombre viene de Zhong Shao (zsh), profesor de Paul Flastad
    \medskip
    \item Incorpora mejoras destacadas: corrección automática
    \item Es completamente configurable
    \medskip
    \item Muy utilizada por programadores
    \medskip
    \item Licencia MIT-like
  \end{itemize}
\end{frame}


%%%%%%%%%%%%%%%%%%%%%%%%%%%%%%%%%%%%%%%%%%%%%%%%%%%%%%%%%%%%%%%%%%%%%%%
\section{Ejercicios}
%%%%%%%%%%%%%%%%%%%%%%%%%%%%%%%%%%%%%%%%%%%%%%%%%%%%%%%%%%%%%%%%%%%%%%%
\begin{frame}
  \begin{center}
    \huge{Ejercicios}
  \end{center}
\end{frame}

\subsection{Ejercicio 1}
%%%%%%%%%%%%%%%%%%%%%%%%%%%%%%%%%%%%%%%%%%%%%%%%%%%%%%%%%%%%%%%%%%%%%%%

\begin{frame}
  \frametitle{Ejercicio 1}
  \begin{itemize}
    \item Instalación de algunas de estas shells:
    \begin{itemize}
      \item csh
      \item tcsh
      \item zsh
      \item pdksh
    \end{itemize}
    \item Análisis de funcionalidades y características particulares
  \end{itemize}
\end{frame}


%%%%%%%%%%%%%%%%%%%%%%%%%%%%%%%%%%%%%%%%%%%%%%%%%%%%%%%%%%%%%%%%%%%%%%%
\section{Scripting}
%%%%%%%%%%%%%%%%%%%%%%%%%%%%%%%%%%%%%%%%%%%%%%%%%%%%%%%%%%%%%%%%%%%%%%%
\begin{frame}
  \begin{center}
    \huge{Scripting}
  \end{center}
\end{frame}

\begin{frame}
  \frametitle{Scripting}
  Cuando se programan scripts hay que tener en cuenta:
  \begin{itemize}
    \item hacerlos multiplataforma
    \item invocar al intérprete correcto en la cabecera
    \item las peculiaridades de cada shell
    \medskip
    \item hay vida más allá de \texttt{bash} !!!
  \end{itemize}
\end{frame}


%%%%%%%%%%%%%%%%%%%%%%%%%%%%%%%%%%%%%%%%%%%%%%%%%%%%%%%%%%%%%%%%%%%%%%%
\section{Referencias}
%%%%%%%%%%%%%%%%%%%%%%%%%%%%%%%%%%%%%%%%%%%%%%%%%%%%%%%%%%%%%%%%%%%%%%%
\begin{frame}
  \begin{center}
    \huge{Referencias}
  \end{center}
\end{frame}

\begin{frame}
  \frametitle{Referencias}
  \begin{itemize}
    \item \url{http://en.wikipedia.org/wiki/Comparison\_of\_command\_shells}
  \end{itemize}
\end{frame}


\end{document}
