\documentclass[a4paper]{article}
%\usepackage[T1]{fontenc}
\usepackage[utf8]{inputenc}
\usepackage{url}
\usepackage[pdfborder=0 0 0]{hyperref}
%\usepackage{html}
%\usepackage{hthtml}
\usepackage{geometry}

% Comments (optional argument is author of comment)
\newcommand{\comments}[2][?]{
  \begin{quote}
    \textbf{Comment (#1):} {\em #2}
  \end{quote}
  }

% \name is for ``special names'', like procedure or variable names.
\newcommand{\name}[1]{\texttt{\hturl{#1}}}

% Just a shortcut for links where the url appears as footnote.
\newcommand{\htflink}[2]{\htmladdnormallinkfoot{#1}{#2}}

\title{MSWL Economic Aspects \\
Master on libre software \\
URJC - GSyC/Libresoft \\
\url{http://master.libresoft.es}}

\author{Jesus M. Gonzalez-Barahona, David López}
\date{November 2011}

\sloppy
\begin{document}
\maketitle

\begin{abstract}
Course syllabus and learning program for the course ``Economic aspects of libre software'', of the Master on libre software of the Universidad Rey Juan Carlos (Móstoles, Spain).

[This is an evolving document, until the course is finished and graded]
\end{abstract}

\tableofcontents

%%----------------------------------------------------------------
%%----------------------------------------------------------------
%%----------------------------------------------------------------
\section{Course topics and schedule}

%%----------------------------------------------------------------
%%----------------------------------------------------------------
\subsection{00 - Presentation of the course}

Presentation of the main aspects of the course, and specially those related to administrative issues, evaluation, etc.

%%----------------------------------------------------------------
\subsubsection{November 17, 2011 (0.5 hours)}

\begin{itemize}
\item \textbf{Lecturers:} Jesus M. Gonzalez-Barahona
\item \textbf{Presentation:} ``Presentation of the course''
  \begin{itemize}
  \item \textbf{Supporting material:} Slides ``Presentation of the course''
  \end{itemize}
\end{itemize}

%%----------------------------------------------------------------
%%----------------------------------------------------------------
\subsection{01 - Introduction and motivation}

Introduction to the economic and business aspects of libre software, and motivation about their importance and relevance.

%%----------------------------------------------------------------
\subsubsection{November 17, 2011 (3.5 hours)}

\begin{itemize}
\item \textbf{Lecturers:} Jesus M. Gonzalez-Barahona and Felipe Ortega
\item \textbf{Presentation:} ``Introduction and motivation''
  \begin{itemize}
  \item \textbf{Discussion:} Facts about the current state of deployment and market share of libre software products
  \item \textbf{Supporting material:} Slides ``Introduction and motivation''
  \end{itemize}
\item \textbf{Discussion and assignment (results in website):} Comments on ``Yockai Benkler on the new open source economics'' (assignment~\ref{sub:comments-benkler})
\item \textbf{Assignment (results in website):} Numbers about libre software use and deployment (assignment~\ref{sub:numbers-use})
\end{itemize}

%%----------------------------------------------------------------
\subsubsection{November 24, 2011 (0.5 hours)}

\begin{itemize}
\item \textbf{Discussion and assignment (results in website):} Statements about economic aspects of libre software (assignment~\ref{sub:statements-eco})
\item \textbf{Assignment (results in website):} Discussion of ``The new (commercial) open source'' (assignment~\ref{sub:new-oss})
\end{itemize}

%%----------------------------------------------------------------
%\subsubsection{December 15, 2011 (0.5 hours)}

%Review and discussion of pending assignments.

%\begin{itemize}
%\item \textbf{Lecturers}: Jesus M. Gonzalez-Barahona.
%\item \textbf{Discussion of assignment:} Comments on ``Yockai Benkler on the new open source economics'' (assignment~\ref{sub:comments-benkler})
%\end{itemize}

%%----------------------------------------------------------------
%%----------------------------------------------------------------
\subsection{02 - Impact on stakeholders and markets, strategic aspects}

%%----------------------------------------------------------------
\subsubsection{November 24, 2011 (1.5 hours)}

\begin{itemize}
\item \textbf{Lecturers}: Jesus M. Gonzalez-Barahona.
\item \textbf{Presentation:} ``Impact on companies using software and in the production fabric''
  \begin{itemize}
  \item \textbf{Supporting material:} Slides ``Economic impact''
  \end{itemize}
\item \textbf{Assignment and discussion (results in website):} The role of libre software (assignment~\ref{sub:impact-role}).
\item \textbf{Assignment (results in website):} Libre software as an strategic tool  (assignment~\ref{sub:impact-strategic-tool}).
\item \textbf{Supporting materials:} 
  \begin{itemize}
  \item Video: ``Open for business: Building successful commerce around open source'' \\
    \url{http://www.parc.com/event/1092/open-for-business.html}
  \item Text: ``FLOSS adoption models (FLOSS guide for SMEs)'', by Carlo Daffara \\
    \url{http://guide.conecta.it/index.php/3._Basic_FLOSS_adoption_models}
  \item Text: ``Best practices for FLOSS adoption (FLOSS guide for SMEs)'', by Carlo Daffara \\
    \url{http://guide.conecta.it/index.php/5._Best_practices_for_FLOSS_adoption}
  \item Text: ``The Economic Motivation of Open Source Software: Stakeholder Perspectives'' \\
    \url{http://dirkriehle.com/2008/07/20/the-economic-motivation-of-open-source-software-stakeholder-perspectives/}
  \end{itemize}
  
\end{itemize}

%%----------------------------------------------------------------
%%----------------------------------------------------------------
\subsection{03 - Cost/benefit analysis}

%%----------------------------------------------------------------
\subsubsection{November 24, 2011 (2 hours)}

\begin{itemize}
\item Lecturers: David López
\item Presentation: Estimating value of IT projects with WiBe methodology
\item \textbf{Supporting materials:} 
  \begin{itemize}
  \item Website: WiBe Methodology \\ \url{http://www.eu.wibe.de/}
  \item WiBe case study. Server Migration: Economic Efficiency Assessment (Red Hat Enterprise Linux vs. Microsoft Windows Server) \\
  \url{www.eu.wibe.de/wibe_consulting/example/WIBE-STUDY-SERVER_MIGRATION_EN.1.0.pdf}
  \item Spreadsheet for implementing WiBe methodology.
  \end{itemize}
\item \textbf{Assignment (results in website):} TCO analysis, Red Hat Enterprise Linux vs. Microsoft Windows Server
\end{itemize}

%%----------------------------------------------------------------
\subsubsection{December 1, 2011 (2 hours)}

\begin{itemize}
\item Lecturers: David López
\item \textbf{Assignment discussion:} TCO analysis, Red Hat Enterprise Linux vs. Microsoft Windows Server
\item Presentation: Specific libre software aspects of TCO estimations
\item \textbf{Assignment (results in website):} Reelaboration of the TCO analysis, Red Hat Enterprise Linux vs. Microsoft Windows Server
\end{itemize}

%%----------------------------------------------------------------
\subsubsection{December 15, 2011 (2 hours)}

\begin{itemize}
\item Lecturers: David López
\item \textbf{Assignment discussion:} Reelaboration of the TCO analysis, Red Hat Enterprise Linux vs. Microsoft Windows Server
\item \textbf{Discussion:} The Open Bravo business model
\item \textbf{Assignment (results in website):} The Open Bravo use case
\end{itemize}

%% \item Debate: strategic aspects of FLOSS outside the software development business.
%% \item Presentation: Adoption of FLOSS in companies: TCO and other financial and 
%%   operational perspectives for SMEs and firms.
%% \item Criteria for an informed selection of FLOSS solutions.
%%   \begin{itemize}
%%   \item Video: Charles Leadbeater on open innovation.
%%   \end{itemize}
%% \item Open debate. Concrete case examples: Android.
%% \item \textit{Assignment}: Choose a specific sector and try to introduce a novel business strategy based on FLOSS.

%% \item Macroeconomics aspects.
%%   \begin{itemize}
%%   \item Implications for global and country-level finances.
    
%%   \end{itemize}



%%----------------------------------------------------------------
%%----------------------------------------------------------------
\subsection{03 - Business models}

%%----------------------------------------------------------------
\subsubsection{December 1, 2011 (4 hours)}

\begin{itemize}
\item \textbf{Lecturers}: Jesus M. Gonzalez-Barahona
\item \textbf{Presentation:} ``FLOSS business models''
  \begin{itemize}
  \item \textbf{Supporting material:} Slides ``FLOSS business models''
  \end{itemize}
\item \textbf{Discussion:} Case examples of companies running FLOSS business models (assignment~\ref{sub:business-cases}).
\item \textbf{Discussion and assignment (results in website):} Identify FLOSS
business models (assignment~\ref{sub:business-models}). First three companies were discussed in class, the rest are for answering in the website.
\item \textbf{Discussion and assignment (results in website):} The open core debate (assignment~\ref{sub:business-opencore}). General discussion in class, specific questions for answering in the website.
\item \textbf{Assignment (results in website):} The Magic Cauldron (assignment~\ref{sub:business-magic-cauldron}). General discussion in class, specific questions for answering in the website.
\item \textbf{Supporting material (read before session):}
  \begin{itemize}
  \item Text ``FLOSS-based business models (FLOSS guide for SMEs)'', by Carlo Daffara \\
    \url{http://guide.flossmetrics.org/index.php/6._FLOSS-based_business_models}
  \end{itemize}
\end{itemize}

%%----------------------------------------------------------------
%\subsubsection{December 15, 2011 (0.25 hours)}

%% \begin{itemize}
%% \item \textbf{Lecturers}: Jesus M. Gonzalez-Barahona.
%% \item \textbf{Assignment (results in website):} Your very own FLOSS
%%   business plan (assignment~\ref{sub:business-plan}).
%% \end{itemize}


%%----------------------------------------------------------------
%%----------------------------------------------------------------
\subsection{04 - Libre software and open innovation}

How libre software can be used to support open innovation practices, and how open innovation practices are used in libre software projects and communities.

%%----------------------------------------------------------------
\subsubsection{December 15, 2011 (4 hours)}

\begin{itemize}
\item \textbf{Lecturers}: Jesus M. Gonzalez-Barahona
\item \textbf{Guest presentation:} ``Open innovation basics'', by Sergio Ramos (UPM)
  \begin{itemize}
  \item \textbf{Discussion:} How open innovation and libre software can be included in the strategy of CeDInt, an R\&D Institute.
  \item \textbf{Supporting material:} Slides ``Open innovation basics'', by Sergio Ramos and Claudio Feijoo (UPM)
  \end{itemize}
\item \textbf{Discussion and assignment (results in website):} Free software and R\&D Institutes (assignment~\ref{sub:openinnova-rd})
\item \textbf{Presentation:} ``Supporting innovation with free software''
  \begin{itemize}
  \item \textbf{Supporting material:} Slides ``Supporting innovation with free software''
  \end{itemize}
\item \textbf{Discussion and assignment (results in website):} Cases of free software and open innovation (assignment~\ref{sub:openinnova-cases})
\item \textbf{Assignment (results in website):} Charles Leadbeater on innovation
\end{itemize}

%% %%----------------------------------------------------------------
%% \subsubsection{December 15, 2011 (1.25 hours)}

%% Review and discussion of pending assignments.

%% \begin{itemize}
%%  \item \textbf{Lecturers}: Jesus M. Gonzalez-Barahona.
%% \item \textbf{Discussion of assignment:} Cases of free software and open innovation (assignment~\ref{sub:openinnova-cases})
%% \item \textbf{Discussion of assignment:} Charles Leadbeater on innovation (assignment~\ref{sub:openinnova-leadbater})
%% \end{itemize}

%%----------------------------------------------------------------
%%----------------------------------------------------------------
\subsection{05 - Sustainability of communities}


%%----------------------------------------------------------------
\subsubsection{December 22, 2011 (4 hours)}

\begin{itemize}
\item \textbf{Lecturers}: Jesus M. Gonzalez-Barahona
  
\item \textbf{Presentation:} ``Funding models for free software development''

    %% \begin{itemize}
    %%  \item Creating and joining FLOSS communities.
    %%     \begin{itemize}
    %%      \item A framework for management styles in FLOSS projects and their impact on FLOSS ecosystems.
    %%     \end{itemize}

    %%  \item Case study: Oracle acquires Sun Microsystems.
    %%     \begin{itemize}
    %%      \item Former strategies and FLOSS policies in Sun Microsystems.
    %%      \item Acquisition by Oracle. Implications for several FLOSS ecosystems.
    %%      \item Case example: the OpenSolaris community and the Illumos project.
    %%      \item Case example: MySQL and implications for market shares.
    %%      \item \textit{Forum assignment}: Analyze the rationale and implications of the Document Foundation and LibreOffice.
    %%     \end{itemize}
    %% \end{itemize}

\item \textbf{Supporting materials:} 
  \begin{itemize}
  \item Text: ``The Economic Case for Open Source Foundations''\\
    \url{http://dirkriehle.com/publications/2010/the-economic-case-for-open-source-foundations/}
  \item Text: ``Control Points and Steering Mechanisms in Open Source Software Projects''\\
    \url{http://dirkriehle.com/2010/11/24/control-points-and-steering-mechanisms-in-open-source-software-projects/} 
  \end{itemize}
\end{itemize}



\section{Grading}

This section details the criteria for grading the course, the deadlines for the different activities, and the submission details for th activities that require them.

\subsection{Evaluation criteria}
\label{sub:evaluation-criteria}

Each activity contributing to the grading of the course has its own evaluation criteria, as described below. Each of these activities has a minimum and maximum grading. If the minimum grading is 0, the activity is optional. Otherwise, the activity is mandatory, and has to be graded at least with the minimum to pass the course. Each activity has also a description, and when possible, some general grading criteria. In any case, the final grade for the course will also depend on the continuous observation of the instructors on the outcomes and progress of students.

Students should ask instructors about any detail which may not be clear to them, either about the general grading plan, or about specific aspects of the activities. As a general rule, evaluation will have into account how the activity and its results show that the student has come close to the competences, knowledge and skills expected for the course.

The student can consider that the next table will be used as a (minimum) guideline for assigning marks:

\begin{itemize}
\item Pass (``aprobado''): 150
\item Good (``notable''): 250
\item Excellent (``sobresaliente''): 350
\end{itemize}

\begin{itemize}
\item \textbf{Exercises (answered in forum)}. \\
  Minimum: 20 points, maximum: 100 points.

  Exercises proposed and answered in the the forum of the course.

\item \textbf{Blog entries}. \\
  Minimum: 20 points, maximum: 80 points

  Blog entries specifically related to the course, and marked as such. The tag used for that is mswl-eco.

\item \textbf{Collaborative notebook}. \\
  Minimum: 0 points, maximum: 40 points

  Based on work in class (in real time) and afterwards (complementing the work, using git).

\item \textbf{Business plan}. \\
  Minimum: 20 points, maximum: 80 points

  Business plan for a company in which libre software clearly has an impact. The plan has to be detailed enough. Detailed description in assignment~\ref{sub:business-plan}.

\item \textbf{Specific report}. \\
  Minimum: 20 points, maximum: 80 points

Specific report about a certain business model or strategy based on FLOSS, showing its general aspects, but also analyzing companies already putting it into place, discussing advantages and drawbacks of the model, etc. A detailed DAFO analysis has to be a part of the report.

As a result of this activity, the student should produce:

\begin{itemize}
\item A `traditional' written report.
\item A video or audio presentation (10 min. maximum).
\item A set of slides supporting the presentation.
\end{itemize}

 It is important to detail all the references, and to heavily root the report on data and/or specific works publicly available. The video or audio presentation  wil be uploaded to some audio or video web hosting site which allows for download of the whole audio or video file, not only streaming (eg, blip.tv)

\item \textbf{Other activities}. \\
  Minimum: 0 points, maximum: 100 points

  These activities have to be agreed in advance with the instructors.
\end{itemize}

\subsection{Submission deadlines}

All activities to be graded in January must be completed and submitted by December 31st 2011.

\subsection{Submission details}

Please, consider the details below for submitting the different activities for evaluation (for those not specified in this list, nothing special is needed for submission).

\begin{itemize}
\item As a summary of all the activities, a ``Summary of activities for evaluation'' should be sent. This summary should be uploaded to the corresponding resource in the Moodle site for this course, and should include the following data:
  \begin{itemize}
  \item \textbf{Name:} Full name of the student (as ``family name'', ``given name'')
  \item \textbf{Blog entries:} Url of the blog entries for this course (HTML, not RSS version).
  \item \textbf{Contributions to the collaborative notebook:} Id for commits to the repository where the collaborative notebook is hosted, and summary of the main contributions to it related to this course, including links to the repository and commit ids if appropriate.
  \item \textbf{List of other activities:} If any, list of other activities submitted for evaluation (those that would fit in the ``other activities'' item in the ``Evaluation criteria'' (subsection~\ref{sub:evaluation-criteria}). The results of those activities should be uploaded to the ``other activities'' resource in the Moodle site for this course, when appropriate. In some specific cases (such as streaming videos) it will be enough to include in this list the url to the external site where the result is hosted.
  \end{itemize}
\end{itemize}

%%----------------------------------------------------------------
%%----------------------------------------------------------------
%%----------------------------------------------------------------
\section{Assignments and activities}

%%----------------------------------------------------------------
%%----------------------------------------------------------------
\subsection{01 - Introduction and motivation}


%%----------------------------------------------------------------
\subsubsection{Statements about economic aspects of libre software}
\label{sub:statements-eco}

Review of 10 statements about economic aspects of FLOSS. They could be true, or not, or arguable. Working in groups, prepare and consolidate your answers collaboratively. Each group can elaborate in more detail about 2-3 statements, but should comment on all of them.

\textbf{Supporting material}

\begin{itemize}
\item Slides ``10 statements about economic aspects of FLOSS''
\end{itemize}

%%----------------------------------------------------------------
\subsubsection{Comments on ``Yochai Benkler on the new open source economics''}
\label{sub:comments-benkler}

Benkler presents a diagram showing different positions for product or services according to the type of management (centralized / decentralized) and economic profitability (market / non-market). He characterizes all his examples in this presentation falling in the non-market/decentralized space. Find counterexamples of enterprises like Crowdspring, making economic profit from a decentralized, crowdsourced platform (if possible, outside the FLOSS sector).

\textbf{Supporting material}

\begin{itemize}
\item Video ``Yockai Benkler on the new open source economics'' \\
  \url{http://www.ted.com/talks/lang/eng/yochai_benkler_on_the_new_open_source_economics.html}
\item Slides for the assignment 
\end{itemize}

%%----------------------------------------------------------------
\subsubsection{Discussion of ``The new (commercial) open source''}
\label{sub:new-oss}

Read the paper ``The new (commercial) open source'' and reflect on the statements and proofs offered to show the supposed negative effects of FLOSS introduction into the market. Contribute any comments or remarks about the points you found to be the most polemic or interesting.

\textbf{Supporting material}

\begin{itemize}
\item Paper `` The New (Commercial) Open Source: Does It Really Improve Social Welfare?'', Engelhardt and Maurer \\
  \url{http://ssrn.com/abstract=1542180}
\end{itemize}

%%----------------------------------------------------------------
\subsubsection{Numbers about libre software use and deployment}
\label{sub:numbers-use}

Look for numbers showing quantitatively the usage or deployment of libre software. Those can represent market shares, number of users, number of companies, etc. They can refer to a market niche, to a country or region, or even to a company or public administration; to a specific point in time, or to evolution over several years. They can be only numbers, but of course can also be graphs or other representations of the data.

%%----------------------------------------------------------------
%%----------------------------------------------------------------
\subsection{02 - Libre software and open innovation}

%%----------------------------------------------------------------
\subsubsection{Free software and R\&D Institutes}
\label{sub:openinnova-rd}

After the presentation by Sergio Ramos about CeDInt-UPM, and its challenges when considering open innovation strategies, and based on it and your own experience and your knowledge of free software communities and projects, please elaborate on either:

\begin{itemize}
\item How could an R\&D Institute adapt some of the innovation practices found in free software projects and communities.
\item How could an R\&D Institute use free software as an integral part of an open innovation strategy.
\end{itemize}


%%----------------------------------------------------------------
\subsubsection{Cases of free software and open innovation}
\label{sub:openinnova-cases}

Find some more cases of free software and open innovation, and try to classify them according to the categories discussed in the slides (the West-Gallagher models of open innovation with free software). Discuss any of them.


%%----------------------------------------------------------------
\subsubsection{Charles Leadbeater on innovation}
\label{sub:openinnova-leadbater}

View "Charles Leadbeater on innovation", and comment about it. In addition to a general comment on the presentation, address the issues detailed in the slides for this assignment.

\textbf{Supporting material}

\begin{itemize}
\item Video ``Charles Leadbeater on innovation'' \\
  \url{http://www.ted.com/talks/charles_leadbeater_on_innovation.html}
\item Slides for the assignment
\end{itemize}

\textbf{Comments}

At first sight, the presentation by Leadbeater may seem quite rational. However, it is very strange to find industries that really take into account customer's participation in the design of their core products (of course, except for their purchasing power). In fact, it is important to remark that Leadbeater is not referring to how industries follow the money of their customers, which they usually do (if a product is not bought by clients, it has little future). Leadbeater is referring to a deeper implication of customers (users) in the design and evolution of products and services. The mountain bike case is quite representative of this.

From this point of view, consider how big industries, such as car manufacturing, or fashion, or foods industries are introducing new products in which customers had very little to say except for ``yes'' or ``no''. Compare that to the mountain bike or rap cases, where customers were actually innovating and designing goods, and the industry just followed their lead when they realized the innovations.
%%----------------------------------------------------------------
%%----------------------------------------------------------------
\subsection{03 - Business models}

%%----------------------------------------------------------------
\subsubsection{Case examples of companies running FLOSS business models}
\label{sub:business-cases}

Discussion on the business models of some of the most well known, or illustrating, companies in the area of libre software.

%%----------------------------------------------------------------
\subsubsection{Identify FLOSS business models}
\label{sub:business-models}

Identify the main (and complementary) business models adopted by the following companies:

\begin{itemize}
\item Liferay
\item Eucalyptus
\item Acquia
\item BlackDuck.
   \url{http://www.blackducksoftware.com/}
 \item Zimbra.
   \url{http://www.zimbra.com/}
 \item IBM (libre software projects).
   \footnotesize{\url{http://www.ibm.com/developerworks/views/opensource/projects.jsp}}
 \item Jaspersoft.
   \url{http://www.jaspersoft.com/}
 \item Funambol.
   \url{http://www.funambol.com/}
\end{itemize}

In every case, document your answers as much as possible. In all the cases, there is relevant information in the Net that of course you should use.

\textbf{Supporting material}

\begin{itemize}
\item Slides for the assignment
\end{itemize}

%%----------------------------------------------------------------
\subsubsection{The open core debate}
\label{sub:business-opencore}

``Open core'' is one of the frequent, yet most controversial business model strategies around FLOSS. Open core companies usually defend their right to maintain proprietary licenses on strategic features and modules. Some FLOSS advocates, on the contrary, argue that this approach benefits from FLOSS without contributing back to communities and projects.

In addition to a general discussion on the topic, answer the following questions:

\begin{itemize}
\item Find three examples of companies following open core strategies.
\item Could any of these companies switch to a different business model
  and ensure sustainability? How?
\item Do these companies make any claims on this issue? Summarize
  their arguments.
\item After learning the arguments from both sides of the story, what is your
  own opinion on this issue?
\end{itemize}

\textbf{References}

\begin{itemize}
\item Lampitt. ``Open core licensing...''
  \footnotesize{\url{http://alampitt.typepad.com/lampitt_or_leave_it/2008/08/open-core-licen.html}}
\item Carlo Daffara. ``Relationships between open core, dual licensing and contributions''
   \url{http://carlodaffara.conecta.it/?p=460}
 \item Simon Phipps. ``Open core is bad for you''.
y   \footnotesize{\url{http://blogs.computerworlduk.com/simon-says/2010/06/open-core-is-bad-for-you/}}
 \item OSI. ``A simple declaration about open core''
   \url{http://www.opensource.org/blog/OpenCore}
\end{itemize}

\textbf{Supporting material}

\begin{itemize}
\item Slides for the assignment
\end{itemize}

%%----------------------------------------------------------------
\subsubsection{The Magic Cauldron}
\label{sub:business-magic-cauldron}

Answer the following 2 questions on Raymond's essay:

\begin{itemize}
\item What does the 'information wants to be free' myth imply for the price of FLOSS products? Does it have any connections to the open core business model?
\item Why does Raymond argue that the so-called 'gift culture' is not a good model to explain the economic incentives behind FLOSS development?
\end{itemize}

\textbf{References}

\begin{itemize}
\item Paper ``The magic cauldron'', by Eric Raymond \\
  \url{http://www.catb.org/~esr/writings/magic-cauldron/}
\end{itemize}

%%----------------------------------------------------------------
\subsubsection{Your very own FLOSS business plan}
\label{sub:business-plan}

Think about an hypothetical start-up company, with an activity based libre software, or for which libre software may mean a difference. Create a short business plan for it.

A previous discussion can be done in groups of up to three persons. Once the discussion has dealt with all the topics, and the plan (or plans) is defined, each student should write her own business plan for it. It should include:

\begin{itemize}
\item Name of project.
\item Main goals and brief description of the start-up.
\item License type and overall strategy.
\item Briefly detail some strategic areas.
\item Fill in Osterwalder's canvas, in the following (suggested) order (you don't have to actually write on the poster, but just include information for each of the following topics):
  \begin{itemize}
  \item Clients segments
  \item Value proposal
  \item Channels
  \item Key resources
  \item Cost structure
  \item Revenues streams.
  \item Customer relationships
  \item Key activities
  \item Key partners
  \end{itemize}
\end{itemize}

\textbf{References}

\begin{itemize}
\item Business model alchemist \\
  \url{http://www.businessmodelalchemist.com/} 
\item Book: ``Business Model Generation: A Handbook for Visionaries, Game Changers, and Challengers'', A. Osterwalder and Yves Pigneur. Wiley, July 2010.
\item How to analyze an OSS business model (part 1 to 5).
  \begin{itemize}
  \item \url{http://carlodaffara.conecta.it/?p=372}
  \item \url{http://carlodaffara.conecta.it/?p=379}
  \item \url{http://carlodaffara.conecta.it/?p=387}
  \item \url{http://carlodaffara.conecta.it/?p=395}
  \item \url{http://carlodaffara.conecta.it/?p=413}
  \end{itemize}
\item Business model canvas \\
  \url{http://www.businessmodelgeneration.com/downloads/business_model_canvas_poster.pdf}
\end{itemize}

\textbf{Supporting material}

\begin{itemize}
\item Slides for the assignment
\end{itemize}


%%----------------------------------------------------------------
\subsubsection{The role of libre software}
\label{sub:impact-role}

Discuss the slide ``The (business, economic) role of libre software'', from ``Economic impact''. In particular, for each of the dimensions presented (see below), explain why libre software could be useful for a company which uses software. The discussion should be in business terms, that is, how that dimension could help to improve profit, or reduce risk, or increase competitiveness, etc. In addition to explain the pros, show also the risks involved, or even why in certain circumstances libre software could be a bad option.

The dimensions are:

\begin{itemize}
\item Functionality
\item Acquisition of technology
\item Economic efficiency
\item New opportunities
\item Service economy
\item Adaptability, conformance to needs
\item Impact
\end{itemize}


%%----------------------------------------------------------------
\subsubsection{Libre software as an strategic tool}
\label{sub:impact-strategic-tool}

View the video ``El software libre como herramienta de estrategia empresarial'', and summarize how, in the opinion of the speaker, libre software can be an strategic tool for a company.

\textbf{Supporting material}

\begin{itemize}
\item \textbf{Video}: ``El software libre como herramienta de estrategia empresarial'', by Juanjo Hierro \\
  \url{http://www.eoi.es/mediateca/video.php?videoid=376}
\end{itemize}

\end{document}
