%
% $Id: $

\section{History of libre (free, open source) software}

%%---------------------------------------------------------------

\begin{frame}
\frametitle{Before 1970s}

\begin{itemize}
\item Software was born libre
\item Source code was shared by developers
\item Software was a kind of add-on to hardware
\item Communities of users, promoted by vendors
\item No specific market for software: vendor catalog
\end{itemize}
\end{frame}

\begin{frame}
\frametitle{1970s, early 1980s}

\begin{itemize}
\item Richard Stallman, GNU, FSF.
  \begin{itemize}
  \item Legal (GPL) and philosophical foundations.
  \item Basic infrastructure: text editor (Emacs), compiler (GCC),
    debugger (GDB), etc.
  \item Goal: building a complete system, alternative to Unix.
  \item Structured work, with clear goals.
  \end{itemize}
\item Isolated efforts: TeX, Spice, etc.
\end{itemize}
\end{frame}

 %%---------------------------------------------------------------

 \begin{frame}
 \frametitle{1970s, early 1980s (2)}

 \begin{itemize}
 \item Berkeley CSRG:
   \begin{itemize}
   \item Importance of sharing source code (``original'' Unix culture),
   \item Limited by ATT license (but not from a practical point of
     view, everyone had it).
   \item Focus on the operating system (kernel, utilities, etc.)
   \item Used in many proprietary software (SunOS, Ultrix, etc.)
   \end{itemize}
 \item Early Internet:
   \begin{itemize}
   \item Reference implementations, available for everyone.
   \item The Net as a tool for cooperation (News, ftp, email).
   \item User community provides the best support.
   \end{itemize}
 \end{itemize}

 \end{frame}

 %%---------------------------------------------------------------

 \begin{frame}
 \frametitle{Late 1980s, early 1990s}

 \begin{itemize}
 \item Complete environments around Unix (SunOS, Solaris, etc.):
   \begin{itemize}
   \item Many applications are the best ones in their field (Unix
     utilities, GCC compiler, etc.)
   \item Specially interesting: X~Window.
   \item Only a kernel is missing...
   \end{itemize}
 \item 386BSD, NetBSD, FreeBSD, OpenBSD:
   \begin{itemize}
   \item Bill Jolitz builds the parts missing in the kernel.
   \item Quickly afterwards: complete systems, similar to SunOS in
     functionality.
   \item BSD license, can be redistributed as proprietary software.
   \end{itemize}
 \end{itemize}

 \end{frame}


 %%---------------------------------------------------------------

 \begin{frame}
 \frametitle{Late 1980s, early 1990s (2)}

 \begin{itemize}
 \item GNU/Linux:
   \begin{itemize}
   \item Linus Torvalds builds his ``libre Minix''.
   \item Hundreds of developers jump on it, integrating GNU software.
   \item Porting of applications, new applications...
   \item Most code under GPL: remains libre after redistribution
   \item One kernel, many distributions (Slackware, Debian, RedHat,
     Suse, etc.).
   \item Large ``popular'' success.
   \end{itemize}
 \end{itemize}

 \end{frame}

 %%---------------------------------------------------------------

 \begin{frame}
 \frametitle{Late 1990s}

 \begin{itemize}
 \item Netscape announcement.
 \item GNU/Linux and FreeBSD compete with Windows NT.
 \item Closer and closer to the ``regular'' user: KDE, GNOME.
 \item GNU/Linux enters Universities (and student's homes).
 \item Best option is libre in many niches (Apache,
   Internet infrastructure, XFree, GCC, Gnat).
 \item Companies such as RedHat obtain venture capital.
 \item Media starts to pay attention to libre software.
 \item Large corporations (Corel, Apple, IBM) study how to deal with
   libre software.
 \end{itemize}

 \end{frame}

%%---------------------------------------------------------------

\begin{frame}
  \frametitle{Early 2000s}

  \begin{itemize}
  \item Libre software is close to be ready for the desktop
    (GNOME 2.x, KDE 3.x, OpenOffice), and it is easy to install by
    end-users.
  \item Libre software enters the strategy of large corporations (IBM,
    HP, Sun).
  \item Some others (such as Microsoft) prefer an strategy of partial
    clash.
  \item Funding difficulties due to the dotcom crisis.
  \item Penetration in public administrations and large corporations
    starts.
  \item Large increase of number of developers, quantity of available
    libre software, etc.
  \end{itemize}

\end{frame}


%%---------------------------------------------------------------

\begin{frame}
  \frametitle{Early 2000s (2)}

  \begin{itemize}
  \item A new research field studies libre software: path open to
    the understanding of how it works
  \item First effects of location independence: countries in the edges
    start to show interesting developments
  \item Some markets, some sectors, consider libre software as a
    ``natural'' option.
  \item Legal environment is changing: will it became hostile to libre
    software?
  \end{itemize}

\end{frame}


%%---------------------------------------------------------------

\begin{frame}
  \frametitle{Late 2000s}

  \begin{itemize}
  \item Libre software is strategic for many companies (ej: Google).
  \item Complete application suites for many environments.
  \item Companies testing new collaboration models (ej: OW2, Morfeo).
  \item New business models, models for new business.
  \item Libre software as an enabler in many industries.
  \item Libre software is becoming ``business as usual''.
  \end{itemize}

\end{frame}

%%---------------------------------------------------------------

\begin{frame}
  \frametitle{The future: an obstacle race?}

  Future evolution of libre software can find some obstacles:

  \begin{itemize}
  \item FUD (fear, uncertainty, doubt) techniques: up to know have
    caused little harm.
  \item ``Dissolution'' (models which can be mistaken with libre
    software): community split, loss of model advantages
  \item Ignorance (loss of vision): why is libre software interesting?
  \item Legal barriers: software patents, DRM, etc.
  \end{itemize}

\end{frame}

%%---------------------------------------------------------------

\begin{frame}
  %\frametitle{}

  \begin{center}
  {\LARGE \bf What will be the status of libre software in 10 years?}
  \end{center}

\end{frame}
