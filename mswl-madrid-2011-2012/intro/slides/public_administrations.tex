%% public_administrations.tex
%%
%% Presentation of the relationship of public administrations and free software for the Official Master on Libre Software (URJC)
%% http://master.libresoft.es
%%

%%---------------------------------------------------------------------
%%---------------------------------------------------------------------

\section{Public Administrations and Libre Software}


%%%%%%%%%%%%%%%%%%%%%%%%%%%%%%%%%%%%%%%%%%%%%%%%%%%%%%%%%%%%%%%%%%%%%%%

%%---------------------------------------------------------------

\begin{frame}
\frametitle{Impacto en las administraciones p�blicas}

\begin{itemize}
\item Aprovechamiento m�s adecuado de recursos: m�s impacto para la
  misma inversi�n (ejemplo: localizaci�n)
\item Fomento del tejido tecnol�gico local
\item Independencia del proveedor
\item Adaptaci�n a las necesidades exactas
\item Escrutinio p�blico (en seguridad, rendimiento, etc.)
\item Disponibilidad a largo plazo
\end{itemize}

\end{frame}

%%---------------------------------------------------------------

\begin{frame}
\frametitle{Dificultades de adopci�n}

\begin{itemize}
\item Desconocimiento y falta de decisi�n pol�tica
\item Poca adecuaci�n de los mecanismos de contrataci�n
\item Falta de estrategia de implantaci�n
\item Escasez o ausencia de productos libres en ciertos segmentos
\end{itemize}

\end{frame}

%%---------------------------------------------------------------

\begin{frame}
\frametitle{Actuaciones de las administraciones p�blicas}

\begin{itemize}
\item Compra de servicios y sistemas basados en software libre
  (gen�ricos, a medida, etc.)
\item Promoci�n de la sociedad de la informaci�n (uso de software a
  gran escala)
\item Fomento de la investigaci�n
\end{itemize}

\end{frame}

%%---------------------------------------------------------------

\begin{frame}
\frametitle{Algunos escenarios ejemplo}

\begin{itemize}
\item Concurso p�blico para la adquisici�n de un juego de aplicaciones
  ofim�ticas con ciertas especificaciones
  \begin{itemize}
  \item proyecto a dos a�os,
  \item decenas o centenas de miles de puestos
  \item varios ganadores que
  compitan entre ellos por tener m�s usuarios
\end{itemize}
\item Consorcio de administraciones que contratan un software a medida
  con la condici�n de que el resultado sea software libre
  \begin{itemize}
  \item despliegue y otros servicios incluidos en el contrato
  \item otras administraciones interesadas pueden entrar en el
    consorcio o contratar por su cuenta
  \end{itemize}
\end{itemize}

\end{frame}

%%---------------------------------------------------------------

\begin{frame}
\frametitle{Algunos casos reales}

\begin{itemize}
\item Proyecto LiMux en Munich
\item Estrategia en Extremadura, basada en LinEx
\item Software libre en Brasil
\end{itemize}

\begin{flushright}
  \url{http://www.muenchen.de/Rathaus/referate/dir/limux/89256/} \\
  \url{http://www.linex.org/} \\
  \url{http://www.softwarelivre.org/}
\end{flushright}
\end{frame}

