%%
%% Consequences of the libre software model
%%
%% Master on libre software
%% http://master.libresoft.es

\section{Some consequences of the model}

%%---------------------------------------------------------------

\begin{frame}
\frametitle{How libre software affects...}

\begin{itemize}
\item End user (individual or company)
\item Developer (or software producer)
\item Integrator
\item Maintenance and services
\end{itemize}
\end{frame}


%%---------------------------------------------------------------

\begin{frame}
\frametitle{End user}

End users can forget about...

\begin{itemize}
\item ...company monopolies \\
  (real competition, best products and services)
\item ...producer `reliability' \\
  (future path ensured by product acceptance, source code availability, community dynamics)
\item ...decision taking with few elements \\
  (software can be tested in real environments, with near-zero cost)
\item ...dependence on provider's strategies \\
  (many providers, community strategies, strategies follow clients)
\item ...black boxes \\
  (no longer ``blind confidence'')
\end{itemize}
\end{frame}

%%---------------------------------------------------------------

\begin{frame}
\frametitle{End user (2)}

What if users could...

\begin{itemize}
\item ...adapt/customize the product at will?
\item ...have the latest release with (very) low cost?
\item ...fix all the problems (or hire someone to fix them)?
\item ...decide on the future evolution of the product?
\item ...contract the (complete) integration of the best products in a given area?
\item ...buy complete auditing for each product by independent third parties?
\end{itemize}
\end{frame}

%%---------------------------------------------------------------

\begin{frame}
\frametitle{End user (3)}

\begin{center}
{\LARGE A large portion of the control moves to the user \\
(from the producer of the software)}
\end{center}
\end{frame}

%%---------------------------------------------------------------

\begin{frame}
\frametitle{Developer (or software producer)}

Libre software changes the rules of the game:

\begin{itemize}
\item Opportunities for competing while being small
\item Easier (and cheaper) to acquire front-wave technology
\item Can take advantage of the work of your competitors (but they can do the same!)
\item External contributors can be found (in many cases, at a fraction of the usual cost, because of win-win relationships)
\item Distribution channels are cheaper, and truly global
\item Feasible to become reference application in a niche
\end{itemize}
\end{frame}


%%---------------------------------------------------------------

\begin{frame}
\frametitle{Developer (or software producer) (2)}

Where does the money come from? (sustainability)
\begin{itemize}
\item The producer enjoys the best knowledge about its product
\item Producer can be the ``most visible point'', if image is cared of
\item Custom-made development, modifications, customizations
\item ``In depth'' support (bug fixing, preference in access to new releases, new features)
\end{itemize}

\begin{center}
{\large Assuming there is a need for a software product \\
  and there is money ready for supporting that need \\
  some developer/producer will benefit from the situation \\
  (if both parts are put in contact)
}
\end{center}
\end{frame}


%%---------------------------------------------------------------

\begin{frame}
\frametitle{Integrator}

Maybe the best placed actor:

\begin{itemize}
\item All libre software products available (without the constraints of proprietary licences!)
\item If products ``don't fit'' you can adapt them (source code is available, interoperability is always possible)
\item Pieces of products, or full products, or anything in the middle, can be integrated
\item No more black boxes: everything is transparent 
\end{itemize}

\begin{center}
{\large They can build on top of the work of others, with similar constraints and possibilities to those others}
\end{center}
\end{frame}


%%---------------------------------------------------------------

\begin{frame}
\frametitle{Services and maintenance}

\begin{itemize}
\item Similar conditions than the producer
\item Competition in the maintenance business
\item Added-value of services is better appreciated (the base cost of the program is low)
\item Good knowledge of the state of the art is important (good idea to have good links with libre software projects)
\item New business models: advising on releases and combination of programs, information about new development, project management, etc.
\item The most diverse and massive kind of business right now
\end{itemize}
\end{frame}

%%---------------------------------------------------------------

\begin{frame}
\frametitle{Some conclusions}

\begin{itemize}
\item Libre software changes the rules of the game
\item It is important to understand (and get advantage) of those rules
\item Still learning effects and mechanisms
\item Many opportunities to discover new effects, and take advantage of them
\end{itemize}
\end{frame}

%%---------------------------------------------------------------

\begin{frame}
\frametitle{To probe further}

\begin{itemize}
\item ``SME Guide to Free Software'', by Carlo Daffara \\
  \url{http://guide.flossmetrics.org}
\end{itemize}
\end{frame}
