\documentclass[a4paper]{article}
%\usepackage[T1]{fontenc}
\usepackage[utf8]{inputenc}
\usepackage{url}
\usepackage[hypertex,colorlinks]{hyperref}
%\usepackage{html}
%\usepackage{hthtml}
\usepackage{geometry}

% Comments (optional argument is author of comment)
\newcommand{\comments}[2][?]{
  \begin{quote}
    \textbf{Comment (#1):} {\em #2}
  \end{quote}
  }

% \name is for ``special names'', like procedure or variable names.
\newcommand{\name}[1]{\texttt{\hturl{#1}}}

% Just a shortcut for links where the url appears as footnote.
\newcommand{\htflink}[2]{\htmladdnormallinkfoot{#1}{#2}}

\title{MSWL Introduction \\
Master on libre software \\
(2011-2012) \\
URJC - GSyC/Libresoft \\
\url{http://master.libresoft.es}}

\author{Jesus M. Gonzalez-Barahona, Gregorio Robles}
\date{September 2011}

\sloppy
\begin{document}
\maketitle

\begin{abstract}
Course syllabus and learning program for the course ``Introduction to libre software'', of the Master on libre software of the Universidad Rey Juan Carlos (Móstoles, Spain).

[This is an evolving document, until the course is finished and graded]
\end{abstract}

\tableofcontents

%%----------------------------------------------------------------
%%----------------------------------------------------------------
%%----------------------------------------------------------------
\section{Administrivia}

\begin{tabular}{ll}
\textbf{Credits:} 3 ECTS \\
\textbf{Schedule:} \\
& 8th September, Thu, 17:00-21:00 \\
& 15th September, Thu, 17:00-21:00 \\
& 22th September, Thu, 17:00-21:00 \\
& 29th September, Thu, 17:00-21:00 \\
& 6th October, Thu, 17:00-21:00 \\
\textbf{Lecturers:} \\
& Jesús M. González-Barahona \\
& \hspace{1cm} email: jgb @ gsyc.es \\
& \hspace{1cm} Twitter: @jgbarah, \url{http://twitter.com/jgbarah} \\
& \hspace{1cm} Identi.ca: @jgbarah, \url{http://identi.ca/jgbarah} \\
& \hspace{1cm} Diaspora: jgbarah@diasp.org \\
& \hspace{1cm} Room 003, Biblioteca (Library), campus at Fuenlabrada \\
& Gregorio Robles \\
& \hspace{1cm} email: grex @ gsyc.es \\
& \hspace{1cm} Room 110, Departamental II, campus at Fuenlabrada \\
\end{tabular}{l}

%%----------------------------------------------------------------
%%----------------------------------------------------------------
%%----------------------------------------------------------------
\section{Course topics and schedule}

%%----------------------------------------------------------------
%%----------------------------------------------------------------
\subsection{00 - Presentation of the course}

Presentation of the main aspects of the course, and specially those related to administrative issues, evaluation, etc.

%%----------------------------------------------------------------
\subsubsection{September 8th, 2011 (1 hour)}

\begin{itemize}
\item Lecturers: Jesus, Gregorio
\item Activity: Presentation by lecturers, questions and answers \\
  The subject, its schedule, its syllabus, its website, and other details are presented. Since this is the first presentation of a subject in the master's program, some generalities are also presented about how most subjects in the program work.
\item Material: Slides (``Presentation'')
\end{itemize}

%%----------------------------------------------------------------
%%----------------------------------------------------------------
\subsection{01 - Introduction to libre software}

Introduction to libre (free, open source) software. Motivation, origins, definition.

%%----------------------------------------------------------------
\subsubsection{September 8th, 2011 (1 hour)}

\begin{itemize}
\item Lecturers: Jesus, Gregorio
\item Activity: ``What is free software?'' \\
  Work in groups to agree on a definition of free software at most two sentences long. Present and defend your definition to the class. Discussion.
\item Exercise (to be delivered in the website): ``Your definition of libre software'' (\ref{exe:definition-libresoftware}).
\item Exercise (to be delivered in the website): ``The most prominent impact of libre software'' (\ref{exe:impact-libresoftware}).
\end{itemize}

%%----------------------------------------------------------------
\subsubsection{September 15th, 2011 (4 hours)}

\begin{itemize}
\item Activity: Video and discussion \\
  El software libre (colección ``Las TIC en un CLIC''), por Fundación CTIC \\
  \url{http://www.youtube.com/watch?v=9NrPGDtzv50}
\item Activity: Presentation by lecturers, questions and answers \\
  Introduction to libre software
\item Activity: Discussion \\
  What is the most relevant of the four freedoms, in your opinion?
\item Material: Slides (``Introduction'')
\item Recommended reading: chapter ``Introduction'' of ``Introduction to free software''.
\item Exercise (to be delivered in the website): ``Differences between Debian and OSI definitions'' (\ref{exe:debian-osi-differences}).
\item Exercise (to be delivered in the website): ``GNU Manifesto'' (\ref{exe:gnu-manifesto}).
\end{itemize}

%%----------------------------------------------------------------
\subsubsection{September 29th, 2011 (4 hours)}

\begin{itemize}
\item Activity: Presentation by lecturers, questions and answers \\
  The open source definition
\item Material: Slides (``Introduction'')
\item Recommended reading: ``Open Source Definition''
\end{itemize}

%%----------------------------------------------------------------
%%----------------------------------------------------------------
\subsection{02 - History of libre software}

History of libre software.

%%----------------------------------------------------------------
\subsubsection{September 22nd, 2011 (4 hours)}

\begin{itemize}
\item Activity: Presentation by lecturers, questions and answers \\
  History of libre software
\item Material: Slides (``History of libre software'')
\item Complementary material: Video (``Free software'', by Richard Stallman) \\
  \url{http://blip.tv/thalskarth/richard-m-stallman-en-el-flisol-09-2424331} \\
  (in Spanish, other similar talks in English are available in the Internet)
\item Complementary material: Article ``Tanenbaum - Torvalds debate'' in Wikipedia \\
  \url{http://en.wikipedia.org/wiki/Tanenbaum-Torvalds_debate}
\item Complementary material: Article ``Open Letter to Hobbyists, by Bill Gates'' in Wikipedia \\
\url{http://en.wikipedia.org/wiki/Open_Letter_to_Hobbyists}
\item Complementary material: Initial GNU Announcement \\
  \url{http://www.gnu.org/gnu/initial-announcement.html}
\item Complementary material: ``Red Hat shares triple in IPO'', cnet News \\
  \url{http://news.cnet.com/2100-1001-229679.html}
\item Exercise (to be delivered in the website): ``Events important in the history of libre software'' (\ref{exe:events-history}).
\item Exercise (to be delivered in the website): ``Netscape announcement'' (\ref{exe:netscape-announcement}).
\item Exercise (to be delivered in the website): ``Free software in ten years from now'' (\ref{exe:10-years}).
\end{itemize}

%%----------------------------------------------------------------
%%----------------------------------------------------------------
%\subsection{03 - Consequences of the model}

%Consequences of the definition of libre software, its development and its business models.

% Possible: free software in several areas (mobile, embedded, etc.)
% Some prominent projects, applications, etc. (GNU, Android, Webkit, Mozilla, ...)
% Some names of actors in free software


%%----------------------------------------------------------------
%\subsubsection{September 29th, 2011 (4 hours)}


%%----------------------------------------------------------------
%%----------------------------------------------------------------
%\subsection{04 - Libre software and public administrations}

%Specific situation of libre software in public administrations

%%----------------------------------------------------------------
%\subsubsection{September 29th, 2011 (3 hours)}

%%----------------------------------------------------------------
%%----------------------------------------------------------------
\subsection{03 - Free cultural works}

Other intellectual goods from the `libre' perspective.

% Wrap-up activities?

%%----------------------------------------------------------------
\subsubsection{October 6th, 2011 (4 hours)}


\section{Grading}

This section details the criteria for grading the course, the deadlines for the different activities, and the submission details for th activities that require them.

\subsection{Evaluation criteria}
\label{sub:evaluation-criteria}

Each activity contributing to the grading of the course has its own evaluation criteria, as described below. Each of these activities has a minimum and maximum grading. If the minimum grading is 0, the activity is optional. Otherwise, the activity is mandatory, and has to be graded al least with the minimum to pass the course. Each activity has also a description, and when possible, some general grading criteria. In any case, the final grade for the course will also depend on the continuous observation of the lecturers on the outcomes and progress of students.

Students should ask lecturers about any detail which may not be clear to them, either about the general grading plan, or about specific aspects of the activities. As a general rule, evaluation will have into account how the activity and its results show that the student has come close to the competences, knowledge and skills expected for the course.

The student can consider that the next table will be used as a (minimum) guideline for assigning marks:

\begin{itemize}
\item Pass (``aprobado''): 150
\item Good (``notable''): 250
\item Excellent (``sobresaliente): 350
\end{itemize}

\begin{itemize}
\item \textbf{Exercises (answered in forum)}. \\
  Minimum: 20 points, maximum: 80 points.

  Exercises proposed and answered in the the forum of the course.

\item \textbf{Blog entries}. \\
  Minimum: 20 points, maximum: 80 points

  Blog entries specifically related to the course, and marked as such. The tag used for that is mswl-intro.

\item \textbf{Collaborative notebook}. \\
  Minimum: 0 points, maximum: 40 points

  Based on work in class (in real time) and afterwards (complementing the work, using git).

\item \textbf{Video presentation}. \\
  Minimum: 20 points, maximum: 80 points

  Can be an screencast, or a more ellaborate kind of video. It has to explain some topic covered by the course. The most focused, the better: try to explain only one issue, but explain it as well as possible. The video wil be uploaded to some video web hosting site which allows for download of the whole video file, not only streaming (eg, blip.tv).

\item \textbf{Specific report}. \\
  Minimum: 20 points, maximum: 120 points

  Specific report about a relevant aspect of libre software, related to the topics dealt with in this course. Can be a traditional written report, but can also be a presentation (recorded in video, in this case), a video, a podcast, etc. It is important to detail all the references, and to heavily root the report on data and/or specific works publicly available.

\item \textbf{Other activities}. \\
  Minimum: 0 points, maximum: 100 points

  These activities have to be agreed with the lecturers.
\end{itemize}

\subsection{Submission deadlines}

All activities to be graded in January must be completed and submitted by December 1st 2011.

\subsection{Submission details}

Please, consider the details below for submitting the different activities for evaluation (for those not specified in this list, nothing special is needed for submission).

[These details could change before the actual submission dates, please check for updates before submitting.]

\begin{itemize}
\item As a summary of all the activities, a ``Summary of activities for evaluation'' should be sent. This summary should be uploaded to the corresponding resource in the Moodle site for this course, and should include the following data:
  \begin{itemize}
  \item \textbf{Name:} Full name of the student (as ``family name'', ``given name'')
  \item \textbf{Blog entries:} Url of the blog entries for this course (HTML, not RSS version).
  \item \textbf{Contributions to the collaborative notebook:} Id for commits to the repository where the collaborative notebook is hosted, and summary of the main contributions to it related to this course, including links to the repository and commit ids if appropriate.
  \item \textbf{Video presentation:} Title and url where the video can be seen/downloaded (remember that the video should be uploaded to some video hosting site).
  \item \textbf{List of other activities:} If any, list of other activities submitted for evaluation (those that would fit in the ``other activities'' item in the ``Evaluation criteria'' (subsection~\ref{sub:evaluation-criteria}). The results of those activities should be uploaded to the ``other activities'' resource in the Moodle site for this course, when appropriate. In some specific cases (such as streaming videos) it will be enough to include in this list the url to the external site where the result is hosted.
  \end{itemize}
\end{itemize}

%%----------------------------------------------------------------
%%----------------------------------------------------------------
%%----------------------------------------------------------------
\section{Exercises}

%%----------------------------------------------------------------
%%----------------------------------------------------------------
\subsection{01 - Introduction to libre software}


%%----------------------------------------------------------------
\subsubsection{Your definition of libre software}
\label{exe:definition-libresoftware}

Write in one or two sentences your definition of libre (free, open source) software. Try to be personal: we're not asking for and academic, neither a mainstream definitio. We ask for a definition which captures your view about what is libre software (we'll have time to talk about the most common definitions later in this subject). 

%%----------------------------------------------------------------
\subsubsection{The most prominent impact of libre software}
\label{exe:impact-libresoftware}

Which one is the most prominent impact of libre software? (obviously enough, from your point of view). The impact may be of any kind (economic, legal, social, technological, etc.), and be produced by any combination of factors, as long as they are clearly related to libre software (free software projects, companies, developers, organizations, etc.).

%%----------------------------------------------------------------
\subsubsection{Differences between Debian and OSI definitions}
\label{exe:debian-osi-differences}

Compare the Debian Free Software Definition with the OSI Open Source Definition. Do they have any difference? If so, which ones? In case there are differences, do you think they are meaningful, and have important practical consequences or not? Could you track back the OSI and Debian definitions to find out when those differences, if any, were introduced? (as a side-exercise, determine the date in which both definitions where published, and their primary authors).

%%----------------------------------------------------------------
\subsubsection{GNU Manifesto}
\label{exe:gnu-manifesto}

The GNU Manifesto is one of the first documents stating the goals of the free software movement. Can you determine when it was published, its context, and its immediate consequences? When did the GNU project started to produce usable programs?

%%----------------------------------------------------------------
\subsubsection{OSI and FSF definitions}
\label{exe:osi-fsf-definitions}

Compare the OSI and the FSF defintions. Try to map concepts (and even paragraphs) from one to the other. Explain their main differences, and also the key points in which they match.

%%----------------------------------------------------------------
\subsubsection{The Firefox and Debian case}
\label{exe:firefox-debian-case}

The Firefox web browser is one of the packages in the Debian main distribution (which only includes software conformant to the Debian Free Software Guidelines, DFSG). However, although the source code of the package is clearly a slight modification from the orginal, Mozilla's Firefox, it is named Iceweasel. Why is Debian not using the name "Firefox"? Has Mozilla given Debian permission to use that name? Can anyone use that name? Please, comment and document all of this also in the context of the DFSG and the Open Source Definition.

%%----------------------------------------------------------------
\subsubsection{Code obfuscation}
\label{exe:code-obfuscation}

One of the clauses of the Open Source Definition comments on "code obfuscation". Which one? Could you provide a definition for "code obfuscation"? Can you point to any free software tool that can do code obfuscation (name and reference url, if possible)? 

%%----------------------------------------------------------------
\subsubsection{OSI and FSF recognized licenses}
\label{exe:osi-fsf-licenses}

OSI maintains a list of software licenses recognized as conformat with their definition of ``open source software''. FSF maintains a list of software licenses recognized as conformant with their definition of ``free software''. Please, analyze both list, and focus in the differences. Try to know whether they correspond with licenses that were not analyzed by one of the two bodies, or if they have been rejected as not conformant with their definition. Explain to which extent these differences are important or not (in terms of how similar the interpretation of both bodies is of ``open source software'' and ``free software''). 

%%----------------------------------------------------------------
\subsubsection{Provisions 5-10 of the Open Source Definition}
\label{exe:}

Provisions 5-10 of the Open Source Definition are clauses that try to clarify "corners" of the definition, delimiting its borders, and leaving outside the open source software realm some licenses or software that could be "borderline". In some cases, these provisions have been included for practical reasons, in others to avoid the circumvention of the spirit of the definition, in others to clarify some details.

In this exercise you must describe one borderline case that at least one of those provisions would clarify, explaining why it is inside or outside the realms of the Open Source Definition, and the rationale for it. Discuss also what kind of reasons have lead to that provision. If possible, do it with a real case, but hypothetical cases are also allowed.


%%----------------------------------------------------------------
%%----------------------------------------------------------------
\subsection{02 - History of libre software}


%%----------------------------------------------------------------
\subsubsection{Events important in the history of libre software}
\label{exe:events-history}

In the textbook for this course, you can find a list of events important in the history of libre software. Can you mention some others? (if possible, mention one before January 1st 2006, and one after that date). For each event, try to date it as accurately as possible, describe it, and explain why do you think that it is important for the history of libre software.

%%----------------------------------------------------------------
\subsubsection{Historical context of the OSI definition}
\label{exe:context-osi-definition}

The OSI Open Source Definition was produced in a very particular context. Try to find out why the term "open source" was coined, who first proposed it, who agreed with it and who didn't. Also, who wrote the Definition, and who decided to found the OSI, and composed its first board. Try to put this in the technological and business context of those times. 

%%----------------------------------------------------------------
\subsubsection{Netscape announcement}
\label{exe:netscape-announcement}

In 1998, Netscape announced that the next release of it web navigator was to be free software. Please, read the announcement, and comment on it. In particular, explain what happened to the software after the announcement, until the first releases of Firefox, and to which extent the plans by Netscape, as stated in that announcement, became real.

Reference: The Netscape Announcement \\
\url{http://blog.lizardwrangler.com/2008/01/22/january-22-1998-the-beginning-of-mozilla/}

%%----------------------------------------------------------------
\subsubsection{Free software in ten years from now}
\label{exe:10-years}

How do you imagine free software in year 2021? In particular, do you think it will be a mainstream model for producing software? Or maybe it will be reduced to some specific niches? (which ones?) Or maybe it will be reduced to irrelevant? And in any case, why?

As a side question, for the main IT device in use by that year, how much free software will be delivered in it, typically? (Just say a percentage)


%%----------------------------------------------------------------
%%----------------------------------------------------------------
\subsection{03 - Consequences of the model}


%%----------------------------------------------------------------
%%----------------------------------------------------------------
\subsection{04 - Libre software and public administrations}


%%----------------------------------------------------------------
%%----------------------------------------------------------------
\subsection{05 - Free cultural works}

%%----------------------------------------------------------------
\subsubsection{Analysis of the definition of Free Cultural Works}
\label{exe:defintion-free-cultural-works}

The definition of Free Cultural Works is obviously based on the FSF definition of free software. It also incorporates some aspects related to the OSI definition of open source software. Please, produce a comparative analysis of the definition of free cultural works and these two definitios of free software and open source software. In addition to more general issues, be sure of answering the following questions:

\begin{itemize}
\item According to these definitions, can all free software and open source software be considered as a free cultural work?
\item In addition to the basic freedoms, some other items are taken into account when defining a free cultural work. Why do you think this has been considered in the case of cultural works, but not in the case of free/open source software? Or maybe it is, but in some other way?
\item Which Creative Commons licenses could be considered as suitable for free cultural works?
\item Why is the Creative Commons ``nc'' (non-commercial) clause considered outside the free cultural works realm, according to this definition?
\end{itemize}

\end{document}
