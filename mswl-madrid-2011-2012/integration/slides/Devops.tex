\begin{frame}
  \frametitle{Contexto}
  \begin{itemize}
    \item Los departamentos de sistemas y desarrollo trabajan aislados
% Los desarrolladores dan el código y los requisitos pero cuando algo falla dicen "en mi local funciona"
% Los desarrolladores nos piden instalar algo no paquetizado y los sysadmins dicen "no!"
    \item Cada uno de los departamentos considera que hace los correcto para el negocio
% Seguro que, analizado de forma aislada, ambos tienen razón
    \item Los desarrolladores no tienen ``conciencia'' de sistemas
% No saben del impacto en los sistemas; ellos tienen herramientas locales para el desarrollo
    \item Ambos departamentos son imprescindibles para el negocio
% Los desarrolladores implementan los requisitos funcionales
% Los sysadmins implementan seguridad, estabilidad y rendimiento
% Están condenados a entenderse porque son objetivos que entran en conflicto
    \item Aparecen las metodologías ágiles de desarrollo: muchos cambios muy pequeños
% Los cambios son los que generan problemas de seguridad, rendimiento y estabilidad
% Los sysadmins a menudo intentan no hacer cambios y los desarrolladores necesitan muchos cambios
  \end{itemize}
\end{frame}

\begin{frame}
  \frametitle{Necesidades}
  \begin{itemize}
    \item Romper las barreras entre los departamentos de sistemas y desarrollo
    \item Implantar metodologías ágiles en sistemas
    \item Usar frameworks de automatización: puppet, chef, cfengine...
    \item Mecanismos de comunicación efectivos entre ambos departamentos
    \item Sistemas ha de estar involucrado en el diseño de la aplicación desde el principio
  \end{itemize}
  \begin{itmemize}
    \item El objetivo final es permitir al negocio que reaccione tan rápido, eficiente y fiable como marca el mercado
  \end{itemize}
\end{frame}

\begin{frame}
  \frametitle{Qué es DevOps}
  \begin{center}
    Filosofía apoyada en procedimientos, herramientas y métodos para permitir una colaboración efectiva entre sysadmins y developers que posibiliten eficientemente el objetivo del negocio.
  \end{center}
  \begin{itemize}
    \item No es un perfil de empleado
% Nadie dice voy a contratar a un Scrum o ITIL, sino un desarrollador que comparta esas metodologías
    \item No es una nueva forma de llamar a los sysadmins
    \item No se trata de superponer ni menospreciar puestos; al contrario
% No es cosa de los desarrolladores para librarse de los sysadmins
% No es de los sysadmins para librarse de los desarrolladoras
% Son las dos cosas a la vez
    \item No es un problema tecnológico, es un problema del negocio
  \end{itemize}
\end{frame}
