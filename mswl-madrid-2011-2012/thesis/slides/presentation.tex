%% presentation.tex
%%
%% Presentation of the course ``Master Thesis" of the Official Master on Libre Software (URJC)
%% http://master.libresoft.es
%%

%%---------------------------------------------------------------------
%%---------------------------------------------------------------------

\section{Presentation of the Master Thesis Course}

%%---------------------------------------------------------------

\begin{frame}
\frametitle{Administrative data}

\begin{itemize}
\item Both semesters, 12 ECTS credits
\item Teachers:
  \begin{itemize}
  \item Gregorio Robles (grex at gsyc.urjc.es)
  \item Departamento de Sistemas Telem�ticos y Computaci�n (GSyC)
  \item Room 110 Departamental III (Fuenlabrada campus)
  \end{itemize}
\item Moodle course (please, join it as soon as possible): \\
  \url{http://docencia.etsit.urjc.es/moodle/course/view.php?id=134}
\end{itemize}
\end{frame}

%%---------------------------------------------------------------

\begin{frame}
\frametitle{Goals of the Master Thesis}

\begin{center}
Primary Goal of the Master Thesis

\vspace{0.25cm}

{\LARGE {\bf To apply the lessons and practices learned in this master
to a real problem}}

\vspace{1cm}

Secondary goal

\vspace{0.25cm}

{\large {\bf To do it in one academic year}}
\end{center}

\end{frame}

%%---------------------------------------------------------------

%%---------------------------------------------------------------

\begin{frame}
\frametitle{Recommended paths}

Case study

\begin{itemize}
    \item Specific software sector
    \item Large project
    \item Ecosystem
\end{itemize}

from several points of view.

\vspace{0.25cm}

Development

\begin{itemize}
    \item (Small) development of software program in a community-like fashion
\end{itemize}

\vspace{0.25cm}

We are open to ways to link master thesis and \emph{Practicum}.

\end{frame}

%%---------------------------------------------------------------

\begin{frame}
\frametitle{Points of view}

Community and goals to be presented by student and approved by mentor. LaTeX template will be provided.

The case study should have several points of view:

\begin{itemize}
  \item History
  \item Philosophy
  \item Main stakeholders and important people
  \item Legal aspects
  \item Organization and community
  \item Technical, process, etc.
  \item Business models
  \item Interviews
\end{itemize}

\end{frame}


%%---------------------------------------------------------------

\begin{frame}
\frametitle{Example I: LMS and libre software}

LMS: learning management systems

\begin{itemize}
  \item History
  \item (Philosophy)
  \item Main stakeholders and important people
  \item Legal aspects
  \item Organization and community
  \item Study of solutions
  \item Technical, process, etc.
  \item Business models
  \item Interviews
  \item Use of LMS
  \item Literature
\end{itemize}

\end{frame}


%%---------------------------------------------------------------

\begin{frame}
\frametitle{Example II: The MySQL Ecosystem}

MySQL is far more than a software product...

\begin{itemize}
  \item History
  \item (Philosophy)
  \item Main stakeholders and important people
  \item Legal aspects
  \item Organization and community
  \item Companies
  \item Technical, process, etc.
  \item Business models
  \item Interviews
  \item Literature
\end{itemize}

\end{frame}

%%---------------------------------------------------------------

\begin{frame}
\frametitle{Development}

\begin{itemize}
\item Master thesis in the ``traditional'' way
\item Software and objectives to be presented by student and approved by mentor. LaTeX template will be provided.
\item Mind the dates! Creating software is more time-consuming than one usually thinks
\item May have some of the previous parts from the recommended path
\item It is not only about programming! There should be at least efforts
to gain a community around the software
\item Integration in an already existing project is possible
\item Interactions with the community have to be documented
\end{itemize}

\end{frame}

%%---------------------------------------------------------------

\begin{frame}
\frametitle{Personal path}

\begin{itemize}
  \item We are open to other ideas
  \item Mind the dates!
  \item May have some of the previous parts from the recommended path
  \item Could be research on some topic
  \item Has to be presented by the student and approved by mentor. LaTeX template will be provided.
\end{itemize}

\end{frame}

%%---------------------------------------------------------------

\begin{frame}
\frametitle{Evaluation}

\begin{itemize}
\item The output of the master thesis will be:
  \begin{itemize}
     \item A report (mandatory): in \LaTeX, with a free license
     \item A software (optional): with a free license, a repository, some website and other community-development tools
     \item A video presentation (mandatory): up to 6 minutes, with a free license
  \end{itemize}
\item All this will be evaluated by a committee (lecturers and external expert)
\item All materials will be published and publicized in the master website
\end{itemize}

\end{frame}

%%---------------------------------------------------------------

\begin{frame}
\frametitle{Looking for excellence!}

The master thesis supposes the best opportunity for master students
to show their acquired knowledge and abilities during the master

\begin{itemize}
\item Award to the best master thesis of the year (to be given in the graduation gala)
\item If it is software, to be presented look at the ``Concurso Universitario de Software Libre''
\item Submission to scientific workshops and conferences is encouraged (and well seen)
\end{itemize}

\end{frame}

%%---------------------------------------------------------------

\begin{frame}
\frametitle{Calendar}

\begin{itemize}
\item 14444.10: Today
\item 01.11: Deadline for proposals
\item 15.11: Deadline for assignments
\item June 2012: Defense
   \begin{itemize}
   \item Important: A student has to have all other master subjects passed in order to defense his thesis!
    \end{itemize}
\item Other dates for the defense: November 2012, March 2013
\end{itemize}

\end{frame}

%%---------------------------------------------------------------

\begin{frame}
\frametitle{Some references}

\begin{itemize}
\item Master Thesis Moodle Course \\
  \url{http://docencia.etsit.urjc.es/moodle/course/view.php?id=134}
\item Desarrollo de Proyectos de Software Libre (chapter 5) \\
  \url{http://ocw.uoc.edu/informatica-tecnologia-y-multimedia/desarrollo-de-proyectos-de-software-libre/Course_listing}
\item Introduction to libre software (book) \\
  \url{http://curso-sobre.berlios.de/introsobre}
\item Concurso Universitario de Software Libre \\
  \url{http://www.concursosoftwarelibre.org/}
\end{itemize}

\end{frame}
