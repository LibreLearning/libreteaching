\documentclass[10pt]{article}

% Puedes entregar la plantilla en inglés, si así lo deseas.
\usepackage[spanish]{babel}
\usepackage[utf8]{inputenx}

\begin{document}

\title{Título tentativo Proyecto Fin de Máster}
\author{Tu nombre aquí}
\date{\today}


\maketitle

\begin{abstract}
Resumen de hasta 150 palabras del trabajo fin de máster que deseas realizar.
\end{abstract}

\section{Objetivos}

El objetivo principal de este proyecto es:

\begin{center}
\bf{Objetivo del proyecto fin de máster en una frase. P.ej.: Estudio completo del ecosistema MySQL}
\end{center}

Este objetivo se puede dividir en los siguientes subobjetivos:

\begin{enumerate}
  \item Subobjetivo 1. P.ej. identificar los productos en el ecosistema MySQL
  \item Subobjetivo 2. P.ej. identificar los modelos de negocio de las empresas con MySQL
  \item Subobjetivo 3
  \item ...
\end{enumerate}


\section{Motivación}

Indica por qué quieres realizar este trabajo fin de máster (desde un punto de vista personal). ¿Por qué lo has elegido? ¿Qué quieres aprender? etc.

\section{Tecnologías involucradas}

Especifica las tecnologías involucradas en este proyecto (y tu grado de conocimiento de las mismas.

\begin{itemize}
  \item Tecnología 1. P.ej.: Java. Tengo alguna experiencia
  \item Tecnología 2. P.ej.: API X. No la conozco
  \item ...
\end{itemize}


\section{Planificación tentativa}

Incluye una planificacitón temporal tentativa. 

\begin{enumerate}
  \item Paso 1: Fecha
  \item Paso 2: Fecha
  \item Paso 3: Fecha
  \item Paso 4: Fecha
  \item ...
  \item Entrega memoria: Fecha
\end{enumerate}

\section{Otros}

Espacio para indicar otras cuestiones que consideres de interés, como por ejemplo, si te gustaría tener otro tutor, necesitas material especial, etc.

\end{document}
