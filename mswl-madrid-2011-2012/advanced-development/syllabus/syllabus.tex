\documentclass[a4paper]{article}
%\usepackage[T1]{fontenc}
\usepackage[utf8]{inputenc}
\usepackage{url}
\usepackage[hypertex,colorlinks]{hyperref}
%\usepackage{html}
%\usepackage{hthtml}
\usepackage{geometry}

% Comments (optional argument is author of comment)
\newcommand{\comments}[2][?]{
  \begin{quote}
    \textbf{Comment (#1):} {\em #2}
  \end{quote}
  }

% \name is for ``special names'', like procedure or variable names.
\newcommand{\name}[1]{\texttt{\hturl{#1}}}

% Just a shortcut for links where the url appears as footnote.
\newcommand{\htflink}[2]{\htmladdnormallinkfoot{#1}{#2}}

\title{MSWL Advanced Development \\
Master on libre software \\
URJC - GSyC/Libresoft \\
\url{http://master.libresoft.es}}

\author{Roberto Calvo}
\date{February 2012}

\sloppy
\begin{document}
\maketitle

\begin{abstract}
Course syllabus and learning program for the course ``Advanced
Development'', of the Master on libre software of the Universidad Rey
Juan Carlos.


[This is an evolving document, until the course is finished and graded]
\end{abstract}

\tableofcontents

%%----------------------------------------------------------------
%%----------------------------------------------------------------
%%----------------------------------------------------------------
\section{Course topics and schedule}

%%----------------------------------------------------------------
%%----------------------------------------------------------------
\subsection{00 - Presentation of the course}

Presentation of the main aspects of the course, and specially those
related to administrative issues, evaluation, etc.

%%----------------------------------------------------------------
\subsubsection{February 17, 2012 (0.5 hours)}


\subsection{01 - Android Beginning}

Add Description

\subsubsection{February 17, 2012 (1 hours)}

\begin{itemize}
\item \textbf{Lecturers:} Roberto Calvo
\item \textbf{Presentation:} ``The Android Project''
  \begin{itemize}
  \item \textbf{Discussion:} The history and evolution of the Android
    project
  \item \textbf{Requirements:} 
  \item \textbf{Supporting material:} Slides ``Android Project''
  \end{itemize}
\item \textbf{Assignment (results in website):}
  \begin{itemize}
  \item Any kind of activity
    related to this talk (blog post, video, etc)
  \end{itemize}
\end{itemize}


%%----------------------------------------------------------------                                                                                                                     
\subsubsection{February 17, 2012 (0.5 hours)}

\begin{itemize}
\item \textbf{Lecturers:} Roberto Calvo
\item \textbf{Presentation:} ``Install Android SDK \& Eclipse''
  \begin{itemize}
  \item \textbf{Discussion:} How install the Android SDK in GNU/Linux
    systems.
  \item \textbf{Requirements:} It's necessary get all the files
    already download in the laptops.
  \item \textbf{Supporting material:}
  \end{itemize}
\item \textbf{Assignment (results in website):}
  \begin{itemize}
  \item Any kind of activity
    related to this talk (blog post, video, etc)
  \end{itemize}
\end{itemize}

%%----------------------------------------------------------------

\subsubsection{February 17, 2012 (2 hours)}

\begin{itemize}
\item \textbf{Lecturers:} Roberto Calvo
\item \textbf{Presentation:} ``Hello World and Layouts''
  \begin{itemize}
  \item \textbf{Discussion:} First steps with Android environment.
  \item \textbf{Requirements:} 
  \item \textbf{Supporting material:}
  \end{itemize}
\item \textbf{Assignment (results in website):}
  \begin{itemize}
  \item Any kind of activity
    related to this talk (blog post, video, etc)
  \end{itemize}
\end{itemize}


\subsection{02 - Activity, ListActivity and Layouts}

Python is a programming language, very popular in libre software
community. We will see the main concepts of the language and will
create a small program, to be stored in the Git repository.

%%----------------------------------------------------------------
\subsubsection{February 24, 2012 (5 hours)}

\begin{itemize}
\item \textbf{Lecturers:} Roberto Calvo
\item \textbf{Presentation:} ``''
  \begin{itemize}
  \item \textbf{Discussion:} Python vs. other programming
    languages. What are the advantages and drawbacks of dynamic
    typing?
  \item \textbf{Requirements:} Running Python 2.x development
    environment and Git
  \item \textbf{Supporting material:} Slides ``Introduction to Python''
  \end{itemize}
\item \textbf{Assignment (results in website):} 
  \begin{itemize}
  \item Creation of a small program which reads its command line
    arguments.
  \item Upload of the program to the Git repository
  \item Any kind of activity
    related to this talk (blog post, video, etc)
  \end{itemize}
\end{itemize}


\subsection{03 - Maps and GeoLocation}

We will some advanced Python coding techniques and libraries, to
develop a web crawler. The web crawler  will track updates of a web
page.

%%----------------------------------------------------------------
\subsubsection{March 2, 2011 (4 hours)}

\begin{itemize}
\item \textbf{Lecturers:} Israel Herraiz
\item \textbf{Presentation:} ``Developing a web crawler in Python''
  \begin{itemize}
  \item \textbf{Requirements:} Running Python 2.X installation and Git
  \item \textbf{Supporting material:} Slides ``Developing a web
    crawler in Python'', sample source code
  \end{itemize}
\item \textbf{Assignment (results in website):} 
  \begin{itemize}
  \item First functional iteration of the web crawler
  \item Upload of the program to the Git repository
  \item Any kind of activity
    related to this talk (blog post, video, etc)
  \end{itemize}
\end{itemize}


\subsection{04 - Working in background }

How to use the android services to work in background. Also we will
see the correct way to use AsynTask structure to sync ActivityList.

%%----------------------------------------------------------------
\subsubsection{March 9, 2011 (4 hours)}

\begin{itemize}
\item \textbf{Lecturers:} Micael Gallego
\item \textbf{Presentation:} ``Introduction to Eclipse''
  \begin{itemize}
\item \textbf{Requirements:} Running Python 2.x development
  environment and Eclipse
  \item \textbf{Supporting material:} Slides ``Introduction to Eclipse''
  \end{itemize}
\item \textbf{Assignment (results in website):} Eclipse project for
  the web crawler application
\end{itemize}

\subsection{05 - Preferences, Dialogs, Menus, Sign Application}

How to use preference sdk to save and load data of the
application. Dialogs and Menus are two important pieces of the
application. Also we study the process to sign applications.

%%----------------------------------------------------------------
\subsubsection{March 16, 2011 (4 hours)}

\begin{itemize}
\item \textbf{Lecturers:} Roberto Calvo
\item \textbf{Presentation:} ``Introduction to GNOME''
  \begin{itemize}
  \item \textbf{Requirements:} GCC and Autotools
  \item \textbf{Supporting material:} Slides ``Introduction to
    GNOME'', sample source code 
  \end{itemize}
\item \textbf{Assignment (results in website):} 
  \begin{itemize}
  \item GNOME desktop application (optional assignment)
  \end{itemize}
\end{itemize}


\section{Grading}

This section details the criteria for grading the course, the
deadlines for the different activities, and the submission details for
the activities that require them.

\subsection{Evaluation criteria}
\label{sub:evaluation-criteria}

Each activity contributing to the grading of the course has its own
evaluation criteria, as described below. Each of these activities has
a minimum and maximum grading. If the minimum grading is 0, the
activity is optional. Otherwise, the activity is mandatory, and has to
be graded al least with the minimum to pass the course. Each activity
has also a description, and when possible, some general grading
criteria. In any case, the final grade for the course will also depend
on the continuous observation of the instructors on the outcomes and
progress of students.

Students should ask instructors about any detail which may not be
clear to them, either about the general grading plan, or about
specific aspects of the activities. As a general rule, evaluation will
have into account how the activity and its results show that the
student has come close to the competences, knowledge and skills
expected for the course.

The student can consider that the next table will be used as a
(minimum) guideline for assigning marks:

\begin{itemize}
\item Pass (``aprobado''): 150
\item Good (``notable''): 250
\item Excellent (``sobresaliente''): 350
\end{itemize}

\begin{itemize}
\item \textbf{Blog entries}. \\
  Minimum: 0 points, maximum: 50 points

  Blog entries specifically related to the course, and marked as
  such. 10 points for each post. The tag used for that is mswl-ad.
  You should notify the post and url in a subject forum.

\item \textbf{Android weekly exercises }. \\
  Minimum: 80 points, maximum: 250 points

There is a programming exercise per week about Android. The maximum
value of each exercise is 50 points. Usually, the exercise must be
submitted before the next session. If you decide submit the exercise
after next session, the maximum value of exercise is 25 points.

All exercises are mandatory and the hard deadline is one week after the
last session (March 23, 2012).


\item \textbf{Android application (optional)}. \\
  Minimum: 0 points, maximum: 100 points

Development of a Android application: RSS reader 

\end{itemize}

\subsection{Submission deadlines}

All activities to be graded in May must be completed and submitted by
March 31th 2011. If you decide develop an Android application the
deadline is May ?

\subsection{Submission details}

Please, consider the details below for submitting the different
activities for evaluation (for those not specified in this list,
nothing special is needed for submission).


\begin{itemize}
\item As a summary of all the activities, a ``Summary of activities
  for evaluation'' should be sent. This summary should be uploaded to
  the corresponding resource in the Moodle site for this course, and
  should include the following data: 
  \begin{itemize}
  \item \textbf{Name:} Full name of the student (as ``family name'', ``given name'')
  \item \textbf{Blog entries:} Url of the blog entries for this course (HTML, not RSS version).
  \item \textbf{Source code and project files in the Git repository:} Id for commits
    to the repository where the source code is hosted, and summary of
    the main contributions to it related to this course, including
    links to the repository and commit ids if appropriate. 
  \end{itemize}
\end{itemize}

%%----------------------------------------------------------------
%%----------------------------------------------------------------
%%----------------------------------------------------------------
\section{Assignments and activities}

%%----------------------------------------------------------------
%%----------------------------------------------------------------
\subsection{Spider to track the updates of a web page}
\label{sub:python}

You will have to write a Python application that get the current
version of a web page, compare against a local cache of the page, and
if changed, retrieve the new version of the page and write in the
standard output a summary of the changes.

The spider must visit all the links below the current page. The log of
changes displayed in the standard output will contain a list of all
the links that have been changed, and the number of lines of
difference between the two versions.

The application must be easily installable using Python standard
deployment methods, must be properly document and must include a
battery of tests to check that it is working as expected.

All the development will be done using Git version control, and all
the code will be publicly available in a Git repository, with frequent
commits.

\textbf{Supporting material}

\begin{itemize}
\item Slides about advanced Android development
\item Snippets of sample source code
\end{itemize}


\subsection{GNOME desktop application}
\label{sub:gnome}

You will have to develop an android application written in JAVA.

All the development will be done using Git version control, and all
the code will be publicly available in a Git repository, with frequent
commits.

We provide you the necessary libraries for JSON and XML parser.

This assignment is optional.


\textbf{Supporting material}

\begin{itemize}
\item Slides about Android development
\item Sample source code
\end{itemize}


%%----------------------------------------------------------------
%\subsubsection{Statements about economic aspects of libre software}
%\label{sub:statements-eco}

\end{document}
