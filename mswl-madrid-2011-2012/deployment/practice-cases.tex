\documentclass[a4paper]{article}

% Paquetes utilizados
\usepackage[utf8]{inputenc}
\usepackage[spanish]{babel}
\usepackage{url}
% Fueras a que las notas al pie aparezcan abajo
\usepackage[bottom]{footmisc}

% Metadatos
\title{\textbf{Casos de estudio para la práctica final}}
\author{Jose Gato Luis, Jose Castro}
\date{March 2012}
\vspace{0.5cm}
%--- \\
% \vspace{0.5cm}
%\normalsize{URJC - GSyC/LibreSoft} \\

% Inicio del documento
\begin{document}

 \section{Pritek Solutions}
La empresa priteksolutions es líder en el mercado de soluciones informáticas. La empresa cuenta
a
con alrededor de 30 trabajadores, de los cuales 20 son de perfil puramente técnico. Su producto
e
estrella e-Services se trata de un portal electrónico para los ciudadanos. e-Services está desarrollado usando .net con varias tecnologías web y Oracle para el soporte de base de datos. En su momento, un comercial de Oracle les ofreció esta solución al tener como objetivo un proyecto con una muy fuerte carga de datos, ya que debía ser capaz de manejar ágilmente la información de millones de usuarios. Desde que se formó la empresa, el IDE oficial de desarrollo es Microsoft Visual Studio y Microsoft Visual Source Safe como herramienta de desarrollo colaborativa o control de versiones.

Con la entrada en la empresa del gerente de soluciones informáticas Jesus Nahoraba, un nuevo aire fresco entra en la empresa. Este nuevo gerente conoce perfectamente las virtudes del software libre y quiere introducirlo de forma progresiva en la empresa. De manera que las soluciones ofrecidas por pritekSolutions tiendan a usar software libre. De hecho, e-Services está
licenciado de forma privativa y Nahoraba quiere plantear la posibilidad de ofrecerlo con alguna licencia libre, ya sea el producto entero, o alguna de sus partes.


\section{Empresa gestora/oficinas LiberSol}

Una empresa gestora, con oficinas en el centro de madrid, trabaja desde hace muchos años con
las siguientes herramientas:
\begin{itemize}
\item Microsoft Office: Word, PowerPoint, Excell y Outlook
\item Internet Explorer 6
\end{itemize}

Ultimamente han oído hablar de las virtudes del software libre y del ahorro en licencias que podrín conseguir. Desgraciadamente, nadie en LiberSol tiene un conocimiento mínimo como
para poder acometer una migración sin problemas. Por ello, se ponen en contacto con el grupo
Libresoft para que les asesore con los pasos a seguir
 

\section{Design4Future}
Design4Future es una empresa de marketing y diseño con sede en Sevilla. Esta empresa se
encarga de ofrecer distintas soluciones de diseó: logos, diseños web, anuncios comerciales e incluso soporte en películas y cortos de animación. En la actualidad, quieren introducirse en el mundo de la animación en 3D.

Esto les llevaría un fuerte cambio en su modelo de negocio tratándose de una gran apuesta. Desgraciadamente, no poseen nuevas fuentes de ingresos para acometer tal labor, ya que se trata de una apuesta personal. Una forma de abaratar costes para entrar en este mundo, es el ahorro en las carísimas licencias que poseen y el software libre les ha abiertos los ojos. Algunas de las
herramientas que utilizan en la actualidad:

\begin{itemize}
\item Photoshoop
\item 3D Studio
\item Adobe Premiere
\item Adobe Flash
\item Adobe Ilustrator
\item Pro tools
\end{itemize}

Design4Future ha puesto a Jose ManStall, uno de sus trabajadores, a cargo de realizar un
estudio de alternativas libres, que cumplan con todos los requisitos que tienen en la actualidad.
Con un ahorro en licencias suficiente, podrían apostar por el mundo de la animación en 3D.



  \section{El druida auténtico}
  La asociación \textit{El druida auténtico} es una asociación de jugadores de juegos de mesa y juegos de rol de Barcelona.
  Esta asociación cuenta con 5 trabajadores con responsabilidades administrativas.
  
  Afortunadamente, esta asociación cuenta con 1 servidor y 6 ordenadores de sobremesa.
  
  El servidor tiene como sistema operativo un Windows Server ofreciendo varios servicios: web, correo electrónico, FTP,
  compartición de ficheros con Samba y es donde está conectada la impresora compartida.
  
  Los puestos de trabajo tienen instalado como sistema operativo Windows 7, la suite ofimática Micosoft Office, el cliente
  de correo Outlook y el navegador Explorer 7.
  
  La asociación se está planteando contratar un informático para migrar los servicios a software libre.
  
  
  \section{Instituto Francisco de Goya}
  El instituto \textit{Francisco de Goya} está situado en Zamora y actualmente imparte cursos de primaria, secundaria y algunos
  ciclos formativos de formación profesional.
  
  Este instituto cuenta con servidores propios y una aula con 25 equipos de sobremesa.
  
  Para ponerse al día en las nuevas tecnologías en la educación, el equipo directivo apuesta por tener una plataforma de
  \textit{e-learning} que facilite la comunicación entre profesores y alumnos.
  
  También se han planteado tener su página web y el servidor de correo coorporativo del instituto.
  
  Por suerte, los alumnos del ciclo formativo de Informática tuvieron una sesión sobre software libre el mes pasado y se
  han ofrecido voluntarios para hacer el diseño y el despliegue.
  
  
  \section{Libre Hosting}
  La empresa \textit{Libre Hosting} será un empresa que proveerá servicios de hosting. Aún está en fase de creación y
  el equipo técnico está haciendo el diseño de la plataforma.
  
  Cuentan con un parque de 200 servidores de tipo \textit{blade} y 5 cabinas de almacenamiento con fibra óptica.
  
  Es ahora cuando están evaluando la posibilidad de desplegar una infraestructura virtualizada (y quizás con 
  \textit{cloud computing}) para minimizar los costes en licencias y poder adaptar las soluciones a sus necesidades
  específicas.
  
  Para poder ofrecer un servicio de hosting de calidad también se ven en la necesidad de un sistema de monitorización
  de cada una de las máquinas tanto físicas como virtuales.
  
  
 

\end{document}
