\documentclass[a4paper]{article}
%\usepackage[T1]{fontenc}
\usepackage[utf8]{inputenc}
\usepackage{url}
\usepackage[hypertex,colorlinks]{hyperref}
%\usepackage{html}
%\usepackage{hthtml}
\usepackage{geometry}

% Comments (optional argument is author of comment)
\newcommand{\comments}[2][?]{
  \begin{quote}
    \textbf{Comment (#1):} {\em #2}
  \end{quote}
  }

% \name is for ``special names'', like procedure or variable names.
\newcommand{\name}[1]{\texttt{\hturl{#1}}}

% Just a shortcut for links where the url appears as footnote.
\newcommand{\htflink}[2]{\htmladdnormallinkfoot{#1}{#2}}

\title{MSWL Deployment of Libre Software \\
Master on libre software \\
URJC - GSyC/Libresoft \\
\url{http://master.libresoft.es}}

\author{Jose Gato Luis, Jose Francisco Castro}
\date{March 2012}

\sloppy
\begin{document}
\maketitle



\section { Final Practice}

After listen the experience of Eduardo Romero the students will have to make a similar exercise. This work will consist in a "complete" study to present to a company that wants to migrate his infrastructures to a libre software environment. 

The students will work in pairs and they will have to choose between six different scenarios (companies, public administrations, etc). These different scenarios will show the "realty" of an environment needing to change to a better (libre) alternative. 

Each pair of students will write a document to present to the selected company with a complete study of "How to make a Libre Software Deployment and succeed". Each document will contain the next important points:

\begin{itemize}
\item Analysys of the software used in the company and the requirements. Ex: "They use a graphical editor because they need to design logos"
\item Software costs
\item Study of libre software alternatives. For each piece of libre software alternative:
	\begin{itemize}
	\item Software description
	\item Requirements that covers this software
	\item Requirements that does not covers and it should be
	\item Solutions to mitigate this fails
	\end{itemize}
\item New economic plan
\item Training plans 
\item Conclusion

\end{itemize}


%%----------------------------------------------------------------
%\subsubsection{Statements about economic aspects of libre software}
%\label{sub:statements-eco}

\end{document}
