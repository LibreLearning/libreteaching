\documentclass[a4paper]{article}
%\usepackage[T1]{fontenc}
\usepackage[utf8]{inputenc}
\usepackage{url}
\usepackage[hypertex,colorlinks]{hyperref}
%\usepackage{html}
%\usepackage{hthtml}
\usepackage{geometry}

% Comments (optional argument is author of comment)
\newcommand{\comments}[2][?]{
  \begin{quote}
    \textbf{Comment (#1):} {\em #2}
  \end{quote}
  }

% \name is for ``special names'', like procedure or variable names.
\newcommand{\name}[1]{\texttt{\hturl{#1}}}

% Just a shortcut for links where the url appears as footnote.
\newcommand{\htflink}[2]{\htmladdnormallinkfoot{#1}{#2}}

\title{MSWL Case Studies II \\
Master on libre software \\
URJC - GSyC/Libresoft \\
\url{http://master.libresoft.es}}

\author{Pedro García and Santiago Dueñas}
\date{January 2012}

\sloppy
\begin{document}
\maketitle

\begin{abstract}
Course syllabus and learning program for the course ``Case Studies II'', 
of the Master on libre software of the Universidad Rey Juan Carlos (Fuenlabrada, Spain).

[This is an evolving document, until the course is finished and graded]
\end{abstract}

\tableofcontents

\section{Grading plan}

Each activity contributing to the grading of the course has its own evaluation criteria, 
as described below. Each of these activities has a minimum and maximum grading. If the 
minimum grading is 0, the activity is \textit{optional}. Otherwise, the activity is \textit{mandatory}, and 
has to be graded al least with the minimum to pass the course. Each activity has also a 
description, and when possible, some general grading criteria. In any case, the final grade 
for the course will also depend on the continuous observation of the instructors on the 
outcomes and progress of students.

Students should ask instructors about any detail which may not be clear to them, 
either about the general grading plan, or about specific aspects of the activities. 
As a general rule, evaluation will have into account how the activity and its results 
show that the student has come close to the competences, knowledge and skills expected 
for the course.

The student can consider that the next table will be used as a (minimum) guideline for 
assigning marks:

\begin{itemize}
\item Pass (``aprobado''): 90
\item Good (``notable''): 135
\item Excellent (``sobresaliente''): 170
\end{itemize}

\subsection{Graded activities}

\begin{itemize}

\item \textbf{Blog entries}. \\
  Minimum: 30 points, maximum: 70 points

  Blog entries specifically related to each session of the course. The articles 
should summarize the session or describe specific topics covered during the talks.
A minimun of 6 articles is required for passing the course. Only one article 
per talk. 

 Each one of these articles will be posted, at most, 10 days after the 
given talk. If not, the maximum calification for that article will be the half
of those published on time.
  
 The tag used to mark these entries is \textit{mswl-cases}.

\item \textbf{Video presentation}. \\
  Minimum: 10 points, maximum: 90 points 

  The video has to explain a specific topic of any project/community shown during the course. 
It can be an screencast, or a more ellaborate kind of video and will be uploaded 
to some video web hosting site which allows for download of the whole video file, 
not only streaming (eg, blip.tv).
  
  Each student will sent, before \textbf{March 1st} a script describing the topic and contents 
that will be covered on the video.

\item \textbf{Interview}. \\
  Minimum: 0 points, maximum: 80 points
  
  Interview with a relevant figure of any Free/Libre/Open Source Software project or community. 


\end{itemize}

\subsection{Deadline}

The deadline for completing all the activities, described on the previuos section is \textbf{May 2nd, 2011}.


\section{Sessions}

\subsection{Session 1 (Friday. January 13, 2012)}

\begin{itemize}
 \item \textbf{Speakers}:

  \begin{enumerate}
   \item Ludovic Hirlimann.
  \end{enumerate}

 \item \textbf{Content}:

  \begin{enumerate}
   \item Quality Assurance, Thunderbird and Mozilla.
  \end{enumerate}

\end{itemize}

\subsection{Session 2 (Friday. January 20, 2012)}

\begin{itemize}
 \item \textbf{Speakers}: 

  \begin{enumerate}
   \item Carlos García Campos.
   \item Albert Astals Cid.
  \end{enumerate}

 \item \textbf{Content}:

  \begin{enumerate}
   \item GNOME (Carlos García Campos).
   \item KDE (Albert Astals Cid).
  \end{enumerate}

\end{itemize}

\subsection{Session 3 (Friday. January 27, 2012)}

\begin{itemize}
 \item \textbf{Speakers}:

  \begin{enumerate}
   \item Roberto Andradas.
    \item Rodrigo Moya.
  \end{enumerate}

 \item \textbf{Content}:

  \begin{enumerate}
   \item OSOR (Roberto Andradas)
    \item Por determinar (Rodrigo Moya)
  \end{enumerate}

\end{itemize}
\subsection{Session 4 (Friday. February 3, 2012)}

\begin{itemize}
 \item \textbf{Speakers}:

  \begin{enumerate}
   \item Guillermo Lopez Leal
  \end{enumerate}

 \item \textbf{Content}:

  \begin{enumerate}
   \item Mozilla Hispano (Guillermo Lopez Leal) 
  \end{enumerate}

\end{itemize}

\subsection{Session 5 (Friday. February 10, 2012)}

\begin{itemize}
 \item \textbf{Speakers}:

  \begin{enumerate}
    \item Felipe Ortega. 
    \item Miquel Vidal. 
  \end{enumerate}

 \item \textbf{Content}:

  \begin{enumerate}
   \item Wikipedia (Felipe Ortega)
   \item Wikipedia (Miquel Vidal)
  \end{enumerate}

\end{itemize}

\end{document}
