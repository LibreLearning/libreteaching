%%
%% $Id$
%%

\documentclass[a4paper,12pt]{article}
\usepackage[utf8]{inputenc}
\usepackage[spanish]{babel}
\usepackage{geometry}
\usepackage[pdfborder=0 0 0]{hyperref}
\usepackage{url}

\title{Obras libres \\
Módulo 1: Nuevas tecnologías \\
Master de Propiedad Intelectual: \\
Retos y nuevos modelos de negocio en la sociedad de la información \\
EOI Escuela de Organización Industrial \\
Programa del curso abril 2010}
\author{Jesús M. González Barahona y Gregorio Robles}

%\date{}

\begin{document}
\maketitle

\newpage

\tableofcontents

\newpage

\section{Equipo de profesores}

\begin{itemize}
\item Jesús M. González Barahona
  \begin{itemize}
  \item Correo electrónico: jgb@libresoft.es
  \end{itemize}
\item Gregorio Robles
  \begin{itemize}
  \item Correo electrónico: grex@libresoft.es
  \end{itemize}
\item Miquel Vidal
  \begin{itemize}
  \item Correo electrónico: mvidal@libresoft.es
  \end{itemize}
\item Israel Herraiz
  \begin{itemize}
  \item Correo electrónico: herraiz@uax.es
  \end{itemize}
\end{itemize}

%\section{Objetivos}

%\begin{itemize}
%\item 
%\end{itemize}

%\section{Metodología}


\section{Programa}

Programa de la asignatura (tentativo)

%--------------------------------------------------------
%--------------------------------------------------------
\subsection{01 - Presentación de la asignatura}

Presentación de los principales aspectos de la asignatura, temas que se tratarán, forma de evaluación, etc.

%--------------------------------------------------------
\subsubsection{Sesión del 24 de abril (2 horas)}

 \begin{itemize}
 \item Profesores: Jesús
 \item Presentación: Presentación de la asignatura.
 \item Ejercicio (foro): Conocimientos sobre temas relacionados con la asignatura. \\
   Incluso sin conocer en detalle la definición de obra libre, comenta:
   \begin{itemize}
   \item Cómo conociste la existencia de las obras intelectuales libres (si fue el caso)
   \item Sitios de obras libres que conoces
   \item Sitios de obras libres en los que has contribuido
   \end{itemize}
 \item Material: Transparencias (tema ``01 - Introducción'')
 \end{itemize}

%--------------------------------------------------------
%--------------------------------------------------------
\subsection{02 - Motivación}

¿Por qué hay autores que distribuyen sus trabajos como obras libres? ¿Qué proyectos y sitios web relacionados con las obras libres son más conocidos? ¿Por qué todo esto es importante?

%--------------------------------------------------------
\subsubsection{Sesión del 24 de abril (2 horas)}

 \begin{itemize}
 \item Profesores: Jesús
 \item Comentario de video: ``Larry Lessig on laws that choke creativity''
 \item Presentación: Motivación
 \item Ejercicio (foro): Comentarios al video. \\
   Elige un aspecto del video de Larry Lessig, relacionado con la asignatura, que te haya llamado especialmente la atención, y coméntalo brevemente.
 \item Material: Transparencias (tema ``02 - Motivación'')
 \item Material: video ``Larry Lessig on laws that choke creativity'', Ted.com, \\
   \url{http://www.ted.com/talks/lang/eng/larry_lessig_says_the_law_is_strangling_creativity.html}
 \end{itemize}

%--------------------------------------------------------
%--------------------------------------------------------
\subsection{03 - Definiciones de obras libres}

%--------------------------------------------------------
%--------------------------------------------------------
\subsection{04 - Licencias de software libre}

\subsubsection{Sesión del 22 de mayo (5 horas)}

Aunque el software libre se asocia normalmente a proyectos abiertos desarrollados en comunidad, en realidad el único rasgo común a todo el software libre es su licencia. En esta lección vemos qué tipos de licencias libres existen, y cómo elegir la más adecuada para cada caso. 

 \begin{itemize}
 \item Profesores: Miquel e Israel
 \item Presentación: Conceptos y Marco legal 
 \item Presentación: Tipos de licencias libres para el software 
 \item Ejercicio (foro): Cuestionario de autoevaluación sobre licencias libres \\
 \item Material: Transparencias (tema ``03 - Licencias-SL'')
 \item Material: vídeo ``Freedom Fry — “Happy birthday to GNU'', gnu.org, \\
   \url{http://www.gnu.org/fry/}
 \item Material: vídeo: ``Qué es el software libre'' - Stallman en español
\url{http://video.google.es/videoplay?docid=9055201737702580581#}
 \end{itemize}

\subsection{05 - Licencias para obras culturales libres}

\subsubsection{Sesión del 29 de mayo (5 horas)}

Existe un heterogéneo movimiento social basado en la libertad de distribuir y modificar trabajos y obras creativas. Su base jurídica son las licencias para obras culturales libres. Funcionan de una forma análoga a como lo hacen las licencias de software libre, pero se aplican a cualquier otro tipo de obras intelectuales (como audiovisuales, literarias, musicales, gráficas, etc.) que no son programas de ordenador.

 \begin{itemize}
 \item Profesores: Miquel e Israel
 \item Presentación: Conceptos y Marco legal
 \item Presentación: Definición de Obras Culturales Libres 
 \item Presentación: El universo Creative Commons
 \item Presentación: Tipos de licencias para obras culturales libres
 \item Presentación: Patentes, marcas, DRM...
 \item Ejercicio (foro): Cuestionario de autoevaluación sobre licencias Creative Commons \\
 \item Material: Transparencias (tema ``04 - Licencias-CC'')
 \item Material: vídeo ``Creative Commons - Sé creativo'', cc.org, \\
   \url{http://www.youtube.com/watch?v=OUo3KMkOETY}
 \end{itemize}

%--------------------------------------------------------
%--------------------------------------------------------
\subsection{06 - Las licencias libres en la práctica. Casos de estudio. Conclusiones}

\subsubsection{Sesión del 5 de junio (5 horas)}

En esta sesión final analizamos los aspectos prácticos de las licencias libres, cómo escoger y aplicar una licencia libre. Compatibilidad entre ellas. Análisis de casos. Repaso general.

 \begin{itemize}
 \item Profesores: Miquel y Grex
 \item Presentación: Aspectos prácticos de las licencias libres
 \item Presentación: Cómo escoger y aplicar una licencia libre
 \item Presentación: Cláusulas de enlazado, cláusulas anti-elusión, bifurcación (forks), marcas, garantía...
 \item Presentación: Obstáculos (patentes, DRM...). Respuestas (cláusulas anti-elusión...). 
 \item Ejercicio (clase): Diagrama de Venn con los conceptos y tipos de licencias.  \\
 \item Ejercicio (clase): Juicio (copyleft/no-copyleft y adaptación/no-adaptación).  \\
 \item Ejercicio (foro): análisis de una licencia libre (la EUPL).  \\
 \item Material: Transparencias (tema ``05 - lic-prac'')
 \item Material: Informe sobre licencias libres (Miguel Vidal)
 \item Material: ``La GPLv3: Copyleft para el siglo XXI'' (Miguel Vidal)
 \item Material: Libro de texto: Aspectos legales y de explotación del software libre (UOC)
 \end{itemize}

\subsection{07 - Modelos de negocio y obras libres}

Panorámica general de las diferentes alternativas de modelos de negocio disponibles para empresas o emprendedores que decidan trabajar con software libre u obras culturales libres.

\subsubsection{Sesión del 10 de julio (5 horas)}

\begin{itemize}
\item Profesores: Jesús
\item Presentación: Modelos de negocio con software libre 
\item Presentación: Modelos de negocio para obras libres

 \item Ejercicio (a entregar en el foro): Creación de un modelo de negocio. \\
   Basándose en lo que se indica en las transparencias correspondientes, entregar un ``Business Model Canvas'' para un negocio relacionado con la producción de obras libres. La actividad se realizará en grupos de dos o tres alumnos.
 \item Material: Transparencias (tema ``Modelos de negocio con software libre'')
\item Material: Transparencias (tema ``Modelos de negocio para obras libres'')
\item Material: Transparencias (actividad ``Creación de un modelo de negocio'')
\item Material: Business Model Canvas Poster
\item Referencias adicionales:
  \begin{itemize}
  \item Open source business models and strategies
  \item The Magic Cauldron
  \item The commercial open source business model
  \item The economic case of open source foundations
  \item The economic motivation of open source: stakeholder perspectives
  \item How to analyze an OSS business model
  \item FLOSS: A guide for SMEs archivo  
  \end{itemize}
\end{itemize}


\section{Evaluación}

Ver apartado de evaluación en el documento de programación del módulo completo, pues será común para todas las asignaturas de este módulo.


\end{document}
