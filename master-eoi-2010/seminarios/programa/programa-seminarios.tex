%%
%% $Id$
%%

\documentclass[a4paper,12pt]{article}
\usepackage[utf8]{inputenc}
\usepackage[spanish]{babel}
\usepackage{geometry}
\usepackage[pdfborder=0 0 0]{hyperref}
\usepackage{url}

\title{Seminarios \\
Módulo 1: Nuevas tecnologías \\
Master de Propiedad Intelectual: \\
Retos y nuevos modelos de negocio en la sociedad de la información \\
EOI Escuela de Organización Industrial \\
Programa del curso abril 2010}
\author{Jesús M. González Barahona y Gregorio Robles}

%\date{}

\begin{document}
\maketitle

\newpage

\tableofcontents

\newpage

%--------------------------------------------------------
%--------------------------------------------------------
%--------------------------------------------------------
\section{Equipo de profesores coordinadores}

\begin{itemize}
\item Jesús M. González Barahona
  \begin{itemize}
  \item Correo electrónico: jgb@libresoft.es
  \end{itemize}
\item Gregorio Robles
  \begin{itemize}
  \item Correo electrónico: grex@libresoft.es
  \end{itemize}
\end{itemize}

%--------------------------------------------------------
%--------------------------------------------------------
%--------------------------------------------------------
%\section{Objetivos}

%\begin{itemize}
%\item 
%\end{itemize}

%\section{Metodología}


%--------------------------------------------------------
%--------------------------------------------------------
%--------------------------------------------------------
\section{Programa}

Programa de seminarios (tentativo)

%--------------------------------------------------------
%--------------------------------------------------------
\subsection{01 - Software libre en entorno empresarial}

Sesión abierta.

%--------------------------------------------------------
\subsubsection{Sesión del 24 de junio (1 hora)}

\begin{itemize}
\item Ponente: Manel Sarasa Serra (Openbravo)
\item Presentación: Openbravo
  %\item Material: Transparencias (tema ``01 - Presentación'')
\item Algunos temas a comentar: 
  \begin{itemize}
  \item Openbravo Public License: ¿es una licencia libre? ¿compatible con la GPL?
  \item Diferentes licencias para Openbravo ERP y Openbravo POS. ¿Por qué?
  \item ¿Qué medidas ha tomado Openbravo para fomentar la colaboración de desarrolladores y otros actores externos?
  \item ¿Qué ventajas obtiene Openbravo (la empresa) de la comunidad de desarrolo que se ha creado a su alrededor?
  \item ¿Cuáles son las fuentes de ingresos fundamentales de Openbravo?
  \item ¿En qué aspectos del modelo de negocio se diferencia Openbravo de los modelos tradicionales, privativos, creados alrededor de otros ERPs?
  \end{itemize}
\item Más información:
  \begin{itemize}
  \item Sitio web de Openbravo: \\
    \url{http://openbravo.com}
  \item Openbravo en Wikipedia: \\
    \url{http://en.wikipedia.org/wiki/Openbravo}
  \end{itemize}
\end{itemize}

%--------------------------------------------------------
\subsubsection{Sesión del 24 de junio (1 hora)}

\begin{itemize}
\item Ponente: José María Olmo Millán (Andago)
\item Presentación: Andago
\item Comentario: la presentación fue suspendida
\end{itemize}

%--------------------------------------------------------
\subsubsection{Sesión del 24 de junio (1 hora)}

\begin{itemize}
\item Ponente: Juan José Hierro (Telefónica I+D, Morfeo Community)
\item Presentación: El software libre como arma estratégica
\item Algunos temas a comentar: 
  \begin{itemize}
  \item 
  \end{itemize}
\end{itemize}


\end{document}
