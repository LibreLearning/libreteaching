%%
%% $Id$
%%

\documentclass[a4paper,12pt]{article}
\usepackage[utf8]{inputenc}
\usepackage[spanish]{babel}
\usepackage{geometry}
\usepackage[pdfborder=0 0 0]{hyperref}
\usepackage{url}

\title{Módulo 1: Nuevas tecnologías \\
Master de Propiedad Intelectual: \\
Retos y nuevos modelos de negocio en la sociedad de la información \\
EOI Escuela de Organización Industrial \\
Programa genral del curso abril 2010}
\author{Jesús M. González Barahona y Gregorio Robles}

%\date{}

\begin{document}
\maketitle

\newpage

\tableofcontents

\newpage

%--------------------------------------------------------
%--------------------------------------------------------
%--------------------------------------------------------
\section{Equipo de profesores coordinadores}

\begin{itemize}
\item Jesús M. González Barahona
  \begin{itemize}
  \item Correo electrónico: jgb@libresoft.es
  \end{itemize}
\item Gregorio Robles
  \begin{itemize}
  \item Correo electrónico: grex@libresoft.es
  \end{itemize}
\end{itemize}

%--------------------------------------------------------
%--------------------------------------------------------
%--------------------------------------------------------
%\section{Objetivos}

%\begin{itemize}
%\item 
%\end{itemize}

%\section{Metodología}


%--------------------------------------------------------
%--------------------------------------------------------
%--------------------------------------------------------
\section{Programa de seminarios y sesión de clausura}

Programa de seminarios y sesión de clausura del módulo.

%--------------------------------------------------------
%--------------------------------------------------------
\subsection{01 - Seminario: Software libre en entorno empresarial}

Sesión abierta.

%--------------------------------------------------------
\subsubsection{Sesión del 24 de junio (1,5 horas)}

\begin{itemize}
\item Ponente: Manel Sarasa Serra (Openbravo)
\item Presentación: Openbravo
  %\item Material: Transparencias (tema ``01 - Presentación'')
\item Algunos temas a comentar (entre los muchos que han salido durante la presentación): 
  \begin{itemize}
  \item Openbravo Public License: ¿es una licencia libre? ¿compatible con la GPL?
  \item Diferentes licencias para Openbravo ERP y Openbravo POS. ¿Por qué?
  \item ¿Qué medidas ha tomado Openbravo para fomentar la colaboración de desarrolladores y otros actores externos?
  \item ¿Qué ventajas obtiene Openbravo (la empresa) de la comunidad de desarrolo que se ha creado a su alrededor?
  \item ¿Cuáles son las fuentes de ingresos fundamentales de Openbravo?
  \item ¿En qué aspectos del modelo de negocio se diferencia Openbravo de los modelos tradicionales, privativos, creados alrededor de otros ERPs?
  \end{itemize}
\item Más información:
  \begin{itemize}
  \item Sitio web de Openbravo: \\
    \url{http://openbravo.com}
  \item Openbravo en Wikipedia: \\
    \url{http://en.wikipedia.org/wiki/Openbravo}
  \end{itemize}
\end{itemize}

%--------------------------------------------------------
%\subsubsection{Sesión del 24 de junio (1 hora)}

%\begin{itemize}
%\item Ponente: José María Olmo Millán (Andago)
%\item Presentación: Andago
%\item Comentario: la presentación fue suspendida
%\end{itemize}

%--------------------------------------------------------
\subsubsection{Sesión del 24 de junio (1,5 horas)}

\begin{itemize}
\item Ponente: Juan José Hierro (Telefónica I+D, Morfeo Community)
\item Presentación: El software libre como arma estratégica
\item Algunos temas a comentar (entre los muchos que han salido durante la presentación): 
  \begin{itemize}
  \item ¿Cómo se aplica la ley de la conservación de los beneficios atractivos, de Christensen, al software libre?
  \item Busca otros ejemplos (además de los que ha comentado el ponente) donde la ``comoditización'' de un producto haya supuesto reparto de benficios en otros actores de la cadena de valor. Si estos ejemplos están el ámbito del software (o más específicamente del software libre), mejor, pero si se te ocurren en otros campos, adelante.
  \end{itemize}
\end{itemize}

%--------------------------------------------------------
%--------------------------------------------------------
\subsection{02 - Seminario: The Wealth of Networks}

Sesiones abiertas. Seminario sobre el libro ``The Wealth of Networks'', por su autor, Yohai Benkler. Las sesiones tendrán lugar en Medialab Prado.

\begin{itemize}
\item Medialab Prado, \\
  \url{http://medialab-prado.es}
\item Seminar with Yochai Benkler about The Wealth of Networks, \\
  \url{http://medialab-prado.es/article/seminario_benkler}
\end{itemize}

%--------------------------------------------------------
\subsubsection{Sesión del 29 de junio (1,5 horas)}

\begin{itemize}
\item Ponente: Yochai Benkler
\item Presentación: Keynote speech
\item Hora: 19:00
%\item Material: 
\end{itemize}

%--------------------------------------------------------
\subsubsection{Sesión del 30 de junio (1,5 horas)}

\begin{itemize}
\item Ponentes: Yochai Benkler, Langdon Winner
\item Presentación: Dialogue Yochai Benkler--Langdon Winner on The Wealth of Networks
\item Hora: 19:00
%\item Material: 
\end{itemize}

%--------------------------------------------------------
%--------------------------------------------------------
\subsection{03 - Seminario: Software libre en primera persona}

%--------------------------------------------------------
\subsubsection{Sesión del 16 de septiembre (1 hora)}

\begin{itemize}
\item Ponente: Jorge Ferrer, Liferay
\item Presentación: Los retos de un modelo de negocio basado en software libre: la experiencia de Liferay
\item Hora: 17:00
\end{itemize}

%--------------------------------------------------------
\subsubsection{Sesión del 16 de septiembre (1 hora)}

\begin{itemize}
\item Ponente: Álvaro López Ortega, Octality
\item Presentación: Mi experiencia en el mundo del software libre
\item Hora: 18:00
\end{itemize}

%--------------------------------------------------------
%--------------------------------------------------------
\subsection{04 - Seminario: Obras libres}

%--------------------------------------------------------
\subsubsection{Sesión del 23 de septiembre (1,5 horas)}

%--------------------------------------------------------
\subsubsection{Sesión del 23 de septiembre (1,5 horas)}


\subsection{05 - Sesión de clausura}

%--------------------------------------------------------
\subsubsection{Sesión del 11 de septiembre (5 horas)}

\begin{itemize}
\item Profesores: Jesús
\item Debate sobre el ejercicio de modelos de negocio (ver programa de ``Obras libres''
\item Repaso del estado de las actividades evaluables y aclaración de dudas
\item Análsis de casos:
  \begin{itemize}
  \item RadarVirtuel.com. \\
    Comunidad basada en datos de disposición pública (si se dispone del aparato adecuado). ¿Pueden los ciudadanos organizarse para tener información? ¿Hay modelos de negocio? \\
    \url{http://www.radarvirtuel.com/}
  \item Identi.ca. \\
    Servicio en sitio único frente a federación de servicios. ¿Cómo sería el microblogging si los sitios interoperaran? Implicaciones \\
    \url{http://identi.ca}
  \item Flatworld \\
    ¿Hay un modelo de negocio? ¿Cómo compara con sitios como Bubok? ¿Es viable con libros bajo licencias libres?\\
    \url{http://www.flatworldknowledge.com/}
  \item The Mongoliad \\
    ¿Será viable? ¿En qué se diferencia de los modelos clásicos de suscripción? \\
    \url{http://mongoliad.com/}
  \item El Cosmonauta \\
    ¿Será viable? ¿Podrían usarse mecanismos de este estilo para realizar grandes producciones? ¿Qué hay de nuevo en sus mecanismos de márketing?\\
    \url{http://elcosmonauta.es/}
  \item Musopen. \\
    Recauda donaciones para grabar música clásica libre. ¿Modelo válido para producir obras libres? \\
    \url{http://musopen.com/} \\
    \url{http://www.kickstarter.com/projects/Musopen/record-and-release-free-music-without-copyrights}
  \item Threshold pledge system \\
    ¿Es generalizable? ¿En qué casos puede funcionar? \\
    \url{http://en.wikipedia.org/wiki/Threshold_pledge_system}
  \item Network neutrality \\
    ¿La hay en telefonía ``tradicional''? ¿Y en telefonía GSM? ¿Y en Internet, en España? \\
    \url{http://en.wikipedia.org/wiki/Network_neutrality}
  \end{itemize}
\item Temas para módulos II y III:
  \begin{itemize}
  \item Jurisdicciones competentes en privacidad en una red social.
  \item Interacción de patentes de software y licencias de software libre.
  \item ¿Puede haber música libre con la actual legislación sobre entidades de gestión de derechos? ¿Y una entidad que se dedicase específicamente a obras libres?
  \item ¿Cuál es la estrategia de Google con Ändroid? ¿Cómo compara con la de Nokia con Meego y la de Apple con iPhone?
  \item ¿Cómo se relaciona la producción de obras libres con la teoría de asignación eficiente por mercados libres? ¿Se comportan las obras intelectuales como un bien público?
  \item ¿Qué regulaciones afectan (directamente o indirectamente) a Wikipedia en la jurisdicción española?
  \end{itemize}
\end{itemize}

%--------------------------------------------------------
%--------------------------------------------------------
%--------------------------------------------------------
\section{Evaluación}

La evaluación final del módulo se realizará a partir de las evaluaciones de las diferentes actividades evaluables realizadas. Cada actividad será evaluada con una serie de puntos, según el parecer de los profesores de la asignatura, que seguirán los criterios generales espeficados más adelante.

La nota final estará en función del número de puntos total obtenido, matizados según la opinión general de los profesores sobre los resultados obtenidos. Hay actividades obligatorias (para aprobar la asignatura, el número de puntos mínimo a conseguir en esa actividad es mayor que cero), y otras voluntarias (se puede aprobar la asignatura teniendo cero puntos en esa actividad).

El número mínimo de puntos que ha de obtener un estudiante (matizado por la opinión general de los profesores) es:

\begin{itemize}
\item Para aprobar la asignatura: 100 puntos.
\item Para obtener un notable: 175 puntos.
\item Para obtener un sobresaliente: 250 puntos.
\end{itemize}

\subsection{Actividades evaluables}

Cada actividad tendrá unos criterios de evaluación propios, seguń se indica a continuación. Para cada actividad habrá una puntuación mínima necesaria (0 si no es obligatoria, mayor que 0 si es obligatoria), una puntuación máxima, una descripción de la actividad y, cuando sea posible definirlos, unos criterios generales de puntuación.

Se aconseja a los alumnos que pregunten a los profesores cualquier detalle que pueda no estar claro sobre en qué consisten las actividades, o sobre cómo se puntuarán. En general la evaluación tendrá siempre en cuenta cómo la realización de la actividad ha acercado al alumno a la consecución de los objetivos de la asignatura, y cómo éste ha mostrado las competencias o habilidades correspondientes.

\begin{itemize}
\item Participación en los foros a cuestiones y ejercicios planteados por los profesores,  Mínimo: 20 puntos, máximo: 80 puntos.

Participación en los foros en el Moodle de cada tema en cuestiones planteadas por los profesores durante las videoconferencias y otras actividades. Además, en las asignaturas se han propuesto varios ejercicios (como por ejemplo, participación en varias redes sociales), que también serán evaluadas.

\item Preguntas de autoevaluación. Mínimo: 0 puntos, máximo: 40 puntos.

En algunos temas se ofrecerá la posibilidad de realizar tests de autoevaluación para que los alumnos puedan cerciorarse de que han comprendido y asimilado correctamente los conceptos presentados en la asignatura.

\item Blog sobre la asignatura. Mínimo: 10 puntos, máximo: 40 puntos.

Mantenimiento de un blog sobre la asignatura. Deberá incluir como mínimo del orden de una nota cada semana o 10 días (preferentemente dos o más cada semana o 10 días). Los temas pueden ser los que se van viendo en la asignatura, temas de actualidad relacionados con software libre, aspectos que le resulten interesantes al alumno sobre software libre, etc. El alumno abrirá el blog donde quiera (o abrirá una categoría en su blog, si ya lo tiene), y enviará la url del feed correspondiente a los profesores.


\item Trabajos sobre temas concretos relacionados con el módulo. Mínimo: 10 puntos, máximo: 75 puntos.

Los estudiantes pueden realizar un trabajo individual sobre algún tema relacionado con el módulo. Se podrá entregar como informe o artículo, o como presentación en video (puede ser, por ejemplo, un screencast). Algunos alumnos pueden ser invitados a exponer su trabajo en clase.


\item Colaborar en documentación de los temas tratados en clase usando Google Docs. Mínimo: 10 puntos, máximo: 40 puntos.

Se ha pedido a los alumnos que colaboren en la creación de documentos con Google Docs, que detallen y expliquen los temas desarrollados en las clases. No es preciso que todos los alumnos colaboren en todos los documentos, ni que en todos los documentos hayan colaborado todos los alumnos. Lo normal será que un alumno comience el documento, alguno más lo continue, y en total como mucho tres o cuatro escriban su mayor parte. Luego otros pueden releerlo, y corregir pequeños errores, mejorar las explicaciones, completar detalles, etc.

\item Realización de trabajos sobre seminarios. Mínimo: 10 puntos, máximo: 30 puntos.

Realización de trabajos cortos sobre aspectos concretos mencionados en los seminarios complementarios del módulo. Cada alumno deberá realizar al menos un trabajo para tres de los seminarios. Las presentaciones de cada seminario normalmente ofrecidas mediante streaming, y grabadas para su publicación posterior en Internet..

\item Colaborar en la redacción de un texto sobre el contenido del módulo en WikiBooks. Mínimo: 0 puntos, máximo: 60 puntos.

Con el contenido sobre los temas mantenido en Google Docs, reailizar un texto en Wiki Books, no sobre el máster, sino sobre sus contenidos. De hecho, puede que tenga mśa sentido hacer dos textos, uno sobre la parte más ``tecnológica'', y otro sobre la parte relacionada con la producción de obras libres. Pero pueden comentarse con los profesores otras posibilidades. El resultado debería tener estructura e intención de libro.

\end{itemize}

\subsection{Fechas}

Evaluación final (todas las actividades habrán de estar terminadas): 15 de septiembre de 2010, salvo las relacionadas con los seminarios que se realizan en septiembre. Para documentos relacionados con ellos, la fecha final será 1 de octubre de 2010.

A partir de la primera fecha, los alumnos no deberán tocar (ni subir nuevas versiones) de los documentos correspondientes con las actividades que no correspondan con estos seminarios de septiembre.

\subsection{Entrega de documentos para evaluación}

Para ser evaluados, los alumnos deberán eviar en el sitio web de la asignatura, mediante el elemento habilitado al efecto, los siguientes documentos:

\begin{itemize}
\item Resumen de actividades evaluables. Fichero en formato texto en el que se presentarán las actividades que el alumno propone para evaluación, elegidas entre la lista siguiente (que corresponde con las actividades especificadas anteriormente):
  \begin{itemize}
  \item Cuestiones y ejercicios. Las cuestiones y ejercicios se habrán entregado como respuesta a las entradas en los foros correspondientes).
  \item Autoevaluación. Los tests correspondientes se habrán hecho en su momento.
  \item Blog. Se indicará la url del blog.
  \item Trabajos sobre temas. Se indicará el nombre de los trabajos y el nombre del fichero de entrega (ver más abajo).
  \item Documentación con Google Docs. Se indicará los documentos en los que se ha colaborado y la cuenta Google desde la que se ha realizado la actividad. Es importante que los documentos en cuestión hayan sido compatidos con los profesores.
  \item Trabajos de seminarios. Se indicará el nombre de los trabajos y el nombre del fichero de entrega (ver más abajo).
  \item Wikibook. Se indicará el o los wikibooks en los que se ha participado, un resumen de en qué ha consistido la participación, la url del documento, y el identificador con el que se ha trabajado.
  \item Otros. Cualquier otra actividad evaluable que se haya podido realizar.
  \end{itemize}
\item Fichero o ficheros de entrega de los trabajos realizados sobre temas. Estos ficheros se entregarán en formato PDF.
\item Fichero o ficheros de entrega de los trabajos de seminarios. Estos ficheros se entregarán en formato PDF.
\end{itemize}



\end{document}
