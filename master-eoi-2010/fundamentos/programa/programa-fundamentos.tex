%%
%% $Id$
%%

\documentclass[a4paper,12pt]{article}
\usepackage[utf8]{inputenc}
\usepackage[spanish]{babel}
\usepackage{geometry}
\usepackage[pdfborder=0 0 0]{hyperref}
\usepackage{url}

\title{Fundamentos de las nuevas tecnologías \\
Módulo 1: Nuevas tecnologías \\
Master de Propiedad Intelectual: \\
Retos y nuevos modelos de negocio en la sociedad de la información \\
EOI Escuela de Organización Industrial \\
Programa del curso abril 2010}
\author{Jesús M. González Barahona y Gregorio Robles}

%\date{}

\begin{document}
\maketitle

\newpage

\tableofcontents

\newpage

%--------------------------------------------------------
%--------------------------------------------------------
%--------------------------------------------------------
\section{Equipo de profesores}

\begin{itemize}
\item Jesús M. González Barahona
  \begin{itemize}
  \item Correo electrónico: jgb@libresoft.es
  \end{itemize}
\item Gregorio Robles
  \begin{itemize}
  \item Correo electrónico: grex@libresoft.es
  \end{itemize}
\item Agustín Santos
  \begin{itemize}
  \item Correo electrónico: asantos@libresoft.es
  \end{itemize}
\end{itemize}

%--------------------------------------------------------
%--------------------------------------------------------
%--------------------------------------------------------
%\section{Objetivos}

%\begin{itemize}
%\item 
%\end{itemize}

%\section{Metodología}


%--------------------------------------------------------
%--------------------------------------------------------
%--------------------------------------------------------
\section{Programa}

Programa de la asignatura (tentativo)

%--------------------------------------------------------
%--------------------------------------------------------
\subsection{01 - Presentación}

Presentación de los aspectos fundamentales de la asignatura. Información sobre las capacidades y habilidades de los alumnos en relación con esta asignatura.

%--------------------------------------------------------
\subsubsection{Sesión del 23 de abril (4 horas)}

\begin{itemize}
\item Profesores: Jesús y Gregorio
\item Presentación: Presentación de la asignatura.
  %\item Material: Transparencias (tema ``01 - Presentación'')
\item Ejercicio : Conocimientos sobre temas relacionados con la asignatura:
  \begin{itemize}
  \item Internet: desde cuándo, tipo de experiencia, navegadores utilizados, servicios utilizados, tiempo por semana (conectado, de uso activo)
  \item Servicios Internet (conoces, tienes cuenta, usas habitualmente): Facebook, Identi.ca, Tuenti, Twitter, Picotea, Flickr, Youtube, Blip.tv, Vimeo, de.li.cious, LinkedIn, Infojobs, Wikipedia, blogs, otros
  \item Software: programas más usados (entre los libres: OpenOffice, Firefox, Thunderbird, GIMP, Linux, otros)
  \item Hardware: sobremesa, portátil, smartphone, ebook, otros
  \end{itemize}
\item Presentación: Evaluación de la asignatura
\item Ejercicio: Exploración muy resumida del entorno Moodle
\end{itemize}

%--------------------------------------------------------
%--------------------------------------------------------
\subsection{02 - Aplicaciones útiles}

%--------------------------------------------------------
\subsubsection{Sesión del 23 de abril (1 hora)}

Introducción práctica a FaceBook.

\begin{itemize}
\item Profesores: Jesús
\item Ejercicio (para terminar no presencialmente): Sitio en FaceBook de alumnos del máster. \\
  Vamos a crear el sitio de los alumnos del máster en FaceBook. Puede ser una página, un grupo, y/o algún otro elemento que os resulte adecuado. Naturalmente, lo primero será crear una cuenta en FaceBook, si no la tienes ya. Luego, crear el sitio del máster, invitar al resto de alumnos (y profesores, al menos de esta asignatura) a ser administradores en él.

Podéis usar este foro para coordinaros, pero en cuanto esté lista la infraestructura en FaceBook, usadla preferentemente. Decidid no sólo qué elementos usaréis, sino cómo los configuraréis, teniendo en cuenta la visibilidad que queráis darle, la privacidad que queráis mantener, cómo reutilizar otras informaciones sobre el master, etc.

Por favor, indica también cuál es tu usuario en FaceBook, para que te podamos identificar.

La actividad es colaborativa, entre todos vostros, pero procurad participar todos...
\end{itemize}

%--------------------------------------------------------
\subsubsection{Sesión del 29 de abril (2 horas)}

Introducción práctica a Twitter y Google Docs.

\begin{itemize}
\item Profesores: Jesús
\item Ejercicio (para terminar no presencialmente): Canal Twitter para el master. \\
Vamos a crear el canal Twitter para el master. Para ello, todos los alumnos se crearán cuentas en Twitter, e intercambiarán mensajes con el hashtag ``\#eoipi'', que será el que usaremos en el máster. Además, cada alumno creará una lista Twitter con todos los alumnos y profesores del master que pueda incluir en ella.

Una vez terminado el ejercicio, seguiremos usando el mismo hashtag para cualquier mensaje corto relacionado con el máster. Se aconseja a los alumnos que lo sigan periódicamente.

Por favor, indica también cuál es tu usuario en Twitter, para que te podamos identificar.
\item Ejercicio (foro): Aplicaciones para Twitter \\
  El sitio de Twitter permite poner mensajes cortos vía web. Pero hay muchas aplicaciones que lo usan, bien para visualizar los mensajes de distintas formas, o bien para ponerlos. Estas aplicaciones pueden funcionar en el móvil, en el escritorio de un ordenador, o ser aplicaciones web. Elige una, y:

  \begin{itemize}
  \item Dinos su nombre y la url donde podemos conseguir más información (y ponlo también en el canal Twitter del máster).
  \item Explica brevemente su funcionamiento, y para qué sirve.
  \end{itemize}

\item Ejercicio (para terminar no presencialmente): Introducción a Google Docs. \\
Usando la cuenta creada para cada alumno por la EOI en Google Docs, crear un documento, invitar a otros alumnos para que lo compartan, modificar los de otros alumnos que lo hayan compartido, etc. El objetivo de la práctica es familiarizarse con Google Docs, que se usará en otras actividades del máster.

Compartir los documentos creados con el profesor, para que pueda verlos.
\end{itemize}

%--------------------------------------------------------
\subsubsection{Sesión del 7 de mayo (1 hora)}

Trabajo con blogs.

\begin{itemize}
\item Profesores: Jesús
\item Ejercicio (para terminar no presencialmente): Blog para el master. \\
Creación de un blog en WordPress.com o similar. Edición de una primera nota en el blog. Envío de enlace para sindicación de todos los de los alumnos del master.
\item Enlaces: WordPress.com, \url{http://wordpress.com}
\end{itemize}

%--------------------------------------------------------
\subsubsection{Sesión del 28 de mayo (1 hora)}

Trabajo con Identi.ca, LinkedIn.

\begin{itemize}
\item Profesores: Jesús
\item Ejercicio (para terminar no presencialmente): Trabajo con LinkedIn. \\
  Crear una cuenta en LinkedIn, y completar las partes que se consideren adecuadas del perfil. Incluir, en la medida de lo posible, información vuestra de otros lugares (Twitter, blog, etc.) Crear, colaborativamente, un grupo para la asignatura del máster. Documentar todo, colaborativamente, usando Google Docs. Para coordinaros, podéis usar el foro cafetería del curso general del Moodle del máster. Contestad a este ejercicio con un enlace a vuestro perfil en LinkedIn.
\item Ejercicio (para terminar no presencialmente): Trabajo con Identi.ca. \\
  Crear una cuenta en Identi.ca. Suscríbete a los demás miembros del máster, enlaza con tu cuentas en Twitter, FaceBook, etc. Cread un grupo para la temática del máster, y apuntaros a él (podéis llamarlo !eoipi). Documentar todo, colaborativamente, usando Google Docs. Para coordinaros, podéis usar el foro cafetería del curso general del Moodle del máster. Contestad a este ejercicio con un enlace a vuestra página en Identi.ca.
\item Enlaces:
  \begin{itemize}
  \item LinkedIn, \url{http://linkedin.com}
  \item Identi.ca, \url{http://identi.ca/}
  \item Identi.ca help page, \url{http://identi.ca/doc/help}
  \end{itemize}
\end{itemize}

%--------------------------------------------------------
%--------------------------------------------------------
\subsection{03 - Introducción y motivación}

Introducción a la asignatura, y motivación de su interés en el contexto del máster.

%--------------------------------------------------------
\subsubsection{Sesión del 29 de abril (5 horas)}

\begin{itemize}
\item Profesores: Agustín
\item Presentación: Motivación
\item Ejercicio: ¿Por qué los conceptos de la asignatura son interesantes? \\
  Los alumnos se organizan en grupos de dos o tres, y eligen entre una lista de conceptos de la asignatura. En unos 15-20 min. crean unas transparencias en Google Docs explicando porqué ese concepto es, en opinión del grupo, interesante. Luego se exponen las transparencias, y se comentan entre todos los alumnos.
\item Materiales: Transparencias, tema ``Motivación''
\end{itemize}

% \item Ejercicio: Exploración del entorno Google Apps
% \item Ejercicio: Exploración del entorno Facebook
% \item Ejercicio: Exploración del entorno Twitter
% \item Ejercicio: Edición colaborativa: resumen de la asignatura en Google Docs
% \item Ejercicio: Edición en Wikipedia

%--------------------------------------------------------
%--------------------------------------------------------
\subsection{04 - Representación de la información y almacenamiento}

En este tema se verán los conceptos básicos sobre la representación de la información (bit, bytes, etc.), estandares, formatos y otros conceptos que influyen en el almacenamiento y transmición de la información.

\subsubsection{Sesión del 29 de abril (2 horas)}

Introducción práctica al concepto de Bit.

\begin{itemize}
\item Profesores: Agustín
\item Se muestra un applet java en el que el alumno puede ver un truco de adivinación. El primer truco se encuentra en:
\item \url{http://www.cut-the-knot.org/blue/Cards.shtml}
\item Tras repetir el proceso una serie de veces, el alumno descubre que el truco se basa en el concepto de bit. Se ofrece una segunda versión del mismo truco en:
\item \url{http://gwydir.demon.co.uk/jo/numbers/binary/how.htm}
\item Utilizando estos ejemplos se introducen otros conceptos asociados (capacidad de representación de la información, orden, etc.)
\end{itemize}

%--------------------------------------------------------
\subsubsection{Sesión del 7 de mayo (1 hora)}

\begin{itemize}
\item Profesores: Agustín
\item Ejercicio: codificación de caracteres (ASCII) \\
Se pide a los alumnos que hagan su propia propuesta para codificar un conjunto de caracteres (para no hacerlo muy complejo se limita el conjunto). Se pide que escriban su nombre utilizando su propia codificaón y que se lo pasen a otro alumno para que lo descodifique. El segundo alumno utiliza para ello su propio esquema de codificación, con lo que, probablemente, no pueda encontrar el texto. La idea del ejercicio es que descubran que la representación de la información depende de varios factores, entre ellos el convenio.
\item Otro material: \url{http://nickciske.com/tools/binary.php}
\end{itemize}

\subsubsection{Sesión del 7 de mayo (3 horas)}
\begin{itemize}
\item Profesores: Agustín
\item Se introducen los conceptos de:
\begin{itemize}
\item Byte, Kb, Mb.
\item Representación de caracteres
\item formatos y extensiones
\item compresión
\end{itemize}
\item Materiales: Transparencias, tema ``Representación de la información y almacenamiento''
\end{itemize}
%--------------------------------------------------------
%--------------------------------------------------------
\subsection{05 - Hardware}

%--------------------------------------------------------
%--------------------------------------------------------
\subsection{06 - Software}

%--------------------------------------------------------
%--------------------------------------------------------
\subsection{07 - Internet y el/la web}

%--------------------------------------------------------
%--------------------------------------------------------
\subsection{08 - Casos}

%--------------------------------------------------------
%--------------------------------------------------------
\subsection{09 - Seguridad y privacidad}



%--------------------------------------------------------
%--------------------------------------------------------
%--------------------------------------------------------
\section{Evaluación}

La evaluación final de la asignatura se realizará a partir de las evaluaciones de las diferentes actividades evaluables realizadas. Cada actividad será evaluada con una serie de puntos, según el parecer de los profesores de la asignatura, que seguirán los criterios generales espeficados más adelante. La nota final estará en función del número de puntos total obtenido, matizados según la opinión general de los profesores sobre los resultados obtenidos. Hay actividades obligatorias (para aprobar la asignatura, el número de puntos mínimo a conseguir en esa actividad es mayor que cero), y otras voluntarias (se puede aprobar la asignatura teniendo cero puntos en esa actividad).

%El número mínimo de puntos que un estudiante ha de tener para aprobar la asignatura es de 100 puntos. La puntuación mínima para obtener un notable es de 175 puntos, mientras que para sobresaliente el mínimo es de 250 puntos\footnote{Nota para alumnos matriculados en el máster de Sistemas Telemáticos ``antiguo'', a extinguir. Al ser la ``nueva'' asignatura de 3 créditos y la ``antigua'' de 4 créditos, requiere aproximadamente un 33\% de esfuerzo acidional. Por ello para los alumnos que cursan esta asignatura como parte del programa ``antiguo'' el número mínimo de puntos para aprobar la asignatura es de 130 puntos, para obtener un notable de 230 puntos, y para sobresaliente de 330 puntos. Estos alumnos podrán optar a tutorías adicionales si lo precisan para completar sus actividades, o ser orientados sobre ellas}.

%% \subsection{Actividades evaluables}

%% Cada actividad tendrá unos criterios de evaluación propios, seguń se indica a continuación. Para cada actividad habrá una puntuación mínima necesaria (0 si no es obligatoria, mayor que 0 si es obligatoria), una puntuación máxima, una descripción de la actividad y, cuando sea posible definirlos, unos criterios generales de puntuación. Se aconseja a los alumnos que pregunten a los profesores cualquier detalle que pueda no estar claro sobre en qué consisten las actividades, o sobre cómo se puntuarán. En general la evaluación tendrá siempre en cuenta cómo la realización de la actividad ha acercado al alumno a la consecución de los objetivos de la asignatura, y cómo éste ha mostrado las competencias o habilidades correspondientes.

%% \begin{itemize}
%% \item Examen final. Mínimo: 25 puntos, máximo: 50 puntos.

%% Examen sobre el temario de la asignatura.

%% \item Participación en los foros a cuestiones planteadas por los profesores. Mínimo: 20 puntos, máximo: 75 puntos.

%% Participación en los foros en el Moodle de cada tema en cuestiones planteadas por los profesores durante las videoconferencias.

%% \item Preguntas de autoevaluación. Mínimo: 0 puntos, máximo: 30 puntos.

%% Moodle ofrece la posibilidad de realizar tests de autoevaluación para que los alumnos puedan cerciorarse de que han comprendido y asimilado correctamente los conceptos presentados en la asignatura. Esta actividad consiste en realizar los tests creados por compañeros vuestros en cursos pasados.

%% \item Blog sobre la asignatura. Mínimo: 0 puntos, máximo: 40 puntos.

%% Mantenimiento de un blog sobre la asignatura. Deberá incluir como mínimo del orden de una nota cada semana o 10 días (preferentemente dos o más cada semana o 10 días). Los temas pueden ser los que se van viendo en la asignatura, temas de actualidad relacionados con software libre, aspectos que le resulten interesantes al alumno sobre software libre, etc. El alumno abrirá el blog donde quiera (o abrirá una categoría en su blog, si ya lo tiene), y enviará la url del feed correspondiente a los profesores.

%% \item Juego docente con e-Adventure. Mínimo: 0 puntos, máximo: 100 puntos.

%% e-Adventure es una plataforma para crear juegos docentes. Esta actividad, que se puede realizar en grupo, tiene como objetivo la creación de un juego docente sobre algún tema relacionado con la asignatura.

%% \item Recrear un artículo de investigación sobre software libre. Mínimo: 0 puntos, máximo: 100 puntos.

%% Hay muchos artículos científicos que estudian el software libre utilizando fuentes de datos públicas, tal y como se enseña en esta asignatura. Esta tarea consiste en recrear esos artículos con datos actualizados para comprobar la veracidad de los artículos y si sus conclusiones siguen siendo válidas años después.

%% \item Caso de estudio un proyecto de software libre. Mínimo: 15 puntos, máximo: 75 puntos.

%% En esta asignatura se presentan métodos y herramientas para auditar y estudiar proyectos de software libre. Esta tarea consiste en elegir un proyecto y realizar un caso de estudio.

%% \item Trabajos sobre temas concretos de software libre. Mínimo: 0 puntos, máximo: 75 puntos.

%% Los estudiantes pueden realizar un trabajo, individual o colectivo sobre algún tema relacionado con la asignatura. Se podrá entregar como informe o artículo, o como presentación en video (puede ser, por ejemplo, un screencast). Algunos alumnos pueden ser invitados a exponer su trabajo en clase. Ejemplos de temas de trabajos: línea histórica de las licencias de software libre, clasificación de empresas según modelos de negocio, tipos de comunidades de desarrollo de software libre.

%% \item Participar en concurso sobre la asignatura en el IRC. Mínimo: 0 puntos, máximo: 50 puntos. \[Propuesta tentativa\]

%% Se trata de un concurso tipo trivial a través del IRC. Dos alumnos compiten por responder correctamente a preguntas sobre el temario, ganando el punto aquél que responda antes correctamente. Se está implementando una plataforma para llevar este tipo de actividades de manera autónoma. Si el proyecto está listo para ser probado, se podra realizar esta actividad.
%% \end{itemize}

%% \subsection{Fechas}

%% Evaluación final (todas las actividades habrán de estar terminadas): 12 mayo 2010

%% Segunda convocatoria de evaluación: junio 2010



\end{document}
