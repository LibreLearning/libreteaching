
%%---------------------------------------------------------------
%%---------------------------------------------------------------
\section{Presentaci�n de la asignatura}


%%---------------------------------------------------------------

\begin{frame}
\frametitle{�De qu� trata esta asignatura?}

\begin{quote}
Hay muchos movimientos, comunidades, proyectos, que est�n cambiando la relaci�n entre la sociedad y la tecnolog�a. Tienen caracter�sticas comunes, entre las que suele aparecer la palabra ``open''.
\end{quote}

\end{frame}



%%---------------------------------------------------------------
\begin{frame}
\frametitle{Programa (tentativo)}

\begin{itemize}
\item Presentaci�n \\
  Presentaci�n de la asignatura
\item Casos de estudio \\
\begin{itemize}
\item Open Government
\item Open Access Scientific Publishing
\item Redes ciudadanas
\item Open Innovation
\item Crowd funding
\item Wikipedia
\end{itemize}
\end{itemize}

\end{frame}


%%---------------------------------------------------------------
\begin{frame}
\frametitle{Actividades base: clases}

\begin{itemize}
\item Motivaci�n, debate
\item Transparencias en el sitio de la asignatura
\item Importante verlas antes de cada sesi�n
\item Materiales complementarios para cada sesi�n
\item Test de autoaprendizaje antes de cada sesi�n
\item Discusi�n de casos, ejercicios, debates...
\item Propuestas de los alumnos
\end{itemize}

\end{frame}

%%---------------------------------------------------------------
\begin{frame}
\frametitle{Actividades base: sitio web, foros}

\begin{itemize}
\item Difusi�n de informaci�n, debate
\item Temas propuestos por profesores:
  \begin{itemize}
  \item Informaci�n general
  \item Preguntas concretas
  \item Comentarios y preguntas sobre el texto base
  \item Propuestas de peque�os trabajos
  \end{itemize}
\item Los alumnos tambi�n pueden proponer temas
  \begin{itemize}
  \item Preguntas de clarificaci�n
  \item Inicio de debates
  \item Propuestas, informaci�n interesante
  \end{itemize}
\end{itemize}

\end{frame}

%%---------------------------------------------------------------
\begin{frame}
\frametitle{Evaluaci�n}

\begin{itemize}
\item Cada actividad, entre un m�nimo y un m�ximo de puntos
\item Puntuaci�n m�nima en cada actividad para aprobar: \\
  0 si es optativa, mayor que 0 si es obligatoria
\item Puntuaci�n m�nima total para aprobar: 100 puntos
\item Puntuaci�n para sobresaliente: 200 puntos
\item Detalles en el programa de la asignatura
\end{itemize}

\end{frame}

%%---------------------------------------------------------------
\begin{frame}
\frametitle{Comentarios finales}

\begin{itemize}
\item Hay evaluaci�n de la asignatura...
\item ...y trabajo a realizar para ella

\item Esperamos que os sea agradable...
\item ...y que todos aprendamos durante estos meses

\item En esta asignatura hay especialmente cancha para vuestros intereses
\end{itemize}

\end{frame}
