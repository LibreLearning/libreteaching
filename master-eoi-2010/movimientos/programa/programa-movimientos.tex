%%
%% $Id$
%%

\documentclass[a4paper,12pt]{article}
\usepackage[utf8]{inputenc}
\usepackage[spanish]{babel}
\usepackage{geometry}
\usepackage[pdfborder=0 0 0]{hyperref}
\usepackage{url}

\title{Movimiento Open Access \\
Módulo 1: Nuevas tecnologías \\
Master de Propiedad Intelectual: \\
Retos y nuevos modelos de negocio en la sociedad de la información \\
EOI Escuela de Organización Industrial \\
Programa del curso abril 2010}
\author{Jesús M. González Barahona y Gregorio Robles}

%\date{}

\begin{document}
\maketitle

\newpage

\tableofcontents

\newpage

%--------------------------------------------------------
%--------------------------------------------------------
%--------------------------------------------------------
\section{Equipo de profesores}

\begin{itemize}
\item Jesús M. González Barahona
  \begin{itemize}
  \item Correo electrónico: jgb@libresoft.es
  \end{itemize}
\item Gregorio Robles
  \begin{itemize}
  \item Correo electrónico: grex@libresoft.es
  \end{itemize}
\item Miquel Vidal
  \begin{itemize}
  \item Correo electrónico: mvidal@libresoft.es
  \end{itemize}
\end{itemize}

%--------------------------------------------------------
%--------------------------------------------------------
%--------------------------------------------------------
%\section{Objetivos}

%\begin{itemize}
%\item 
%\end{itemize}

%\section{Metodología}


%--------------------------------------------------------
%--------------------------------------------------------
%--------------------------------------------------------
\section{Programa}

Programa de la asignatura (tentativo)

%--------------------------------------------------------
%--------------------------------------------------------
\subsection{01 - Presentación}

Presentación de los aspectos fundamentales de la asignatura. Información sobre las capacidades y habilidades de los alumnos en relación con esta asignatura.

%--------------------------------------------------------
\subsubsection{Sesión del 7 de mayo (0.5 horas)}

\begin{itemize}
\item Profesores: Jesús
\item Presentación: Presentación de la asignatura.
  %\item Material: Transparencias (tema ``01 - Presentación'')
\end{itemize}

%--------------------------------------------------------
%--------------------------------------------------------
\subsection{02 - Historia del software libre}

%--------------------------------------------------------
\subsubsection{Sesión del 8 de mayo (5 horas)}

Repaso de la historia del software libre.

\begin{itemize}
\item Profesores: Miquel
\end{itemize}


%--------------------------------------------------------
%--------------------------------------------------------
%--------------------------------------------------------
\section{Evaluación}

La evaluación final de la asignatura se realizará a partir de las evaluaciones de las diferentes actividades evaluables realizadas. Cada actividad será evaluada con una serie de puntos, según el parecer de los profesores de la asignatura, que seguirán los criterios generales espeficados más adelante. La nota final estará en función del número de puntos total obtenido, matizados según la opinión general de los profesores sobre los resultados obtenidos. Hay actividades obligatorias (para aprobar la asignatura, el número de puntos mínimo a conseguir en esa actividad es mayor que cero), y otras voluntarias (se puede aprobar la asignatura teniendo cero puntos en esa actividad).

%El número mínimo de puntos que un estudiante ha de tener para aprobar la asignatura es de 100 puntos. La puntuación mínima para obtener un notable es de 175 puntos, mientras que para sobresaliente el mínimo es de 250 puntos\footnote{Nota para alumnos matriculados en el máster de Sistemas Telemáticos ``antiguo'', a extinguir. Al ser la ``nueva'' asignatura de 3 créditos y la ``antigua'' de 4 créditos, requiere aproximadamente un 33\% de esfuerzo acidional. Por ello para los alumnos que cursan esta asignatura como parte del programa ``antiguo'' el número mínimo de puntos para aprobar la asignatura es de 130 puntos, para obtener un notable de 230 puntos, y para sobresaliente de 330 puntos. Estos alumnos podrán optar a tutorías adicionales si lo precisan para completar sus actividades, o ser orientados sobre ellas}.

%% \subsection{Actividades evaluables}

%% Cada actividad tendrá unos criterios de evaluación propios, seguń se indica a continuación. Para cada actividad habrá una puntuación mínima necesaria (0 si no es obligatoria, mayor que 0 si es obligatoria), una puntuación máxima, una descripción de la actividad y, cuando sea posible definirlos, unos criterios generales de puntuación. Se aconseja a los alumnos que pregunten a los profesores cualquier detalle que pueda no estar claro sobre en qué consisten las actividades, o sobre cómo se puntuarán. En general la evaluación tendrá siempre en cuenta cómo la realización de la actividad ha acercado al alumno a la consecución de los objetivos de la asignatura, y cómo éste ha mostrado las competencias o habilidades correspondientes.

%% \begin{itemize}
%% \item Examen final. Mínimo: 25 puntos, máximo: 50 puntos.

%% Examen sobre el temario de la asignatura.

%% \item Participación en los foros a cuestiones planteadas por los profesores. Mínimo: 20 puntos, máximo: 75 puntos.

%% Participación en los foros en el Moodle de cada tema en cuestiones planteadas por los profesores durante las videoconferencias.

%% \item Preguntas de autoevaluación. Mínimo: 0 puntos, máximo: 30 puntos.

%% Moodle ofrece la posibilidad de realizar tests de autoevaluación para que los alumnos puedan cerciorarse de que han comprendido y asimilado correctamente los conceptos presentados en la asignatura. Esta actividad consiste en realizar los tests creados por compañeros vuestros en cursos pasados.

%% \item Blog sobre la asignatura. Mínimo: 0 puntos, máximo: 40 puntos.

%% Mantenimiento de un blog sobre la asignatura. Deberá incluir como mínimo del orden de una nota cada semana o 10 días (preferentemente dos o más cada semana o 10 días). Los temas pueden ser los que se van viendo en la asignatura, temas de actualidad relacionados con software libre, aspectos que le resulten interesantes al alumno sobre software libre, etc. El alumno abrirá el blog donde quiera (o abrirá una categoría en su blog, si ya lo tiene), y enviará la url del feed correspondiente a los profesores.

%% \item Juego docente con e-Adventure. Mínimo: 0 puntos, máximo: 100 puntos.

%% e-Adventure es una plataforma para crear juegos docentes. Esta actividad, que se puede realizar en grupo, tiene como objetivo la creación de un juego docente sobre algún tema relacionado con la asignatura.

%% \item Recrear un artículo de investigación sobre software libre. Mínimo: 0 puntos, máximo: 100 puntos.

%% Hay muchos artículos científicos que estudian el software libre utilizando fuentes de datos públicas, tal y como se enseña en esta asignatura. Esta tarea consiste en recrear esos artículos con datos actualizados para comprobar la veracidad de los artículos y si sus conclusiones siguen siendo válidas años después.

%% \item Caso de estudio un proyecto de software libre. Mínimo: 15 puntos, máximo: 75 puntos.

%% En esta asignatura se presentan métodos y herramientas para auditar y estudiar proyectos de software libre. Esta tarea consiste en elegir un proyecto y realizar un caso de estudio.

%% \item Trabajos sobre temas concretos de software libre. Mínimo: 0 puntos, máximo: 75 puntos.

%% Los estudiantes pueden realizar un trabajo, individual o colectivo sobre algún tema relacionado con la asignatura. Se podrá entregar como informe o artículo, o como presentación en video (puede ser, por ejemplo, un screencast). Algunos alumnos pueden ser invitados a exponer su trabajo en clase. Ejemplos de temas de trabajos: línea histórica de las licencias de software libre, clasificación de empresas según modelos de negocio, tipos de comunidades de desarrollo de software libre.

%% \item Participar en concurso sobre la asignatura en el IRC. Mínimo: 0 puntos, máximo: 50 puntos. \[Propuesta tentativa\]

%% Se trata de un concurso tipo trivial a través del IRC. Dos alumnos compiten por responder correctamente a preguntas sobre el temario, ganando el punto aquél que responda antes correctamente. Se está implementando una plataforma para llevar este tipo de actividades de manera autónoma. Si el proyecto está listo para ser probado, se podra realizar esta actividad.
%% \end{itemize}

%% \subsection{Fechas}

%% Evaluación final (todas las actividades habrán de estar terminadas): 12 mayo 2010

%% Segunda convocatoria de evaluación: junio 2010



\end{document}
