\documentclass[a4paper]{article}
%\usepackage[T1]{fontenc}
\usepackage[utf8]{inputenc}
\usepackage{url}
\usepackage[hypertex,colorlinks]{hyperref}
%\usepackage{html}
%\usepackage{hthtml}
\usepackage{geometry}

% Comments (optional argument is author of comment)
\newcommand{\comments}[2][?]{
  \begin{quote}
    \textbf{Comment (#1):} {\em #2}
  \end{quote}
  }

% \name is for ``special names'', like procedure or variable names.
\newcommand{\name}[1]{\texttt{\hturl{#1}}}

% Just a shortcut for links where the url appears as footnote.
\newcommand{\htflink}[2]{\htmladdnormallinkfoot{#1}{#2}}

\title{Introduction to Libre Software Communities\\
Master on Free Software Projects Development and Management \\
URJC - GSyC/Libresoft \\
\url{http://master.libresoft.es}}

\author{Daniel Izquierdo Cortázar}
\date{February 2011}

\sloppy
\begin{document}
\maketitle

\begin{abstract}
Course syllabus and learning program for the course ``Dynamics of Libre Software Communities'', of the Master on Free Software Projects Development and Management of the Universidad Rey Juan Carlos in collaboration with Igalia.

[This is an evolving document, until the course is finished and graded]
\end{abstract}

\tableofcontents

%%---------------------------------------------------------------------
%%---------------------------------------------------------------------
%%---------------------------------------------------------------------
\section{Schedule}



%%---------------------------------------------------------------------
\subsection{Session 1 - February 18-19, 2011}

\begin{itemize}
\item \textbf{In place session}
\item \textbf{Lecturer:} Daniel Izquierdo Cortázar
\item \textbf{Topics:}
  \begin {itemize}
    \item Presentation of the subject
    \item Introduction to libre software communities
    \item Introduction to LibreSoft tools
  \end{itemize}
\item \textbf{Exercises:}
  \begin{itemize}
    \item Questions in forum 
    \item Analysing first set of projects
  \end{itemize}
\end{itemize}

%%---------------------------------------------------------------------
\subsection{Session 2 - February 25-26, 2011}

\begin{itemize}
\item \textbf{In place session}
\item \textbf{Lecturer:} Felipe Ortega
\item \textbf{Topics:}
  \begin{itemize}
    \item Introduction to R 
    \item Statistical approach
    \item Other statements about communities
  \end{itemize}
\item \textbf{Exercises:}
  \begin{itemize}
    \item Questions in forum 
  \end{itemize}
\end{itemize}

%%---------------------------------------------------------------------
\subsection{Session 3 - March 4-5, 2011}

\begin{itemize}
\item \textbf{Distance session}
\item \textbf{Lecturer:} Igalia and Daniel Izquierdo Cortázar
\item \textbf{Topics:}
  \begin{itemize}
   \item Introduction to Latex
   \item Reviewing final work
  \end{itemize}
\end{itemize}

%%---------------------------------------------------------------------
\subsection{Session 4 - March 11-12, 2011}

\begin{itemize}
\item \textbf{In place session}
\item \textbf{Lecturer:} José Gato and Daniel Izquierdo Cortázar
\item \textbf{Topics:}
  \begin{itemize}
    \item Case study: the Linux kernel
    \item Academic studies
  \end{itemize}
\item \textbf{Exercises:}
  \begin{itemize}
    \item Questions in forum 
  \end{itemize}

\end{itemize}


\subsection{Session 5 - March 25-26, 2011}

\begin{itemize}
\item \textbf{In place session}
\item \textbf{Lecturer:} Daniel Izquierdo Cortázar
\item \textbf{Topics:}
  \begin{itemize}
    \item Quality in Open Source
  \end{itemize}
\item \textbf{Exercises:}
  \begin{itemize}
    \item Questions in forum 
  \end{itemize}

\end{itemize}


\subsection{Session 6 - April 1-2, 2011}

\begin{itemize}
\item \textbf{In place session}
\item \textbf{Lecturer:} Felipe Ortega and Miquel Vidal
\item \textbf{Topics:}
  \begin{itemize}
    \item Wikipedia
  \end{itemize}
\item \textbf{Exercises:}
  \begin{itemize}
    \item Questions in forum 
    \item Presentation of final work
  \end{itemize}

\end{itemize}

%%---------------------------------------------------------------------
%%---------------------------------------------------------------------
%%---------------------------------------------------------------------
 \section{Grading plan}

 Each activity contributing to the grading of the course has its own evaluation 
criteria, as described below. Each of these activities has a minimum and maximum 
grading. If the minimum grading is 0, the activity is optional. Otherwise, the 
activity is mandatory, and has to be graded al least with the minimum to pass 
the course. Each activity has also a description, and when possible, some general 
grading criteria. In any case, the final grade for the course will also depend on 
the continuous observation of the instructors on the outcomes and progress of students.

 Students should ask instructors about any detail which may not be clear to them, 
either about the general grading plan, or about specific aspects of the activities. 
As a general rule, evaluation will have into account how the activity and its results 
show that the student has come close to the competences, knowledge and skills expected 
for the course.

 The student can consider that the next table will be used as a (minimum) guideline for assigning marks:

 \begin{itemize}
 \item Pass (``aprobado''): 60
 \item Good (``notable''): 80
 \item Excellent (``sobresaliente): 105
 \end{itemize}

 \begin{itemize}
  \item \textbf{Exercises (answered in forum)}. \\
   Minimum: 5 points, maximum: 20 points.




  \item \textbf{Specific report}. \\
   Minimum: 50 points, maximum: 100 points

 \end{itemize}


 \begin{itemize}
  \item The specific report is also divided in:
  \begin{itemize}
    \item English 5
    \item Community 5
    \item Latex 10
    \item Presentation 10
    \item Work itself 70
  \end{itemize}

 \end{itemize}



\end{document}
