\documentclass[a4paper]{article}
%\usepackage[T1]{fontenc}
\usepackage[utf8]{inputenc}
\usepackage{url}
%\usepackage[hypertex,colorlinks]{hyperref}
%\usepackage{html}
%\usepackage{hthtml}
\usepackage{geometry}

% Comments (optional argument is author of comment)
%\newcommand{\comments}[2][?]{
%  \begin{quote}
%    \textbf{Comment (#1):} {\em #2}
%  \end{quote}
%  }

% \name is for ``special names'', like procedure or variable names.
\newcommand{\name}[1]{\texttt{\hturl{#1}}}

% Just a shortcut for links where the url appears as footnote.
\newcommand{\htflink}[2]{\htmladdnormallinkfoot{#1}{#2}}

\title{MSWL Advanced Development \\
Master on libre software \\
URJC - GSyC/Libresoft \\
\url{http://master.libresoft.es}}

\author{Roberto Calvo}
\date{April 2013}

\sloppy
\begin{document}
\maketitle

\begin{abstract}
Course's syllabus and learning program for the course ``Advanced
Development'', of the Master on libre software of the Universidad Rey
Juan Carlos.


[This is an evolving document, until the course is finished and graded]
\end{abstract}

\tableofcontents

%%----------------------------------------------------------------
%%----------------------------------------------------------------
%%----------------------------------------------------------------
\section{Course topics and schedule}

%%----------------------------------------------------------------
%%----------------------------------------------------------------
\subsection{00 - Presentation of the course}

Presentation of the main aspects of the course, and specially those
related to administrative issues, evaluation, etc.

%%----------------------------------------------------------------
\subsubsection{4th April, 2013 (0.5 hours)}


\subsection{01 - Android Beginning}


\subsubsection{4th April, 2013 (1 hours)}

\begin{itemize}
\item \textbf{Lecturers:} Roberto Calvo
\item \textbf{Presentation:} ``The Android Project''
  \begin{itemize}
  \item \textbf{Discussion:} The history and evolution of the Android
    project
  \item \textbf{Requirements:} 
  \item \textbf{Supporting material:} Slides ``Android Project''
  \end{itemize}
\item \textbf{Assignment (results in website):}
  \begin{itemize}
  \item You can write a post in the blog describing your opininon
    about the community and policy android project
  \end{itemize}
\end{itemize}


%%----------------------------------------------------------------                                                                                                                     
\subsubsection{4th April, 2013 (0.5 hours)}

\begin{itemize}
\item \textbf{Lecturers:} Roberto Calvo
\item \textbf{Presentation:} ``Install Android SDK \& Eclipse''
  \begin{itemize}
  \item \textbf{Discussion:} How install the Android SDK in GNU/Linux
    systems.
  \item \textbf{Requirements:} It's necessary get all the files
    already download in the laptops.
  \item \textbf{Supporting material:}
  \end{itemize}
\item \textbf{Assignment (results in website):}
  \begin{itemize}
  \item Any kind of activity
    related to this talk (blog post, video, etc)
  \end{itemize}
\end{itemize}

%%----------------------------------------------------------------

\subsubsection{4th April, 2013 (2 hours)}

\begin{itemize}
\item \textbf{Lecturers:} Roberto Calvo
\item \textbf{Presentation:} ``Hello World and Layouts''
  \begin{itemize}
  \item \textbf{Discussion:} First steps with Android environment.
  \item \textbf{Requirements:} 
  \item \textbf{Supporting material:}
  \end{itemize}
\item \textbf{Assignment (results in website):}
  \begin{itemize}
  \item Any kind of activity
    related to this talk (blog post, video, etc)
  \end{itemize}
\end{itemize}


\subsection{02 - Activity, ListActivity and Layouts}

There are many activities type in Android. We will see the most
popular as Activity and ListActivity. Also, we will study the way to
create graphic interface in Android, Layouts

%%----------------------------------------------------------------
\subsubsection{11th April, 2013 (4 hours)}

\begin{itemize}
\item \textbf{Lecturers:} Roberto Calvo
\item \textbf{Presentation:} `` Activity, ListActivity and Layouts''
  \begin{itemize}
  \item \textbf{Discussion:}
  \item \textbf{Requirements:} 
  \item \textbf{Supporting material:} Slides ``Anatomy of Android
    Application'' and ``Layouts''
  \end{itemize}
\item \textbf{Assignment (results in website):} 
  \begin{itemize}
  \item All the exercises must be uploaded to GIT repository
  \end{itemize}
\end{itemize}


\subsection{03 - Maps and GeoLocation}

Maps and Geolocation is a great feature that every app must have. We
will learning to development apps that use GPS and Google Maps.


%%----------------------------------------------------------------
\subsubsection{18th April, 2013 (4 hours)}

\begin{itemize}
\item \textbf{Lecturers:} Roberto Cavo
\item \textbf{Presentation:} ``Maps and Geolocation''
  \begin{itemize}
  \item \textbf{Requirements:} AVD with Google APIs
  \item \textbf{Supporting material:} Slides ``Maps and Geolocation''
  \end{itemize}
\item \textbf{Assignment (results in website):} 
  \begin{itemize}
  \item All the exercises must be uploaded to GIT repository
  \end{itemize}
\end{itemize}


\subsection{04 - Working in background }

How to use the android services to work in background. Also we will
see the correct way to use AsynTask structure to sync ActivityList.

%%----------------------------------------------------------------
\subsubsection{25th April, 2013 (4 hours)}

\begin{itemize}
\item \textbf{Lecturers:} Roberto Calvo
\item \textbf{Presentation:} ``Working in background''
  \begin{itemize}
\item \textbf{Requirements:} 
  \item \textbf{Supporting material:} Slides ``Working in background''
  \end{itemize}
\item \textbf{Assignment (results in website):} 
 \begin{itemize}
   \item All the exercises must be uploaded to GIT repository
 \end{itemize}
\end{itemize}

\subsection{05 - Preferences, Dialogs, Menus, Sign Application}

How to use preference sdk to save and load data of the
application. Dialogs and Menus are two important pieces of the
application. Also we study the process to sign applications.

%%----------------------------------------------------------------
\subsubsection{9th May, 2013 (4 hours)}

\begin{itemize}
\item \textbf{Lecturers:} Roberto Calvo
\item \textbf{Presentation:} ``Preferences, Dialogs, Menus, Sign Application''
  \begin{itemize}
  \item \textbf{Requirements:} 
  \item \textbf{Supporting material:} Slides ``Application Environment'', sample source code 
  \end{itemize}
\item \textbf{Assignment (results in website):} 
  \begin{itemize}
  \item All the exercises must be uploaded to GIT repository
  \end{itemize}
\end{itemize}


\section{Grading}

This section details the criteria for grading the course, the
deadlines for the different activities, and the submission details for
the activities that require them.

\subsection{Evaluation criteria}
\label{sub:evaluation-criteria}

Each activity contributing to the grading of the course has its own
evaluation criteria, as described below. Each of these activities has
a minimum and maximum grading. If the minimum grading is 0, the
activity is optional. Otherwise, the activity is mandatory, and has to
be graded al least with the minimum to pass the course. Each activity
has also a description, and when possible, some general grading
criteria. In any case, the final grade for the course will also depend
on the continuous observation of the instructors on the outcomes and
progress of students.

Students should ask instructors about any detail which may not be
clear to them, either about the general grading plan, or about
specific aspects of the activities. As a general rule, evaluation will
have into account how the activity and its results show that the
student has come close to the competences, knowledge and skills
expected for the course.

The student can consider that the next table will be used as a
(minimum) guideline for assigning marks:

\begin{itemize}
\item Pass (``aprobado''): 200
\item Good (``notable''): 300
\item Excellent (``sobresaliente''): 400
\end{itemize}

\begin{itemize}
\item \textbf{Blog entries}. \\
  Minimum: 0 points, maximum: 50 points

  Blog entries specifically related to the course, and marked as
  such. 10 points for each post. The tag used for that is mswl-ad.
  You should notify the post and url in a subject forum.

\item \textbf{Android weekly exercises }. \\
  Minimum: 80 points, maximum: 250 points

There is a programming exercise per week about Android. The maximum
value of each exercise is 50 points. Usually, the exercise must be
submitted before the next session. If you decide submit the exercise
after next session, the maximum value of exercise is 25 points.

We will decide if the exercise are submited in time looking at the
datetime of git commit. Each exercise must be approved individually
(to approve the exercise is necessary to obtain half of the maximum score)

All exercises are mandatory and the hard deadline is 9th of June 2013

\item \textbf{Android application}. \\
  Minimum: 75 points, maximum: 150 points

Development of an Android Application with the requeriments that are 
described in this document.

\end{itemize}

\subsection{Submission deadlines}

All activities to be graded in June must be completed and submitted 
the 9th of June

\subsection{Submission details}

Please, consider the details below for submitting the different
activities for evaluation (for those not specified in this list,
nothing special is needed for submission).


\begin{itemize}
\item As a summary of all the activities, a ``Summary of activities
  for evaluation'' should be sent. This summary should be uploaded to
  the corresponding resource in the Moodle site for this course, and
  should include the following data: 
  \begin{itemize}
  \item \textbf{Name:} Full name of the student (as ``family name'', ``given name'')
  \item \textbf{Blog entries:} Url of the blog entries for this course (HTML, not RSS version).
  \item \textbf{Source code and project files in the Git repository:} Id for commits
    to the repository where the source code is hosted, and summary of
    the main contributions to it related to this course, including
    links to the repository and commit ids if appropriate. 
  \end{itemize}
\end{itemize}

%%----------------------------------------------------------------
%%----------------------------------------------------------------
%%----------------------------------------------------------------
\section{Assignments and activities}

%%----------------------------------------------------------------
%%----------------------------------------------------------------

\subsection{Android Application }
\label{sub:gnome}

You will have to develop an android application written in JAVA.
\\
All the development will be done using Git version control, and all
the code will be publicly available in a Git repository, with frequent
commits.
\\
We provide you the necessary libraries for JSON and XML parser.
\\
\subsubsection{Requeriments}

\begin{itemize}
\item The app must have at least once activity, once list activity and one map activity.
\item The app must incorporate GPS and maps.
\item The app must save preferences at internal storage.
\item The app must get information using WIFI/3G connection.
\end{itemize}

Optional items
\begin{itemize}
\item The app could have localization support in Spanish and English.
\item The app could have interactions with external or internal activities.
\end{itemize}

\vspace{0.4cm}

\textbf{Supporting material}

\begin{itemize}
\item Slides about Android development
\item Video tutorials
\item Samples of source code
\end{itemize}


%%----------------------------------------------------------------
%\subsubsection{Statements about economic aspects of libre software}
%\label{sub:statements-eco}

\end{document}
