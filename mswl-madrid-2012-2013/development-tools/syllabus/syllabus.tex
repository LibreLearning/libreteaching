\documentclass[a4paper]{article}
%\usepackage[T1]{fontenc}
\usepackage[utf8]{inputenc}
\usepackage{url}
\usepackage[hypertex,colorlinks]{hyperref}
%\usepackage{html}
%\usepackage{hthtml}
\usepackage{geometry}

% Comments (optional argument is author of comment)
\newcommand{\comments}[2][?]{
  \begin{quote}
    \textbf{Comment (#1):} {\em #2}
  \end{quote}
  }

% \name is for ``special names'', like procedure or variable names.
\newcommand{\name}[1]{\texttt{\hturl{#1}}}

% Just a shortcut for links where the url appears as footnote.
\newcommand{\htflink}[2]{\htmladdnormallinkfoot{#1}{#2}}

\title{MSWL Development Tools \\
Master on libre software \\
URJC - GSyC/Libresoft \\
\url{http://master.libresoft.es}}

\author{}
\date{September 2012}

\sloppy
\begin{document}
\maketitle

\begin{abstract}
Course syllabus and learning program for the course ``Development
Tools'', of the Master on libre software of the Universidad Rey Juan
Carlos.

[This is an evolving document, until the course is finished and graded]
\end{abstract}

\tableofcontents

%%----------------------------------------------------------------
%%----------------------------------------------------------------
%%----------------------------------------------------------------
\section{Course topics and schedule}

%%----------------------------------------------------------------
%%----------------------------------------------------------------
\subsection{00 - Presentation of the course}

Presentation of the main aspects of the course, and specially those
related to administrative issues, evaluation, etc.

%%----------------------------------------------------------------
\subsubsection{September 13, 2012 (0.5 hours)}

\subsection{01 - Version control with Git}

Git is a distributed version control system. This class will introduce
the main concepts about version control and about Git. This is a
pre-requisite for the rest of the course.

%%----------------------------------------------------------------
\subsubsection{September 6, 2012 (2 hours)}

\begin{itemize}
\item \textbf{Lecturers:} Gregorio Robles
\item \textbf{Presentation:} ``Git''
  \begin{itemize}
  \item \textbf{Discussion:} Existing services to setup a Git
    repository.
  \item \textbf{Requirements:} Running Git installation
  \item \textbf{Supporting material:} Slides ``Git''
  \end{itemize}
\item \textbf{Assignment (results in website):}
  \begin{itemize}
  \item Registration in a Git
    public service and creation of a repository for the rest of the
    course.
  \item Any kind of activity
    related to this talk (blog post, video, etc)
  \end{itemize}
\end{itemize}

\subsection{02 - Introduction to Python}

Python is a programming language, very popular in libre software
community. We will see the main concepts of the language and will
create a small program, to be stored in the Git repository.

%%----------------------------------------------------------------
\subsubsection{September 6, 2012 (2 hours)}

\begin{itemize}
\item \textbf{Lecturers:} Gregorio Robles
\item \textbf{Presentation:} ``Introduction to Python''
  \begin{itemize}
  \item \textbf{Discussion:} Python vs. other programming
    languages. What are the advantages and drawbacks of dynamic
    typing?
  \item \textbf{Requirements:} Running Python 2.x development
    environment and Git
  \item \textbf{Supporting material:} Slides ``Introduction to Python''
  \end{itemize}
\item \textbf{Assignment (results in website):} 
  \begin{itemize}
  \item Creation of a small program which reads its command line
    arguments.
  \item Upload of the program to the Git repository
  \item Any kind of activity
    related to this talk (blog post, video, etc)
  \end{itemize}
\end{itemize}


\subsection{03 - Developing a web crawler in Python}

We will some advanced Python coding techniques and libraries, to
develop a web crawler. The web crawler  will track updates of a web
page.

%%----------------------------------------------------------------
\subsubsection{September 27, 2012 (4 hours)}

\begin{itemize}
\item \textbf{Lecturers:} Israel Herraiz
\item \textbf{Presentation:} ``Developing a web crawler in Python''
  \begin{itemize}
  \item \textbf{Requirements:} Running Python 2.X installation and Git
  \item \textbf{Supporting material:} Slides ``Developing a web
    crawler in Python'', sample source code
  \end{itemize}
\item \textbf{Assignment (results in website):} 
  \begin{itemize}
  \item First functional iteration of the web crawler
  \item Upload of the program to the Git repository
  \item Any kind of activity
    related to this talk (blog post, video, etc)
  \end{itemize}
\end{itemize}


\subsection{04 - Introduction to Integrated Development Environments}

How to use Python with the Eclipse IDE and to Redmine to support and
assist the development of the web crawler started in lecture 03.
%%----------------------------------------------------------------
\subsubsection{September 13, 2012 (4 hours)}

\begin{itemize}
\item \textbf{Lecturers:} Micael Gallego
\item \textbf{Presentation:} ``Introduction to Eclipse''
  \begin{itemize}
\item \textbf{Requirements:} Running Python 2.x development
  environment and Eclipse
  \item \textbf{Supporting material:} Slides ``Introduction to Eclipse''
  \end{itemize}
\item \textbf{Assignment (results in website):} Eclipse project for
  the web crawler application
\end{itemize}

\subsubsection{September 20, 2012 (4 hours)}

\begin{itemize}
\item \textbf{Lecturers:} Francisco Gort\'azar
\item \textbf{Requirements:} Running Python 2.x development
  environment, Eclipse
\item \textbf{Presentation:} ``Web-based IDEs''
  \begin{itemize}
  \item \textbf{Supporting material:} Slides ``Web-based IDEs''
  \end{itemize}
\item \textbf{Assignment (results in website):} Code reviews assisted
  by the Redmine tickets system
\end{itemize}


\subsection{05 - Introduction to GNOME}

GNOME development. Libraries and tools used in the development of
desktop applications.

%%----------------------------------------------------------------
\subsubsection{October 4, 2012 (4 hours)}

\begin{itemize}
\item \textbf{Lecturers:} Carlos Garc\'ia Campos
\item \textbf{Presentation:} ``Introduction to GNOME''
  \begin{itemize}
  \item \textbf{Requirements:} GCC and Autotools
  \item \textbf{Supporting material:} Slides ``Introduction to
    GNOME'', sample source code 
  \end{itemize}
\item \textbf{Assignment (results in website):} 
  \begin{itemize}
  \item GNOME desktop application (optional assignment)
  \end{itemize}
\end{itemize}


\section{Grading}

This section details the criteria for grading the course, the
deadlines for the different activities, and the submission details for
the activities that require them.

\subsection{Evaluation criteria}
\label{sub:evaluation-criteria}

Each activity contributing to the grading of the course has its own
evaluation criteria, as described below. Each of these activities has
a minimum and maximum grading. If the minimum grading is 0, the
activity is optional. Otherwise, the activity is mandatory, and has to
be graded al least with the minimum to pass the course. Each activity
has also a description, and when possible, some general grading
criteria. In any case, the final grade for the course will also depend
on the continuous observation of the instructors on the outcomes and
progress of students.

Students should ask instructors about any detail which may not be
clear to them, either about the general grading plan, or about
specific aspects of the activities. As a general rule, evaluation will
have into account how the activity and its results show that the
student has come close to the competences, knowledge and skills
expected for the course.

The student can consider that the next table will be used as a
(minimum) guideline for assigning marks:

\begin{itemize}
\item Pass (``aprobado''): 150
\item Good (``notable''): 250
\item Excellent (``sobresaliente''): 350
\end{itemize}

\begin{itemize}
\item \textbf{Blog entries}. \\
  Minimum: 0 points, maximum: 50 points

  Blog entries specifically related to the course, and marked as
  such. 10 points for each post.. The tag used for that is mswl-devtools.

\item \textbf{Python application}. \\
  Minimum: 60 points, maximum: 250 points

Source code of the ``spider'' assignment. Grading will vary taking in
account the different topics covered in the lectures (easy deployment,
documentation, testing, clarity of the code, etc). The application
should be delivered through the Git repository. The activity in the
Git repository will also be used for grading. Correct usage of the Git
repository is mandatory.

\item \textbf{Eclipse and Redmine integration}. \\
  Minimum: 60 points, maximum: 100 points

Creation of the Eclipse project, usage of Redmine for maintenance and
code reviews. Usage of Git through Eclipse and availability of the
project files in the Git repository.


\item \textbf{GNOME application (optional)}. \\
  Minimum: 0 points, maximum: 100 points

Development of a GNOME desktop application, in C.

\end{itemize}

\subsection{Submission deadlines}

All activities to be graded in May must be completed and submitted by December 23rd 2012.

\subsection{Submission details}

Please, consider the details below for submitting the different activities for evaluation (for those not specified in this list, nothing special is needed for submission).

\begin{itemize}
\item As a summary of all the activities, a ``Summary of activities
  for evaluation'' should be sent. This summary should be uploaded to
  the corresponding resource in the Moodle site for this course, and
  should include the following data: 
  \begin{itemize}
  \item \textbf{Name:} Full name of the student (as ``family name'', ``given name'')
  \item \textbf{Blog entries:} Url of the blog entries for this course (HTML, not RSS version).
  \item \textbf{Source code and project files in the Git repository:} Id for commits
    to the repository where the source code is hosted, and summary of
    the main contributions to it related to this course, including
    links to the repository and commit ids if appropriate. 
  \item \textbf{Redmine integration:} URL to the activity summary in
    the Redmine site.   
  \end{itemize}
\end{itemize}

%%----------------------------------------------------------------
%%----------------------------------------------------------------
%%----------------------------------------------------------------
\section{Assignments and activities}

%%----------------------------------------------------------------
%%----------------------------------------------------------------
\subsection{Spider to track the updates of a web page}
\label{sub:python}

You will have to write a Python application that get the current
version of a web page, compare against a local cache of the page, and
if changed, retrieve the new version of the page and write in the
standard output a summary of the changes.

The spider must visit all the links below the current page. The log of
changes displayed in the standard output will contain a list of all
the links that have been changed, and the number of lines of
difference between the two versions.

The application must be easily installable using Python standard
deployment methods, must be properly document and must include a
battery of tests to check that it is working as expected.

All the development will be done using Git version control, and all
the code will be publicly available in a Git repository, with frequent
commits.

\textbf{Supporting material}

\begin{itemize}
\item Slides about advanced Python development
\item Snippets of sample source code
\end{itemize}

\subsection{Integration with Eclipse and Redmine}
\label{sub:eclipse}

You will have to use Eclipse for the development of the spider/web
crawler. You will have to interact with the Git repository through
Eclipse, and take advantage of the IDE features of Eclipse for Python
(code completion, etc).

The Eclipse project files will be added to the Git repository, either
in a separate branch or in a selected directory.

The maintenance of the spider, tasks management (bugs, new features),
documentation of the application and code reviews will be done using
the web tools available in the Redmine site (Redmine provided by the
lecturers).

\textbf{Supporting material}

\begin{itemize}
\item Slides about Eclipse and Redmine
\item Redmine site at URJC
\end{itemize}


\subsection{GNOME desktop application}
\label{sub:gnome}

You will have to develop an application written in C for the GNOME
desktop.

All the development will be done using Git version control, and all
the code will be publicly available in a Git repository, with frequent
commits.

This assignment is optional.


\textbf{Supporting material}

\begin{itemize}
\item Slides about GNOME development
\item Sample source code
\end{itemize}


%%----------------------------------------------------------------
%\subsubsection{Statements about economic aspects of libre software}
%\label{sub:statements-eco}

\end{document}
