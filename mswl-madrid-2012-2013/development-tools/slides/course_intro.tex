% Created 2011-03-09 Wed 23:43
\documentclass[11pt]{beamer}

      \mode<presentation>

      \usetheme{PaloAlto}

      \usecolortheme{whale}

      \usefonttheme{structurebold}

%\usebackgroundtemplate{\includegraphics[width=\paperwidth]{format/libresoft-bg-soft.png}}
      \beamertemplateballitem

      \setbeameroption{show notes}
      \usepackage[utf8]{inputenc}

      \usepackage[T1]{fontenc}

      \usepackage{hyperref}

      \usepackage{color}
      \usepackage{listings}
      \lstset{numbers=none,language=[ISO]C++,tabsize=4,
  frame=single,
  basicstyle=\small,
  showspaces=false,showstringspaces=false,
  showtabs=false,
  keywordstyle=\color{blue}\bfseries,
  commentstyle=\color{red},
  }

      \usepackage{verbatim}

      \institute{israel.herraiz@upm.es \\ Universidad Politécnica de Madrid}

       \subject{MSWL Development and Tools}

\usepackage[utf8]{inputenc}
\usepackage[T1]{fontenc}
\usepackage{fixltx2e}
\usepackage{graphicx}
\usepackage{longtable}
\usepackage{float}
\usepackage{wrapfig}
\usepackage{soul}
\usepackage{textcomp}
\usepackage{marvosym}
\usepackage{wasysym}
\usepackage{latexsym}
\usepackage{amssymb}
\usepackage{hyperref}
\tolerance=1000
\providecommand{\alert}[1]{\textbf{#1}}

\title{Course: MSWL Development and Tools}
\author{Israel Herraiz}

\begin{document}

\maketitle


\section{Goal of this course}
\label{sec-1}

   
\begin{frame}[fragile]\frametitle{Goal of this course}
\label{sec-1_1}

To get to know what tools, technologies, libraries, frameworks,
languages are being used in the libre software community for software
development.

This includes:

\begin{itemize}

\item Version control with Git
\item Python programming
\item Eclipse, Redmine and IDEs
\item The case of GNOME: developing an application for the Desktop

\end{itemize}
\end{frame}

\begin{frame}[fragile]\frametitle{Course sessions}
\label{sec-1_2}

\begin{enumerate}
\item Version control with Git (Sep 9th)
\item Introduction to Python (Sep 9th)
\item Advance Python (developing a web spider) (Sep 16th)
\item Eclipse and Redmine (Sep 23rd and Oct 7th)
\item GNOME development (Sep 30th)
\end{enumerate}


\end{frame}
\begin{frame}[fragile]\frametitle{Assignments}
\label{sec-1_3}

The following assignments are mandatory for this course
\begin{itemize}
\item Developing a web spider in Python
\item Setting up and using a Git repository
\item Eclipse and Redmine integration in the development process
\end{itemize}


The following assignments are optional
\begin{itemize}
\item GNOME desktop application written in C
\item Blogging about topics highlighted during the course
\end{itemize}
\end{frame}

\begin{frame}[fragile]\frametitle{Grading}
\label{sec-1_4}

The following scale will be used for grading
\begin{itemize}
\item Pass (``aprobado''): 150 points
\item Good (``notable''): 250 points
\item Excellent (``sobresaliente''): 350 points
\end{itemize}

These points can be earned through the following assignments
\begin{itemize}
\item Blog entries (up to 50 points)
\item Python application (up to 250 points, 60 points minimum)
\item Eclipse and Redmine (up to 100 points, 60 points minimum)
\item GNOME application (up to 100 points)
\end{itemize}
\end{frame}
\end{document}