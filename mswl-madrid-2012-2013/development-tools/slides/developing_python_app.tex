% Created 2011-03-09 Wed 23:43
\documentclass[11pt]{beamer}

      \mode<presentation>

      \usetheme{Warsaw}

      \usecolortheme{lily}

\usebackgroundtemplate{\includegraphics[width=\paperwidth]{format/libresoft-bg-soft.png}}
      \beamertemplateballitem

      \setbeameroption{show notes}
      \usepackage[utf8]{inputenc}

      \usepackage[T1]{fontenc}

      \usepackage{hyperref}

      \usepackage{color}
      \usepackage{listings}
      \lstset{numbers=none,language=[ISO]C++,tabsize=4,
  frame=single,
  basicstyle=\small,
  showspaces=false,showstringspaces=false,
  showtabs=false,
  keywordstyle=\color{blue}\bfseries,
  commentstyle=\color{red},
  }

      \usepackage{verbatim}

      \institute{israel.herraiz@upm.es \\ Universidad Politécnica de Madrid}

       \subject{MSWL Development and Tools}

\usepackage[utf8]{inputenc}
\usepackage[T1]{fontenc}
\usepackage{fixltx2e}
\usepackage{graphicx}
\usepackage{longtable}
\usepackage{float}
\usepackage{wrapfig}
\usepackage{soul}
\usepackage{textcomp}
\usepackage{marvosym}
\usepackage{wasysym}
\usepackage{latexsym}
\usepackage{amssymb}
\usepackage{hyperref}
\tolerance=1000
\providecommand{\alert}[1]{\textbf{#1}}

\title{Developing a web crawler written in Python}
\author{Israel Herraiz}

\begin{document}

\maketitle

\begin{frame}
\frametitle{Outline}
\setcounter{tocdepth}{3}
\tableofcontents
\end{frame}

\lstset{ %
language=python,
basicstyle=\footnotesize\ttfamily,       % the size of the fonts that are used
numbers=none,                   % where to put the line-numbers
numberstyle=\ttfamily,      % the size of the fonts that are used
stepnumber=2,                   % the step between two
numbersep=5pt,                  % how far the line-numbers are from
showspaces=false,               % show spaces adding particular
showstringspaces=false,         % underline spaces within strings
showtabs=false,                 % show tabs within strings adding
frame=none,                   % adds a frame around the code
tabsize=2,                      % sets default tabsize to 2 spaces
captionpos=b,                   % sets the caption-position to bottom
breaklines=true,                % sets automatic line breaking
breakatwhitespace=false,        % sets if automatic breaks should only
title=\lstname,                 % show the filename of files included
escapeinside={\%*}{*)},         % if you want to add a comment within
morekeywords={*,...}            % if you want to add more keywords to
}

\section{How to properly deploy using \texttt{setuptools}}
\label{sec-1}

   
\begin{frame}[fragile]\frametitle{What is \texttt{setuptools}?}
\label{sec-1_1}


Python package to install and deploy Python applications. 


Features

\begin{itemize}
\item Register and upload your package in PyPi
\item Take care of dependencies

\begin{itemize}
\item Generate dependencies to build dist packages
\item Retrieve missing deps on installing
\end{itemize}

\item Properly install your app in the system

\begin{itemize}
\item Windows, GNU/Linux, etc
\item Easy creation of packages for distributions
\end{itemize}

\end{itemize}
\end{frame}
\begin{frame}[fragile]\frametitle{What is PyPi?}
\label{sec-1_2}


The Python Package Index, Official repository for Python apps.


Database with information about Python programs. Example:


\begin{itemize}
\item \href{http://pypi.python.org/pypi/mlstats/}{http://pypi.python.org/pypi/mlstats/}
\end{itemize}

Why should you care?

\begin{itemize}
\item Visibility for your app
\item If it is a library, it can be used in a standard way with other
  libraries and apps
\end{itemize}
\end{frame}
\begin{frame}[fragile]\frametitle{Using \texttt{setuptools} with your own programs}
\label{sec-1_3}


The main file of your app must be named \texttt{setup.py}. We have also the
main file of your app, called \texttt{mycraaawler} and a directory with own
modules for your app called \texttt{pymycraaawler}.

So this is the sources tree
\begin{itemize}
\item \texttt{mycraaawler} (Python script)
\item \texttt{setup.py}
\item \texttt{pymycraaawler}

\begin{itemize}
\item \texttt{\_\_init\_\_.py}
\end{itemize}

\item And other files (\texttt{COPYING}, \texttt{README}, etc)
\end{itemize}
\end{frame}
\begin{frame}[fragile]\frametitle{\texttt{setup.py}}
\label{sec-1_4}

\begin{lstlisting}
from setuptools import setup, find_packages

setup( name = "MyCraaawler",
       version = "0.1",
       packages = find_packages(),
       scripts = ['mycraaaawler'],
       install_requires = ['BeatifulSoup'],
       package_data = { 'pymycraawler': [''], },
       author = "Sponge Bob",
       author_email = "bob@undersea.pineapple.com",
       description = "Scrapper for The New Atlantis",
       license = "",
       keywords = "",
       url = "",
       long_description = "",
       download_url = "", )
\end{lstlisting}
\end{frame}
\begin{frame}[fragile]\frametitle{Installing and creating dist packages}
\label{sec-1_5}


\texttt{python setup.py [command]}

\begin{itemize}
\item \texttt{install}
\item \texttt{easy\_install}
\item \texttt{bdist}
\item \texttt{sdist}
\item \texttt{check}
\item \texttt{register}
\item \texttt{upload}
\end{itemize}
\end{frame}
\begin{frame}[fragile]\frametitle{Configuration for automatic register and upload}
\label{sec-1_6}


You have to create a file named \texttt{\textasciitilde{}/.pypirc}
\begin{verbatim}
[distutils]
index-servers =
    pypi

[pypi]
username:your_username
password:your_password
\end{verbatim}
\end{frame}
\section{How do I get a file from the Internet?}
\label{sec-2}
\begin{frame}[fragile]\frametitle{Python modules for Internet communications}
\label{sec-2_1}


There are two different modules in Python to connect to a host in the Internet:
\begin{itemize}
\item \texttt{urllib}
\item \texttt{urllib2}
\end{itemize}
\end{frame}
\begin{frame}[fragile]\frametitle{Making sense of \texttt{urllib}, \texttt{urllib2} and co.}
\label{sec-2_2}

   
\begin{itemize}
\item \texttt{urllib} is deprecated in favor of \texttt{urllib2} (although in Python 3, both modules are present).
\item \texttt{urllib} provides \texttt{urlopen} which is similar to \texttt{open} but will  accept URLs
\item \texttt{urllib2} also provides \texttt{urlopen}
\item \texttt{urlparse} is useful to handle URLs (but we will not use it)
\end{itemize}
\end{frame}
\begin{frame}[fragile]\frametitle{Reading a file located in a remote host}
\label{sec-2_3}


\begin{lstlisting}
 import urllib2
 user_agent = "Mozilla/5.0 (X11; U; Linux x86_64; en-US) AppleWebKit/534.7 (KHTML, like Gecko) Chrome/7.0.517.41 Safari/534.7"

 _opener = urllib2.build_opener()
 _opener.addheaders = [('User-agent', user_agent)]
 raw_code = self._opener.open(article.url).read()
\end{lstlisting}
\end{frame}
\section{Parsing \texttt{HTML} using \texttt{BeatifulSoup}}
\label{sec-3}
\begin{frame}[fragile]\frametitle{About \texttt{BeatifulSoup}}
\label{sec-3_1}


\texttt{BeatifulSoup} is a \texttt{HTML} parser for Python. You didn't write that
awful page. You're just trying to get some data out of it. Right now,
you don't really care what \texttt{HTML} is supposed to look like.

Neither does this parser.
\begin{itemize}
\item Find out more at \href{http://www.crummy.com/software/BeautifulSoup/}{http://www.crummy.com/software/BeautifulSoup/}
\end{itemize}
\end{frame}
\begin{frame}[fragile]\frametitle{Example: getting all the \texttt{<div>} of a particular kind (called \texttt{article} in this case)}
\label{sec-3_2}


\begin{lstlisting}
from BeautifulSoup import BeautifulSoup as Soup

soup_code = Soup(raw_code)
divs = [div for div 
            in soup_code.findAll('div') 
            if div.get('id') == 'article']
\end{lstlisting}
\end{frame}
\begin{frame}[fragile]\frametitle{Example: get all links}
\label{sec-3_3}



\begin{lstlisting}
from BeautifulSoup import BeautifulSoup as Soup

soup_code = Soup(raw_code)
links = [link['href'] for link 
                      in soup_code.findAll('a') 
                      if link.has_key('href')]
\end{lstlisting}
\end{frame}
\begin{frame}[fragile]\frametitle{Example: Decode \texttt{HTML} entities}
\label{sec-3_4}



\begin{lstlisting}
from BeautifulSoup import BeautifulStoneSoup as StoneSoup

html_entity = "Cami&oacute;n"

python_str = StoneSoup(html_entity,
              convertEntities = StoneSoup.HTML_ENTITIES)
\end{lstlisting}
\end{frame}
\section{Command line arguments made easy with \texttt{argparse}}
\label{sec-4}
\begin{frame}[fragile]\frametitle{What is \texttt{argparse}?}
\label{sec-4_1}


The argparse module makes it easy to write user friendly command line interfaces
\begin{itemize}
\item New in version 2.7
\item Find out more at
   \href{http://docs.python.org/release/2.7/library/argparse.html}{http://docs.python.org/release/2.7/library/argparse.html}
\end{itemize}
\end{frame}
\begin{frame}[fragile]\frametitle{Example}
\label{sec-4_2}


Command line application \texttt{myapp [-n N] URL}

\begin{lstlisting}
import argparse
parser = argparse.ArgumentParser(description="Let's craaaawl the Internet")

parser.add_argument('url', nargs=1,
     help='target URL')

parser.add_argument('-n', '--number-of-levels', type=int, default=1,
     help='how deep the craaaawl will go')

args = parser.parse_args()

target_url = args.url.pop()
deep = args.number_of_levels
\end{lstlisting}
\end{frame}
\begin{frame}[fragile]\frametitle{Another example}
\label{sec-4_3}


Command line application \texttt{myapp [-w]}

\begin{lstlisting}
import argparse
parser = argparse.ArgumentParser(description='World domination 0.1')

parser.add_argument('-w', '--start-nuclear-war', action='store_true',
    help='should I start a thermonuclear war?')

args = parser.parse_args()

if args.launch_nuclear_war:
   world.dominate()
else:
   sleep()

\end{lstlisting}
\end{frame}

\end{document}