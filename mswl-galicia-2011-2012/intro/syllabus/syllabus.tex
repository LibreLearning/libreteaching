\documentclass[a4paper]{article}
%\usepackage[T1]{fontenc}
\usepackage[utf8]{inputenc}
\usepackage{url}
\usepackage[hypertex,colorlinks]{hyperref}
%\usepackage{html}
%\usepackage{hthtml}
\usepackage{geometry}

% Comments (optional argument is author of comment)
\newcommand{\comments}[2][?]{
  \begin{quote}
    \textbf{Comment (#1):} {\em #2}
  \end{quote}
  }

% \name is for ``special names'', like procedure or variable names.
\newcommand{\name}[1]{\texttt{\hturl{#1}}}

% Just a shortcut for links where the url appears as footnote.
\newcommand{\htflink}[2]{\htmladdnormallinkfoot{#1}{#2}}

\title{Introduction to Libre Software\\
Master on Free Software Projects Development and Management \\
URJC - GSyC/Libresoft \\
\url{http://docencia.etsit.urjc.es/moodle/course/view.php?id=39}}

\author{Jesus M. Gonzalez-Barahona, Gregorio Robles, Miquel Vidal, Teo Romera, Israel Herraiz}
\date{October 2011}

\sloppy
\begin{document}
\maketitle

\begin{abstract}
Course syllabus and learning program for the course ``Introduction to libre software'', of the Master on Free Software Projects Development and Management of the Universidad Rey Juan Carlos in collaboration with Igalia.

[This is an evolving document, until the course is finished and graded]
\end{abstract}

\tableofcontents

%%---------------------------------------------------------------------
%%---------------------------------------------------------------------
%%---------------------------------------------------------------------
\section{Schedule}

%%---------------------------------------------------------------------
%%---------------------------------------------------------------------
\subsection{Introduction}

%%---------------------------------------------------------------------
\subsubsection{October 21, 2011}

\begin{itemize}
\item \textbf{Lecturer:} Gregorio Robles
\item Introduction to the Master (by Chema Casanova, 1h)
\item Introduction to the subject, including Moodle and evaluation (1h)
\item Activity: "explain libre software to your grandmother" (1h)

Students are asked in groups to explain the concept of libre software as if
they had to explain it to their grandmother. The teacher writes down the 
ideas that arise and there is discussion. The importance of this activity
is to show the important facets of free software, especially those that are
related to the cooperation-competition and individual-collective dimensions.

\item Definition of libre software (slides) (30 min)

\item Venn diagram with concepts related to software (slides) (1 h)

\end{itemize}

%%---------------------------------------------------------------------
\subsubsection{October 22, 2011}

\begin{itemize}
\item \textbf{Lecturer:} Gregorio Robles

\item The Definition of Open Source (slides) (2h 30 min)

\item Activity: trial of two 'polemic' statements (slides) (1h 30 min)

\item Activity: Myths and other interesting debates (slides) (30 min)

\end{itemize}

%%---------------------------------------------------------------------
%%---------------------------------------------------------------------
\subsection{Consequences}


%%---------------------------------------------------------------------
\subsubsection{October 29, 2011}

\begin{itemize}
\item \textbf{Lecturer:} Jesus Gonzalez-Barahona

\item Consequences of libre software for the software industry (2:00). \\
  Talk and discussion about the main ways in which libre software is changing and shaping the software industry \\
  Documents: slides.

\item Video: ``Charles Leadbeater on innovation'' (TED talk) (1:00) \\
  We watch the video, and discuss about it, and its relationship with libre software \\
  \url{http://www.ted.com/talks/charles_leadbeater_on_innovation.html}
\item Consequences for main actors (1:30) \\
  Talk and discussion about the main consquences of libre software for several actors (users, developers, integrators, etc.)
  Documents: slides.

\item Exercise: ``Main consequences for education in schools'' (to answer as an exercise in Moodle) \\
Consider that a school decides to use only libre software for supporting teaching. This would include software for the interaction of teachers and students, for supporting the explanation of the concepts in the curricula, etc. Explain the main consequences (bot positive and negative) of this decision. If possible, find documents or texts in the Internet dealing with this issue, to document your answers.
\item Exercise: ``Reasons for forking LibreOffice'' (to answer as an exercise in Moodle) \\
  Some months ago, LibreOffice forked from the well known OpenOffice.org. Find information about that event, and explain when, how and why it happened, and the main reasons adduced by the forkers.
\end{itemize}

%%---------------------------------------------------------------------
%%---------------------------------------------------------------------
\subsection{History}


%%---------------------------------------------------------------------
\subsubsection{November 4, 2011}



%%---------------------------------------------------------------------
%%---------------------------------------------------------------------
\subsection{Legal issues}


%%---------------------------------------------------------------------
\subsubsection{November 5, 2011}


%%---------------------------------------------------------------------
\subsubsection{November 11, 2011}


%%---------------------------------------------------------------------
\subsubsection{November 12, 2011}


%%---------------------------------------------------------------------
\subsubsection{November 18, 2011}


%%---------------------------------------------------------------------
%%---------------------------------------------------------------------
\subsection{Economic issues}



%%---------------------------------------------------------------------
\subsubsection{November 19, 2011}


%%---------------------------------------------------------------------
\subsubsection{November 25, 2011}


%%---------------------------------------------------------------------
\subsubsection{November 26, 2011}



%%---------------------------------------------------------------------
%%---------------------------------------------------------------------
%%---------------------------------------------------------------------
\section{Grading plan}

Each activity contributing to the grading of the course has its own evaluation criteria, as described below. Each of these activities has a minimum and maximum grading. If the minimum grading is 0, the activity is optional. Otherwise, the activity is mandatory, and has to be graded al least with the minimum to pass the course. Each activity has also a description, and when possible, some general grading criteria. In any case, the final grade for the course will also depend on the continuous observation of the instructors on the outcomes and progress of students.

Students should ask instructors about any detail which may not be clear to them, either about the general grading plan, or about specific aspects of the activities. As a general rule, evaluation will have into account how the activity and its results show that the student has come close to the competences, knowledge and skills expected for the course.

The student can consider that the next table will be used as a (minimum) guideline for assigning marks:

\begin{itemize}
  \item Pass (``aprobado''): 100
  \item Good (``notable''): 175
   \item Excellent (``sobresaliente): 250
\end{itemize}

\begin{itemize}
 \item \textbf{Exercises (answered in forum)}. \\
   Minimum: 25 points, maximum: 60 points.

   Exercises proposed and answered in the the forum of the course.

 \item \textbf{Blog entries}. \\
   Minimum: 25 points, maximum: 60 points

   Blog entries specifically related to the course, and marked as such. The tag used for that is mswl-intro.

%% \item \textbf{Collaborative notebook}. \\
%%   Minimum: 0 points, maximum: 40 points

%%   Based on work in class (in real time) and afterwards (complementing the work, using git).

 \item \textbf{Video presentation}. \\
   Minimum: 0 points, maximum: 80 points

   Can be an screencast, or more ellaborate kinds of videos. Has to explain some topic covered by the course. The most focused, the better: try to explain only one issue, but explain it as well as possible.

 \item \textbf{Specific report}. \\
   Minimum: 0 points, maximum: 80 points

   Specific report about a relevant aspect of libre software, related to the topics dealt with in this course. Can be a traditional written report, but can also be a presentation (recorded in video, in this case), a video, a podcast, etc. It is important to detail all the references, and to heavily root the report on data and/or specific works publicly available.

 \item \textbf{Participation in the Quizzbowl}. \\
   Minimum: 0 points, maximum: 60 points

  This activity consists in participating in a quizzbowl held on the IRC. Given that a minimum number of students participate, the quizzbowl will allow them to have "duels" on an IRC channel. Further information will be provided in the forum. Points will be obtained for participating in the matches, the final classification and filling out a final questionnaire about the experience.

 \item \textbf{Other activities}. \\
   Minimum: 0 points, maximum: 80 points

   These activities have to be agreed with the instructors.

 \end{itemize}

\end{document}
