%% introduction.tex
%% 

%%---------------------------------------------------------------------
%%---------------------------------------------------------------------

\section{Introduction}

%%---------------------------------------------------------------------
\begin{frame}
\frametitle{There is a new guy in town}

\begin{itemize}
\item GNU/Linux, Apache, GNOME, KDE, OpenOffice, etc. are very
  important, but...
\item The really new thing is the libre software model:
  \begin{itemize}
  \item Unprecedented combination of collaboration and competition.
  \item Shift in emphasis from marketing to support and quality.
  \item Classical assumptions about intellectual propriety are
    questioned.
  \item End-users recover the control (instead of big software providers)
  \item A new model for a new (global, networked) world?
  \end{itemize}
\item Last years have shown the feasibility of the model.
\end{itemize}

\end{frame}


%%---------------------------------------------------------------

\begin{frame}
\frametitle{What is libre software?}

In short free software guarantees:

\begin{itemize}
\item Freedom to use
\item Freedom to study, and to adapt
\item Freedom to redistribute
\item Freedom to improve and release improvements
\end{itemize}

In other words, if you get it, you can...

\begin{itemize}
\item use it
\item study and adapt it
\item redistribute it
\item improve it and release improvements
\end{itemize}
\vspace{.2cm}
\begin{flushright}
\url{http://www.gnu.org/philosophy/free-sw.html}
\end{flushright}

\end{frame}


%%---------------------------------------------------------------

\begin{frame}
\frametitle{Free / libre / open source}

\begin{itemize}
\item The definition is from Free Software Foundation, for free
  software (``Free Software Definition'')

\item But same applies to open source software (``Open Source Definition'')

\item To avoid discussion and missinterpretations, we will ``libre software''

\item Important consequence: \\
  To be able of modifying source code, it must be available.

\item Lots of licenses: GPL, LGPL, BSD, Apache, MPL, etc.
\end{itemize}

\vspace{.2cm}

\begin{flushright}
\url{http://opensource.org/docs/osd}
\end{flushright}

\end{frame}


%%---------------------------------------------------------------

\begin{frame}
\frametitle{Why this definition?}

\begin{itemize}
\item Ethical concerns: the world should work this way
\item Practical concerns: some actors benefit this way
\end{itemize}

Long discussions, that have reached some level of consensus:

\begin{itemize}
\item Free Software Definition (FSF)
\item Debian Free Software Guidelines (Debian)
\item Open Source Definition (OSI)
\end{itemize}

\end{frame}


%%---------------------------------------------------------------

\begin{frame}
\frametitle{When libre software enters a new niche...}

\begin{itemize}
\item It can become one of the first choices (GNU/Linux in operating
  systems, Apache in WWW servers, OpenOffice in office applications, etc.)
\item It benefits from a lot of synergy (reuse of code, reuse of
  knowledge, reuse of distribution channels, etc.)
\item Users gain competitive advantage:
  \begin{itemize}
  \item Availability of source code makes improvements and
    customization possible in large scale (by in-house or
    subcontracted teams).
  \item Standardization, but maintaining competition between
    providers.
  \item No more per-use licenses.
  \item Much more and better support (ensured by competition).
  \end{itemize}
\item Competition is the name of the game.
\end{itemize}

\end{frame}


%%---------------------------------------------------------------

\begin{frame}
\frametitle{Consequences for the software industry}

The software business is changing upside down (still slowly, but
gaining momentum):

\begin{itemize} 
\item Traditional software ``manufacturers'' will have to reinvent
  themselves completely (no more per-copy incomes).
\item A whole new industry (based in support and libre development)
  will be needed as libre software gains market acceptance.
\item It allows for (and encourages) competition in support, and even
  in the evolution of a piece of software.
\item Users are benefited in several ways. Therefore, big pressure
  from end-users (including big companies) to switch to libre software.
\end{itemize}

\end{frame}

%%---------------------------------------------------------------

\begin{frame}
\frametitle{Some specific impacts}

\begin{itemize} 
\item Cost: cost model radically different from proprietary software
\item Openess: can be modified, can be inspected, can be studied
\item Distribution: new distribution channels, new methods
\item Development: ``surprising'' development models
\item Maintenance and support: true competition
\end{itemize}

Mixture of two powerful mechanisms:

\begin{itemize}
\item Competition (using the same souce base)
\item Cooperation (even non-voluntary)
\end{itemize}

\end{frame}

%%---------------------------------------------------------------

\begin{frame}
\frametitle{Different actors, different visions}

\begin{itemize}
\item End users (persons, companies, public administrations, etc.)
\item Developer (or software producer)
\item Software integration
\item Service provider
\end{itemize}

\end{frame}


%%---------------------------------------------------------------

\begin{frame}
\frametitle{Libre software for (large) end users}

\begin{itemize}
\item Libre software is not necessarily better or worse. It is 
  just different
\item In several niches, we have already excelent products and
  companies supporting them.
\item In many cases, the most cost-effective way of producing
  software.
\item Special advantages when there is interest in long-term life
  cycles, vendor independence, multiplatform support, adaption to
  evolving technologies.
\item If a powerful enough user (or group of users) needs to drive the
  technology, this is probably the best way to go.
\item Many things can be done to promote a competitive
  libre software industry in a given niche. Many benefits
  are derived of such a promotion.
\end{itemize}
\end{frame}

%%---------------------------------------------------------------

\begin{frame}
\frametitle{Conclusions?}

\begin{itemize}
\item Still too few cases to be sure about future trends
\item But there are interesting expectations
\item The model seems to be economically and technically sound
\item The model favors to the most competitive
\item The model levels the field for smaller actors
\item There is a lot of experimentation: new development, bussiness,
  user care, technology policy models
\item Field with a lot of innovation: a good (and updated) knowledge
  about the environment is needed
\end{itemize}

Still too many issues to solve... or are they bussiness opportunities?
\end{frame}


%%---------------------------------------------------------------

\begin{frame}

\begin{center}
{\LARGE
Why not learn about libre software?
}
\end{center}


\end{frame}


\section{Open source definition (annotated)}

%%---------------------------------------------------------------

\begin{frame}
\frametitle{Foreword}

{\bf Open source doesn't just mean access to the source code.

The distribution terms of open-source software must comply with the
following criteria:}

\vspace{.2cm}
\begin{flushright}
[The text in all these slides was copied verbatim from the Open Source
Definition (annotated version), as published by the Open Source
Initiative]
\end{flushright}

\end{frame}

%%---------------------------------------------------------------

\begin{frame}
\frametitle{1. Free Redistribution}

{\bf 
The license shall not restrict any party from selling or giving away
the software as a component of an aggregate software distribution
containing programs from several different sources. The license shall
not require a royalty or other fee for such sale.}

\vspace{.2cm}
Rationale: By constraining the license to require free redistribution,
we eliminate the temptation to throw away many long-term gains in
order to make a few short-term sales dollars. If we didn't do this,
there would be lots of pressure for cooperators to defect.

\end{frame}

%%---------------------------------------------------------------

\begin{frame}
\frametitle{2. Source Code}

{\bf
The program must include source code, and must allow distribution in
source code as well as compiled form. Where some form of a product is
not distributed with source code, there must be a well-publicized
means of obtaining the source code for no more than a reasonable
reproduction cost preferably, downloading via the Internet without
charge. The source code must be the preferred form in which a
programmer would modify the program. Deliberately obfuscated source
code is not allowed. Intermediate forms such as the output of a
preprocessor or translator are not allowed.}

\vspace{.2cm}
Rationale: We require access to un-obfuscated source code because you
can't evolve programs without modifying them. Since our purpose is to
make evolution easy, we require that modification be made easy.

\end{frame}

%%---------------------------------------------------------------

\begin{frame}
\frametitle{3. Derived Works}

{\bf
The license must allow modifications and derived works, and must allow
them to be distributed under the same terms as the license of the
original software.}

\vspace{.2cm}
Rationale: The mere ability to read source isn't enough to support
independent peer review and rapid evolutionary selection. For rapid
evolution to happen, people need to be able to experiment with and
redistribute modifications.

\end{frame}

%%---------------------------------------------------------------

\begin{frame}
\frametitle{4. Integrity of The Author's Source Code}

{\bf
The license may restrict source-code from being distributed in modified form only if the license allows the distribution of "patch files" with the source code for the purpose of modifying the program at build time. The license must explicitly permit distribution of software built from modified source code. The license may require derived works to carry a different name or version number from the original software.}

\vspace{.2cm}
Rationale: Encouraging lots of improvement is a good thing, but users have a right to know who is responsible for the software they are using. Authors and maintainers have reciprocal right to know what they're being asked to support and protect their reputations.

Accordingly, an open-source license must guarantee that source be
readily available, but may require that it be distributed as pristine
base sources plus patches. In this way, "unofficial" changes can be
made available but readily distinguished from the base source.

\end{frame}

%%---------------------------------------------------------------

\begin{frame}
\frametitle{5. No Discrimination Against Persons or Groups}

{\bf
The license must not discriminate against any person or group of persons.}

\vspace{.2cm}
Rationale: In order to get the maximum benefit from the process, the
maximum diversity of persons and groups should be equally eligible to
contribute to open sources. Therefore we forbid any open-source
license from locking anybody out of the process.

Some countries, including the United States, have export restrictions
for certain types of software. An OSD-conformant license may warn
licensees of applicable restrictions and remind them that they are
obliged to obey the law; however, it may not incorporate such
restrictions itself.

\end{frame}

%%---------------------------------------------------------------

\begin{frame}
\frametitle{6. No Discrimination Against Fields of Endeavor}

{\bf 
The license must not restrict anyone from making use of the program in
a specific field of endeavor. For example, it may not restrict the
program from being used in a business, or from being used for genetic
research.} 

\vspace{.2cm}
Rationale: The major intention of this clause is to prohibit license
traps that prevent open source from being used commercially. We want
commercial users to join our community, not feel excluded from it.

\end{frame}

%%---------------------------------------------------------------

\begin{frame}
\frametitle{7. Distribution of License}

{\bf
The rights attached to the program must apply to all to whom the
program is redistributed without the need for execution of an
additional license by those parties.}

\vspace{.2cm}
Rationale: This clause is intended to forbid closing up software by
indirect means such as requiring a non-disclosure agreement.

\end{frame}

%%---------------------------------------------------------------

\begin{frame}
\frametitle{8. License Must Not Be Specific to a Product}

{\bf
The rights attached to the program must not depend on the program's
being part of a particular software distribution. If the program is
extracted from that distribution and used or distributed within the
terms of the program's license, all parties to whom the program is
redistributed should have the same rights as those that are granted in
conjunction with the original software distribution.}

\vspace{.2cm}
Rationale: This clause forecloses yet another class of license traps.

\end{frame}

%%---------------------------------------------------------------

\begin{frame}
\frametitle{9. License Must Not Restrict Other Software}

{\bf
The license must not place restrictions on other software that is
distributed along with the licensed software. For example, the license
must not insist that all other programs distributed on the same medium
must be open-source software.}

\vspace{.2cm}
Rationale: Distributors of open-source software have the right to make
their own choices about their own software.

Yes, the GPL is conformant with this requirement. Software linked with
GPLed libraries only inherits the GPL if it forms a single work, not
any software with which they are merely distributed.

\end{frame}

%%---------------------------------------------------------------

\begin{frame}
\frametitle{10. License Must Be Technology-Neutral}

{\bf
No provision of the license may be predicated on any individual
technology or style of interface.}

\vspace{.2cm}
Rationale: This provision is aimed specifically at licenses
which require an explicit gesture of assent in order to establish a
contract between licensor and licensee. Provisions mandating so-called
"click-wrap" may conflict with important methods of software
distribution such as FTP download, CD-ROM anthologies, and web
mirroring; such provisions may also hinder code re-use. Conformant
licenses must allow for the possibility that (a) redistribution of the
software will take place over non-Web channels that do not support
click-wrapping of the download, and that (b) the covered code (or
re-used portions of covered code) may run in a non-GUI environment
that cannot support popup dialogues.

\end{frame}
