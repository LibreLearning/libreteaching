\documentclass[a4paper]{article}

% Paquetes utilizados
\usepackage[utf8]{inputenc}
\usepackage[spanish]{babel}
\usepackage{url}
% Fueras a que las notas al pie aparezcan abajo
\usepackage[bottom]{footmisc}

% Metadatos
\title{\textbf{Curso ``Arquitectura de servidores \\ con software libre''} \\
Virtualización, alta disponibilidad \\ y ``cloud computing'' \\
%\author{Miguel Vidal \& Jose Castro}
\date{Marzo - Junio 2011}
\vspace{0.5cm}
--- \\
% \vspace{0.5cm}
\normalsize{URJC - GSyC/LibreSoft} \\
% \vspace{0.75cm}
\small{\url{http://formacion.libresoft.es/cursos/sistemas}}}



% Inicio del documento
\begin{document}

  % Título
  \maketitle

  % Resumen
  \begin{abstract}
Lo que sigue es el programa de "<Arquitectura de servidores con software libre">, un curso de formación continua y título propio de la Universidad Rey Juan Carlos (Madrid, España). 

\medskip

Este documento, a modo de \textit{syllabus}, se encuentra en desarrollo hasta que el curso finalice.
  \end{abstract}
  \newpage

  % Índice
  \tableofcontents

  \newpage

  \section{Objetivos}
  %%%%%%%%%%%%%%%%%%%
  Este curso pretende ofrecer una panorámica de las tareas relacionadas con el diseño, la planificación, el despliegue y la administración de servidores con software libre desde un punto de vista profesional.

Se trata de mostrar las diversas destrezas, conocimientos y habilidades que actualmente requiere conocer un especialista en sistemas –desde las tareas de gestión de distintas plataformas de tipo Unix (Linux, BSD, derivados libres de Solaris) al diseño y despliegue de complejas arquitecturas virtualizadas–, y que normalmente no reciben atención académica específica. Puede resultar provechoso tanto a administradores noveles como a sysadmins experimentados que deseen ponerse al día con tecnologías en el estado del arte.

  \section{Contenidos}
  %%%%%%%%%%%%%%%%%%%
  El temario del curso es el siguiente:
  \begin{enumerate}
    \item \textbf{Presentación del curso:} motivaciones y metodología.
    \item \textbf{Tareas esenciales de la administración de sistemas:} gestión de incidencias, políticas, procedimientos, ética, aspectos legales, estándares, ITIL...
    \item \textbf{Sistemas operativos libres para servidores}: Linux, BSD y familia Solaris (illumos, Nexenta, OpenSolaris).
    \item \textbf{Homogeneización del CPD}: automatización, monitorización y mantenimiento del sistema
    \item \textbf{Seguridad proactiva} (acciones preventivas, políticas de seguridad, control de desastres) y \textbf{herramientas criptográficas} (OpenSSH, GnuPG, CAcert, cifrado de particiones)
    \item \textbf{Gestión de redes}: cortafuegos con OpenBSD Packet Filter, \textit{load balancing} y \textit{traffic managing}.
    \item \textbf{Storage as a Service}: SAN/NAS, iSCSI, ZFS, backups, arrays (RAIDs, RAID-Z).
    \item \textbf{Servicios de Internet}: servidores web, DNS, MTAs, listas de correo, UCE...
    \item \textbf{Soluciones de virtualización} del sistema operativo (Xen, KVM, jails, zonas/containers, LDOMs/Sparc) y de red (Crossbow)
    \item \textbf{Plataformas de alta disponibilidad}: clustering (Linux HA, OHAC) y open storage
    \item \textbf{De la virtualización al cloud computing}: IaaS, PaaS y SaaS. OpenNebula.
    \item \textbf{Estrategias de Cloud Computing} y Green IT
  \end{enumerate}

  \section{Horario y lugar}
  %%%%%%%%%%%%%%%%%%%%%%%%%
  Las clases se impartirán los \textbf{viernes} en horario de \textbf{16h30 a 20h30} en las instalaciones de \textbf{Madrid On Rails}\footnote{Las instalaciones de Madrid On Rails están en la C/Lago Titicaca, 10. 28032 Vicálvaro (Madrid)}.

  \section{Metodología de trabajo}
  %%%%%%%%%%%%%%%%%%%%%%%%%%%%%%%%
    Temas a pensar y desarrollar:
    \begin{itemize}
      \item Moodle
      \item Entradas en un blog personal
      \item Etiqueta \#caslu para Twitter
      \item Entorno de prácticas: máquinas, bts...
    \end{itemize}
    \subsection{Calendario}
    El calendario del curso es el siguiente: 

\medskip

    \begin{tabular}{| c | c | c | c |}
      \hline
      Marzo & Abril & Mayo & Junio \\
      \hline
      25 & 1 & 6 & 3 \\
      & 8 & 13 & 10 \\
      & 29 & 20 & \\
      & & 27 & \\
      \hline
    \end{tabular}

    \subsection{Sesiones}
      Cada una de las sesiones se dividirá en varias partes:
      \begin{itemize}
        \item Test previo de autoevaluación (no evaluable)
        \item Corrección de ejercicios propuestos (30 minutos)
        \item Ponencia de un profesional (1 hora)
        \item Descanso (15 minutos)
        \item Sesión teórica (2 horas y 15 minutos)
        \item Ejercicios y prácticas durante la semana (no llevará más de 1 hora)
      \end{itemize}

      \subsubsection{Sesión 1 - Introducción - Tareas esenciales de la administración de sistemas}
         \begin{itemize}
  \item \textbf{Fecha:} Viernes, 25 de marzo, 16:30h
  \item \textbf{Profesores:} Miguel Vidal y Jose Castro
  \item \textbf{Ponencia:} Gregorio Robles (Director del curso)
    \begin{itemize}
      \item ``Presentación del grupo LibreSoft''
      \item ``Introducción al software libre'' (2h.)   
    \end{itemize}
  \item Descanso (15')
  \item \textbf{Exposición:} ``Presentación del curso'' (1h.)
    \begin{itemize}
%      \item \textbf{Profesores:} Miguel Vidal y Jose Castro
      \item \textit{Material de apoyo:} Transparencias ``Presentación del curso'' 
    \end{itemize}
  \item \textbf{Exposición:} ``Introducción a la administración de sistemas'' (1h.)
    \begin{itemize}
      \item \textit{Material de apoyo:} Transparencias ``Introducción a la administración de sistemas''
    \end{itemize}
  \item \textbf{Ejercicios y prácticas:}
\end{itemize}



      \subsubsection{Sesión 2 -- Seguridad}
         %\input{lesson2}
      \subsubsection{Sesión 3 -- SOs libres, homogeneización}
         %\input{lesson3}
      \subsubsection{Sesión 4 -- Redes}
         %\input{lesson4}
      \subsubsection{Sesión 5 -- Storage as a Service}
         %\input{lesson5}
      \subsubsection{Sesión 6 -- Servicios de Internet}
         %\input{lesson6}
      \subsubsection{Sesión 7 -- Virtualización I}
         %\input{lesson7}
      \subsubsection{Sesión 8 -- Virtualización II}
         %\input{lesson8}
      \subsubsection{Sesión 9 -- Cloud Computing}
         %\input{lesson9}
      \subsubsection{Sesión 10 -- Clusters de Alta Disponibilidad}
         %\input{lesson10}

    \subsection{Práctica final}
      Creación de una plataforma de HA virtualizada con un servidor web (podría ser un Cherokee y que tuvieran que desarrollar ellos el agente de recurso).
      

  \section{Evaluación}
  %%%%%%%%%%%%%%%%%%%%

Cada una de las actividades tiene una calificación mínima y máxima. Si la calificación mínima es de 0, la actividad es \textit{opcional}. De lo contrario, la actividad es 
\textit{obligatoria} y tiene que alcanzarse al menos la mínima puntuación para aprobar el curso.

\begin{itemize}
\item \textbf{Ejercicios (respondidos en el Moodle)}. \\
  Mínimo: 10 puntos, máximo: 40 puntos.

  Ejercicios propuestos y respondidos en el Moodle a lo largo del curso. Se valorará también la participación general.

\item \textbf{Práctica final}. \\
  Mínimo: 10 puntos, máximo: 50 puntos

  Práctica final. 

\item \textbf{Entradas de blog}. \\
  Mínimo: 0 puntos, máximo: 10 puntos

  Entradas de blog relacionadas con la materia del curso y marcadas como tales. La etiqueta usada es \#caslu. Actividad opcional. 
\end{itemize} 

\begin{itemize}
\item Aprobado: 50-69
\item Notable: 70-84
\item Sobresaliente: 85-100
\end{itemize}

% Fin del documento
\end{document}
