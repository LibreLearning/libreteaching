\documentclass[hyperref={pdfpagelabels=true},utf8x]{beamer}

%\usetheme{Warsaw}
\usetheme{GSyC}


%\usebackgroundtemplate{\includegraphics[width=\paperwidth]{../format/blank-bg.png}}


\usepackage[spanish]{babel}
\usepackage[utf8x]{inputenc}
\usepackage{graphics}
\usepackage{amssymb} % Simbolos matematicos
\usepackage{fancyvrb}
\usepackage{multicol}
\usepackage{alltt}
\usepackage{listings}




\definecolor{darkred}{rgb}  {1.0, 0.0, 0.0}
\definecolor{darkgreen}{rgb}{0.0, 0.4, 0.0}
\definecolor{darkblue}{rgb} {0.0, 0.0, 0.8}

% for resalted text
\newcommand{\res}[1]{\textcolor{darkred}{#1}}
% for different text
\newcommand{\dif}{\textsl}
% for reserved words
\newcommand{\rw}[1]{\textrm{\textbf{#1}}}
% for commands
\newcommand{\com}[1]{\textrm{\textbf{#1}}}

% for java code. Usage: \begin{java} java code... \end{java}
\lstnewenvironment{java} 
{\lstset{language=java,numbers=left,stepnumber=2}} 
{} 

% for xml code. Usage: \begin{xml} xml code... \end{xml}
\lstnewenvironment{xml} 
{\lstset{language=xml,numbers=left,stepnumber=2}} 
{} 




%% Metadatos del PDF.
\hypersetup{
  pdftitle={Curso de Desarrollo en Android},
  pdfauthor={GSyC/LibreSoft},
  pdfcreator={GSyC},
  pdfproducer=PDFLaTeX,
  pdfsubject={Entorno y Anatomía de una aplicación},
}
%%



%\pgfdeclareimage[height=0.5cm]{gsyc-logo}{../format/gsyc}
%\logo{\pgfuseimage{gsyc-logo}}



\AtBeginSection[]
{
  \begin{frame}<beamer>{Contenidos}
    \tableofcontents[currentsection,hideallsubsections]
  \end{frame}
}

\AtBeginSubsection[]
{
  \begin{frame}<beamer>{Contenidos}
    \tableofcontents[currentsection,currentsubsection]
  \end{frame}
}




\begin{document}

%% Entre corchetes como argumento opcional un título o autor abreviado
%% para los pies de transpa

\title{Entorno de una App Android: Hello World!}
\subtitle{Curso de Desarrollo en Android}
\author[GSyC/LibreSoft]{GSyC/LibreSoft}
\date[2011]{Marzo de 2011}
%\author{Departamento de Sistemas Telemáticos y Computación (GSyC)}
%\date{Noviembre de 2008}

\frame{
\titlepage
\begin{center}
%\includegraphics[width=2cm]{../format/gsyc}
\end{center}
}




%% LICENCIA DE REDISTRIBUCIÓN DE LAS TRANSPAS
\frame{
~
\vspace{4cm}
\begin{flushright}
{\tiny
\copyright 2011 GSyC/LibreSoft \\
  Algunos derechos reservados. \\
  Este trabajo se distribuye bajo la licencia \\
  Creative Commons Attribution Share-Alike\\
  disponible en http://creativecommons.org/licenses/by-sa/3.0/es\\
}
\end{flushright}
}
%%
%%%%%%%%%%%%%%%%%%%%%%%%%%%%%%%%%%%%%%%%%%%%%%%%%%%%%%%%%%%%%%%%%%%

\begin{frame}
  \frametitle{Contenidos}
  \tableofcontents[hideallsubsections]
\end{frame}


%%%%%%%%%%%%%%%%%%%%%%%%%%%%%%%%%%%%%%%%%%%%%%%%%%%%%%%%%%%%%%%%%%%
\section{Creación y ejecución de  un primer proyecto: Hello World}
%%%%%%%%%%%%%%%%%%%%%%%%%%%%%%%%%%%%%%%%%%%%%%%%%%%%%%%%%%%%%%%%%%%

%%%%%%%%%%%%%
\begin{frame}[shrink=55.9]
\frametitle{Pasos para la creación y ejecución del proyecto Hello World}

\begin{block}{1) Crear un proyecto para la primera aplicación:}
Se creará automáticamente un proyecto que no está vacío: imprime una
cadena de texto cuando arranca la aplicación.
\end{block}

\begin{itemize}
    \item Seleccionar ''File'' $\rightarrow$ ''New'' $\rightarrow$ ''Project''
    \item Seleccionar el wizard ``Android Project'' en la carpeta ``Android''
    \item Pulsar ''Next''. Rellenar los datos siguientes:
    \begin{itemize}
          \item Project Name: Hello World
          \item Build Target: Android 2.2
          \item Application Name: Hello World
          \item Package Name: com.clau.helloworld
          \item Create Activity: HelloWorld
    \end{itemize}
    \item Pulsar ''Next'', pulsar ''Finish''.
\end{itemize}

\begin{block}{2) Crear la configuración de ejecución para este proyecto:}
Se pueden crear varias configuraciones de ejecución, y configuraciones
de depuración. Creamos una configuración de ejecución:
\end{block}

\begin{itemize}
    \item Seleccionar ''Run'' $\rightarrow$ ''Run Configurations''
    \item Pulsar con el botón derecho sobre ''Android Application'' y seleccionar ''New''. Rellenar los datos siguientes:
      \begin{itemize}
      \item Name: Hello World
      \item Project: Hello World
      \end{itemize}
    \item Pulsar la pestaña ''Target''
    \item Pulsar ''Manager...''
    \item Pulsar ''New'' para crear un nuevo Android Virtual Device (AVD). Rellenar los siguientes datos
      \begin{itemize}
      \item Name: MyAndroid
      \item Target: Android 2.2 - API Level 8
      \end{itemize}
    \item Pulsar ''Create AVD'' y esperar a que una ventana confirme la creación
    \item Cerrar la ventana ''Android SDK''
    \item Esperar a que en el panel aparezca el nuevo AVD con nombre ''MyAndroid''
    \item Activar la casilla del AVD ''MyAndroid'', pulsar ''Apply'', pulsar ''Close''.
\end{itemize}

\begin{block}{3) Ejecutar la aplicación:}
Se puede pulsar directamente el botón Run, o hacerlo a través de una de las configuraciones de ejecución preexistentes:
\end{block}

Seleccionar la configuración de ejecución ''Hello World'' recién creada en ''Run'' $\rightarrow$ ''Run Configurations''


\end{frame}



%%%%%%%%%%%%%
\begin{frame}[fragile]
 \frametitle{Pasos para la creación y ejecución}

 \begin{itemize}
 \item Cuando desde Eclipse, con el plugin de Android instalado, se
   ejecuta una aplicación a través de una configuración de ejecución o
   de depuración, Eclipse hace lo siguiente:
   \begin{enumerate}
     \item Compila la aplicación, generando un ejecutable para dalvik (\verb|.dex|)
     \item Empaqueta el ejecutable y otros recursos externos en un paquete Android (\verb|.apk|)
     \item Arranca el emulador (si no estaba ya arrancado)
     \item Instala el paquete de la aplicación (\verb|.apk|) en el emulador
     \item Arranca la aplicación en el emulador
   \end{enumerate}

 \item Si se utiliza una configuración de depuración el depurador de
   Eclipse se conecta a la aplicación, pudiéndose depurar entonces
   desde la \res{perspectiva de depuración de Eclipse}

 \end{itemize}

\end{frame}


%%%%%%%%%%%%%
\begin{frame}[fragile]
\frametitle{Carpetas y ficheros generados para un proyecto}

\begin{tiny}
\begin{tabular}{|p{3cm}|p{7cm}|}
\hline
\res{Ficheros Android}           &	\res{Descripción} \\
&\\\hline\hline
\verb|AndroidManifest.xml| & Fichero que describe la aplicación: permisos, capacidades que exporta, cómo correrá.\\\hline
&\\
\verb|default.properties| & Fichero generado automáticamente. Define cómo construir la aplicación.  \\\hline
&\\
Carpeta \verb|src|  & Código fuente de la aplicación.\\\hline
&\\
\verb|src/com.clau.helloworld/|
\verb|HelloWorld.java| & Fichero  con el código de la \dif{Activity} \verb|HelloWorld| de esta aplicación. Es el punto de entrada a la aplicación.\\\hline
&\\                         

Carpeta \verb|gen|  &  Carpeta en la que se almacenan  ficheros relacionados con recursos autogenerados.\\\hline
&\\
\verb|gen/com.clau.helloworld/|
\verb|R.java|  & Fichero fuente para manejar recursos desde la aplicación: no debe modificarse.\\\hline
&\\
Carpeta \verb|res| 	&  Recursos de la aplicación: animaciones, imágenes, ficheros de \emph{layout}, ficheros XML, strings, y otros ficheros.\\\hline
&\\
\verb|res/drawable/icon.png|	& Icono de la aplicación que se muestra en el lanzador de aplicaciones del teléfono.\\\hline
&\\
\verb|res/layout/main.xml|	& Fichero XML que define el \emph{layout}.\\\hline
&\\
\verb|res/values/strings.xml|	& Fichero XML con los \emph{Strings} de la aplicación.\\\hline
\end{tabular}
\end{tiny}

\end{frame}





%%%%%%%%%%%%%
\begin{frame}[fragile, shrink=27]
\frametitle{Explicación de la aplicación Hello World}

\begin{block}{Actividad HelloWorld ({\tiny src/com.clau.helloworld/HelloWorld.java})}
\begin{java}
package com.clau.helloworld;

import android.app.Activity;
import android.os.Bundle;

public class HelloWorld extends Activity {
  /** Called when the activity is first created. */
  @Override
  public void onCreate(Bundle savedInstanceState) {
    super.onCreate(savedInstanceState);
    setContentView(R.layout.main);
  }
}
\end{java}
\end{block}

\begin{itemize}
\item En la línea 6 (l-6) se extiende la clase \res{Activity}, clase
  utilizada en una aplicación para soportar tareas que requieren una
  interfaz gráfica
\item El punto de entrada es el método \verb|onCreate()|, que se
  redefine (l-9)
\item En el código Java no vemos ni el \dif{string} ni la \res{View}
  (el componente visual que se utiliza para mostrarlo en pantalla): se
  han definido como \res{recursos} externos
\item En l-11 se \dif{infla} la interfaz gráfica a partir de los
  recursos definidos en el fichero \verb|res/layout/main.xml|
\end{itemize}

\end{frame}


%%%%%%%%%%%%%
\begin{frame}[fragile,shrink=17.18]
\frametitle{Recursos de la aplicación Hello World}

\begin{itemize}
\item Definir en XML los aspectos visuales permite desacoplarlos de la
  lógica de la aplicación
\item Los recursos están en la carpeta \verb|res| del proyecto, con
  subcarpetas \verb|drawable|, \verb|layout| y \verb|values|
\item Desde el programa se puede acceder a los recursos externos a
  través de la \res{variable \com{R}}
\item El fichero \verb|res/layout/main.xml| define el \dif{layout} de
  la vista de la actividad:
\end{itemize}

\begin{block}{res/layout/main.xml}
\begin{tiny}
\begin{xml}
<?xml version=''1.0'' encoding=''utf-8''?> 
<LinearLayout xmlns:android=''http://schemas.android.com/apk/res/android''
  android:orientation=''vertical''
  android:layout_width=''fill_parent''
  android:layout_height=''fill_parent''> 
  <TextView  
    android:layout''width=''fill_parent'' 
    android:layout_height=''wrap_content''
    android:text=''@string/hello''
  /> 
</LinearLayout> 
\end{xml}
\end{tiny}
\end{block}

En la l-9 de \verb|main.xml| se referencia un \dif{string} definido en la l-3 de \verb|res/values/strings.xml|:

\begin{block}{res/values/strings.xml}
\begin{tiny}
\begin{xml}
<?xml version=''1.0'' encoding=''utf-8''?>
<resources>
  <string name=''hello''>Hello World, HelloWorld!</string>
  <string name=''app_name''>Hello World</string>
</resources>
\end{xml}
\end{tiny}
\end{block}


\end{frame}



%%%%%%%%%%%%%
\begin{frame}[fragile]
\frametitle{Identificación de recursos de la aplicación Hello World}

Para acceder a los elementos de la interfaz en el código se pueden añadir atributos \verb|id| en XML (ver l-3):


\begin{tiny}
\begin{block}{res/layout/main.xml (modificado respecto al original)}
\begin{xml}
...
<TextView  
  android:id=''@+id/myTextView''
  android:layout_width=''fill_parent''
  android:layout_height=''wrap_content''
  android:text=''Hello World, HelloWorld''
/> 
...
\end{xml}
\end{block}
\end{tiny}


Luego, desde el código Java se podría utilizar el método \verb|findViewById|
para obtener una referencia al  elemento cuyo id se ha definido en la l-3:


\begin{tiny}
\begin{block}{HelloWorld.java (modificado respecto al original)}
\begin{java}
...
TextView myTextView = (TextView)findViewById(R.id.myTextView); 
...
\end{java}
\end{block}
\end{tiny}

\end{frame}



%%%%%%%%%%%%%
\begin{frame}[fragile,shrink=32.47]
\frametitle{HelloWorld sin la interfaz definida externamente en los recursos}

Alternativamente se podrían crear los elementos gráficos de la
actividad HelloWorld en Java (no recomendado):


\begin{block}{HelloWorld.java (modificado respecto al original)}
\begin{java}
public void onCreate(Bundle savedInstanceState) { 
  super.onCreate(savedInstanceState); 
  LinearLayout.LayoutParams lp; 
  lp = new LinearLayout.LayoutParams(LayoutParams.FILL_PARENT, 
                                     LayoutParams.FILL_PARENT); 
  LinearLayout.LayoutParams textViewLP; 
  textViewLP = new LinearLayout.LayoutParams
                           (LayoutParams.FILL_PARENT, 
                            LayoutParams.WRAP_CONTENT); 
  LinearLayout ll = new LinearLayout(this); 
  ll.setOrientation(LinearLayout.VERTICAL); 
  TextView myTextView = new TextView(this); 
  myTextView.setText(``Hello World, HelloWorld''); 
  ll.addView(myTextView, textViewLP); 
  this.addContentView(ll, lp); 
} 
\end{java}
\end{block}



\end{frame}

%%%%%%%%%%%%%%%%%%%%%%%%%%%%%%%%%%%%%%%%%%%%%%%%%%%%%%%%%%%%%%%%%%%
\section{Depuración de aplicaciones en el emulador}
%%%%%%%%%%%%%%%%%%%%%%%%%%%%%%%%%%%%%%%%%%%%%%%%%%%%%%%%%%%%%%%%%%%

%%%%%%%%%%%%%
\begin{frame}[fragile]
\frametitle{Las perspectivas de Eclipse para depuración}

\begin{itemize}
\item Para depurar se selecciona una configuración preexistente en el
  menú Run $\rightarrow$ Debug Configurations. También puedes pulsar
  directamente el botón de depuración (dibujo de un \emph{bug})

\item Eclipse tiene varias \res{perspectivas}, cada una con diferentes
paneles. La perspectiva por omisión es la de Java. 

\item Arriba a la derecha aparecen botones para acceder a otras
  perspectivas, como la de depuración, en la que se pueden poner
  puntos de parada (\dif{breakpoints}), ver la información de Log
  (LogCat) y depurar (ejecución paso a paso,...)

\item Hay otra perspectiva: DDMS (Dalvik Debug Monitor Service)
  permite monitorizar y manipular el estado del emulador (procesos
  arrancados en el emulador p.ej.)

\end{itemize}
\end{frame}


%%%%%%%%%%%%%
\begin{frame}[fragile,shrink=33.83]
\frametitle{Ejercicio de depuración}

\begin{itemize}
\item Añade en un fichero el siguiente método:

\lstset{language=java}
\begin{lstlisting}
public void forceError() {
    if(true) {
        throw new Error("Error generado adrede");
    }
}
\end{lstlisting}

\item Añade una llamada a \verb|forceError()| en algún lugar del código (por
ejemplo cuando se va a añadir un elemento en la lista, tras haber
pulsado el botón del teléfono).

\item Ejecuta la aplicación: se producirá una excepción en el lugar en el
que hayas incluido esta llamada, y la aplicación en el emulador parará
con un mensaje en pantalla.

\item Depura la aplicación: para depurar se selecciona una configuración
preexistente en el menú Run $\rightarrow$ Debug
Configurations. También puedes pulsar directamente el botón de
depuración (dibujo de un \emph{bug})

Aparecerá un mensaje en Android para pasar a la perspectiva de depuración

\item Desde la perspectiva de depuración se puede observar en el panel
LogCat este mensaje (dentro del LogCat, mira el filtro que tiene un E
en rojo): \verb|AndroidRuntime error: java.lang.Error: Error generado adrede|

\item Añade ahora un breakpoint en la línea desde la que llamas a
\verb|forceError()|: con el botón derecho pulsa en la columna de la
izda. de la línea

\item Vuelve a depurar. Cuando la ejecución pare en el breakpoint, ejecuta
paso a paso hasta llegar a la línea que lanza la excepción. 

\item Puedes ver el contenido de la excepción en el panel de Variables de la
perspectiva de depuración
\end{itemize}

\end{frame}

%%%%%%%%%%%%%
\begin{frame}[fragile,shrink=33.83]
\frametitle{Generación de mensajes en LogCat}

\begin{itemize}
\item En el panel LogCat aparecen mensajes informativos. Hay varios filtros representados por círculos arriba a la derecha. 
\item Desde la aplicación se pueden generar mensajes de logging
  utilizando la clase \verb|android.util.Log|:

\begin{tabular}{|c|c|}
\hline
\textbf{Método} &	\textbf{Propósito}\\\hline\hline
\verb|Log.e()|  & Log errors\\\hline
\verb|Log.w()|	& Log warnings\\\hline
\verb|Log.i()|	& Log informational messages\\\hline
\verb|Log.d()|	& Log Debug messages\\\hline
\verb|Log.v()|	& Log Verbose messages\\\hline
\end{tabular}
\end{itemize}

\begin{block}{Ejercicio}
\begin{itemize}
\item Añade la siguiente ĺínea: \verb|import android.util.Log;| 
\item Añade a la clase este String:
  \verb|private static final String DEBUG_TAG= "MiTag";|
\item Añade llamadas del tipo
  \verb|Log.i(DEBUG_TAG, "Testing informational message 1 ");|
\item Corre la aplicación y busca el el panel del LogCat (filtro I) tus mensajes
\end{itemize}
\end{block}

\end{frame}

% Run -> Debug Configurations -> Debug

% Aparece la perspectiva de depuración de Eclipse android (2: android y ddms). Puedes conmutar entre ésta y la perspectiva de Java con botones arriba a la derecha

% LogCat te da info

% Sale lista de procesos


% Añadimos fallo aposta: 
% public void forceError() {
%     if(true) {
%         throw new Error("Whoops");
%     }
% }

% Añadimos llamada en algún sitio: 

% Prueba a poner un breakpoint pulsando con el botón derecho en la 1ª columna de la fila seleccionada





%% Incorporar a las transpas las explicaciones sobre cómo crear proyecto HelloWorld (de hoja moodle)
%% Explicación de Hello World (del PAAD)
%% 
%% TODO list example (hacerlo, contarlo)
%%
%% Depuración y Logging de TODO list (del WAAD)
%%
%% Tipos de apps Android (del PAAD)
%% Desarrollando para móviles (del PAAD)
%% Componentes: DDMS... del PAAD y WAAD
%%
%% Deberes: depura todo_list, carga aplicaciones de samples (snake,...)


%%%%%%%%%%%%%%%%%%%%%%%%%%%%%%%%%%%%%%%%%%%%%%%%%%%%%%%%%%%%%%%%%%%
\section{Bibliografía}
%%%%%%%%%%%%%%%%%%%%%%%%%%%%%%%%%%%%%%%%%%%%%%%%%%%%%%%%%%%%%%%%%%%

%%%%%%%%%%%%%
\begin{frame}[fragile]
\frametitle{Bibliografía}

\begin{itemize}
\item Capítulos 1 y 2 de \dif{Professional Android Application
  Development}. Reto Meier. Ed. Wrox, 2009.
\item Capítulos 1 y 2 de \dif{Wireless Android Application
  Development}. Shane Conder, Lauren Darcey. Ed. Addison Wesley
  Professional, 2009.
\item Documentación del Android SDK: en la carpeta \verb|docs| del
  directorio del SDK, o en
  \verb|http://developer.android.com/guide/index.html|
\item Documentación sobre Android (tutoriales, vídeos,...): 
  \verb|http://developer.android.com|
\item Hay instrucciones para la instalación del SDK de Android y de
  Eclipse en la página web de la asignatura
\end{itemize}
\end{frame}






\end{document}





%%% Local Variables: 
%%% mode: latex
%%% TeX-master: t
%%% End: 
