%%
%% $Id$
%%

\documentclass[a4paper,12pt]{article}
\usepackage[utf8]{inputenc}
\usepackage[spanish]{babel}
\usepackage{geometry}
\usepackage[pdfborder=0 0 0]{hyperref}
\usepackage{url}

\title{Tecnologías de la Información y las Comunicaciones y Herramientas \\
Master en Economía Digital e Industrias Creativas \\
EOI Escuela de Organización Industrial \\
Programa del curso mayo-octubre 2011}
\author{Jesús M. González Barahona y Gregorio Robles}

%\date{}

\begin{document}
\maketitle

\newpage

\tableofcontents

\newpage

%--------------------------------------------------------
%--------------------------------------------------------
%--------------------------------------------------------
\section{Equipo de profesores}

\begin{itemize}
\item Jesús M. González Barahona
  \begin{itemize}
  \item Correo electrónico: jgb@libresoft.es
  \end{itemize}
\item Gregorio Robles
  \begin{itemize}
  \item Correo electrónico: grex@libresoft.es
  \end{itemize}
\item Agustín Santos
  \begin{itemize}
  \item Correo electrónico: asantos@libresoft.es
  \end{itemize}
\item Miquel Vidal
  \begin{itemize}
  \item Correo electrónico: mvidal@libresoft.es
  \end{itemize}
\item Felipe Ortega
  \begin{itemize}
  \item Correo electrónico: jfelipe@libresoft.es
  \end{itemize}
\item Israel Herraiz
  \begin{itemize}
  \item Correo electrónico: israel.herraiz@upm.es
  \end{itemize}
\end{itemize}

%--------------------------------------------------------
%--------------------------------------------------------
%--------------------------------------------------------
%\section{Objetivos}

%\begin{itemize}
%\item 
%\end{itemize}

%\section{Metodología}


%--------------------------------------------------------
%--------------------------------------------------------
%--------------------------------------------------------
\section{Programa: Marco general}

Programa del módulo I del máster, en lo que tiene que ver con el área de TIC (tentativo)

%--------------------------------------------------------
%--------------------------------------------------------
\subsection{Introducción}

%--------------------------------------------------------
\subsubsection{Sesión del 20 de mayo (2 horas)}

Presentación de los aspectos fundamentales de la asignatura. Información sobre las capacidades y habilidades de los alumnos en relación con esta asignatura. Motivación e introducción a la infraestructura TIC básica del máster.

\begin{itemize}
\item Profesores: Jesús
\item Presentación: Presentación del área de Tecnologías de la Información y las Comunicaciones en el ámbito del máster.
  %\item Material: Transparencias (tema ``01 - Presentación'')
\item Ejercicio : Conocimientos sobre temas relacionados con la asignatura:
  \begin{itemize}
  \item Internet: desde cuándo, tipo de experiencia, navegadores utilizados, servicios utilizados, tiempo por semana (conectado, de uso activo)
  \item Servicios Internet (conoces, tienes cuenta, usas habitualmente): Facebook, Identi.ca, Tuenti, Twitter, Picotea, Flickr, Youtube, Blip.tv, Vimeo, de.li.cious, LinkedIn, Infojobs, Wikipedia, blogs, otros
  \item Software: programas más usados (entre los libres: OpenOffice, Firefox, Thunderbird, GIMP, Linux, otros)
  \item Hardware: sobremesa, portátil, smartphone, ebook, otros
  \item Idiomas: ¿cómo va el inglés?
  \item Vocablos varios: HTTP, TCP, IP, Mashup, AJAX
  \end{itemize}
\item Explicación: Servidores y federaciones SIP, Skype como caso p2p. Explicación técnica, de negocio y legal sobre Skype (introducción)
\item Ejercicio planteado (para trabajar no presencialmente): Skype: implicaciones técnicas, legales, de negocio (ejercicio~\ref{sub:skype-implicaciones}).
\item Pregunta para reflexión (Skype): Implicaciones del modelo p2p para el negocio \\
  ¿Por qué los costes de infraestructura son muy bajos en Skype comparado con otros servicios de VoIP? ¿Quién y cómo está en realidad proporcionando gran parte de la infraestructura necesaria para que funcione el servicio?
\end{itemize}

%--------------------------------------------------------
%--------------------------------------------------------
\subsection{Fenómenos emergentes}

Fenómenos emergentes, en lo que tienen que ver con el área TIC del máster.

%--------------------------------------------------------
\subsubsection{Sesión del 21 de mayo (2 horas)}

Obras libres. ¿Por qué hay autores que distribuyen sus trabajos como obras libres? ¿Qué proyectos y sitios web relacionados con las obras libres son más conocidos? ¿Por qué todo esto es importante?

\begin{itemize}
\item Profesores: Jesús
\item Comentario de video: ``Larry Lessig on laws that choke creativity''
\item Presentación: Motivación
\item Ejercicio (foro): Comentarios al video. \\
  Elige un aspecto del video de Larry Lessig, relacionado con la asignatura, que te haya llamado especialmente la atención, y coméntalo brevemente.
\item Material: Transparencias (tema ``Motivación'')
\item Material: video ``Larry Lessig on laws that choke creativity'', Ted.com, \\
  \url{http://www.ted.com/talks/lang/eng/larry_lessig_says_the_law_is_strangling_creativity.html}
\item Ejercicio (planteado en clase, para terminar en el foro) : Música, aviones y granjeros (ejercicio~\ref{sub:musica-aviones}).
\item Pregunta para reflexión (obras libres): Modelos de sostenibilidad de las obras libres \\
  Si en las obras libres la distribución no está ``monopolizada'' por los creadores, ¿cómo pueden conseguir ingresos para seguir dedicándose a la creación? ¿Qué modelos de sostenibilidad pueden utilizarse? ¿Qué efecto tienen los actores que redistribuyen las obras libres con ánimo de lucro?
 \end{itemize}

%--------------------------------------------------------
\subsubsection{Sesión del 28 de mayo (2 horas)}

Obras construidas en colaboración: el caso de Wikipedia

%%TODO: Enlaces y subsecciones con breve descripción sobre los ejercicios y vídeos

\begin{itemize}
\item Profesor: Felipe
\item Presentación: Wikipedia
\item Ejercicio(clase): Introducción a edición en wikis.
\item Ejercicio(foro): Realizar una mejora en una artículo de Wikipedia.
\item Ejercicio(foro): Buscar información sobre el fork de Wikipedia en español.
\item Material: Transparencias (tema ``Wikipedia'').
\item Material: Video: Wikipedia entre las 10 páginas más consultadas por los 
españoles (RTVE, 20-05-2011).
\item Material (video): ¿Por qué se colabora de forma altruista en Wikipedia? 
(RTVE, 15-01-2011).
\end{itemize}

%--------------------------------------------------------
\subsubsection{Sesión del 28 de mayo (2 horas)}

Ecosistemas tecnológicos: el caso de Android

%%TODO: Enlaces y subsecciones con breve descripción sobre los ejercicios y vídeos

\begin{itemize}
\item Profesor: Israel
\item Presentación: Android y su ecosistema
\item Material: Transparencias sobre Android
\item Ejercicio (foro): ROMs customizadas para Android: ¿cuáles
  conoces o has encontrado?
\item Ejercicio (foro): ¿Cómo se arreglaron los problemas legales
  entre Google y Cyanogen?
\item Pregunta para reflexión (ecosistema Android): control por parte de Google \\
  ¿Cómo trata de mantener Google control sobre el ecosistema Android? ¿En qué medida es posible que aparezcan ``forks'' de Android en los que Google no pueda ejercer ese control? ¿Qué medidas legales, económicas y tecnológicas está usando Google para evitar que esto ocurra?
\end{itemize}

%--------------------------------------------------------
\subsubsection{Sesión del 3 de junio (3 horas)}

Implicaciones del tratamiento masivo de datos: recogida y análisis sobre personas por parte de empresas y otras instituciones.

\begin{itemize}
\item Profesor: Jesús
\item Presentación: ¿Qué datos se pueden recoger sobre nosotros cuando navegamos por Internet?
\item Material de apoyo (lectura previa recomendada): ``Data Mining: How Companies Now Know Everything About You'', Joel Stein, Time Magazine, 10 Mar 2011 \\
  \url{http://www.time.com/time/business/article/0,8599,2058114-1,00.html}
\item Material de apoyo: ``Web bugs'', Wikipedia \\
  \url{http://en.wikipedia.org/wiki/Web_bug}
\item Material de apoyo: ``Betrayed by our own data'', Kai Biermann, Zeit Online, 26 Mar 2011 \\
  \url{http://www.zeit.de/digital/datenschutz/2011-03/data-protection-malte-spitz} \\
  \url{http://www.zeit.de/datenschutz/malte-spitz-vorratsdaten}
\item Ejercicio (clase): Funcionamiento de Ghostery
\end{itemize}

%--------------------------------------------------------
\subsubsection{Sesión del 11 de junio (2 horas)}

``Cloud computing'' y ``software as a service''.

\begin{itemize}
\item Profesor: Jesús
\item Material de apoyo (video): ``Cloud Computing Explained'', Michael Sheehan y Tim Wayne\\
  \url{http://www.youtube.com/watch?v=QJncFirhjPg}
\item Material de apoyo (video): ``Cloud Computing in Plain English'', Common Craft \\
\url{http://www.commoncraft.com/cloud-computing-video}
\item Material de apoyo (video): ``The Top Free Cloud Applications'', Christopher Barnatt \\
\url{http://www.youtube.com/watch?v=IJcs7muN9XE}
\item Material de apoyo (transparencias): ``Disadvantages of Cloud Computing'', Maikel Mardjan \\
  \url{http://www.slideshare.net/maikelm/disadvantages-of-cloud-computing}
\item Ejercicio (clase): Funcionamiento de EyeOS \\
  \url{http://try.eyeos.org/}
\end{itemize}
%--------------------------------------------------------
\subsubsection{Sesión del 18 de junio (2 horas)}

Realidad virtual, realidad aumentada

\begin{itemize}
\item Profesor: Jesús
\item Material de apoyo (video, visionado previo recomendado): ``Augmented Reality Explained by Common Craft'', Common Craft \\
  \url{http://www.commoncraft.com/augmented-reality-video}
\item Material de apoyo: ``Vuzix STAR 1200 Augmented Reality Glasses: Terminator Vision Ready to Go'' \\
\url{http://technabob.com/blog/2011/06/06/vuzix-star-1200-augmented-reality-glasses/}
\item Material de apoyo: ``Augmented Reality on iPad seen via Vuzix glasses'' \\
  \url{http://applehotnews.net/2011/05/23/augmented-reality-on-ipad-seen-via-vuzix-glasses/}
\item Material de apoyo: ``ARvertising news'' \\
\url{http://www.arvertising.com/news/}
\item Material de apoyo: ``Who Owns the Advertising Space in an Augmented Reality World?'' \\
\url{http://technewscast.com/technology/tech-buzz/who-owns-the-advertising-space-in-an-augmented-reality-world/}
\item Material de apoyo: ``Real estate (Second Life)'' (Wikipedia) \\
\url{http://en.wikipedia.org/wiki/Real_estate_%28Second_Life%29}
\item Material de apoyo: ``Second Life Virtual Pets Copyright Dispute'' \\
\url{http://onecoolsitebloggingtips.com/2011/01/04/second-life-virtual-pets-copyright-dispute/}
\end{itemize}

%--------------------------------------------------------
\subsubsection{Sesión del 25 de junio (3 horas)}

Dinero electrónico, micropagos: Paypal, Bitcoin

\begin{itemize}
\item Profesor: Jesús

\item PayPal: \\
  \url{http://www.paypal.com/}

\item PayPal España: \\
  \url{http://www.paypal.es/}

\item PayPal en Wikipedia: \\
  \url{http://en.wikipedia.org/wiki/PayPal}

\item BitCoin: \\
  \url{http://www.bitcoin.org/}

\item BitCoin en Wikipedia: \\
  \url{http://en.wikipedia.org/wiki/Bitcoin}

\item Documento: ``A Wild Few Weeks for Bitcoin'' \\
  \url{http://howestreet.com/2011/06/wild-weeks-bitcoin/}

\item Ejercicio (a terminar en el foro): ¿Qué es PayPal? (ejercicio~\ref{sub:paypal-quees}).

\item Ejercicio: ¿Cómo se compara la privacidad de PayPal con la de las tarjetas de crédito?

\item Ejercicio: ¿Cómo se compara la privacidad de BitCoin con la del dinero en efectivo?

\item Ejercicio: ¿Cuáles son los cargos de PayPal por usar sus servicios de pago?
\end{itemize}

Comunidades de experiencia: Open Street Maps

\begin{itemize}
\item Profesor: Jesús

\item OpenStreetMap: \\
\url{http://wiki.openstreetmap.org}

\item OpenStreetMap (España): \\
\url{http://www.openstreetmap.es}

\item OpenStreetMap en Wikipedia: \\
\url{http://en.wikipedia.org/wiki/OpenStreetMap}

\item Licencias de OpenStreetMap: \\
\url{http://www.osmfoundation.org/wiki/License}

\item Video: ``2008 Year of Edits OpenStreetMap'' \\
\url{http://www.youtube.com/watch?v=hwlqUremx4Y}

\item Video: ``Animation of Openstreetmap Haïti coverage after the 2010 earthquake''
\url{http://www.youtube.com/watch?v=V7-7zVyRXIY&NR=1}

\item Video: ``Richard Weait - Introduction to Open Street Map @ COSSFest 2010'' \\
\url{http://www.youtube.com/watch?v=tWCGkAZtnME}

\item Documentación: ``OpenStreetMap: Beginners' Guide for Business'' \\
\url{http://wiki.openstreetmap.org/wiki/Beginners%27_guide/Business}

\item OpenCycleMap: ``http://www.opencyclemap.org'' \\
\url{http://www.opencyclemap.org/}

\item Programas en Android: \\
\url{https://market.android.com/search?q=openstreetmap&c=apps}

\end{itemize}

Web hosting:

\begin{itemize}
\item Profesor: Jesús

\item Documento: Web hosting service (Wikipedia): \\
  \url{http://en.wikipedia.org/wiki/Web_hosting_service}

\item FindMyHosting.com: \\
  \url{http://www.findmyhosting.com/}

\item Ejemplo de web hosting simple con aplicaciones: DreamHost \\
  \url{http://www.dreamhost.com/}

\item Ejemplo de web hosting simple con aplicaciones: eApps \\
  \url{http://www.eapps.com/}

\item Ejemplo de web hosting simple con aplicaciones: LimeDomains \\
  \url{http://apps.limedomains.com/}

\item Google App Engine (Wikipedia):\\
  \url{http://en.wikipedia.org/wiki/Google_App_Engine}

\item Amazon S3 (Wikipedia): \\
  \url{http://en.wikipedia.org/wiki/Amazon_S3}

\item Google Sites: \\
  \url{http://sites.google.com/}
\end{itemize}


%--------------------------------------------------------
\subsubsection{Sesión del 9 de julio (3 horas)}

Caso de estudio conjunto (sesión conjunta con el área jurídica y de negocio). Como cierre de la primera parte del máster, dedicada sobre todo a plantear problemas, se analizará con cierto detalle un caso de estudio que permita explorar aspectos tecnológicos, jurídicos y de negocio relevantes.

\begin{itemize}
\item Profesores: Ana Noguerol, Alberto Abellá, Jesús
\item Ejercicio (a realizar en grupos en clase, y terminar en línea individualmente): Redes sociales para la empresa (ejercicio~\ref{sub:redes-sociales-empresa})
\end{itemize}

%--------------------------------------------------------
\subsubsection{Sesión del 9 de julio (1 hora)}

Presentación de la segunda parte del máster (sesión conjunta con el área jurídica y de negocio).

\begin{itemize}
\item Profesores: Ana Noguerol, Alberto Abellá, Jesús
\end{itemize}

%--------------------------------------------------------
\subsubsection{Sesión del 9 de julio (1 hora)}

Reflexión en el aula sobre los temas de los proyectos fin de máster, que a estas alturas del calendario deberían estar definidos.

\begin{itemize}
\item Profesores: Ana Noguerol, Alberto Abellá, Jesús
\end{itemize}


%--------------------------------------------------------
%--------------------------------------------------------
\subsection{Herramientas y servicios}

Aplicaciones y servicios útiles en el contexto del máster.

%--------------------------------------------------------
\subsubsection{Sesión del 20 de mayo (1 hora)}

Moodle e introducción rápida a otras aplicaciones básicas:

\begin{itemize}
\item Profesores: Jesús
\item Demo: Exploración muy resumida del entorno Moodle.
\item Explicación: Menciones rápidas al entorno Google Apps y al blog.
\item Ejercicio planteado (para trabajar no presencialmente): Exploración del entorno Google Apps (ejercicio~\ref{sub:googleapps}).
\item Ejercicio planteado (para trabajar no presencialmente): Exploración de funcionamiento del blog de la EOI (ejercicio~\ref{sub:blog}).
\end{itemize}

%--------------------------------------------------------
\subsubsection{Sesión del 21 de mayo (2 horas)}

Google Docs y apuntes colaborativos:

\begin{itemize}
\item Profesores: Jesús
\item Ejercicio (para terminar no presencialmente): Introducción a Google Docs (ejercicio~\ref{sub:googledocs}).
\item Planteamiento de actividad: Edición colaborativa: resumen de la asignatura en Google Docs
\end{itemize}

Micronotas (microblogging): Twitter:

\begin{itemize}
\item Profesores: Jesús
\item Presentación y debate: Introducción al sevicio de Twitter y su API. Diferencia entre servicio y clientes. Impliaciones para el rendimiento y para la innovación. La publicidad y cómo se comportan los clientes frente a ella. La comercialización de los datos de Twitter.
\item Ejercicio (para terminar no presencialmente): Canal Twitter para el máster (ejercicio~\ref{sub:canaltwitter}).
\item Ejercicio (foro): Aplicaciones para Twitter (ejercicio~\ref{sub:apptwitter}).
\item Pregunta para reflexión (Twitter): Implicaciones para la venta de publicidad de que el servicio sea accesible mediante API \\
  ¿Qué efectos tiene el que se puedan evitar los twits patrocinados desde una aplicación?
\end{itemize}

%--------------------------------------------------------
\subsubsection{Sesión del 28 de mayo (1 hora)}

%%TODO: Enlaces y subsecciones con breve descripción sobre los ejercicios y vídeos

Wikipedia

\begin{itemize}
\item Profesor: Felipe
\item Ejercicio (clase): Introducción a edición en wikis.
\item Ejercicio (foro): Realizar una mejora en una artículo de Wikipedia.
\item Pregunta para reflexión (Wikipedia): Implicaciones de la ``autoridad'' de Wikipedia \\
  ¿Qué ocurre si hay artículos incorrectos en Wikipedia? ¿Quién tiene responsabilidad? ¿Puede eso plantear problemas legales? ¿A quién? (Fundación Wikimedia, autores, bibliotecarios, etc.)
\item Pregunta para reflexión (Wikipedia): Modelos de sostenibilidad \\
  ¿Será sostemible la Wikipedia a medio y largo plazo? ¿Qué modelos de negocio puede haber que aprovechen el valor creado por Wikipedia? ¿Contribuyen estos negocios de alguna forma a la sostenibilidad de Wikipedia?
\end{itemize}

%--------------------------------------------------------
\subsubsection{Sesión del 3 de junio (2 horas)}

Microblogging (Twitter, Identi.ca, servicios relacionados):

\begin{itemize}
\item Profesor: Jesús
\item Ejercicio (para terminar no presencialmente): Trabajo con Identi.ca
\item Enlaces:
  \begin{itemize}
  \item Identi.ca, \url{http://identi.ca/}
  \item Identi.ca help page, \url{http://identi.ca/doc/help}
  \end{itemize}
\item Demos y ejercicios con servicios relacionados con Twitter
\item Pregunta para reflexión (Twitter): Implicaciones legales cuando se hacen retwits, cuando se citan twits. \\
  ¿Da permiso el autor? ¿De quien es el contenido?
\end{itemize}

%--------------------------------------------------------
\subsubsection{Sesión del 11 de junio (1.5 horas)}

Introducción práctica a FaceBook.

\begin{itemize}
\item Profesores: Jesús
\item Ejercicio (para terminar no presencialmente): Exploración del entorno Facebook
\item Ejercicio (para terminar no presencialmente): Sitio en FaceBook de alumnos del máster (ejercicio~\ref{sub:facebook}).

\item Pregunta para la reflexión (Facebook): Implicaciones para la privacidad. \\
  ¿En qué aspectos hay implicaciones para la privacidad? Muestra escenarios concretos

\item Material complementario: Video ``Qué vida más triste: Mi vida a lo Facebook'' \\
\url{http://www.youtube.com/watch?v=IwwAr0fyMjE}
\item Material complementario: ``Facebook Analytics: The Measurable Ecosystem'' \\
  \url{http://john.webanalyticsdemystified.com/2010/03/14/facebook-analytics-part-i-%E2%80%93-the-measurable-ecosystem/}
\item Material complementario: ``On Platform Ecosystems- A Coming Explosion of the Facebook Ecosystem?''\\
  \url{http://blog.openviewpartners.com/on-platform-ecosystems-a-coming-explosion-of-the-facebook-ecosystem/}
\item Material complementario: ``Facebook Open Graph: What it Means for Privacy'' \\
  \url{http://mashable.com/2010/04/21/open-graph-privacy/}
\item Material complementario: ``A Tale of Two Developer Ecosystems: iPhone versus Facebook'' \\
  \url{http://blogs.gartner.com/ray_valdes/2010/04/21/iphone-vs-facebook-ecosystem/}
\item Enlaces:
  \begin{itemize}
  \item Facebook, \url{http://facebook.com}
  \end{itemize}
\end{itemize}


%--------------------------------------------------------
\subsubsection{Sesión del 11 de junio (1.5 horas)}

Trabajo con LinkedIn.

\begin{itemize}
\item Profesores: Jesús
\item Ejercicio (para terminar no presencialmente): Exploración del entorno LinkedIn.
\item Ejercicio (para terminar no presencialmente): Trabajo con LinkedIn (ejercicio~\ref{sub:linkedin}).
\item Ejercicio (para terminar no presencialmente): Cuentas premium en LinkedIn (ejercicio~\ref{sub:linkedin-premium}).
\item Enlaces:
  \begin{itemize}
  \item LinkedIn, \url{http://linkedin.com}
  \end{itemize}
\end{itemize}

%--------------------------------------------------------
\subsubsection{Sesión del 18 de junio (2 horas)}

Aplicaciones Android de realidad aumentada:

\begin{itemize}
\item Profesores: Jesús
\item Demo: Aplicaciones de realidad aumentada sobre Android. Entre otras, damos un rápido repaso a:
  \begin{itemize}
  \item Wikitude
  \item Layar
  \item Tweeps Around
  \item Augment
  \item Google Googles
  \item Google Googles
  \item Open Signal Maps
  \item GPS Test
  \item Google Sky
  \item QR Droid
  \end{itemize}
\item Ejercicio (foro): Capas de realidad aumentada (ejercicio~\ref{sub:capas-realidad-aumentada}).
\item Ejercicio (foro): Entornos de realidad virtual (ejercicio~\ref{sub:entornos-realidad-virtual}).
\end{itemize}

%--------------------------------------------------------
\subsubsection{Sesión del 25 de junio (2 horas)}

Servicios de video:

\begin{itemize}
\item Videoconferencia: fring.
\item Archivo de videos: YouTube, Blip.tv, Vimeo, etc.
\item Streaming de video: Ustream.
\end{itemize}

%--------------------------------------------------------
\subsection{Fundamentos informáticos}

\subsubsection{Sesión del 16 de septiembre (5 horas)}

\begin{itemize}
\item Profesores: Agustín
\item Presentación: Motivación (transpas: Motivacion\_fundamentos.tex)
\item Ejercicio: ¿Por qué los conceptos de la asignatura son interesantes? \\
  Los alumnos se organizan en grupos de dos o tres, y eligen entre una lista de conceptos de la asignatura. En unos 15-20 min. crean unas transparencias en Google Docs explicando porqué ese concepto es, en opinión del grupo, interesante. Luego se exponen las transparencias, y se comentan entre todos los alumnos.
\item Materiales: Transparencias, tema ``Motivación''
\end{itemize}

%--------------------------------------------------------
%--------------------------------------------------------
\subsubsection{Representación de la información y almacenamiento}

En este tema se verán los conceptos básicos sobre la representación de la información (bit, bytes, etc.), estandares, formatos y otros conceptos que influyen en el almacenamiento y transmición de la información.

Introducción práctica al concepto de Bit.
\begin{itemize}
\item Profesores: Agustín y Gregorio
\item Transparencias: representacion\_info.tex
\item Ejercicio: Introducción a los bits \\
  Se muestra un applet Java en el que el alumno puede ver un truco de adivinación. Tras repetir el proceso una serie de veces, el alumno descubre que el truco se basa en el concepto de bit. Se ofrece una segunda versión del mismo truco. Utilizando estos ejemplos se introducen otros conceptos asociados (capacidad de representación de la información, orden, etc.)
\item Enlaces:
  \begin{itemize}
  \item Primer truco: \url{http://www.cut-the-knot.org/blue/Cards.shtml}
  \item Segndo truco: \url{http://gwydir.demon.co.uk/jo/numbers/binary/how.htm}
  \end{itemize}
\end{itemize}

\begin{itemize}
\item Profesores: Agustín y Gregorio
\item Ejercicio: codificación de caracteres (ASCII) \\
Se pide a los alumnos que hagan su propia propuesta para codificar un conjunto de caracteres (para no hacerlo muy complejo se limita el conjunto). Se pide que escriban su nombre utilizando su propia codificaón y que se lo pasen a otro alumno para que lo descodifique. El segundo alumno utiliza para ello su propio esquema de codificación, con lo que, probablemente, no pueda encontrar el texto. La idea del ejercicio es que descubran que la representación de la información depende de varios factores, entre ellos el convenio.
\item Otro material: \url{http://nickciske.com/tools/binary.php}
\end{itemize}

%--------------------------------------------------------
\subsubsection{Sesión del 17 de septiembre (5 horas)}

\begin{itemize}
\item Profesores: Agustín
\item Se introducen los conceptos de:
\begin{itemize}
\item Byte, Kb, Mb.
\item Representación de caracteres
\item formatos y extensiones
\item compresión
\end{itemize}

\item Materiales: Transparencias, tema ``Representación de la información y almacenamiento''
(representacion\_info.tex)
\end{itemize}


\begin{itemize}
\item Profesores: Gregorio, Agustín
\item Presentación: Color, audio y vídeo. Fundamentos
\item Presentación: Codificación de una imagen: píxeles, resolución, tamaño. Compresión, con pérdidas y sin pérdidas. Capacidad humana de percepción. Audio. Ejemplos.

\item Actividad: fin de fiesta (transparencias: Problema\_ENIAC.tex)

E N I A C

E: Le damos el valor de una nota codificada al codificar con 16 bits en un rango de 0 a 20kHz. Si transforman el valor en Hz a la nota musical, obtienen que es el Do. Le indicamos que lo pase a notación anglosajona (C) y que tome la segunda letra después de la misma (la E).

N: Misma posición que un estándar conocido

I: Compresión

A: Primera letra del nombre de pila de la persona que describió la
máquina analítica diseñada por un matemático inglés que murió 101 años
antes de la invención del lenguaje de programación que es la evolución
de otro lenguaje de programación que originaba de BCPL.

C: Número romano de lo que ocupa en memoria en kilobytes la composición
de un texto en ASCII de 64.000 caracteres (de los cuales 16.250 son con
tilde), a la que se le unen 24.000 instrucciones de un procesador que
puede computar 64 instrucciones diferentes, más 3.200 estados binarios y
20.000 números en ASCII.

\end{itemize}



%--------------------------------------------------------
%--------------------------------------------------------
\subsection{Hardware}

\begin{itemize}
\item Profesores: Gregorio y Agustín

\item Repaso de materiales básicos de la asignatura (documento de programa, sitio web, etc.)

\item Presentación: Procesador, memoria, buses...
  \begin{itemize}
  \item Material: transparencias
  \end{itemize}

\item Actividad: El procesador.

  Cada grupo de alumnos (de tres en tres) diseña una arquitectura de procesador (y lo ponene en twitter). Luego se la pasa a otro grupo que ejecute una secuencia (y pondrán el resultado en twitter como respuesta).

  Las arquitecturas pueden usar los siguientes operadores: sumar, multiplicar, dame memoria (le pasamos dos valores), mete a memoria (le pasamos dos valores), and, or, if, mostrar en pantalla.
\end{itemize}

\begin{itemize}
\item Profesor: Agustín y Gregorio
\item Actividad: Desmontaje de un ordenador.
\item Presentación: memorias, etc.
\end{itemize}


%--------------------------------------------------------
%--------------------------------------------------------
\subsection{Software}

%--------------------------------------------------------
\subsubsection{Sesión del 23 de septiembre (3 horas)}

\begin{itemize}
\item Profesores: Gregorio y Agustín
\item Actividad: Datos e instrucciones

  Se reparten a todos los estudiantes una serie de bits (del 0000 al
  1111) y se les dice que están almacenados en el disco duro. Se les
  pregunta que qué creen que son. Lo normal es que digan que son algo
  parecido a datos.

  Hablaremos entonces de datos e instrucciones. Mostraremos una transpa
  (transpa A) donde se puede ver que el mismo valor tiene un significado
  diferente si se trata de un dato (que en nuestro juego de rol será una
  cifra) o instrucciones (que será una operación matemática).

  En otra transparencia (transpa B), mostraremos una secuencia y les
  haremos salir diciendo qué es lo que son (si dato o instrucción) y qué
  es lo que pasa.

  Realizaremos la misma secuencia, pero empezando por detrás. Los alumnos
  verán que puede que ahora, por haber cambiado el orden, son una cosa
  diferente (antes datos, ahora instrucción, p.ej.).

  Para explicar lo que es una arquitectura hardware, mostraremos una nueva
  variante de las transpas A y B y discutiremos si son posibles y se
  pueden dar. La respuesta es que claro que sí.


\item Actividad: Introducción a algoritmos.

  En grupos (de tres) tendrán que realizar una receta con los siguientes
  ingredientes: pan, mahonesa, atún en aceite enlatado, aceitunas

  \begin{itemize}
  \item Material: foto \\
    \url{http://1.bp.blogspot.com/_b5T-Rd5V_SM/STvNk-tk02I/AAAAAAAABQA/qDleMVIJf6c/s400/atun+nav+8.jpg}
  \end{itemize}

  Los profesores cogerán una de las recetas y uno de ellos hará de robot, tomándose literalmente las instrucciones. Esto mostrará los problemas de los algoritmos y su imprecisión.  Luego, cada grupo realiza un algoritmo, que pasa al siguiente grupo que lo traduce a pseudocódigo, que lo pasa al siguiente que hace de compilador y el último hace de procesador.

\item Presentación: algoritmos, lenguaje natural, lenguaje de programación
etc.
  \begin{itemize}
  \item Material: transparencias
  \end{itemize}
\end{itemize}


\begin{itemize}
\item Profesores: Gregorio y Agustín
\item Presentación y actividad: Nociones de programación: Scratch (2h 30 min)

\item Presentación y actividad: Conceptos avanzados en programación: (2 horas). Compilación, enlazamiento, biblioteca, errata, parche

\item Actividad: Repaso con un trivial en IRC (30 min)

\end{itemize}

\begin{itemize}
\item Profesor: Agustín
\item Presentación: Interprete-compilador y la alternativa tipo Java o C\#. Ventajas e
inconvenientes, sobre todo en lo relativo a otros módulos del master (acceso a código fuente y protección, velocidad de ejecución, complejidad en el mantenimiento, ing. inversa, etc.).

\end{itemize}

%--------------------------------------------------------
\subsection{Redes e Internet}

\subsubsection{Sesión del 24 de septiembre (5 horas)}

Introducción a las redes de ordenadores. 

\begin{itemize}
\item Actividad: Dinámica de los yogures. Dinámica de descubrimiento de cómo se creó la red telefónica básica (centralitas, nodos, etc.)

\item Transparencias: redes.tex

\item Presentación: Creación de Internet. Diseño. Diferencias con la RTB

\item Presentación: Conceptos clave en redes: encaminamiento, router, paquete, firewall, puerto, IP (dirección IP). protocolo.

\item Presentación: Arquitecturas p2p y cliente-servidor
\end{itemize}

\begin{itemize}
\item Presentación y actividades:
  \begin{itemize}
  \item repaso de arquitectura cliente-servidor
  \item un servidor web estático
  \item un servidor web dinámico - servidores de aplicaciones
  \item ejemplo de servidores de aplicaciones. Ventajas e inconvenientes (debate)
  \item HTTP
  \item HTML
  \item wikis y ejemplo de sintaxis wiki - Ejercicio: participación en wikipedia
  \item cookies
  \item HTTP(s) - Certificado
  \item Cómo funcionan los motores de búsqueda
  \item Mash-ups
  \item La nube
  \end{itemize}
\end{itemize}



%% Trabajo con Flickr, SlideShare, Blip.tv, Technorati, HomePuzz, FriendFeed, Google Wave, Skype, Wayback Machine, Jamendo, Magnatune, Last.fm, Meneame, Digg, Delicious, StumbleUpon, OneRiot, Socialmention, WhosTalking, SecondLife, OpenSimulator y OSGrid.

%% \begin{itemize}
%% \item Profesores: Jesús
%% \item Ejercicio (para terminar no presencialmente): Trabajo con Flickr.
%% \item Ejercicio (para terminar no presencialmente): Trabajo con SlideShare.
%% \item Ejercicio (para terminar no presencialmente): Trabajo con Blip.tv.
%% \item Ejercicio (para terminar no presencialmente): Trabajo con Technorati.
%% \item Ejercicio planteado: Exploración de SecondLife o OpenSimulator
%% \item Youtube: Derechos de autor de las televisiones. \\
%% \item Pregunta para reflexión (Youtube): Utilización de material con licencias privativas \\
%% Implicaciones de grabar unos minutos de la televisión en abierto y subirlos a Youtube. El caso particular de los anuncios
%% \item Pregunta para reflexión (SecondLife): Derechos patrimoniales en el mundo virtual \\
%%   \url{http://www.wired.com/gaming/virtualworlds/news/2006/05/70909}
%% \end{itemize}


%% %--------------------------------------------------------
%% \subsubsection{Sesión del 7 de mayo (1 hora)}

%% Trabajo con blogs.

%% \begin{itemize}
%% \item Profesores: Jesús

%% \item Ejercicio (para terminar no presencialmente): Edición en Wikipedia
%% \item Enlaces: WordPress.com, \url{http://wordpress.com}
%% \end{itemize}

%--------------------------------------------------------
%--------------------------------------------------------
\subsection{Otros asuntos}

%--------------------------------------------------------
\subsubsection{Sesión del 18 de junio (1 hora)}

Proyectos fin de máster

\begin{itemize}
\item Profesor: Jesús.
\item Presentaciones de ideas de proyecto fin de máster, y comentarios sobre las mismas.
\end{itemize}

%--------------------------------------------------------
%--------------------------------------------------------
%--------------------------------------------------------
\section{Evaluación}

[Atención: estos criterios aún no han sido fijados completamente para esta edición, puede haber cambios]

La evaluación final del módulo se realizará a partir de las evaluaciones de las diferentes actividades evaluables realizadas. Cada actividad será evaluada con una serie de puntos, según el parecer de los profesores de la asignatura, que seguirán los criterios generales espeficados más adelante.

La nota final estará en función del número de puntos total obtenido, matizados según la opinión general de los profesores sobre los resultados obtenidos. Hay actividades obligatorias (para aprobar la asignatura, el número de puntos mínimo a conseguir en esa actividad es mayor que cero), y otras voluntarias (se puede aprobar la asignatura teniendo cero puntos en esa actividad).

El número mínimo de puntos que ha de obtener un estudiante (matizado por la opinión general de los profesores) es:

\begin{itemize}
\item Para aprobar la asignatura: 100 puntos.
\item Para obtener un notable: 175 puntos.
\item Para obtener un sobresaliente: 250 puntos.
\end{itemize}

\subsection{Actividades evaluables}

Cada actividad tendrá unos criterios de evaluación propios, seguń se indica a continuación. Para cada actividad habrá una puntuación mínima necesaria (0 si no es obligatoria, mayor que 0 si es obligatoria), una puntuación máxima, una descripción de la actividad y, cuando sea posible definirlos, unos criterios generales de puntuación.

Se aconseja a los alumnos que pregunten a los profesores cualquier detalle que pueda no estar claro sobre en qué consisten las actividades, o sobre cómo se puntuarán. En general la evaluación tendrá siempre en cuenta cómo la realización de la actividad ha acercado al alumno a la consecución de los objetivos de la asignatura, y cómo éste ha mostrado las competencias o habilidades correspondientes.

\begin{itemize}
\item Participación en los foros a cuestiones y ejercicios planteados por los profesores,  Mínimo: 20 puntos, máximo: 80 puntos.

Participación en los foros en el Moodle de cada tema en cuestiones planteadas por los profesores durante las videoconferencias y otras actividades. Además, en las asignaturas se han propuesto varios ejercicios (como por ejemplo, participación en varias redes sociales), que también serán evaluadas.

\item Preguntas de autoevaluación. Mínimo: 0 puntos, máximo: 40 puntos.

En algunos temas se ofrecerá la posibilidad de realizar tests de autoevaluación para que los alumnos puedan cerciorarse de que han comprendido y asimilado correctamente los conceptos presentados en la asignatura.

\item Blog sobre la asignatura. Mínimo: 10 puntos, máximo: 40 puntos.

Mantenimiento de un blog sobre la asignatura. Deberá incluir como mínimo del orden de una nota cada semana o 10 días (preferentemente dos o más cada semana o 10 días). Los temas pueden ser los que se van viendo en la asignatura, temas de actualidad relacionados con software libre, aspectos que le resulten interesantes al alumno sobre software libre, etc. El alumno abrirá el blog donde quiera (o abrirá una categoría en su blog, si ya lo tiene), y enviará la url del feed correspondiente a los profesores.


\item Trabajos sobre temas concretos relacionados con el módulo. Mínimo: 10 puntos, máximo: 75 puntos.

Los estudiantes pueden realizar un trabajo individual sobre algún tema relacionado con el módulo. Se podrá entregar como informe o artículo, o como presentación en video (puede ser, por ejemplo, un screencast). Algunos alumnos pueden ser invitados a exponer su trabajo en clase.


\item Colaborar en documentación de los temas tratados en clase usando Google Docs. Mínimo: 10 puntos, máximo: 40 puntos.

Se ha pedido a los alumnos que colaboren en la creación de documentos con Google Docs, que detallen y expliquen los temas desarrollados en las clases. No es preciso que todos los alumnos colaboren en todos los documentos, ni que en todos los documentos hayan colaborado todos los alumnos. Lo normal será que un alumno comience el documento, alguno más lo continue, y en total como mucho tres o cuatro escriban su mayor parte. Luego otros pueden releerlo, y corregir pequeños errores, mejorar las explicaciones, completar detalles, etc.

\item Realización de trabajos sobre seminarios. Mínimo: 10 puntos, máximo: 30 puntos.

Realización de trabajos cortos sobre aspectos concretos mencionados en los seminarios complementarios del módulo. Cada alumno deberá realizar al menos un trabajo para tres de los seminarios. Las presentaciones de cada seminario normalmente ofrecidas mediante streaming, y grabadas para su publicación posterior en Internet..

\item Colaborar en la redacción de un texto sobre el contenido del módulo en WikiBooks. Mínimo: 0 puntos, máximo: 60 puntos.

Con el contenido sobre los temas mantenido en Google Docs, reailizar un texto en Wiki Books, no sobre el máster, sino sobre sus contenidos. De hecho, puede que tenga mśa sentido hacer dos textos, uno sobre la parte más ``tecnológica'', y otro sobre la parte relacionada con la producción de obras libres. Pero pueden comentarse con los profesores otras posibilidades. El resultado debería tener estructura e intención de libro.

\end{itemize}

%% \subsection{Fechas}

%% Evaluación final (todas las actividades habrán de estar terminadas): 15 de septiembre de 2010, salvo las relacionadas con los seminarios que se realizan en septiembre. Para documentos relacionados con ellos, la fecha final será 1 de octubre de 2010.

%% A partir de la primera fecha, los alumnos no deberán tocar (ni subir nuevas versiones) de los documentos correspondientes con las actividades que no correspondan con estos seminarios de septiembre.

%% \subsection{Entrega de documentos para evaluación}


%% Para ser evaluados, los alumnos deberán eviar en el sitio web de la asignatura, mediante el elemento habilitado al efecto, los siguientes documentos:

%% \begin{itemize}
%% \item Resumen de actividades evaluables. Fichero en formato texto en el que se presentarán las actividades que el alumno propone para evaluación, elegidas entre la lista siguiente (que corresponde con las actividades especificadas anteriormente):
%%   \begin{itemize}
%%   \item Cuestiones y ejercicios. Las cuestiones y ejercicios se habrán entregado como respuesta a las entradas en los foros correspondientes).
%%   \item Autoevaluación. Los tests correspondientes se habrán hecho en su momento.
%%   \item Blog. Se indicará la url del blog.
%%   \item Trabajos sobre temas. Se indicará el nombre de los trabajos y el nombre del fichero de entrega (ver más abajo).
%%   \item Documentación con Google Docs. Se indicará los documentos en los que se ha colaborado y la cuenta Google desde la que se ha realizado la actividad. Es importante que los documentos en cuestión hayan sido compatidos con los profesores.
%%   \item Trabajos de seminarios. Se indicará el nombre de los trabajos y el nombre del fichero de entrega (ver más abajo).
%%   \item Wikibook. Se indicará el o los wikibooks en los que se ha participado, un resumen de en qué ha consistido la participación, la url del documento, y el identificador con el que se ha trabajado.
%%   \item Otros. Cualquier otra actividad evaluable que se haya podido realizar.
%%   \end{itemize}
%% \item Fichero o ficheros de entrega de los trabajos realizados sobre temas. Estos ficheros se entregarán en formato PDF.
%% \item Fichero o ficheros de entrega de los trabajos de seminarios. Estos ficheros se entregarán en formato PDF.
%% \end{itemize}


%--------------------------------------------------------
%--------------------------------------------------------
%--------------------------------------------------------
\section{Ejercicios}
\label{sec:ejercicios}

%--------------------------------------------------------
%--------------------------------------------------------
\subsection{Fenómenos emergentes}

%--------------------------------------------------------
\subsubsection{Skype: implicaciones técnicas, legales, de negocio}
\label{sub:skype-implicaciones}

Explora información disponible en Internet (probablmente puedes comenzar por Wikipedia), y trata de entender (y explicar después, como respuesta al ejercicio):

\begin{itemize}
\item Implicaciones técnicas (ancho de banda, consumo de procesador, etc.) para usuarios en puntos bien conectados, y motivos para ello
\item Implicaciones con respecto a la privacidad y seguridad de las comunicaciones
\item Mecanismos de monetización (obtención de ingresos)
\end{itemize}

[Al menos comenta sobre uno de estos asuntos, pero si puedes, hazlo sobre todos ellos]


%--------------------------------------------------------
\subsubsection{Música, aviones y granjeros}
\label{sub:musica-aviones}

Basado en un tema comentado en la presentación de Lawrence Lessig ```Larry Lessig on laws that choke creativity''. ¿Y si en el mundo de la música se hubiera decidido que los ``granjeros'' no tuvieran ``derechos de paso'' sobre sus ``tierras''?

Imagina que en lugar de hablar de los aviones en los años 1920-1930 estuviéramos hablando de la música grabada en los años 1970. En esos años empezó a ser real la posibilidad de que los usuarios pudieran copiar la música grabada, mediante los casettes. Ya desde mucho antes se aplicaba la legislación de derechos de autor heredada de la imprenta, que prohibía la copia (al menos la copia con fines comerciales, y en muchos lugares, también la copia privada). ¿Qué habría ocurrido, en tu opinión, si en esos años se hubiera decidido que copiar música grabada es legal? En la medida de lo posible, trata de razonar tu respuesta comparando con lo que ha ocurrido en el mundo de los aviones.

\subsubsection{ROMs ``customizadas'' para Android}
\label{sec:roms-cust-para}

Busca información sobre ROMs producidas por terceros (diferentes a
Google y fabricantes de móviles) que estén disponibles para
dispositivos Android.

Averigua cómo se relacionan con la comunidad Android y/o Google, y en
qué se diferencia a los sistemas distribuidos por Google u/o los
fabricantes.

\subsubsection{Problemas legales entre CyanogenMod y Android}
\label{sec:probl-legal-entre}

CyanogenMod es una versión personalizada de Android para algunos
teléfonos móviles. Hace algún tiempo, Google envió una petición de
``cease and desist'' a los autores de CyanogenMod para que dejaran de
distribuirla. Tras algunas conversaciones y negociaciones, CyanogenMod
siguió distribuyéndose a los usuarios con el beneplácito de Google.

¿Qué problemas alegaba Google para pedir el cese de la distribución de
CyanogenMod? ¿Cómo arreglaron estos problemas?

%--------------------------------------------------------
\subsubsection{¿Qué es PayPal?}
\label{sub:paypal-quees}

¿Qué es PayPal? Coméntalo al menos desde los siguientes puntos de vista:

\begin{itemize}
\item ¿Por qué hace falta PayPal si ya hay muchos otros medios de pago, como las tarjetas de crédito, que se pueden usar en Internet? En otras palabras, ¿por qué ha tenido éxito PayPal como medio de pago en Internet?
\item ¿Cuál es la operativa normal de PayPal, tanto para pagar como para pagar? ¿En qué se diferencia de otras, como la de la transferencia bancaria o la tarjeta de crédito?
\item ¿Cómo es PayPal como institución? ¿En qué se parece o se diferencia de un banco?
\end{itemize}


%--------------------------------------------------------
\subsection{Redes sociales para la empresa}
\label{sub:redes-sociales-empresa}

Considérense las redes sociales como una herramienta potencialmente útil para empresa, desde dos puntos de vista:

\begin{itemize}
\item Herramienta de gestión de información: Como herramienta para compartir información de diversos tipos (pero siempre en relación con actividades de negocio) entre el personal de la empresa, y con el de otras empresas con las que se tenga relación.
\item Herramienta de apoyo de relaciones con clientes: Como herramienta para compartir información con clientes.
\end{itemize}

En cuanto a redes sociales, considérense fundamentalmente FaceBook, Google+ y Diaspora (complementariamente, pueden considerarse también Twitter e Identi.ca (Status.net), como redes más especializadas en micromensajes).

Para cada una de éstas, estúdiese qué aspectos tecnológicos, jurídicos y de negocio habrá que tener en cuenta para tomar decisiones estratégicas sobre cuál o cuáles serían más apropiadas para las necesidades de una empresa concreta. El ejercicio no trata de resolver un problema concreto, sino de identificar qué elementos serían los fundamentales a analizar para tomar una decisión. Si el alumno quiere, puede hacer ya un análisis preliminar, que permita una primera clasificación de las redes consideradas para cada uno de los aspectos que se consideren.

A modo de ejemplo, se plantean algunas preguntas que probablemente surgirán en cualquier análisis estratégico, y que el alumno ha de tener en cuenta a la hora de identificar los elementos mencionados. Pero el alumno puede identificar también otros elmentos no directamente relacionados con estas preguntas, si los considera relevantes (en ese caso, estaría bien indicar qué tipo de preguntas permitirá responder el elemento mencionado).

Las preguntas planteadas son:

\begin{itemize}
\item ¿Qué tipo de tecnologías clave están implicadas y pueden marcar diferencias?
\item ¿Qué diferencias funcionales hay entre las distintas redes?
\item ¿Qué tecnologías utilizadas son básicas para la monetización y la
sostenibilidad de las redes, y cómo puede eso influir en la evolución de la red?
\item ¿Qué tecnologías, funcionalidades o servicios pueden suponer puntos
de control por las empresas proovedoras de las redes? ¿Cuáles podrían suponer
problemas para las empresas usuarias, incluso siendo útiles? ¿Cuáles favorecen
más la competencia, cuáles situaciones de monopolio?
\item ¿Qué fuentes de ingreso fundamentales se observan (tanto para las redes como para las empresas usuarias)?
\item ¿Qué mecanismos básicos de atracción de usuarios individuales o corporativos?
\item ¿Que relaciones de alianzas / competencia / ecosistemas están
ocurriendo y pueden preverse?
\item ¿Hay impedimentos o barreras potenciales jurídicas para los usos de la red que quiera hacer la empresa?
\item ¿Hay implicaciones para la privacidad, para los secretos comerciales, cómo están tratados tecnológicamente y jurídciamente?
\item ¿Hay implicaciones de propiedad intelectual? ¿En qué sentido? ¿Las
distintas soluciones tecnológicas tienen algún impacto a este respecto?
\item ¿Podría necesitar (o ser conveneinte para) las empresas usuarias algún tipo de regulación, o quizás ya la hay y en ese caso cuál?
\end{itemize}

La mecánica de trabajo será la siguiente:

\begin{itemize}
\item Los profesores expondrán los aspectos más relevantes de las redes sociales consideradas.
\item Los alumnos se reunirán durante un tiempo, en grupos, y tratarán de identificar los elementos pedidos, razonándolos a partir de las preguntas propuestas.
\item Los alumnos realizarán una o varias grabaciones de video, de una duración total máxima de 8 minutos, exponiendo los principales elementos identificados.
\item El aula completa debatirá en el aula, con los profesores, sobre los elementos identificados en los videos.
\item Cada alumno elaborará un informe más detallado sobre los elementos que él considere relevantes, apoyándose en la reflexión en su grupo, en el aula completa, y en su propio análisis personal. Este informe podrá entregarse como un informe tradicional en texto, o como una presentación en video. En ambos casos deberán presentarse también unas transparencias (un máximo de 10) que resuman y ayuden a comprender el informe.
\end{itemize}

%--------------------------------------------------------
%--------------------------------------------------------
\subsection{Herramientas útiles}

%--------------------------------------------------------
\subsubsection{Exploración del entorno Google Apps}
\label{sub:googleapps}

Explora tu cuenta en Google Apps, y las distintas aplicaciones a las que puedes acceder desde ahí.

%--------------------------------------------------------
\subsubsection{Exploración del blog de la EOI}
\label{sub:blog}

Explora el funcionamiento del blog de la EOI. Edita una primera nota en él. Envía un enlace para sindicación de todos los de los alumnos del master.

%--------------------------------------------------------
\subsubsection{Introducción a Google Docs}
\label{sub:googledocs}

Usando la cuenta creada para cada alumno por la EOI en Google Apps, crear un documento con Google Docs, invitar a otros alumnos para que lo compartan, modificar los de otros alumnos que lo hayan compartido, etc. El objetivo de la práctica es familiarizarse con Google Docs, que se usará en otras actividades del máster.

Compartir los documentos creados con el profesor, para que pueda verlos.

%--------------------------------------------------------
\subsubsection{Canal Twitter para el máster}
\label{sub:canaltwitter}

Vamos a crear el canal Twitter para el master. Para ello, todos los alumnos se crearán cuentas en Twitter, e intercambiarán mensajes con el hashtag ``\#ecodigital'', que será el que usaremos en el máster. Además, cada alumno creará una lista Twitter con todos los alumnos y profesores del master que pueda incluir en ella.

Una vez terminado el ejercicio, seguiremos usando el mismo hashtag para cualquier mensaje corto relacionado con el máster. Se aconseja a los alumnos que lo sigan periódicamente.

Por favor, indica también cuál es tu usuario en Twitter, para que te podamos identificar (inclúyelo en tu perfil Moodle).

%--------------------------------------------------------
\subsubsection{Aplicaciones para Twitter}
\label{sub:apptwitter}

El sitio de Twitter permite poner mensajes cortos vía web. Pero hay muchas aplicaciones que lo usan, bien para visualizar los mensajes de distintas formas, o bien para ponerlos. Estas aplicaciones pueden funcionar en el móvil, en el escritorio de un ordenador, o ser aplicaciones web. Elige una, y:

\begin{itemize}
\item Dinos su nombre y la url donde podemos conseguir más información (y ponlo también en el canal Twitter del máster).
\item Explica brevemente su funcionamiento, y para qué sirve.
\end{itemize}

%--------------------------------------------------------
\subsubsection{Trabajo con Identi.ca}
\label{sub:identica}

Crea una cuenta en Identi.ca. Suscríbete a los demás miembros del máster, enlaza con tu cuenta en Twitter. Cread un grupo para la temática del máster, y apuntaros a él (podéis llamarlo !ecodigi). Documentad todo, colaborativamente, usando Google Docs. Para coordinaros, podéis usar el foro cafetería del curso general del Moodle del máster. Contestad a este ejercicio con un enlace a vuestra página en Identi.ca.

%--------------------------------------------------------
\subsubsection{Demos y ejercicios con servicios relacionados con Twitter}
\label{sub:twitter-relacion}

Este ejercicio consiste en probar algunos servicios relacionados con Twitter. Puede comenzarse por los que se ofrecen en los listados siguientes, o bien buscar otros servicios (o aplicaciones) que puedan ser interesantes.

Información y estadisticas sobre temas y hashtags:

\begin{itemize}
\item Información sobre hashtags: \\
   \url{http://www.followthehashtag.com/}

\item Información sobre tendencias (trends) por país, ciudad: \\
   \url{http://whatthetrend.com}

\item Evolución en el tiempo de hashtags: \\
   \url{http://hashtags.org/ecodigital}
   
\item Evolución en el tiempo de hashtags: \\
  \url{http://trendistic.com}

\item Mapa de tendencias: \\
  \url{http://trendsmap.com/}

\item Tweet reach: \\
  \url{http://tweetreach.com/reach?q=ecodigital}

\item Compare two hashtags: \\
  \url{http://hashtagbattle.com}
\end{itemize}

Información y estadísticas sobre autores:

\begin{itemize}
\item Twittercounter: \\
   \url{http://twittercounter.com/}

\item Klout: \\
   \url{http://klout.com}

\item Tweetstats: estadísticas de twits para un usuario (en el tiempo): \\
  \url{http://tweetstats.com/graphs/jgbarah}

\item Xefer: gráficos con estadísticas sobre los patrones temporales de autores: \\
  \url{http://xefer.com/twitter/jgbarah}

\item Tweet cloud: \\
  \url{http://mytweetcloud.com/}

\item Mention map (social network maps of twits) \\
  \url{http://apps.asterisq.com/mentionmap/}

\item Twits in common, etc. \\
  \url{http://twiangulate.com/search/}

\item Simple visualization of the network of a user: \\
  \url{http://www.neuroproductions.be}
\end{itemize}

Presentación de twits:

\begin{itemize}
\item Twitterfall, flujo contínuo de twitts: \\
  \url{http://twitterfall.com/}

\item Monitter, flujo contínuo de twitts: \\
  \url{http://www.monitter.com/}

\item Animated tweets: \\
  \url{http://visibletweets.com}

\item Twitterfountain \\
  \url{http://www.twitterfountain.com/}
\end{itemize}

Alertas, buscadores:

\begin{itemize}
\item Hashtag alert (daily twits, email) \\
  \url{http://www.twilert.com/}

\item Buscador de twits (generates relevant information for the search string) \\
  \url{http://topsy.com/} \url{http://topsy.com/s?q=ecodigital}
\end{itemize}

Organización de información basada en twits:

\begin{itemize}
\item Storify \\
  \url{http://storify.com}

\item Paper.li \\
  \url{http://paper.li}
\end{itemize}

Archivo:

\begin{itemize}
\item 140kit, archivos de twits para investigación (grabación bajo demanda): \\
  \url{http://140kit.com/}

\item TwapperKeeper, archivos de twits para investigación (grabación bajo demanda): \\
  \url{http://twapperkeeper.com/index.php}
\end{itemize}

Otros:

\begin{itemize}
\item Twitometro \\
  \url{http://tweetometro.wordpress.com/}

\item Twits por minuto desde algunas ciudades \\
  \url{http://www.casa.ucl.ac.uk/tom/}
\end{itemize}

%\item Twitstat: \\
%  \url{http://m.twitstat.com/}
%   % Nota: falla (redirige fx4 al sitio para móviles, y ahí no se ven estadísticas)

%\item Twittermeter \\
%  \url{http://twittermeter.com/}
%  % Nota: falla (Sólo datos de una semana de hace tiempo)

%--------------------------------------------------------
\subsubsection{Sitio en FaceBook de alumnos del máster}
\label{sub:facebook}

Vamos a crear el sitio de los alumnos del máster en FaceBook. Puede ser una página, un grupo, y/o algún otro elemento que os resulte adecuado. Naturalmente, lo primero será crear una cuenta en FaceBook, si no la tienes ya. Luego, crear el sitio del máster, invitar al resto de alumnos (y profesores, al menos de esta asignatura) a ser administradores en él.

Podéis usar el foro de cafetería para coordinaros, pero en cuanto esté lista la infraestructura en FaceBook, usadla preferentemente. Decidid no sólo qué elementos usaréis, sino cómo los configuraréis, teniendo en cuenta la visibilidad que queráis darle, la privacidad que queráis mantener, cómo reutilizar otras informaciones sobre el master, etc.

Por favor, indica también cuál es tu usuario en FaceBook, para que te podamos identificar. Documentad todo, colaborativamente, usando Google Docs.

La actividad es colaborativa, entre todos vostros, pero procurad participar todos...

%--------------------------------------------------------
\subsubsection{Trabajo con LinkedIn}
\label{sub:linkedin}

Crea una cuenta en LinkedIn, y completa las partes que se consideren adecuadas del perfil. Incluye, en la medida de lo posible, información vuestra de otros lugares (Twitter, blog, etc.) Crea, colaborativamente, un grupo para la asignatura del máster. Documentad todo, colaborativamente, usando Google Docs. Para coordinaros, podéis usar el foro cafetería del curso general del Moodle del máster. Contestad a este ejercicio con un enlace a vuestro perfil en LinkedIn.

%--------------------------------------------------------
\subsubsection{Cuentas premium en LinkedIn}
\label{sub:linkedin-premium}

En junio de 2011, un usuario de LinkedIn recibe el siguiente mensaje (sólo se muestra el fragmento relevante:

\begin{quotation}
Find and manage high quality contacts with LinkedIn Premium. Get started today and get 1 month free!

LinkedIn Premium helps you be more productive, with tools such as:

\begin{itemize}
\item InMail: Contact anyone on LinkedIn without an introduction - response guaranteed!
\item Expanded Profile Views: See expanded profiles of everyone on LinkedIn, even people outside of your network
\item Who's Viewed My Profile: See a complete list of who's viewed your profile on LinkedIn
\end{itemize}

Get these benefits and many more including the ability to view up to 700 profiles per search.
\end{quotation}

¿Puedes deducir algo sobre el modelo de negocio de LinkedIn a partir de este mensaje? Localiza las condiciones de uso y las de privacidad, y analiza qué puede hacer LinkedIn con tus datos. Busca la entrada en Wikipedia en inglés para el término ``freemium'', y explica si tiene alguna relación con lo que has comentado sobre algún aspecto del modelo de negocio de LinkedIn.

%--------------------------------------------------------
\subsubsection{Capas de realidad aumentada}
\label{sub:capas-realidad-aumentada}

Prueba varias capas (layers) en las aplicaciones de realidad aumentada geolocalizada de Android (Wikitude, Layar, etc.). Elige la que te parezca más interesante, y coméntala.

%--------------------------------------------------------
\subsubsection{Entornos de realidad virtual}
\label{sub:entornos-realidad-virtual}

Prueba algún entorno de realidad virtual tipo Second Life. Para ello tendrás que descargarte el programa que permita acceder al entorno, probablmente crearte una cuenta, y configurar tu avatar. Juega un poco con él en el entorno virtual, y comenta qué te ha parecido. Realiza una captura de pantalla (en la mayoría de los programas para acceder al entorno virtual se puede hacer desde el propio programa) en la parte del mundo virtual que quieras, y anéxala a tu comenatario.

\end{document}
