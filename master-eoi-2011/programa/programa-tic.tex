%%
%% $Id$
%%

\documentclass[a4paper,12pt]{article}
\usepackage[utf8]{inputenc}
\usepackage[spanish]{babel}
\usepackage{geometry}
\usepackage[pdfborder=0 0 0]{hyperref}
\usepackage{url}

\title{Tecnologías de la Información y las Comunicaciones y Herramientas \\
Master en Economía Digital e Industrias Creativas \\
EOI Escuela de Organización Industrial \\
Programa del curso mayo 2011}
\author{Jesús M. González Barahona y Gregorio Robles}

%\date{}

\begin{document}
\maketitle

\newpage

\tableofcontents

\newpage

%--------------------------------------------------------
%--------------------------------------------------------
%--------------------------------------------------------
\section{Equipo de profesores}

\begin{itemize}
\item Jesús M. González Barahona
  \begin{itemize}
  \item Correo electrónico: jgb@libresoft.es
  \end{itemize}
\item Gregorio Robles
  \begin{itemize}
  \item Correo electrónico: grex@libresoft.es
  \end{itemize}
\item Agustín Santos
  \begin{itemize}
  \item Correo electrónico: asantos@libresoft.es
  \end{itemize}
\item Miquel Vidal
  \begin{itemize}
  \item Correo electrónico: mvidal@libresoft.es
  \end{itemize}
\item Felipe Ortega
  \begin{itemize}
  \item Correo electrónico: jfelipe@libresoft.es
  \end{itemize}
\item Israel Herraiz
  \begin{itemize}
  \item Correo electrónico: israel.herraiz@upm.es
  \end{itemize}
\end{itemize}

%--------------------------------------------------------
%--------------------------------------------------------
%--------------------------------------------------------
%\section{Objetivos}

%\begin{itemize}
%\item 
%\end{itemize}

%\section{Metodología}


%--------------------------------------------------------
%--------------------------------------------------------
%--------------------------------------------------------
\section{Programa: Marco general}

Programa del módulo I del máster, en lo que tiene que ver con el área de TIC (tentativo)

%--------------------------------------------------------
%--------------------------------------------------------
\subsection{Introducción}

%--------------------------------------------------------
\subsubsection{Sesión del 20 de mayo (2 horas)}

Presentación de los aspectos fundamentales de la asignatura. Información sobre las capacidades y habilidades de los alumnos en relación con esta asignatura. Motivación e introducción a la infraestructura TIC básica del máster.

\begin{itemize}
\item Profesores: Jesús
\item Presentación: Presentación del área de Tecnologías de la Información y las Comunicaciones en el ámbito del máster.
  %\item Material: Transparencias (tema ``01 - Presentación'')
\item Ejercicio : Conocimientos sobre temas relacionados con la asignatura:
  \begin{itemize}
  \item Internet: desde cuándo, tipo de experiencia, navegadores utilizados, servicios utilizados, tiempo por semana (conectado, de uso activo)
  \item Servicios Internet (conoces, tienes cuenta, usas habitualmente): Facebook, Identi.ca, Tuenti, Twitter, Picotea, Flickr, Youtube, Blip.tv, Vimeo, de.li.cious, LinkedIn, Infojobs, Wikipedia, blogs, otros
  \item Software: programas más usados (entre los libres: OpenOffice, Firefox, Thunderbird, GIMP, Linux, otros)
  \item Hardware: sobremesa, portátil, smartphone, ebook, otros
  \item Idiomas: ¿cómo va el inglés?
  \item Vocablos varios: HTTP, TCP, IP, Mashup, AJAX
  \end{itemize}
\item Explicación: Servidores y federaciones SIP, Skype como caso p2p. Explicación técnica, de negocio y legal sobre Skype (introducción)
\item Ejercicio planteado (para trabajar no presencialmente): Skype: implicaciones técnicas, legales, de negocio (ejercicio~\ref{sub:skype-implicaciones}).
\end{itemize}

%--------------------------------------------------------
%--------------------------------------------------------
\subsection{Fenómenos emergentes}

Fenómenos emergentes, en lo que tienen que ver con el área TIC del máster.

%--------------------------------------------------------
\subsubsection{Sesión del 21 de mayo (2 horas)}

Motivación ¿Por qué hay autores que distribuyen sus trabajos como obras libres? ¿Qué proyectos y sitios web relacionados con las obras libres son más conocidos? ¿Por qué todo esto es importante?

\begin{itemize}
\item Profesores: Jesús
\item Comentario de video: ``Larry Lessig on laws that choke creativity''
\item Presentación: Motivación
\item Ejercicio (foro): Comentarios al video. \\
  Elige un aspecto del video de Larry Lessig, relacionado con la asignatura, que te haya llamado especialmente la atención, y coméntalo brevemente.
\item Material: Transparencias (tema ``Motivación'')
\item Material: video ``Larry Lessig on laws that choke creativity'', Ted.com, \\
  \url{http://www.ted.com/talks/lang/eng/larry_lessig_says_the_law_is_strangling_creativity.html}
\item Ejercicio (planteado en clase, para terminar en el foro) : Música, aviones y granjeros (ejercicio~\ref{sub:musica-aviones}).

 \end{itemize}

\subsubsection{Sesión del 28 de mayo (5 horas)}
\label{sec:sesion-del-28}

Sesiones monográficas sobre Wikipedia y Android

\begin{itemize}
\item Profesores: Felipe, Israel
\item Presentación: Wikipedia
\item Presentación: Android y su ecosistema
\item Dinámica: cuestionario de evaluación de Android (en clase)
\item Ejercicio (foro): ROMs customizadas para Android: ¿cuáles
  conoces o has encontrado?
\item Ejercicio (foro): ¿Cómo se arreglaron los problemas legales
  entre Google y Cyanogen?
\end{itemize}

%--------------------------------------------------------
%--------------------------------------------------------
\subsection{Herramientas y servicios}

Aplicaciones y servicios útiles en el contexto del máster.

%--------------------------------------------------------
\subsubsection{Sesión del 20 de mayo (1 hora)}

Moodle e introducción rápida a otras aplicaciones básicas:

\begin{itemize}
\item Profesores: Jesús
\item Demo: Exploración muy resumida del entorno Moodle.
\item Explicación: Menciones rápidas al entorno Google Apps y al blog.
\item Ejercicio planteado (para trabajar no presencialmente): Exploración del entorno Google Apps (ejercicio~\ref{sub:googleapps}).
\item Ejercicio planteado (para trabajar no presencialmente): Exploración de funcionamiento del blog de la EOI (ejercicio~\ref{sub:blog}).
\end{itemize}

%--------------------------------------------------------
\subsubsection{Sesión del 21 de mayo (2 horas)}

Google Docs y apuntes colaborativos:

\begin{itemize}
\item Profesores: Jesús
\item Ejercicio (para terminar no presencialmente): Introducción a Google Docs (ejercicio~\ref{sub:googledocs}).
\item Planteamiento de actividad: Edición colaborativa: resumen de la asignatura en Google Docs
\end{itemize}

Twitter:

\begin{itemize}
\item Profesores: Jesús
\item Presentación y debate: Introducción al sevicio de Twitter y su API. Diferencia entre servicio y clientes. Impliaciones para el rendimiento y para la innovación. La publicidad y cómo se comportan los clientes frente a ella. La comercialización de los datos de Twitter.
\item Ejercicio (para terminar no presencialmente): Canal Twitter para el máster (ejercicio~\ref{sub:canaltwitter}).
\item Ejercicio (foro): Aplicaciones para Twitter (ejercicio~\ref{sub:apptwitter}).
\item Ejercicio (para terminar no presencialmente): Exploración del entorno Twitter
icio~\ref{sub:identica})..
\end{itemize}

%--------------------------------------------------------
\subsubsection{Sesión del 3 de junio (2 horas)}

Identi.ca:

\begin{itemize}
\item Ejercicio (para terminar no presencialmente): Trabajo con Identi.ca (ejerc
\end{itemize}

%--------------------------------------------------------
%\subsubsection{Sesión del 21 de abril (1 hora)}



%--------------------------------------------------------
%% \subsubsection{Sesión del 20 de abril (2 horas)}

%% Introducción práctica a FaceBook.

%% \begin{itemize}
%% \item Profesores: Jesús
%% \item Ejercicio (para terminar no presencialmente): Exploración del entorno Facebook
%% \item Ejercicio (para terminar no presencialmente): Sitio en FaceBook de alumnos del máster. \\
%%   Vamos a crear el sitio de los alumnos del máster en FaceBook. Puede ser una página, un grupo, y/o algún otro elemento que os resulte adecuado. Naturalmente, lo primero será crear una cuenta en FaceBook, si no la tienes ya. Luego, crear el sitio del máster, invitar al resto de alumnos (y profesores, al menos de esta asignatura) a ser administradores en él.

%% Podéis usar este foro para coordinaros, pero en cuanto esté lista la infraestructura en FaceBook, usadla preferentemente. Decidid no sólo qué elementos usaréis, sino cómo los configuraréis, teniendo en cuenta la visibilidad que queráis darle, la privacidad que queráis mantener, cómo reutilizar otras informaciones sobre el master, etc.

%% Por favor, indica también cuál es tu usuario en FaceBook, para que te podamos identificar.

%% La actividad es colaborativa, entre todos vostros, pero procurad participar todos...
%% \end{itemize}

%% %--------------------------------------------------------
%% \subsubsection{Sesión del 29 de abril (2 horas)}

%% Introducción práctica a Twitter y Google Apps (especialmente, Google Docs).


%% %--------------------------------------------------------
%% \subsubsection{Sesión del 7 de mayo (1 hora)}

%% Trabajo con blogs.

%% \begin{itemize}
%% \item Profesores: Jesús

%% \item Ejercicio (para terminar no presencialmente): Edición en Wikipedia
%% \item Enlaces: WordPress.com, \url{http://wordpress.com}
%% \end{itemize}

%% %--------------------------------------------------------
%% \subsubsection{Sesión del 28 de mayo (1 hora)}

%% Trabajo con Identi.ca, LinkedIn.

%% \begin{itemize}
%% \item Profesores: Jesús
%% \item Ejercicio (para terminar no presencialmente): Trabajo con LinkedIn. \\
%%   Crea una cuenta en LinkedIn, y completa las partes que se consideren adecuadas del perfil. Incluye, en la medida de lo posible, información vuestra de otros lugares (Twitter, blog, etc.) Crea, colaborativamente, un grupo para la asignatura del máster. Documentad todo, colaborativamente, usando Google Docs. Para coordinaros, podéis usar el foro cafetería del curso general del Moodle del máster. Contestad a este ejercicio con un enlace a vuestro perfil en LinkedIn.

%% \item Enlaces:
%%   \begin{itemize}
%%   \item LinkedIn, \url{http://linkedin.com}
%%   \item Identi.ca, \url{http://identi.ca/}
%%   \item Identi.ca help page, \url{http://identi.ca/doc/help}
%%   \end{itemize}
%% \end{itemize}

%% %--------------------------------------------------------
%% \subsubsection{Sesión del 19 de junio (1 hora)}

%% Trabajo con Flickr, SlideShare, Blip.tv, Technorati, HomePuzz, FriendFeed, Google Wave, Skype, Wayback Machine, Jamendo, Magnatune, Last.fm, Meneame, Digg, Delicious, StumbleUpon, OneRiot, Socialmention, WhosTalking, SecondLife, OpenSimulator y OSGrid.

%% \begin{itemize}
%% \item Profesores: Jesús
%% \item Ejercicio (para terminar no presencialmente): Trabajo con Flickr.
%% \item Ejercicio (para terminar no presencialmente): Trabajo con SlideShare.
%% \item Ejercicio (para terminar no presencialmente): Trabajo con Blip.tv.
%% \item Ejercicio (para terminar no presencialmente): Trabajo con Technorati.
%% \item Ejercicio planteado: Exploración de SecondLife o OpenSimulator
%% \end{itemize}

%% Algunas preguntas a responder:

%% \begin{itemize}
%% \item Twitter: Implicaciones legales cuando se hacen retwits. \\
%%   ¿Da permiso el autor? ¿De quien es el contenido?
%% \item Facebook: Implicaciones para la privacidad. \\
%%   ¿En qué aspectos hay implicaciones para la privacidad? Muestra escenarios concretos
%% \item Youtube: Derechos de autor de las televisiones. \\
%%   Implicaciones de grabar unos minutos de la televisión en abierto y subirlos a Youtube. El caso particular de los anuncios
%% \item SecondLife: Derechos patrimoniales en el mundo virtual \\
%%   \url{http://www.wired.com/gaming/virtualworlds/news/2006/05/70909}
%% \end{itemize}


%% %--------------------------------------------------------
%% %--------------------------------------------------------
%% \subsection{03 - Introducción y motivación}

%% Introducción a la asignatura, y motivación de su interés en el contexto del máster.

%% %--------------------------------------------------------
%% \subsubsection{Sesión del 29 de abril (5 horas)}

%% \begin{itemize}
%% \item Profesores: Agustín
%% \item Presentación: Motivación
%% \item Ejercicio: ¿Por qué los conceptos de la asignatura son interesantes? \\
%%   Los alumnos se organizan en grupos de dos o tres, y eligen entre una lista de conceptos de la asignatura. En unos 15-20 min. crean unas transparencias en Google Docs explicando porqué ese concepto es, en opinión del grupo, interesante. Luego se exponen las transparencias, y se comentan entre todos los alumnos.
%% \item Materiales: Transparencias, tema ``Motivación''
%% \end{itemize}

%% %--------------------------------------------------------
%% %--------------------------------------------------------
%% \subsection{04 - Representación de la información y almacenamiento}

%% En este tema se verán los conceptos básicos sobre la representación de la información (bit, bytes, etc.), estandares, formatos y otros conceptos que influyen en el almacenamiento y transmición de la información.

%% \subsubsection{Sesión del 29 de abril (2 horas)}

%% Introducción práctica al concepto de Bit.

%% \begin{itemize}
%% \item Profesores: Agustín
%% \item Ejercicio: Introducción a los bits \\
%%   Se muestra un applet Java en el que el alumno puede ver un truco de adivinación. Tras repetir el proceso una serie de veces, el alumno descubre que el truco se basa en el concepto de bit. Se ofrece una segunda versión del mismo truco. Utilizando estos ejemplos se introducen otros conceptos asociados (capacidad de representación de la información, orden, etc.)
%% \item Enlaces:
%%   \begin{itemize}
%%   \item Primer truco: \url{http://www.cut-the-knot.org/blue/Cards.shtml}
%%   \item Segndo truco: \url{http://gwydir.demon.co.uk/jo/numbers/binary/how.htm}
%%   \end{itemize}
%% \end{itemize}

%% %--------------------------------------------------------
%% \subsubsection{Sesión del 7 de mayo (1 hora)}

%% \begin{itemize}
%% \item Profesores: Agustín
%% \item Ejercicio: codificación de caracteres (ASCII) \\
%% Se pide a los alumnos que hagan su propia propuesta para codificar un conjunto de caracteres (para no hacerlo muy complejo se limita el conjunto). Se pide que escriban su nombre utilizando su propia codificaón y que se lo pasen a otro alumno para que lo descodifique. El segundo alumno utiliza para ello su propio esquema de codificación, con lo que, probablemente, no pueda encontrar el texto. La idea del ejercicio es que descubran que la representación de la información depende de varios factores, entre ellos el convenio.
%% \item Otro material: \url{http://nickciske.com/tools/binary.php}
%% \end{itemize}

%% %--------------------------------------------------------
%% \subsubsection{Sesión del 7 de mayo (3 horas)}

%% \begin{itemize}
%% \item Profesores: Agustín
%% \item Se introducen los conceptos de:
%% \begin{itemize}
%% \item Byte, Kb, Mb.
%% \item Representación de caracteres
%% \item formatos y extensiones
%% \item compresión
%% \end{itemize}
%% \item Materiales: Transparencias, tema ``Representación de la información y almacenamiento''
%% \end{itemize}

%% %--------------------------------------------------------
%% %--------------------------------------------------------
%% \subsubsection{Sesión del 4 de junio (1.5 horas)}

%% \begin{itemize}
%% \item Profesores: Gregorio, Agustín
%% \item Presentación: Color, audio y vídeo. Fundamentos
%% \item Presentación: Codificación de una imagen: píxeles, resolución, tamaño. Compresión, con pérdidas y sin pérdidas. Capacidad humana de percepción. Audio. Ejemplos.
%% \end{itemize}

%% %--------------------------------------------------------
%% %--------------------------------------------------------
%% \subsection{05 - Hardware}

%% %--------------------------------------------------------
%% \subsubsection{Sesión del 14 de mayo (3 horas)}

%% \begin{itemize}
%% \item Profesores: Gregorio y Agustín

%% \item Repaso de materiales básicos de la asignatura (documento de programa, sitio web, etc.)

%% \item Presentación: Procesador, memoria, buses...
%%   \begin{itemize}
%%   \item Material: transparencias
%%   \end{itemize}

%% \item Actividad: El procesador.

%%   Cada grupo de alumnos (de tres en tres) diseña una arquitectura de procesador (y lo ponene en twitter). Luego se la pasa a otro grupo que ejecute una secuencia (y pondrán el resultado en twitter como respuesta).

%%   Las arquitecturas pueden usar los siguientes operadores: sumar, multiplicar, dame memoria (le pasamos dos valores), mete a memoria (le pasamos dos valores), and, or, if, mostrar en pantalla.

%% \item Presentación: Tipos de memoria y otro hardware importante
%%   \begin{itemize}
%%   \item Material: transparencias
%%   \item Videos: \url{http://libresoft.es/grex/IBAS/videos}
%%   \end{itemize}
%% \end{itemize}

%% %--------------------------------------------------------
%% \subsubsection{Sesión del 28 de mayo}

%% \begin{itemize}
%% \item Profesor: Agustín
%% \item Actividad: Desmontaje de un ordenador.
%% \item Presentación: memorias, etc.
%% \end{itemize}


%% %--------------------------------------------------------
%% %--------------------------------------------------------
%% \subsection{06 - Software}

%% %--------------------------------------------------------
%% \subsubsection{Sesión del 14 de mayo (2 horas)}

%% \begin{itemize}
%% \item Profesores: Gregorio y Agustín
%% \item Actividad: Datos e instrucciones

%%   Se reparten a todos los estudiantes una serie de bits (del 0000 al
%%   1111) y se les dice que están almacenados en el disco duro. Se les
%%   pregunta que qué creen que son. Lo normal es que digan que son algo
%%   parecido a datos.

%%   Hablaremos entonces de datos e instrucciones. Mostraremos una transpa
%%   (transpa A) donde se puede ver que el mismo valor tiene un significado
%%   diferente si se trata de un dato (que en nuestro juego de rol será una
%%   cifra) o instrucciones (que será una operación matemática).
  
%%   En otra transparencia (transpa B), mostraremos una secuencia y les
%%   haremos salir diciendo qué es lo que son (si dato o instrucción) y qué
%%   es lo que pasa.
  
%%   Realizaremos la misma secuencia, pero empezando por detrás. Los alumnos
%%   verán que puede que ahora, por haber cambiado el orden, son una cosa
%%   diferente (antes datos, ahora instrucción, p.ej.).
  
%%   Para explicar lo que es una arquitectura hardware, mostraremos una nueva
%%   variante de las transpas A y B y discutiremos si son posibles y se
%%   pueden dar. La respuesta es que claro que sí.


%% \item Actividad: Introducción a algoritmos.

%%   En grupos (de tres) tendrán que realizar una receta con los siguientes
%%   ingredientes: pan, mahonesa, atún en aceite enlatado, aceitunas
  
%%   \begin{itemize}
%%   \item Material: foto \\
%%     \url{http://1.bp.blogspot.com/_b5T-Rd5V_SM/STvNk-tk02I/AAAAAAAABQA/qDleMVIJf6c/s400/atun+nav+8.jpg}
%%   \end{itemize}
  
%%   Los profesores cogerán una de las recetas y uno de ellos hará de robot, tomándose literalmente las instrucciones. Esto mostrará los problemas de los algoritmos y su imprecisión.  Luego, cada grupo realiza un algoritmo, que pasa al siguiente grupo que lo traduce a pseudocódigo, que lo pasa al siguiente que hace de compilador y el último hace de procesador.
  
%% \item Presentación: algoritmos, lenguaje natural, lenguaje de programación
%% etc.
%%   \begin{itemize}
%%   \item Material: transparencias
%%   \end{itemize}

%% \item Actividad: fin de fiesta

%% E N I A C

%% E: Le damos el programa (y la arquitectura) y sale la representación
%% ASCII de la E

%% N: No se la damos

%% I: Compresión

%% A: Primera letra del nombre de pila de la persona que describió la
%% máquina analítica diseñada por un matemático inglés que murió 101 años
%% antes de la invención del lenguaje de programación que es la evolución
%% de otro lenguaje de programación que originaba de BCPL.

%% C: Número romano de lo que ocupa en memoria en kilobytes la composición
%% de un texto en ASCII de 64.000 caracteres (de los cuales 16.250 son con
%% tilde), a la que se le unen 24.000 instrucciones de un procesador que
%% puede computar 64 instrucciones diferentes, más 3.200 estados binarios y
%% 20.000 números en ASCII.

%% \end{itemize}


%% %--------------------------------------------------------
%% \subsubsection{Sesión del 21 de mayo (5 horas)}


%% \begin{itemize}
%% \item Profesores: Gregorio y Agustín
%% \item Presentación y actividad: Nociones de programación: Scratch (2h 30 min)

%% \item Presentación y actividad: Conceptos avanzados en programación: (2 horas). Compilación, enlazamiento, biblioteca, errata, parche

%% \item Actividad: Repaso con un trivial en IRC (30 min)

%% \end{itemize}

%% %--------------------------------------------------------
%% \subsubsection{Sesión del 28 de mayo}

%% \begin{itemize}
%% \item Profesor: Agustín
%% \item Presentación: Interprete-compilador y la alternativa tipo Java o C\#. Ventajas e
%% inconvenientes, sobre todo en lo relativo a otros módulos del master (acceso a código fuente y protección, velocidad de ejecución, complejidad en el mantenimiento, ing. inversa, etc.).

%% \end{itemize}

%% %--------------------------------------------------------
%% %--------------------------------------------------------
%% \subsection{07 - La industria de la informática}

%% %--------------------------------------------------------
%% %--------------------------------------------------------
%% \subsubsection{Sesión del 4 de junio (3.5 horas)}

%% Android, modelos de negocio industria informática y multimedia

%% \begin{itemize}
%% \item Profesores: Roberto Calvo, Gregorio
%% \item Presentación: Charla introductoria sobre Android (Roberto Calvo)

%% \item Actividad: Presentación-debate sobre la industria informática y modelos de negocio.

%% Se presentan las diferentes "capas" de la industria informática, empezando por el hardware, el sistema operativo, las bibliotecas, las aplicaciones, los compiladores, las utilidades, los sistemas empotrados... y se reflexiona sobre modelos de negocio (directos e indirectos) que se pueden encontrar o se han realizado en la industria informática.

%% \end{itemize}

%% %--------------------------------------------------------
%% %--------------------------------------------------------
%% \subsection{08 - Internet}

%% %--------------------------------------------------------
%% \subsubsection{Sesión del 11 de junio (5 horas)}

%% Introducción a las redes de ordenadores.

%% \begin{itemize}
%% \item Actividad: Dinámica de los yogures. Dinámica de descubrimiento de cómo se creó la red telefónica básica (centralitas, nodos, etc.)

%% \item Presentación: Creación de Internet. Diseño. Diferencias con la RTB

%% \item Presentación: Conceptos clave en redes: encaminamiento, router, paquete, firewall, puerto, IP (dirección IP). protocolo.

%% \item Presentación: Arquitecturas p2p y cliente-servidor
%% \end{itemize}

%% %--------------------------------------------------------
%% %--------------------------------------------------------
%% \subsubsection{Sesión del 18 de junio}


%% \begin{itemize}
%% \item Presentación y actividades:
%%   \begin{itemize}
%%   \item repaso de arquitectura cliente-servidor
%%   \item un servidor web estático
%%   \item un servidor web dinámico - servidores de aplicaciones
%%   \item ejemplo de servidores de aplicaciones. Ventajas e inconvenientes (debate)
%%   \item HTTP
%%   \item HTML
%%   \item wikis y ejemplo de sintaxis wiki - Ejercicio: participación en wikipedia
%%   \item cookies
%%   \item HTTP(s) - Certificado
%%   \item Cómo funcionan los motores de búsqueda
%%   \item Mash-ups
%%   \item La nube
%%   \end{itemize}
%% \end{itemize}

%--------------------------------------------------------
%--------------------------------------------------------
%\subsection{09 - Seguridad y privacidad}



%--------------------------------------------------------
%--------------------------------------------------------
%--------------------------------------------------------
\section{Evaluación}

[Atención: estos criterios aún no han sido fijados completamente para esta edición, puede haber cambios]

La evaluación final del módulo se realizará a partir de las evaluaciones de las diferentes actividades evaluables realizadas. Cada actividad será evaluada con una serie de puntos, según el parecer de los profesores de la asignatura, que seguirán los criterios generales espeficados más adelante.

La nota final estará en función del número de puntos total obtenido, matizados según la opinión general de los profesores sobre los resultados obtenidos. Hay actividades obligatorias (para aprobar la asignatura, el número de puntos mínimo a conseguir en esa actividad es mayor que cero), y otras voluntarias (se puede aprobar la asignatura teniendo cero puntos en esa actividad).

El número mínimo de puntos que ha de obtener un estudiante (matizado por la opinión general de los profesores) es:

\begin{itemize}
\item Para aprobar la asignatura: 100 puntos.
\item Para obtener un notable: 175 puntos.
\item Para obtener un sobresaliente: 250 puntos.
\end{itemize}

\subsection{Actividades evaluables}

Cada actividad tendrá unos criterios de evaluación propios, seguń se indica a continuación. Para cada actividad habrá una puntuación mínima necesaria (0 si no es obligatoria, mayor que 0 si es obligatoria), una puntuación máxima, una descripción de la actividad y, cuando sea posible definirlos, unos criterios generales de puntuación.

Se aconseja a los alumnos que pregunten a los profesores cualquier detalle que pueda no estar claro sobre en qué consisten las actividades, o sobre cómo se puntuarán. En general la evaluación tendrá siempre en cuenta cómo la realización de la actividad ha acercado al alumno a la consecución de los objetivos de la asignatura, y cómo éste ha mostrado las competencias o habilidades correspondientes.

\begin{itemize}
\item Participación en los foros a cuestiones y ejercicios planteados por los profesores,  Mínimo: 20 puntos, máximo: 80 puntos.

Participación en los foros en el Moodle de cada tema en cuestiones planteadas por los profesores durante las videoconferencias y otras actividades. Además, en las asignaturas se han propuesto varios ejercicios (como por ejemplo, participación en varias redes sociales), que también serán evaluadas.

\item Preguntas de autoevaluación. Mínimo: 0 puntos, máximo: 40 puntos.

En algunos temas se ofrecerá la posibilidad de realizar tests de autoevaluación para que los alumnos puedan cerciorarse de que han comprendido y asimilado correctamente los conceptos presentados en la asignatura.

\item Blog sobre la asignatura. Mínimo: 10 puntos, máximo: 40 puntos.

Mantenimiento de un blog sobre la asignatura. Deberá incluir como mínimo del orden de una nota cada semana o 10 días (preferentemente dos o más cada semana o 10 días). Los temas pueden ser los que se van viendo en la asignatura, temas de actualidad relacionados con software libre, aspectos que le resulten interesantes al alumno sobre software libre, etc. El alumno abrirá el blog donde quiera (o abrirá una categoría en su blog, si ya lo tiene), y enviará la url del feed correspondiente a los profesores.


\item Trabajos sobre temas concretos relacionados con el módulo. Mínimo: 10 puntos, máximo: 75 puntos.

Los estudiantes pueden realizar un trabajo individual sobre algún tema relacionado con el módulo. Se podrá entregar como informe o artículo, o como presentación en video (puede ser, por ejemplo, un screencast). Algunos alumnos pueden ser invitados a exponer su trabajo en clase.


\item Colaborar en documentación de los temas tratados en clase usando Google Docs. Mínimo: 10 puntos, máximo: 40 puntos.

Se ha pedido a los alumnos que colaboren en la creación de documentos con Google Docs, que detallen y expliquen los temas desarrollados en las clases. No es preciso que todos los alumnos colaboren en todos los documentos, ni que en todos los documentos hayan colaborado todos los alumnos. Lo normal será que un alumno comience el documento, alguno más lo continue, y en total como mucho tres o cuatro escriban su mayor parte. Luego otros pueden releerlo, y corregir pequeños errores, mejorar las explicaciones, completar detalles, etc.

\item Realización de trabajos sobre seminarios. Mínimo: 10 puntos, máximo: 30 puntos.

Realización de trabajos cortos sobre aspectos concretos mencionados en los seminarios complementarios del módulo. Cada alumno deberá realizar al menos un trabajo para tres de los seminarios. Las presentaciones de cada seminario normalmente ofrecidas mediante streaming, y grabadas para su publicación posterior en Internet..

\item Colaborar en la redacción de un texto sobre el contenido del módulo en WikiBooks. Mínimo: 0 puntos, máximo: 60 puntos.

Con el contenido sobre los temas mantenido en Google Docs, reailizar un texto en Wiki Books, no sobre el máster, sino sobre sus contenidos. De hecho, puede que tenga mśa sentido hacer dos textos, uno sobre la parte más ``tecnológica'', y otro sobre la parte relacionada con la producción de obras libres. Pero pueden comentarse con los profesores otras posibilidades. El resultado debería tener estructura e intención de libro.

\end{itemize}

%% \subsection{Fechas}

%% Evaluación final (todas las actividades habrán de estar terminadas): 15 de septiembre de 2010, salvo las relacionadas con los seminarios que se realizan en septiembre. Para documentos relacionados con ellos, la fecha final será 1 de octubre de 2010.

%% A partir de la primera fecha, los alumnos no deberán tocar (ni subir nuevas versiones) de los documentos correspondientes con las actividades que no correspondan con estos seminarios de septiembre.

%% \subsection{Entrega de documentos para evaluación}


%% Para ser evaluados, los alumnos deberán eviar en el sitio web de la asignatura, mediante el elemento habilitado al efecto, los siguientes documentos:

%% \begin{itemize}
%% \item Resumen de actividades evaluables. Fichero en formato texto en el que se presentarán las actividades que el alumno propone para evaluación, elegidas entre la lista siguiente (que corresponde con las actividades especificadas anteriormente):
%%   \begin{itemize}
%%   \item Cuestiones y ejercicios. Las cuestiones y ejercicios se habrán entregado como respuesta a las entradas en los foros correspondientes).
%%   \item Autoevaluación. Los tests correspondientes se habrán hecho en su momento.
%%   \item Blog. Se indicará la url del blog.
%%   \item Trabajos sobre temas. Se indicará el nombre de los trabajos y el nombre del fichero de entrega (ver más abajo).
%%   \item Documentación con Google Docs. Se indicará los documentos en los que se ha colaborado y la cuenta Google desde la que se ha realizado la actividad. Es importante que los documentos en cuestión hayan sido compatidos con los profesores.
%%   \item Trabajos de seminarios. Se indicará el nombre de los trabajos y el nombre del fichero de entrega (ver más abajo).
%%   \item Wikibook. Se indicará el o los wikibooks en los que se ha participado, un resumen de en qué ha consistido la participación, la url del documento, y el identificador con el que se ha trabajado.
%%   \item Otros. Cualquier otra actividad evaluable que se haya podido realizar.
%%   \end{itemize}
%% \item Fichero o ficheros de entrega de los trabajos realizados sobre temas. Estos ficheros se entregarán en formato PDF.
%% \item Fichero o ficheros de entrega de los trabajos de seminarios. Estos ficheros se entregarán en formato PDF.
%% \end{itemize}


%--------------------------------------------------------
%--------------------------------------------------------
%--------------------------------------------------------
\section{Ejercicios}
\label{sec:ejercicios}

%--------------------------------------------------------
%--------------------------------------------------------
\subsection{Fenómenos emergentes}

%--------------------------------------------------------
\subsubsection{Skype: implicaciones técnicas, legales, de negocio}
\label{sub:skype-implicaciones}

Explora información disponible en Internet (probablmente puedes comenzar por Wikipedia), y trata de entender (y explicar después, como respuesta al ejercicio):

\begin{itemize}
\item Implicaciones técnicas (ancho de banda, consumo de procesador, etc.) para usuarios en puntos bien conectados, y motivos para ello
\item Implicaciones con respecto a la privacidad y seguridad de las comunicaciones
\item Mecanismos de monetización (obtención de ingresos)
\end{itemize}

[Al menos comenta sobre uno de estos asuntos, pero si puedes, hazlo sobre todos ellos]


%--------------------------------------------------------
\subsubsection{Música, aviones y granjeros}
\label{sub:musica-aviones}

Basado en un tema comentado en la presentación de Lawrence Lessig ```Larry Lessig on laws that choke creativity''. ¿Y si en el mundo de la música se hubiera decidido que los ``granjeros'' no tuvieran ``derechos de paso'' sobre sus ``tierras''?

Imagina que en lugar de hablar de los aviones en los años 1920-1930 estuviéramos hablando de la música grabada en los años 1970. En esos años empezó a ser real la posibilidad de que los usuarios pudieran copiar la música grabada, mediante los casettes. Ya desde mucho antes se aplicaba la legislación de derechos de autor heredada de la imprenta, que prohibía la copia (al menos la copia con fines comerciales, y en muchos lugares, también la copia privada). ¿Qué habría ocurrido, en tu opinión, si en esos años se hubiera decidido que copiar música grabada es legal? En la medida de lo posible, trata de razonar tu respuesta comparando con lo que ha ocurrido en el mundo de los aviones.

\subsubsection{ROMs "customizadas" para Android}
\label{sec:roms-cust-para}

Busca información sobre ROMs producidas por terceros (diferentes a
Google y fabricantes de móviles) que estén disponibles para
dispositivos Android.

Averigua cómo se relacionan con la comunidad Android y/o Google, y en
qué se diferencia a los sistemas distribuidos por Google u/o los
fabricantes.

\subsubsection{Problemas legales entre CyanogenMod y Android}
\label{sec:probl-legal-entre}

CyanogenMod es una versión personalizada de Android para algunos
teléfonos móviles. Hace algún tiempo, Google envió una petición de
``cease and desist'' a los autores de CyanogenMod para que dejaran de
distribuirla. Tras algunas conversaciones y negociaciones, CyanogenMod
siguió distribuyéndose a los usuarios con el beneplácito de Google.

¿Qué problemas alegaba Google para pedir el cese de la distribución de
CyanogenMod? ¿Cómo arreglaron estos problemas?

%--------------------------------------------------------
%--------------------------------------------------------
\subsection{Herramientas útiles}

%--------------------------------------------------------
\subsubsection{Exploración del entorno Google Apps}
\label{sub:googleapps}

Explora tu cuenta en Google Apps, y las distintas aplicaciones a las que puedes acceder desde ahí.

%--------------------------------------------------------
\subsubsection{Exploración del blog de la EOI}
\label{sub:blog}

Explora el funcionamiento del blog de la EOI. Edita una primera nota en él. Envía un enlace para sindicación de todos los de los alumnos del master.

%--------------------------------------------------------
\subsubsection{Introducción a Google Docs}
\label{sub:googledocs}

Usando la cuenta creada para cada alumno por la EOI en Google Apps, crear un documento con Google Docs, invitar a otros alumnos para que lo compartan, modificar los de otros alumnos que lo hayan compartido, etc. El objetivo de la práctica es familiarizarse con Google Docs, que se usará en otras actividades del máster.

Compartir los documentos creados con el profesor, para que pueda verlos.

%--------------------------------------------------------
\subsubsection{Canal Twitter para el máster}
\label{sub:canaltwitter}

Vamos a crear el canal Twitter para el master. Para ello, todos los alumnos se crearán cuentas en Twitter, e intercambiarán mensajes con el hashtag ``\#ecodigi'', que será el que usaremos en el máster. Además, cada alumno creará una lista Twitter con todos los alumnos y profesores del master que pueda incluir en ella.

Una vez terminado el ejercicio, seguiremos usando el mismo hashtag para cualquier mensaje corto relacionado con el máster. Se aconseja a los alumnos que lo sigan periódicamente.

Por favor, indica también cuál es tu usuario en Twitter, para que te podamos identificar (inclúyelo en tu perfil Moodle).

%--------------------------------------------------------
\subsubsection{Aplicaciones para Twitter}
\label{sub:apptwitter}

El sitio de Twitter permite poner mensajes cortos vía web. Pero hay muchas aplicaciones que lo usan, bien para visualizar los mensajes de distintas formas, o bien para ponerlos. Estas aplicaciones pueden funcionar en el móvil, en el escritorio de un ordenador, o ser aplicaciones web. Elige una, y:

\begin{itemize}
\item Dinos su nombre y la url donde podemos conseguir más información (y ponlo también en el canal Twitter del máster).
\item Explica brevemente su funcionamiento, y para qué sirve.
\end{itemize}

%--------------------------------------------------------
\subsubsection{Trabajo con Identi.ca}
\label{sub:identica}

Crea una cuenta en Identi.ca. Suscríbete a los demás miembros del máster, enlaza con tu cuenta en Twitter. Cread un grupo para la temática del máster, y apuntaros a él (podéis llamarlo !ecodigi). Documentad todo, colaborativamente, usando Google Docs. Para coordinaros, podéis usar el foro cafetería del curso general del Moodle del máster. Contestad a este ejercicio con un enlace a vuestra página en Identi.ca.


\end{document}
